\documentclass[a4paper,12pt]{article}

\usepackage[utf8]{inputenc}
\usepackage[T1]{fontenc} % per i caratteri accentati corretti in PDF
\usepackage[italian]{babel}
\usepackage{geometry}
\usepackage{setspace}
\usepackage{enumitem}
\usepackage{titlesec}
\usepackage{tocloft}
\usepackage{graphicx}

\renewcommand{\contentsname}{Indice}

\geometry{margin=2.5cm}
\setstretch{1.2}

\titleformat{\section}{\large\bfseries}{\thesection}{1em}{}
\titleformat{\subsection}{\mdseries\bfseries}{\thesubsection}{1em}{}

\begin{document}

\begin{center}
    \includegraphics[width=9.5cm]{../assets/logo7ZPUs.jpeg}\\
    \small\hspace{10cm} 7zpus.swe@gmail.com\\
    \vspace{0.5cm}
    \Large \textbf{Analisi dei Capitolati di Progetto 2025/2026}\\
\end{center}

\vspace{0.3cm}
\hrule
\vspace{0.5cm}

\tableofcontents

\newpage

\section{Tabella di Versionamento}
\begin{tabular}{|c|c|c|c|}
    \hline
    \textbf{Versione} & \textbf{Data} & \textbf{Autore} & \textbf{Descrizione}                       \\
    \hline
    1.0               & 17/10/2025    & Soligo Lorenzo  & Creazione del documento e stesura iniziale \\
    \hline

\end{tabular}

\section{Introduzione}
In questo documento sono analizzati i 9 Capitolati di Progetto proposti per
l'anno accademico 2025/2026, con l'obiettivo di valutarne la complessità, i
rischi associati e le potenzialità in termini di apprendimento e sviluppo delle
competenze del gruppo di progetto 7-ZPUs. Con alcune di queste aziende sono già
stati presi contatti preliminari (ERGON e Sanmarco Informatica). I verbali di
tali incontri sono presenti nella sezione \textit{Verbali/Verbali Esterni}
della repository di documentazione. \vspace{0.5cm}

\section{Analisi dei Capitolati}
I capitolati sono presentati in ordine di interesse decrescente, tenendo conto
degli aspetti tecnici come della preferenza naturale del gruppo.

\subsection{ Capitolato C3 - DIPReader}
\textbf{Azienda Proponente:} Sanmarco Informatica S.p.A.\\
\textbf{Committente:} Prof. Tullio Verdanega e Prof. Riccardo Cardin\\
\textbf{Obiettivo:} Sviluppo di un software per la lettura e ricerca di documenti digitali in formato .ZIP, con funzionalità avanzate di ricerca e verifica dell'autenticità.
L'azienda ha posto enfasi sul fatto che tale strumento risulterebbe molto utile in ambiti legali e giudiziari in cui vi è la necessità di cercare documenti specifici all'interno di una mole importante di dati, garantendone l'integrità, l'autenticità, la leggibilità  e la reperibilità.\\
\textbf{Dominio Tecnico:}
\begin{itemize}
    \item Database: SQLite e/o FAISS. Il primo potrebbe essere scelto per rappresentare relazioni tra i documenti, mantenendo portabile la struttura del DIPReader. FAISS, invece, potrebbe essere utilizzato per implementare funzionalità di ricerca basate su similarità e campi semantici.
    \item Framework Frontend: Angular o React, entrambi validi e sostanzialemnte equivalenti. Consigliato però l'uso di TypeScript per una maggiore robustezza del codice.
    \item Strumenti di Versionamento: GitHub o BitBucket
    \item Piattaforme: Windows, Linux, MacOS
\end{itemize}

\vfill
\begin{flushright}
    \textit{7-ZPUs}
\end{flushright}

\end{document}