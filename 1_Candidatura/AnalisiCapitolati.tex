\documentclass[a4paper,12pt]{article}

\usepackage[utf8]{inputenc}
\usepackage[T1]{fontenc} % per i caratteri accentati corretti in PDF
\usepackage{lmodern}
\usepackage[italian]{babel}
\renewcommand{\rmdefault}{lmss}
\usepackage{float}
\usepackage{geometry}
\usepackage{setspace}
\usepackage{enumitem}
\usepackage{titlesec}
\usepackage{tocloft}
\usepackage{graphicx}

\renewcommand{\contentsname}{Indice}

\geometry{margin=2.5cm}
\setstretch{1.2}

\titleformat{\section}{\large\bfseries}{\thesection}{1em}{}
\titleformat{\subsection}{\mdseries\bfseries}{\thesubsection}{1em}{}

\begin{document}

\begin{center}
    \includegraphics[width=9.5cm]{../assets/logo7ZPUs.jpg}\\
    \small\hspace{10cm} 7zpus.swe@gmail.com\\
    \vspace{0.5cm}
    \Large \textbf{Analisi dei Capitolati di Progetto 2025/2026}\\
\end{center}

\vspace{0.3cm}
\hrule
\vspace{0.5cm}

\tableofcontents

\newpage

\section*{Tabella di Versionamento}
\begin{tabular}{|c|c|c|c|}
    \hline
    \textbf{Versione} & \textbf{Data} & \textbf{Autore} & \textbf{Descrizione}\\
    1.4               & 26/10/2025   & Soligo Lorenzo    & Analisi capitolato C8\\
    \hline
    1.3               & 26/10/2025   & Laoud Zakaria    & Analisi capitolato C7\\
    \hline
    1.2               & 26/10/2025    & Matteo A. Rocco  & Aggiunta capitolato C5 \\
    \hline
    1.1 & 25/10/2025 & Georgescu Diana & Approfondimento del Capitolato C4 \\
    \hline
    1.0               & 17/10/2025    & Soligo Lorenzo  & Creazione del documento e stesura iniziale \\
    \hline

\end{tabular}

\section{Introduzione}
In questo documento sono analizzati i 9 Capitolati di Progetto proposti per
l'anno accademico 2025/2026, con l'obiettivo di valutarne la complessità, i
rischi associati e le potenzialità in termini di apprendimento e sviluppo delle
competenze del gruppo di progetto 7-ZPUs. Con alcune di queste aziende sono già
stati presi contatti preliminari (ERGON e Sanmarco Informatica). I verbali di
tali incontri sono presenti nella sezione \textit{Verbali/Verbali Esterni}
della repository di documentazione. \vspace{0.5cm}

\section{Analisi dei Capitolati}
I capitolati sono presentati in ordine di interesse decrescente, tenendo conto
degli aspetti tecnici come della preferenza naturale del gruppo.

\subsection{ Capitolato C3 - DIPReader}
\subsubsection*{Azienda Proponente:} Sanmarco Informatica S.p.A.
\subsubsection*{Committente:} Prof. Tullio Verdanega e Prof. Riccardo Cardin.
\subsubsection*{Obiettivo:} Sviluppo di un software per la lettura e ricerca di documenti digitali in formato .ZIP, con funzionalità avanzate di ricerca, verifica dell'autenticità. É inoltre necessario consultare i documenti all'interno del DIP permettendone il salvataggio in loco e la rielaborazione.
\subsubsection*{Dominio Applicativo:} L'azienda ha posto enfasi sul fatto che tale strumento risulterebbe molto utile in ambiti legali e giudiziari in cui vi è la necessità di cercare documenti specifici all'interno di una mole importante di dati, garantendone l'integrità, l'autenticità, la leggibilità  e la reperibilità.
\subsubsection*{Dominio Tecnico:}
\begin{itemize}
    \item Database: SQLite e/o FAISS. Il primo potrebbe essere scelto per rappresentare relazioni tra i documenti, mantenendo portabile la struttura del DIPReader. FAISS, invece, potrebbe essere utilizzato per implementare funzionalità di ricerca basate su similarità e campi semantici.
    \item Framework Frontend: Angular o React, entrambi validi e sostanzialemnte equivalenti. Consigliato però l'uso di TypeScript per una maggiore robustezza del codice.
    \item Strumenti di Versionamento: GitHub o BitBucket
    \item Piattaforme: Windows, Linux, MacOS
\end{itemize}
\subsubsection*{Aspetti Positivi:}
\begin{itemize}
    \item Progetto ben definito con obiettivi chiari e raggiungibili.
    \item Caso d'uso realistico e stimolante, con dirette applicazioni nel mondo reale.
    \item Necessità di implementare una struttura efficiente vista la mole di dati da gestire. Una sfida stimolante per il gruppo.
    \item Disponibilità di contatti con l'azienda proponente per chiarimenti e supporto.
    \item Funzionalità AI per la ricerca semantica come plus interessante ma non come focus principale.
    \item Alcune delle tecnologie proposte sono conosciute a parte del gruppo che potrà aiutare nella formazione degli altri membri.
\end{itemize}

\subsubsection*{Aspetti Negativi:}
\begin{itemize}
    \item 
\end{itemize}

\subsubsection*{Possibili Rischi:}
\begin{itemize}
    \item Possibili difficoltà nella gestione della mole di dati e nella loro analisi.
    \item Necessità di bilanciare funzionalità avanzate con la semplicità d'uso.
\end{itemize}

\subsubsection*{Conclusioni:}
Il capitolato C3 rappresenta una scelta solida e stimolante per il gruppo che potrà affrontare sfide tecniche interessanti e sviluppare competenze rilevanti nel campo della gestione dei dati e dell'analisi documentale. 
L'AI svvolge un ruolo secondario ma interessante, lasciando spazio al gruppo per esplorare questa tecnologia senza doverla necessariamente padroneggiare in profondità.
L'azienda si è presentata come molto disponibile e aperta al dialogo, come dimostrato dall'incontro del 2025/10/20, il che è un ulteriore punto a favore per la scelta di questo capitolato.


\subsection{ Capitolato C8 - SmartOrder}
\subsubsection*{Azienda Proponente:} ERGON Informatica S.R.L.
\subsubsection*{Committente:} Prof. Tullio Verdanega e Prof. Riccardo Cardin.
\subsubsection*{Obiettivo:} Sviluppo di un sistema si analisi di contenuti multimodali per la generazione automatica di ordini grazie all'ausilio di LLM. Tale sistema riuscirebbe a garantire un'automatizzazione del sistema di ordine, facilitandone l'avanzamento di richieste lato cliente e delegando il compito di 
\subsubsection*{Dominio Applicativo:} Servizi di assistenza alla vendita. Con un sistema come SmartOrder sarebbe possibile velocizzare la creazione di ordine, semplificando il processo di ordine lato utente in quanto si presuppone che il sistema conosca il contesto di vendita e sia in grado di interpretare richieste complesse.
\subsubsection*{Dominio Tecnico:}
\begin{itemize}
    \item Database: A libera scelta, per esempio SQL Server Express, MySql o MariaDB
    \item Framework Frontend: Angular, React o .NET Blazor, anche in questo caso la scelta è libera.
    \item Strumenti AI: \begin{itemize}
                            \item BERT, RoBERTa o GTP di OpenAi per NLP
                            \item Tesseract OCR, EasyOCR, Reti Convoluzionari (CNN) o Vision Transformer per Riconoscimento Ottico dei Caratteri
                            \item Whisper di OpenAi o Google Speech-to-Text per il riconoscimento vocale
                        \end{itemize}
    \item Comunicazione LLM-UI: API REST
    \item Comunicazione LLM-Database: Connessione standard con connettori da fonti ODBC oppure tramite l'implementazione di un middleware ad hoc
\end{itemize}
\subsubsection*{Aspetti Positivi:}
\begin{itemize}
    \item Progetto improntato sulle tecnologie AI più recenti e interessanti.
    \item Utilizzo dell'AI non banale, con un chiaro valore aggiunto per l'utente finale.
    \item Caso d'uso realistico e stimolante.
    \item L'azienda è certificata dallo standard ISO:9001, il che garantisce una certa qualità nei processi di sviluppo.
    \item Esperienza pregressa con università e studenti.
\end{itemize}

\subsubsection*{Aspetti Negativi:}
\begin{itemize}
    \item La mole di lavoro richiesta potrebbe essere elevata, vista la complessità del dominio applicativo.
    \item Presentazione del capitolato non del tutto chiara anche in sede di incontro con l'azienda.
    \item Necessità di apprendere e padroneggiare diverse tecnologie AI, che potrebbero risultare troppo complesse per il gruppo.
    \item Centralità dell'AI nel progetto, che potrebbe risultare un rischio se non adeguatamente gestita.
\end{itemize}

\subsubsection*{Possibili Rischi:}
\begin{itemize}
    \item Complessità degli argomenti AI proposti che potrebbero risultare troppo difficili da padroneggiare.
    \item Rischio di non riuscire a integrare correttamente le diverse tecnologie AI.
    \item 
\end{itemize}

\subsubsection*{Conclusioni:}
Il capitolato C8 rappresenta una scelta ambiziosa per il gruppo, che potrebbe affrontare sfide tecniche significative e sviluppare competenze avanzate nel campo dell'intelligenza artificiale applicata.
Tale mole di lavoro e tecnologie richieste potrebbero però risultare eccessive per il gruppo, soprattutto considerando la necessità di padroneggiare diverse tecnologie AI complesse come il \textit{"fine tuning"} richiesto per fornire al sistema di LLM il contesto aziendale. 
Seppur l'azienda abbia esperienza studenti e università e potrebbe fornire supporto anche organizzativo vista la certificazione ISO:9001, l'incontro non è riuscito a chiarire tutti i dubbi del gruppo, soprattutto riguardo alla complessità tecnologica.
\vspace{2.0cm}

\subsection{ Capitolato C7 - Sistema di acquisizione dati da sensori}
\subsubsection*{Azienda Proponente:} M31 S.r.l.
\subsubsection*{Committente:} Prof. Tullio Verdanega e Prof. Riccardo Cardin.
\subsubsection*{Obiettivo:} 
L’obiettivo del progetto è la realizzazione di un sistema distribuito per l’acquisizione, la gestione e lo smistamento dei dati provenienti da sensori Bluetooth Low Energy (BLE). Tale sistema intende fornire un’infrastruttura scalabile, sicura e multi-tenant (multi utente) capace di raccogliere informazioni da diversi, dispositivi differenti tra loro, aggregarle e renderle disponibili tramite una piattaforma cloud centralizzata.
Attraverso l’interazione tra sensori BLE, gateway BLE–WiFi e un ambiente cloud dedicato, il progetto mira a garantire comunicazioni sicure, segregazione dei dati e strumenti avanzati di monitoraggio e visualizzazione. L’obiettivo finale è creare una base solida per applicazioni IoT complesse e realistiche, ponendo le fondamenta per future integrazioni con tecniche di analisi e algoritmi predittivi.
\subsubsection*{Dominio Applicativo:} 
Un tale strumento può trovare utilizzo in numerosi ambiti, come l'industria manifatturiera, logistica, healthcare e le smart city. L’Internet of Things (IoT) si basa su tecnologie come il Bluetooth Low Energy (BLE), che consentono la raccolta di dati in modo efficiente e a basso consumo energetico, favorendo la cooperazione in tempo reale tra centinaia di dispositivi.
\subsubsection*{Dominio Tecnico:}
\begin{itemize}
    \item Database: \textbf{MongoDB} per dati non strutturati, \textbf{PostgreSQL} per dati strutturati e \textbf{Redis} (come parte del sistema di caching)
    \item Framework Frontend: \textbf{Angular}, cercando di creare un applicazione mono pagina (SPAs)
    \item Framework Backend: \textbf{Node.js} e \textbf{Nest.js} (con \textbf{Typescript}) per mocroservizi, \textbf{Go} per la sincronizzazione e \textbf{NATS o Apache Kafka} per la comunicazione tra i microservizi
    \item Infrastuttura: \textbf{Google Cloud Platform} (GCP) e \textbf{Kubernetes} per l'orchestrazione dei container e microservizi
    \item Piattaforme: web
\end{itemize}
\subsubsection*{Aspetti Positivi:}
\begin{itemize}
    \item Progetto caratterizzato da una struttura ben definita con una suddivisione in Layer molto chiara.
    \item Casi d’uso realistici e ormai essenziali per il funzionamento della società moderna.
    \item Utilizzo di molte tecnologie con campi d'uso eterogenei tra loro (parzialmente studiate da parte del gruppo).
    \item Libertà nella selezione delle tecnologie da adottare, previa motivazione e approvazione da parte di M31.
\end{itemize}

\subsubsection*{Aspetti Negativi:}
\begin{itemize}
    \item Si ha un elevata complessita tecnica data dalle numerose tecnologie da utilizzare.
    \item Infrastruttura piuttosto complessa da configurare e gestire con conseguente lungo periodo di studio preliminare.
\end{itemize}

\subsubsection*{Possibili Rischi:}
\begin{itemize}
    \item Possibili difficoltà nell'integrazione tra le diverse componenti del sistema.
\end{itemize}

\subsubsection*{Conclusioni:}
Il capitolato C7 rappresenta un’opportunità stimolante per il gruppo, offrendo la possibilità di affrontare sfide legate a sistemi distribuiti e gestione sicura dei dati IoT.
Il progetto consente di esplorare tecnologie moderne e richieste dal mercato, come microservizi, cloud e interfacce web. Nel complesso, C7 offre un percorso formativo completo e motivante, con ricadute concrete sulle competenze tecniche del gruppo, utili nel mondo del lavoro.

\subsection{ Capitolato C5 - NEXUM}
\subsubsection*{Azienda Proponente:} EGGON
\subsubsection*{Committente:} Prof. Tullio Verdanega e Prof. Riccardo Cardin.
\subsubsection*{Obiettivo:}
Sviluppo di moduli software integrativi al prodotto NEXUM, un software di comunicazione interna e gestione HR, in grado di connettere aziende, collaboratori, dipendenti e Consulenti del Lavoro.
\subsubsection*{Dominio Applicativo:}
Il capitolato si inserisce all'interno dell'applicativo NEXUM di Eggon Srl, una piattaforma digitale che include:
\begin{itemize}
    \item Modulo di messaggistica interna per policy, circolari e notifiche
    \item Modulo di timbratura digitale
    \item Gestione anagrafiche, ruoli e permessi
\end{itemize}
prevedendo perciò di integrare ed evolvere il prodotto con nuove funzionalità da sviluppare che permettano di introdurre assistenza coadiuvata dall'IA, generativa per creazione di contenuti formali e informali, e in modalità "Co-Pilot" per facilitazione dei rapporti con gli attori esterni coinvolti (consulenti del lavoro). Inoltre è prevista l'estensione delle funzionalità attualmente esistenti per timbratura e gestione/monitoraggio dei turni.
 
\subsubsection*{Dominio Tecnico:}
\begin{itemize}
    \item Database: Amazon RDS for PostgreSQL
    \item Framework Frontend: Angular per dashboard amministrativa, Next.js per Progressive Web App da destinare agli utenti finali.
    \item Backend: Ruby on Rails, AWS WAF, Sidekiq+SQS
    \item Sicurezza: KMS, Secrets Manager, Amazon Cognito
    \item Strumenti di Versionamento: GitHub.
    \item Piattaforme: Windows, Linux, MacOS.
\end{itemize}
\subsubsection*{Aspetti Positivi:}
\begin{itemize}
    \item Progetto ben definito con obiettivi chiari e raggiungibili.
    \item Caso d'uso realistico e stimolante, con dirette applicazioni nel mondo reale.
    \item Necessità di implementare una struttura efficiente vista la mole di dati da gestire. Una sfida stimolante per il gruppo.
    \item Disponibilità di contatti con l'azienda proponente per chiarimenti e supporto.
    \item Funzionalità AI per la ricerca semantica come plus interessante ma non come focus principale.
    \item Alcune delle tecnologie proposte sono conosciute a parte del gruppo che potrà aiutare nella formazione degli altri membri.
\end{itemize}

\subsubsection*{Aspetti Negativi:}
\begin{itemize}
    \item 
\end{itemize}

\subsubsection*{Possibili Rischi:}
\begin{itemize}
    \item Possibili difficoltà nella gestione della mole di dati e nella loro analisi.
    \item Necessità di bilanciare funzionalità avanzate con la semplicità d'uso.
\end{itemize}

\subsubsection*{Conclusioni:}
. 



\subsection{Analisi Capitolato C4 - L’app che Protegge e Trasforma}
\subsubsection*{Azienda Proponente:} Miriade
\subsubsection*{Committente:} Prof. Tullio Verdanega e Prof. Riccardo Cardin.
\subsubsection*{Obiettivo}
Realizzare un’app mobile multipiattaforma in grado di fornire strumenti di prevenzione, supporto e protezione per le vittime di violenza di genere.  
L’app integra funzionalità di analisi comportamentale basate su intelligenza artificiale, sistemi di allerta discreti, accesso a risorse geo-localizzate e sezioni formative.  
L’idea finale è quella di offrire un ambiente digitale sicuro, conforme al GDPR, che permetta all’utente di sentirsi tutelato e informato.

\subsubsection*{Dominio Applicativo}
L’app è pensata per persone a rischio o vittime di violenza di genere e per i centri che offrono supporto.  
Mira a fornire strumenti concreti di prevenzione e sicurezza, come il rilevamento automatico di situazioni di pericolo, allarmi silenziosi e una modalità stealth per proteggere l’utente.  
Include inoltre un diario criptato, l’accesso ai centri di assistenza, percorsi formativi e una community moderata per condividere esperienze in modo sicuro.


\subsubsection*{Dominio Tecnico}
\begin{itemize}[leftmargin=*]
    \item Frontend: Flutter.
    \item Backend: AWS Lambda, API Gateway, DynamoDB/RDS, S3, Cognito.
    \item AI: AWS SageMaker o Bedrock per modelli NLP e classificazione del rischio.
    \item Architettura: microservizi o approccio serverless per scalabilità e resilienza.
    \item Sicurezza: crittografia AES-256, autenticazione a più fattori, audit trail, conformità GDPR e accessibilità WCAG 2.1.
\end{itemize}

\subsubsection*{Aspetti Positivi}
\begin{itemize}[leftmargin=*]
    \item Capitolo con forte impatto sociale e finalità etica rilevante.
    \item Attenzione alla sicurezza e alla privacy dei dati sensibili.
    \item Supporto diretto e affiancamento da parte dell’azienda.
    \item Presenza di linee guida UX orientate all’accessibilità e alla serenità d’uso.
\end{itemize}

\subsubsection*{Aspetti Negativi}
\begin{itemize}[leftmargin=*]
    \item Numerose funzionalità opzionali che aumentano la complessità progettuale.
    \item Rischi legati all’affidabilità e ai bias dei modelli AI.
    \item Necessità di grande attenzione legale per conformità alle normative su privacy e geolocalizzazione.
\end{itemize}

\subsubsection*{Possibili Rischi}
\begin{itemize}[leftmargin=*]
    \item Fuga di dati sensibili con potenziale esposizione di informazioni o posizioni.
    \item Errori di classificazione AI.
    \item Abuso della community, come uso improprio o divulgazione di dati personali.
    \item Dipendenza da servizi esterni.
    \item Rischi legali a causa di registrazione audio/video e tracciamento GPS soggetti a diverse normative.
\end{itemize}

\subsubsection*{Conclusione}
Il capitolato C4 proposto da Miriade è ambizioso e unisce aspetti tecnici avanzati a un obiettivo sociale significativo.  
La realizzazione dell’app comporta sfide importanti, soprattutto in ambito di sicurezza, AI e conformità legale.
Dopo un’attenta discussione, il gruppo ha concluso di non voler proseguire con il capitolato C4.  
Pur riconoscendone il valore sociale e l’intento positivo, il progetto risulta estremamente vasto e complesso dal punto di vista tecnico e organizzativo.  
La grande quantità di funzionalità previste, unite alla forte dipendenza dall'intelligenza artificiale e all’elevato livello di responsabilità legale e di sicurezza richiesti, rendono il capitolato poco adatto agli obiettivi e alle competenze che il gruppo intende sviluppare in questo percorso.

\vspace{0.5cm}

\vfill
\begin{flushright}
    \textit{7-ZPUs}
\end{flushright}

\end{document}