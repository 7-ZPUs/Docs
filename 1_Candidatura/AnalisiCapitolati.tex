\documentclass[a4paper,12pt]{article}

\usepackage[utf8]{inputenc}
\usepackage[T1]{fontenc} % per i caratteri accentati corretti in PDF
\usepackage{lmodern}
\usepackage[italian]{babel}
\renewcommand{\rmdefault}{lmss}
\usepackage{float}
\usepackage{geometry}
\usepackage{setspace}
\usepackage{enumitem}
\usepackage{titlesec}
\usepackage{tocloft}
\usepackage{graphicx}

\renewcommand{\contentsname}{Indice}

\geometry{margin=2.5cm}
\setstretch{1.2}

\titleformat{\section}{\large\bfseries}{\thesection}{1em}{}
\titleformat{\subsection}{\mdseries\bfseries}{\thesubsection}{1em}{}

\begin{document}

\begin{center}
    \includegraphics[width=9.5cm]{../assets/logo7ZPUs.jpg}\\
    \small\hspace{10cm} 7zpus.swe@gmail.com\\
    \vspace{0.5cm}
    \Large \textbf{Analisi dei Capitolati di Progetto 2025/2026}\\
\end{center}

\vspace{0.3cm}
\hrule
\vspace{0.5cm}

\tableofcontents

\newpage

\section{Tabella di Versionamento}
\begin{tabular}{|c|c|c|c|}
    \hline
    \textbf{Versione} & \textbf{Data} & \textbf{Autore} & \textbf{Descrizione}                       \\
    \hline
    1.0               & 17/10/2025    & Soligo Lorenzo  & Creazione del documento e stesura iniziale \\
    \hline

\end{tabular}

\section{Introduzione}
In questo documento sono analizzati i 9 Capitolati di Progetto proposti per
l'anno accademico 2025/2026, con l'obiettivo di valutarne la complessità, i
rischi associati e le potenzialità in termini di apprendimento e sviluppo delle
competenze del gruppo di progetto 7-ZPUs. Con alcune di queste aziende sono già
stati presi contatti preliminari (ERGON e Sanmarco Informatica). I verbali di
tali incontri sono presenti nella sezione \textit{Verbali/Verbali Esterni}
della repository di documentazione. \vspace{0.5cm}

\section{Analisi dei Capitolati}
I capitolati sono presentati in ordine di interesse decrescente, tenendo conto
degli aspetti tecnici come della preferenza naturale del gruppo.

\subsection{ Capitolato C3 - DIPReader}
\subsubsection*{Azienda Proponente:} Sanmarco Informatica S.p.A.
\subsubsection*{Committente:} Prof. Tullio Verdanega e Prof. Riccardo Cardin.
\subsubsection*{Obiettivo:} Sviluppo di un software per la lettura e ricerca di documenti digitali in formato .ZIP, con funzionalità avanzate di ricerca, verifica dell'autenticità. É inoltre necessario consultare i documenti all'interno del DIP permettendone il salvataggio in loco e la rielaborazione.
\subsubsection*{Dominio Applicativo:} L'azienda ha posto enfasi sul fatto che tale strumento risulterebbe molto utile in ambiti legali e giudiziari in cui vi è la necessità di cercare documenti specifici all'interno di una mole importante di dati, garantendone l'integrità, l'autenticità, la leggibilità  e la reperibilità.
\subsubsection*{Dominio Tecnico:}
\begin{itemize}
    \item Database: SQLite e/o FAISS. Il primo potrebbe essere scelto per rappresentare relazioni tra i documenti, mantenendo portabile la struttura del DIPReader. FAISS, invece, potrebbe essere utilizzato per implementare funzionalità di ricerca basate su similarità e campi semantici.
    \item Framework Frontend: Angular o React, entrambi validi e sostanzialemnte equivalenti. Consigliato però l'uso di TypeScript per una maggiore robustezza del codice.
    \item Strumenti di Versionamento: GitHub o BitBucket
    \item Piattaforme: Windows, Linux, MacOS
\end{itemize}
\subsubsection*{Aspetti Positivi:}
\begin{itemize}
    \item Progetto ben definito con obiettivi chiari e raggiungibili.
    \item Caso d'uso realistico e stimolante, con dirette applicazioni nel mondo reale.
    \item Necessità di implementare una struttura efficiente vista la mole di dati da gestire. Una sfida stimolante per il gruppo.
    \item Disponibilità di contatti con l'azienda proponente per chiarimenti e supporto.
    \item Funzionalità AI per la ricerca semantica come plus interessante ma non come focus principale.
    \item Alcune delle tecnologie proposte sono conosciute a parte del gruppo che potrà aiutare nella formazione degli altri membri.
\end{itemize}

\subsubsection*{Aspetti Negativi:}
\begin{itemize}
    \item 
\end{itemize}

\subsubsection*{Possibili Rischi:}
\begin{itemize}
    \item Possibili difficoltà nella gestione della mole di dati e nella loro analisi.
    \item Necessità di bilanciare funzionalità avanzate con la semplicità d'uso.
\end{itemize}

\subsubsection*{Conclusioni:}
Il capitolato C3 rappresenta una scelta solida e stimolante per il gruppo che potrà affrontare sfide tecniche interessanti e sviluppare competenze rilevanti nel campo della gestione dei dati e dell'analisi documentale. 
L'AI svvolge un ruolo secondario ma interessante, lasciando spazio al gruppo per esplorare questa tecnologia senza doverla necessariamente padroneggiare in profondità.
L'azienda si è presentata come molto disponibile e aperta al dialogo, come dimostrato dall'incontro del 2025/10/20, il che è un ulteriore punto a favore per la scelta di questo capitolato.

\vspace{2.0cm}

\subsection{ Capitolato C8 - SmartOrder}
\subsubsection*{Azienda Proponente:}
\subsubsection*{Committente:} Prof. Tullio Verdanega e Prof. Riccardo Cardin.
\subsubsection*{Obiettivo:} 
\subsubsection*{Dominio Applicativo:} 
\subsubsection*{Dominio Tecnico:}
\begin{itemize}
    \item Database: 
    \item Framework Frontend: Angular o React, entrambi validi e sostanzialemnte equivalenti. Consigliato però l'uso di TypeScript per una maggiore robustezza del codice.
    \item Strumenti di Versionamento: GitHub o BitBucket
    \item Piattaforme: Windows, Linux, MacOS
\end{itemize}
\subsubsection*{Aspetti Positivi:}
\begin{itemize}
    \item Progetto ben definito con obiettivi chiari e raggiungibili.
    \item Caso d'uso realistico e stimolante, con dirette applicazioni nel mondo reale.
    \item Necessità di implementare una struttura efficiente vista la mole di dati da gestire. Una sfida stimolante per il gruppo.
    \item Disponibilità di contatti con l'azienda proponente per chiarimenti e supporto.
    \item Funzionalità AI per la ricerca semantica come plus interessante ma non come focus principale.
    \item Alcune delle tecnologie proposte sono conosciute a parte del gruppo che potrà aiutare nella formazione degli altri membri.
\end{itemize}

\subsubsection*{Aspetti Negativi:}
\begin{itemize}
    \item 
\end{itemize}

\subsubsection*{Possibili Rischi:}
\begin{itemize}
    \item Possibili difficoltà nella gestione della mole di dati e nella loro analisi.
    \item Necessità di bilanciare funzionalità avanzate con la semplicità d'uso.
\end{itemize}

\subsubsection*{Conclusioni:}
Il capitolato C3 rappresenta una scelta solida e stimolante per il gruppo che potrà affrontare sfide tecniche interessanti e sviluppare competenze rilevanti nel campo della gestione dei dati e dell'analisi documentale. 
L'AI svvolge un ruolo secondario ma interessante, lasciando spazio al gruppo per esplorare questa tecnologia senza doverla necessariamente padroneggiare in profondità.
L'azienda si è presentata come molto disponibile e aperta a collaborazioni, il che è un ulteriore punto a favore per la scelta di questo capitolato. 




\vfill
\begin{flushright}
    \textit{7-ZPUs}
\end{flushright}

\end{document}