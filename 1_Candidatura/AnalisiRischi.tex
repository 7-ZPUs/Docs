\documentclass[a4paper,12pt]{article}

\usepackage[utf8]{inputenc}
\usepackage[T1]{fontenc} % per i caratteri accentati corretti in PDF
\usepackage[italian]{babel}
\usepackage{lmodern}
\usepackage{geometry}
\usepackage{float}
\usepackage{verbatim}
\usepackage{setspace}
\usepackage{enumitem}
\usepackage{titlesec}
\usepackage{tocloft}
\usepackage{graphicx}

\renewcommand{\rmdefault}{lmss}
\renewcommand{\contentsname}{Indice}

\geometry{margin=2.5cm}
\setstretch{1.2}

\titleformat{\section}{\large\bfseries}{\thesection}{1em}{}
\titleformat{\subsection}{\mdseries\bfseries}{\thesubsection}{1em}{}

\begin{document}

\begin{center}
    \includegraphics[width=9.5cm]{../assets/logo7ZPUS.jpg}\\
    \small\hspace{10cm} 7zpus.swe@gmail.com\\
    \vspace{0.5cm}
    \Large \textbf{Documento di analisi dei rischi}\\
    \vspace{0.2cm}
\end{center}

\vspace{0.3cm}
\hrule
\vspace{0.3cm}

\tableofcontents

\newpage

\section*{Tabella di Versionamento}
    \begin{tabular}{|c|c|c|c|}
        \hline
        \textbf{Versione} & \textbf{Data} & \textbf{Autore} & \textbf{Descrizione} \\
        \hline
        1.0 & 24/10/2025 & Matteo A. Rocco & Creazione e stesura iniziale \\
        \hline
    \end{tabular}

\section{Descrizione}
Questo documento ha l'obiettivo di riportare le problematiche che potrebbero sorgere durante la realizzazione del capitolato di progetto di Ingegneria del Software, analizzando in modo preciso e realistico quanti più casi possibili, la loro probabilità di verificarsi e come porvi eventualmente rimedio.


\vspace{0.5cm}
\section{Rischi}
Ad ogni rischio verranno assegnati i seguenti parametri:
\begin{itemize}
    \item Probabilità: espressa con scala da 1 a 4 (Improbabile, Poco probabile, Probabile, Altamente probabile)
    \item Conseguenze: indica la gravità e cosa comporta il verificarsi di tale evento
    \item Rimedi: come risolvere o mitigare il più possibile eventuali conseguenze negative
\end{itemize}

\subsection{Rischio 1: Sforamento dei costi preventivati}
\textbf{Probabilità:} 2 - Poco probabile\\
\textbf{Conseguenze:} Il superamento del budget preventivato potrebbe compromettere la fattibilità del progetto, portando a ritardi nella consegna o alla necessità di ridurre alcune funzionalità previste.\\
\textbf{Rimedi:} Monitorare costantemente le spese rispetto al budget iniziale, effettuare revisioni periodiche del piano finanziario e prevedere un margine di sicurezza per coprire eventuali costi imprevisti.

\subsection{Rischio 2:}

\subsection{Rischio 3:}

\subsection{Rischio 4:}



\vspace{0.5cm}


\vfill
\begin{flushright}
    \textit{7-ZPUs}
\end{flushright}

\end{document}