\documentclass[a4paper,12pt]{article}

\usepackage[utf8]{inputenc}
\usepackage[T1]{fontenc} % per i caratteri accentati corretti in PDF
\usepackage[italian]{babel}
\usepackage{lmodern}
\usepackage{geometry}
\usepackage{float}
\usepackage{verbatim}
\usepackage{setspace}
\usepackage{enumitem}
\usepackage{titlesec}
\usepackage{tocloft}
\usepackage{graphicx}

\renewcommand{\rmdefault}{lmss}
\renewcommand{\contentsname}{Indice}

\geometry{margin=2.5cm}
\setstretch{1.2}

\titleformat{\section}{\large\bfseries}{\thesection}{1em}{}
\titleformat{\subsection}{\mdseries\bfseries}{\thesubsection}{1em}{}

\begin{document}

\begin{center}
    \includegraphics[width=9.5cm]{../assets/logo7ZPUS.jpg}\\
    \small\hspace{10cm} 7zpus.swe@gmail.com\\
    \Large \textbf{Documento di analisi dei rischi}\\
    \vspace{0.5cm}
\end{center}

\vspace{0.3cm}
\hrule
\vspace{0.3cm}

\tableofcontents

\newpage

\section*{Tabella di Versionamento}
    \begin{tabular}{|c|c|c|c|}
        \hline
        \textbf{Versione} & \textbf{Data} & \textbf{Autore} & \textbf{Descrizione} \\
        \hline
        1.1.1 & 30/10/2025 & Rocco Matteo A. & Correzione refusi \\
        \hline
        1.1 & 27/10/2025 & Gruppo al completo & Aggiunta rischi \\
        \hline
        1.0 & 24/10/2025 & Rocco Matteo A. & Creazione e stesura iniziale \\
        \hline
    \end{tabular}

\section{Descrizione}
Questo documento ha l'obiettivo di riportare le problematiche che potrebbero sorgere durante la realizzazione del capitolato di progetto di Ingegneria del Software, analizzando in modo preciso e realistico quanti più casi possibili, la loro probabilità di verificarsi e come porvi eventualmente rimedio.


\vspace{0.5cm}
\section{Rischi}
Ad ogni rischio verranno assegnati i seguenti parametri:
\begin{itemize}
    \item Probabilità: espressa con scala da 1 a 4 (Improbabile, Poco probabile, Probabile, Altamente probabile)
    \item Conseguenze: indica la gravità (Lieve, Medio, Grave, Gravissimo) e cosa comporta il verificarsi di tale evento
    \item Rimedi: come risolvere o mitigare il più possibile eventuali conseguenze negative
\end{itemize}

\subsection{Rischio 1: Sforamento dei costi preventivati}
\textbf{Probabilità:} 2 - Poco probabile\\
\textbf{Conseguenze:} Il superamento del budget preventivato potrebbe compromettere la fattibilità del progetto, portando a ritardi nella consegna o alla necessità di ridurre alcune funzionalità previste. (Medio)\\
\textbf{Rimedi:} Monitorare costantemente l'allocazione delle ore rispetto alla pianificazione iniziale, effettuare stand-up meetings periodici e prevedere margini temporali per imprevisti.

\subsection{Rischio 2: Calo di produttività del team}
\textbf{Probabilità:} 4 - Altamente probabile\\
\textbf{Conseguenze:} Il gruppo prevede un calo dell'attività nel periodo natalizio, a causa di impegni personali e festività, che si sovrappone al periodo di sessione d'esame. (Lieve)
\textbf{Rimedi:} Pianificare in anticipo le attività più critiche prima del periodo di calo, e le restanti tenendo conto del periodo di ridotta attività.  

\subsection{Rischio 3: Mancata comunicazione e collaborazione tra i membri del team}
\textbf{Probabilità:} 1 - Improbabile\\
\textbf{Conseguenze:} La mancanza di comunicazione efficace può portare a incomprensioni, duplicazione degli sforzi e ritardi nel completamento delle attività. (Grave)\\
\textbf{Rimedi:} Stabilire canali di comunicazione chiari e regolari e una routine di aggiornamenti pianificati per garantire che tutti i membri del team siano allineati sugli obiettivi e le responsabilità.

\subsection{Rischio 4: Mancata comunicazione con l'azienda proponente}
\textbf{Probabilità:} 2 - Poco probabile\\
\textbf{Conseguenze:} La mancanza di feedback regolari dall'azienda proponente potrebbe portare a uno sviluppo rallentato e implementazione di funzionalità non allineate con le loro aspettative. (Grave)\\
\textbf{Rimedi:} Stabilire un calendario di incontri regolari con l'azienda proponente per garantire un flusso costante di comunicazione e feedback.

\subsection{Rischio 5: Problemi tecnici con gli strumenti di sviluppo}
\textbf{Probabilità:} 3 - Probabile\\
\textbf{Conseguenze:} L'inesperienza del gruppo con le tecnologie da utilizzare potrebbe portare a difficoltà tecniche e rallentamenti nello sviluppo. (Medio)\\
\textbf{Rimedi:} Prevedere sessioni di formazione iniziali con il supporto occasionale dell'azienda proponente per familiarizzare con gli strumenti e le tecnologie.

\subsection{Rischio 6: Mancato rispetto delle norme e documenti di progetto interni}
\textbf{Probabilità:} 2 - Poco probabile\\
\textbf{Conseguenze:} La mancata aderenza ai documenti di progetto potrebbe portare a discrepanze tra i lavori svolti dai componenti del gruppo. (Grave)\\
\textbf{Rimedi:} Ruolo attivo di amministratori e tester per garantire il rispetto delle norme e dei documenti di progetto interni.


\vspace{0.5cm}


\vfill
\begin{flushright}
    \textit{7-ZPUs}
\end{flushright}

\end{document}