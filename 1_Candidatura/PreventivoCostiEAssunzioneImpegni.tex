\documentclass[a4paper,12pt]{article}

\usepackage[utf8]{inputenc}
\usepackage[T1]{fontenc} % per i caratteri accentati corretti in PDF
\usepackage[italian]{babel}
\usepackage{lmodern}
\usepackage{geometry}
\usepackage{float}
\usepackage{setspace}
\usepackage{enumitem}
\usepackage{titlesec}
\usepackage{tocloft}
\usepackage{graphicx}

\renewcommand{\rmdefault}{lmss}
\renewcommand{\contentsname}{Indice}

\geometry{margin=2.5cm}
\setstretch{1.2}

\titleformat{\section}{\large\bfseries}{\thesection}{1em}{}
\titleformat{\subsection}{\mdseries\bfseries}{\thesubsection}{1em}{}

\begin{document}

\begin{center}
    \includegraphics[width=9.5cm]{../assets/logo7ZPUs.jpg}\\
    \small\hspace{10cm} 7zpus.swe@gmail.com\\
    \vspace{0.5cm}
    \Large \textbf{Documento di preventivo costi e assunzione impegni}\\
    \vspace{0.2cm}
\end{center}

\vspace{0.3cm}
\hrule
\vspace{0.3cm}

\tableofcontents

\newpage

\section*{Tabella di Versionamento}
    \begin{tabular}{|c|c|c|c|}
        \hline
        \textbf{Versione} & \textbf{Data} & \textbf{Autore} & \textbf{Descrizione} \\
        \hline
        1.1 & 26/10/2025 & Rocco Matteo A. & Aggiunta ore e importi \\
        \hline
        1.0 & 23/10/2025 & Rocco Matteo A. & Creazione e stesura iniziale \\
        \hline
    \end{tabular}

\section{Descrizione}
Questo documento riporta il piano di lavoro del gruppo \textit{7-ZPUs} (gruppo 11) per il progetto didattico del corso di Ingegneria del Software presso Università degli Studi di Padova (a.a. 2025-2026).

\subsection*{Componenti del gruppo}
\begin{itemize}[noitemsep]
    \item Fattoni Antonio 
    \item Georgescu Diana
    \item Gingillino Aaron
    \item Laoud Zakaria
    \item Rocco Matteo Alberto
    \item Soligo Lorenzo
    \item Vigolo Davide
\end{itemize}

\vspace{0.5cm}
\section{Assunzione di impegno}
I componenti del gruppo si prefiggono di dedicare 95 ore produttive ciascuno allo svolgimento del progetto \textit{DIPReader} di SanMarco Informatica, assumendo a turno i seguenti ruoli:
\begin{itemize} [noitemsep]
    \item Responsabile
    \item Amministratore
    \item Progettista
    \item Analista
    \item Programmatore
    \item Verificatore
\end{itemize}

\subsection*{Responsabile}
La figura del responsabile di progetto software guida e coordina lo sviluppo del prodotto, pianificando, monitorando e adattando le attività del team per garantire il raggiungimento degli obiettivi di progetto, nel rispetto di tempi, costi e qualità, secondo le pratiche e i principi dell'ingegneria del software.

\subsection*{Amministratore}
La figura dell'amministratore nell'ingegneria del software è la figura incaricata della gestione e del mantenimento dell'ambiente operativo del sistema. Supervisiona infrastrutture, configurazioni e procedure per garantire che il software sia installato, funzionante e sicuro, collaborando con il team di sviluppo per supportare la continuità e la qualità del servizio.

\subsection*{Progettista}
Il progettista software è la figura che traduce i requisiti del sistema in una struttura tecnica coerente e realizzabile, definendo architettura, componenti e interfacce. Opera come collegamento tra analisi e sviluppo, garantendo che la soluzione progettata rispetti requisiti, vincoli e principi di qualità del software.

\subsection*{Analista}
La figura dell'analista raccoglie, interpreta e formalizza i bisogni degli stakeholder in requisiti tecnici chiari e verificabili. È il mediatore tra dominio utente e dominio tecnico, e garantisce che il sistema sviluppato soddisfi effettivamente gli obiettivi per cui è stato concepito.

\subsection*{Programmatore}
Il programmatore è la figura che implementa il sistema traducendo il progetto in codice eseguibile, garantendo qualità, correttezza e manutenibilità del software. Collabora con analisti e progettisti per assicurare che la soluzione realizzata soddisfi i requisiti e rispetti gli standard tecnici del progetto.

\subsection*{Analista}
Il verificatore è la figura incaricata di controllare che il prodotto sia stato sviluppato correttamente rispetto ai requisiti e alle specifiche. Attraverso attività di test, analisi e revisione, garantisce la qualità tecnica del software, contribuendo alla sua affidabilità e conformità prima del rilascio.\\

\noindent Ogni ruolo verrà ricoperto da ciascun membro e le modalità specifiche con cui verrà svolto il ruolo saranno definite nel documento dedicato al Way of Working.

\newpage

\subsection{Ripartizione oraria}
La ripartizione oraria stabilita è la seguente:


\vspace{1cm}
\begin{table}[H]
{
\centering
\begin{tabular}{c|c|c|c|c|c|c}

        & \textbf{Resp.} & \textbf{Admin.} & \textbf{Anal.} & \textbf{Proj.} & \textbf{Progr.} & \textbf{Verif.} \\
        \hline
        Fattoni Antonio & 10 & 8 & 13 & 17 & 24 & 23\\
        \hline
        Georgescu Diana & 10 & 10 & 14 & 18 & 20 & 23\\
        \hline
        Gingillino Aaron & 10 & 8 & 14 & 19 & 21 & 23\\
        \hline
        Laoud Zakaria & 9 & 8 & 14 & 20 & 21 & 23\\
        \hline
        Rocco Matteo Alberto & 12 & 9 & 12 & 17 & 22 & 23\\
        \hline
        Soligo Lorenzo & 12 & 9 & 13 & 17 & 21 & 23\\
        \hline
        Vigolo Davide & 9 & 10 & 13 & 17 & 24 & 22\\
\end{tabular}\par
}
\end{table}

\noindent La suddivisione tiene conto delle diverse disponibilità emerse durante riunioni interne e riflette le competenze di ciascun componente del gruppo.

\section{Preventivo costi}
Con riferimento alle regole di progetto, i costi orari per ciascun ruolo sono i seguenti:
\begin{itemize} [noitemsep]
    \item Responsabile: 30€
    \item Amministratore: 20€
    \item Analista: 25€
    \item Progettista: 25€
    \item Programmatore: 15€
    \item Verificatore: 15€
\end{itemize}

\begin{table}[H]
{
\centering
\begin{tabular}{c|c|c}

        & \textbf{Ore totali} & \textbf{Costo totale ruolo (€)}  \\
        \hline
        Responsabile & 72 & 2160,00 \\
        \hline
        Amministratore & 62 & 1240,00 \\
        \hline
        Analista & 93 & 2325,00 \\
        \hline
        Progettista & 125 & 3125,00 \\
        \hline
        Programmatore & 153 & 2295,00 \\
        \hline
        Verificatore & 160 & 2400,00 \\
        \hline
        \textbf{Totale} & 665 & 13545,00 \\

\end{tabular}\par
}
\end{table}
\noindent Il budget di spesa previsto si attesta perciò a \textbf{13545,00€}.

\vspace{0.5cm}


\vfill
\begin{flushright}
    \textit{7-ZPUs}
\end{flushright}

\end{document}