\documentclass[a4paper,12pt]{article}

\usepackage[utf8]{inputenc}
\usepackage[T1]{fontenc} % per i caratteri accentati corretti in PDF
\usepackage[italian]{babel}
\usepackage{lmodern}
\renewcommand{\rmdefault}{lmss}
\usepackage{geometry}
\usepackage{setspace}
\usepackage{enumitem}
\usepackage{titlesec}
\usepackage{tocloft}
\usepackage{graphicx}

\renewcommand{\contentsname}{Indice}

\geometry{margin=2.5cm}
\setstretch{1.2}

\titleformat{\section}{\large\bfseries}{\thesection}{1em}{}
\titleformat{\subsection}{\mdseries\bfseries}{\thesubsection}{1em}{}

\begin{document}

\begin{center}
    \includegraphics[width=9.5cm]{../../../assets/logo7zpus.jpg}\\
    \small\hspace{10cm} 7zpus.swe@gmail.com\\
    \Large \textbf{Verbale Interno Gruppo di Progetto}\\
    \vspace{0.5cm}
\end{center}

\noindent
\textbf{Data:} 14/10/2025 \\
\textbf{Durata:} 1 ora e 15 minuti \\
\textbf{Luogo:} incontro online (Server Discord del gruppo)

\vspace{0.3cm}
\hrule
\vspace{0.5cm}

\tableofcontents

\newpage

\section*{Tabella di Versionamento}
    \begin{tabular}{|c|c|c|c|}
        \hline
        \textbf{Versione} & \textbf{Data} & \textbf{Autore} & \textbf{Descrizione} \\
        \hline
        1.0.2 & 30/10/2025 & Rocco Matteo A. & Correzione refusi \\
        \hline
        1.0.1 & 22/10/2025 & Vigolo Davide & Adattamento a nuovo template di verbale \\
        \hline
        1.0.0 & 14/10/2025 & Vigolo Davide & Creazione del verbale e stesura iniziale \\
        \hline
    \end{tabular}


\section*{Partecipanti}
\begin{itemize}[noitemsep]
    \item Fattoni Antonio
    \item Georgescu Diana
    \item Rocco Matteo Alberto
    \item Soligo Lorenzo
    \item Vigolo Davide
\end{itemize}

\section{Ordine del Giorno}
\begin{enumerate}
    \item Discussione dei vari capitolati e delle relative presentazioni.
    \item Stila di una lista di aziende di interesse
\end{enumerate}
\vspace{0.5cm}
\section{Svolgimento e Discussione}
Il gruppo ha analizzato i vari capitolati, discutendo le principali caratteristiche tecniche e funzionali di ciascuno. 
Durante la discussione, i membri hanno condiviso opinioni e interessi rispetto alle tecnologie coinvolte e agli obiettivi progettuali. 
È stato effettuato un breve confronto sulle modalità di interazione previste con le aziende, al fine di chiarire eventuali aspetti organizzativi. 
Sono state inoltre valutate le competenze pregresse dei componenti nei diversi ambiti tecnologici, con l’obiettivo di individuare aree di miglioramento e di supporto reciproco all’interno del gruppo. 
È stata infine stilata una lista ordinata dei tre capitolati per i quali il gruppo ha mostrato maggiore interesse, tenendo conto delle valutazioni espresse. 
È stato deciso di redigere un documento dichiarativo relativo al \textit{way of working} da adottare. 
Infine, è stato concordato di organizzare un nuovo incontro dedicato alla preparazione delle domande da sottoporre alle aziende proponenti.

\subsection{Ordine di Preferenza dei Capitolati}
A seguire l'ordine di preferenza dei capitolati scelti dal gruppo:
\begin{enumerate}
    \item \textbf{C3}
    \item \textbf{C9}
    \item \textbf{C8}
\end{enumerate}

\vspace{0.5cm}
\section{Decisioni e Conclusione}
\begin{itemize}
    \item Verrà redatto un documento che definisca il \textit{way of working} del gruppo.
    \item Il prossimo incontro sarà dedicato alla formulazione delle domande da rivolgere alle aziende.
\end{itemize}
La riunione si conclude dopo 1 ora e 15 minuti.
Il prossimo incontro verrà pianificato nei prossimi giorni tramite i canali di comunicazione del gruppo.

\subsection{Ordine del giorno Prossimo Incontro}
\begin{enumerate}
    \item Formulazione delle domande delle aziende proponenti.
    \item Stesura delle mail da inviare alle aziende.
    \item Pianificazione del prossimo incontro.
    \item Pianificazione delle attività future del gruppo.
\end{enumerate}

\vfill
\begin{flushright}
    \textit{7-ZPUs}
\end{flushright}

\end{document}
