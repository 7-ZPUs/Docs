\documentclass[a4paper,12pt]{article}

\usepackage[utf8]{inputenc}
\usepackage[T1]{fontenc} % per i caratteri accentati corretti in PDF
\usepackage[italian]{babel}
\usepackage{lmodern}
\renewcommand{\rmdefault}{lmss}
\usepackage{geometry}
\usepackage{setspace}
\usepackage{enumitem}
\usepackage{titlesec}
\usepackage{tocloft}
\usepackage{graphicx}

\renewcommand{\contentsname}{Indice}

\geometry{margin=2.5cm}
\setstretch{1.2}

\titleformat{\section}{\large\bfseries}{\thesection}{1em}{}
\titleformat{\subsection}{\mdseries\bfseries}{\thesubsection}{1em}{}

\begin{document}

\begin{center}
    \includegraphics[width=9.5cm]{../../../assets/logo7ZPUs.jpg}\\
    \small\hspace{10cm} 7zpus.swe@gmail.com\\
    \Large \textbf{Verbale Interno Gruppo di Progetto}\\
    \vspace{0.5cm}
\end{center}

\noindent
\textbf{Data:} 17/10/2025 \\
\textbf{Durata:} 1 ore e 15 minuti \\
\textbf{Luogo:} incontro online (Server Discord del gruppo)

\vspace{0.3cm}
\hrule
\vspace{0.5cm}

\tableofcontents

\newpage

\section*{Tabella di Versionamento}
\begin{tabular}{|c|c|c|c|}
    \hline
    \textbf{Versione} & \textbf{Data} & \textbf{Autore} & \textbf{Descrizione}                     \\
    \hline
    1.0.3 & 30/10/2025 & Rocco Matteo A. & Correzione refusi \\
    \hline
    1.0.2 & 22/10/2025 & Soligo Lorenzo  & Applicazione Nuovo Modello Standard \\
    \hline
    1.0.1 & 17/10/2025 & Georgescu Diana & Correzione refusi \\
    \hline
    1.0 & 17/10/2025 & Soligo Lorenzo  & Creazione del verbale e stesura iniziale \\
    \hline

\end{tabular}

\section*{Partecipanti}
\begin{itemize}[noitemsep]
    \item Fattoni Antonio
    \item Georgescu Diana
    \item Gingilino Aaron
    \item Laoud Zakaria
    \item Rocco Matteo Alberto
    \item Soligo Lorenzo
    \item Vigolo Davide
\end{itemize}

\section{Ordine del Giorno}
\begin{enumerate}
    \item Formulazione delle domande delle aziende proponenti.
    \item Stesura delle mail da inviare alle aziende.
    \item Pianificazione del prossimo incontro.
    \item Pianificazione delle attività future del gruppo.
\end{enumerate}
\vspace{0.5cm}
\section{Svolgimento e Discussione}
Il gruppo, partendo dalle considerazioni emerse nella precedente riunione del
15 Ottobre 2025, ha proceduto alla formulazione delle domande da porre alle
aziende proponenti i capitolati di interesse. Le domande affrontano aspetti
tecnici delle tecnologie richieste/consigliate per lo sviluppo nonché
chiarimenti sui requisiti funzionali. Sono state redatte le due mail da inviare
e una volta approvate sono state spedite in modo da sfruttare al meglio il
\textit{prime-time} delle aziende. Infine è stata discussa la pianificazione
delle prossime attività del gruppo.

\vspace{0.5cm}
\section{Domande Scelte per i Capitolati}

\subsection{Capitolato C3 - DIPReader}
\begin{itemize}
    \item Differenze tra utilizzo di Angular e React (la prima più coesa e robusta e la
          seconda più facile e flessibile?)
    \item Principali differenze tra sviluppo applicazione multiplatform e su browser
    \item Quali informazioni è fondamentale includere nella rappresentazione
          semplificata?
    \item Concetti minimi da apprendere per poter utilizzare FAISS
    \item Cloud e locale come interagiscono? C'è una sorta di cache locale? Fino a che
          punto dovremo sviluppare l'interazione col cloud?
\end{itemize}

\subsection{Capitolato C8 - SmartOrder}
\begin{itemize}
    \item Come implementare l'interfaccia di acquisizione dati?
    \item Quanto effettivamente bisogna implementare nel lato “AI”: è richiesto uno
          sviluppo dell’intelligenza artificiale o la sola integrazione? Qual è il grado
          di complessità richiesto?
\end{itemize}

\vspace{0.5cm}
\section{Decisioni e Conclusione}
\begin{itemize}
    \item Inviata una mail a Sanmarco informatica per richiedere chiarimenti sul
          capitolato C3.
    \item Inviata una mail a Ergon per richiedere chiarimenti sul capitolato C8.
    \item Prossimo incontro pianificato in seguito alla ricezione delle risposte dalle
          aziende per la prima metà della settimana, sperabilmente.
    \item Fissato un meeting settimanale ogni venerdì alle 10 per aggiornamento.
    \item Fissata la necessità di stabilire un Way of Working, da discutere nel prossimi
          incontri, in seguito al contatto con le aziende.
\end{itemize}
Si è deciso inoltre di incontrarsi per prepararsi ad un eventuale colloquio con una chiamata/incontro da fissare con i canali di comunicazione del gruppo.
Infine, è stato concordato che i prossimi verbali seguiranno il modello standard del gruppo basato su questo stesso verbale.

\subsection{Ordine del giorno Prossimo Incontro}
\begin{enumerate}
    \item Discussione delle risposte ricevute dalle aziende.
    \item Preparazione ad un eventuale colloquio con le aziende.
    \item Inizio discussione del Way of Working del gruppo.
\end{enumerate}

\vfill
\begin{flushright}
    \textit{7-ZPUs}
\end{flushright}

\end{document}
