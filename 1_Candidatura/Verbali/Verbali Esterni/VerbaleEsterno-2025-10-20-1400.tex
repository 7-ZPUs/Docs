\documentclass[a4paper,12pt]{article}

\usepackage[utf8]{inputenc}
\usepackage[T1]{fontenc} % per i caratteri accentati corretti in PDF
\usepackage[italian]{babel}
\usepackage{geometry}
\usepackage{setspace}
\usepackage{enumitem}
\usepackage{titlesec}
\usepackage{tocloft}
\usepackage{graphicx}

\renewcommand{\contentsname}{Indice}

\geometry{margin=2.5cm}
\setstretch{1.2}

\titleformat{\section}{\large\bfseries}{\thesection}{1em}{}
\titleformat{\subsection}{\mdseries\bfseries}{\thesubsection}{1em}{}

\begin{document}


\begin{center}
    \includegraphics[width=9.5cm]{../../../assets/logo7zpus.jpeg}\\
    \small\hspace{10cm} 7zpus.swe@gmail.com\\
    \Large \textbf{Verbale Esterno Gruppo di Progetto e ERGON Informatica}\\
    \vspace{0.5cm}
\end{center}


\noindent
\textbf{Data:} 20/10/2025 \\
\textbf{Durata:}  30 minuti con l'azienda, 20 minuti di discussione interna \\
\textbf{Luogo:} Incontro online (Zoom con azienda, Discord internamente)

\vspace{0.3cm}
\hrule
\vspace{0.5cm}

\tableofcontents

\newpage

\section{Tabella di Versionamento}
\begin{center}
\begin{tabular}{|c|c|c|c|}
    \hline
    \textbf{Versione} & \textbf{Data} & \textbf{Autore} & \textbf{Descrizione} \\
    \hline
    1.0.2 & 22/10/2025 & Matteo Alberto Rocco & Revisione struttura \\
    \hline
    1.0.1 & 20/10/2025 & Lorenzo Soligo & Revisione del Modello Standard \\
    \hline
    1.0 & 20/10/2025 & Diana Georgescu & Creazione del verbale e stesura iniziale \\
    \hline
\end{tabular}
\end{center}

\section{Partecipanti}
{\small
\begin{itemize}[noitemsep, topsep=0pt, parsep=0pt, partopsep=0pt, leftmargin=1.8em]
    \item Fattoni Antonio
    \item Georgescu Diana
    \item Gingilino Aaron
    \item Laoud Zakaria
    \item Rocco Matteo Alberto
    \item Soligo Lorenzo
    \item Vigolo Davide
\end{itemize}
}

\section{Introduzione}
Il presente documento riporta il verbale del colloquio tenutosi con l'azienda ERGON in merito al progetto SmartOrder.
L’obiettivo dell’incontro era chiarire i requisiti tecnici, organizzativi e comunicativi relativi a tale progetto.

\section{Domande e Risposte}

\textbf{Domanda 1:} È corretto affermare che l’obiettivo del progetto è creare una piattaforma da utilizzare nei sistemi aziendali, basata sull’utilizzo di modelli di linguaggio di grandi dimensioni (LLM)?\\[0.5em]
\textbf{Risposta:} Sì, è corretto. L’obiettivo è realizzare una piattaforma aziendale che integri LLM per l’automazione e l’assistenza nei processi interni.

\vspace{2em}

\textbf{Domanda 2:} In che modo verrà impiegata l’intelligenza artificiale nel progetto? È previsto l’utilizzo di chiamate API o di dati forniti direttamente?\\[0.5em]
\textbf{Risposta:} Si partirà da modelli LLM pre-addestrati (come ChatGPT, Gemini, LLaMA, ecc.) sui quali verrà effettuato un \textit{fine-tuning} specifico per il contesto aziendale. Il modello sarà quindi adattato ai dati reali forniti dall’azienda, tramite un dataset realistico. Successivamente, le interazioni avverranno tramite chiamate API.

\vspace{2em}

\textbf{Domanda 3:} È prevista una fase di addestramento del modello?\\[0.5em]
\textbf{Risposta:} Non è previsto un addestramento completo, bensì un adattamento (\textit{fine-tuning}) limitato al dominio aziendale, utilizzando un dataset reale fornito dall’azienda stessa.

\vspace{2em}

\textbf{Domanda 4:} Come avverrà la comunicazione con l’azienda e con quale frequenza?\\[0.5em]
\textbf{Risposta:} La comunicazione potrà avvenire tramite call o mail. L’azienda è flessibile e si adatta ai vostri ritmi, garantendo risposte in tempi brevi.

\vspace{2em}

\textbf{Domanda 5:} Qual è il livello di flessibilità nell’utilizzo delle tecnologie?\\[0.5em]
\textbf{Risposta:} L’azienda concede piena libertà nella scelta delle tecnologie, pur fornendo alcune linee guida e strumenti consigliati. Non è richiesto un ciclo Agile rigoroso, ma è possibile adottarlo se preferito. Potete operare in autonomia.

\vspace{2em}

\textbf{Domanda 6:} Per quanto riguarda il \textit{fine-tuning}, lavoreremo su modelli locali o solo tramite API?\\[0.5em]
\textbf{Risposta:} È previsto che il \textit{fine-tuning} venga eseguito da voi, con il nostro supporto. Successivamente, si continuerà a interagire con il modello tramite chiamate API.

\vspace{2em}

\textbf{Domanda 7:} Il sistema finale dovrà essere una web app, un’applicazione desktop o un software locale?\\[0.5em]
\textbf{Risposta:} Il progetto prevede la realizzazione di una web app responsive, accessibile da browser.

\vspace{2em}

\textbf{Domanda 8:} Saranno fornite risorse o supporto per l’apprendimento delle tecnologie necessarie?\\[0.5em]
\textbf{Risposta:} Sì, l’azienda fornirà supporto tecnico e risorse di riferimento. Le principali tecnologie previste sono in linguaggio \texttt{Python}. La comunicazione di supporto avverrà via mail o call (1–2 giorni di risposta).

\vspace{2em}

\textbf{Domanda 9:} La web app dovrà accettare input multimodali (testo, audio, immagini) da trasformare in formati strutturati, ad esempio JSON?\\[0.5em]
\textbf{Risposta:} Sì, è possibile integrare input multimodali. Tale aspetto potrà essere approfondito nel corso dello sviluppo.

\vspace{2em}

\textbf{Domanda 10:} In che modo avverrà la connessione al database aziendale?\\[0.5em]
\textbf{Risposta:} La comunicazione con il database avverrà tramite connettori standard (ODBC) o specifici forniti dall’azienda, a seconda del tipo di database utilizzato. Sarà inoltre integrato un database vettoriale associato al modello LLM, utile per la ricerca semantica e la gestione dei dati testuali.

\vspace{2em}

\textbf{Domanda 11:} Verrà fornita una base di partenza per il database vettoriale?\\[0.5em]
\textbf{Risposta:} Sì, verrà messa a disposizione una configurazione di riferimento basata sul database più comunemente utilizzato nel contesto aziendale.

\section{Conclusione}
Al termine della call con l’azienda, il gruppo si è riunito in una chiamata su Discord per confrontarsi sugli argomenti trattati e discutere i punti emersi durante l’incontro.  
Durante tale confronto interno, sono stati approfonditi i chiarimenti ricevuti, rielaborate le informazioni acquisite e rivalutato il progetto alla luce delle specifiche tecniche e organizzative fornite dall’azienda.  
Questa discussione ha permesso di allineare le idee, chiarire eventuali dubbi residui e definire una visione condivisa sugli sviluppi futuri del lavoro.

\vspace{15em}

\begin{tabular}{@{}p{0.5in}p{2in}@{}}
Data: & \hrulefill \\
\end{tabular}

\vspace{1em}

\begin{tabular}{@{}p{0.5in}p{2in}@{}}
Firma: & \hrulefill \\
& ERGON Informatica Srl \\
\end{tabular}

\vfill
\begin{flushright}
    \textit{7-ZPUs}
\end{flushright}

\end{document}
