% DA UC-37

\usecase{verificaIntegritaDIPCompleto}{Verifica integrità DIP completo}
\begin{figure}[H]
      \centering
      \includegraphics[width=0.6\textwidth]{../assets/uml/UC33.png}
      \caption{UC33 - Verifica integrità DIP completo}
      \label{fig:uc_verificaIntegritaDIPCompleto}
\end{figure}
\begin{itemize}
    \item \textbf{Attore Primario}: Utente
    \item \textbf{Precondizioni}:
        \begin{itemize}
            \item L'Utente ha avviato l'applicazione
        \end{itemize}
    \item \textbf{Postcondizioni}: 
        \begin{enumerate}
            \item Il sistema compone il report di integrità dell'intero DIP
            \item Stato di verifica DIP aggiornato (Non Verificato / Valido / Non Valido).
        \end{enumerate}      
    \item \textbf{Flusso Principale}:
        \begin{enumerate}
            \item L'utente seleziona "Verifica Integrità dell'intero DIP"
            \item Il sistema verifica il DIP completo
            \item Il sistema aggiorna lo stato della verifica del DIP.
        \end{enumerate}
\end{itemize}

\usecase{verificaIntegritaClasseDocumentale}{Verifica integrità classe documentale}
\begin{figure}[H]
      \centering
      \includegraphics[width=0.6\textwidth]{../assets/uml/UC34.png}
      \caption{UC34 - Verifica integrità classe documentale}
      \label{fig:uc_verificaIntegritaClasseDocumentale}
\end{figure}
\begin{itemize}
    \item \textbf{Attore Primario}: Utente
    \item \textbf{Precondizioni}: 
        \begin{itemize}
            \item L'Utente ha avviato l'applicazione
            \item L'Utente si trova nella vista principale del DIP
            \item L'Utente ha selezionato una classe documentale.
        \end{itemize} 
    \item \textbf{Postcondizioni}: 
        \begin{enumerate}
            \item Il sistema compone il Report di integrità classe documentale
            \item Stato di verifica classe aggiornato (Non Verificato / Valido / Non Valido).
        \end{enumerate}      
    \item \textbf{Flusso Principale}:
        \begin{enumerate}
            \item L'utente seleziona "Verifica Integrità" sulla classe documentale
            \item Il sistema verifica la classe documentale
            \item Il sistema aggiorna lo stato della verifica della classe documentale.
        \end{enumerate}
\end{itemize}

\usecase{verificaIntegritaProcesso}{Verifica integrità processo}
\begin{figure}[H]
      \centering
      \includegraphics[width=0.6\textwidth]{../assets/uml/UC35.png}
      \caption{UC35 - Verifica integrità processo}
      \label{fig:uc_verificaIntegritaProcesso}
\end{figure}
\begin{itemize}
    \item \textbf{Attore Primario}: Utente
    \item \textbf{Precondizioni}: 
        \begin{itemize}
            \item L'Utente ha avviato l'applicazione
            \item L'Utente si trova nella vista principale del DIP
            \item L'Utente ha selezionato un processo.
        \end{itemize} 
    \item \textbf{Postcondizioni}: 
        \begin{enumerate}
            \item Il sistema compone il report di integrità processo
            \item Stato di verifica processo aggiornato (Non Verificato / Valido / Non Valido).
        \end{enumerate}      
    \item \textbf{Flusso Principale}:
        \begin{enumerate}
            \item L'utente selezione "Verifica Integrità" sul processo
            \item Il sistema verifica il processo
            \item Il sistema aggiorna lo stato della verifica del processo.
        \end{enumerate}
\end{itemize}

\usecase{verificaIntegritaDocumento}{Verifica integrità documento}
\begin{figure}[H]
      \centering
      \includegraphics[width=0.6\textwidth]{../assets/uml/UC36.png}
      \caption{UC36 - Verifica integrità documento}
      \label{fig:uc_verificaIntegritaDocumento}
\end{figure}
\begin{itemize}
    \item \textbf{Attore Primario}: Utente
    \item \textbf{Precondizioni}: 
        \begin{itemize}
            \item L'Utente ha avviato l'applicazione
            \item L'Utente si trova nella vista principale del DIP
            \item L'Utente ha selezionato un documento.
        \end{itemize} 
    \item \textbf{Postcondizioni}: 
        \begin{enumerate}
            \item Il sistema compone il report di integrità documento
            \item Stato di verifica documento aggiornato (Non Verificato / Valido / Non Valido).
        \end{enumerate}      
    \item \textbf{Flusso Principale}:
        \begin{enumerate}
            \item L'utente selezione "Verifica Integrità" sul documento
            \item Il sistema esegue la verifica del documento
            \item Il sistema aggiorna lo stato della verifica del documento.
        \end{enumerate}
\end{itemize}

\usecase{visualizzazioneReportIntegritaDIPCompleto}{Visualizzazione report integrità DIP completo}
\begin{figure}[H]
    \centering
    \includegraphics[width=0.6\textwidth]{../assets/uml/UC37.png}
      \caption{UC37 - Visualizzazione report integrità DIP completo}
    \label{fig:uc_visualizzazioneReportIntegritaDIPCompleto}
\end{figure}
\begin{itemize}
    \item \textbf{Attore Primario}: Utente
    \item \textbf{Precondizioni}: 
        \begin{itemize}
            \item L'Utente ha avviato l'applicazione
            \item È stata completata la verifica di integrità del DIP completo.
        \end{itemize} 
    \item \textbf{Postcondizioni}: Il report dettagliato del DIP viene visualizzato.     
    \item \textbf{Flusso Principale}:
        \begin{enumerate}
            \item L'utente visualizza il report della verifica del DIP completo
            \item L'utente visualizza un pannello con le informazioni aggregate del DIP.
        \end{enumerate}
    \item \textbf{Inclusioni}: 
        \begin{itemize}
            \item \ref{visualizzazioneNumeroClassiVerificate} Visualizzazione numero classi verificate
            \item \ref{visualizzazioneNumeroClassiIntegre} Visualizzazione numero classi integre
            \item \ref{visualizzazioneNumeroClassiCorrotte} Visualizzazione numero classi corrotte
            \item \ref{visualizzazioneListaClassiCorrotte} Visualizzazione lista classi corrotte
            \item \ref{visualizzazioneDataEOraVerificaDIP} Visualizzazione data e ora verifica DIP.
        \end{itemize}
\end{itemize}
\begin{figure}[H]
    \centering
    \includegraphics[width=1\textwidth]{../assets/uml/IncUC37.png}
      \caption{Inclusioni UC37 - Visualizzazione report integrità DIP completo}
    \label{fig:inclusioniVisualizzazioneReportIntegritaDIPCompleto}
\end{figure}

\subusecase{visualizzazioneNumeroClassiVerificate}{Visualizzazione numero classi verificate}
\begin{itemize}
    \item \textbf{Attore Primario}: Utente
    \item \textbf{Precondizioni}: 
        \begin{itemize}
            \item L'Utente ha avviato l'applicazione
            \item È stata completata la verifica di integrità del DIP completo
            \item È stato generato un report di verifica del DIP completo.
        \end{itemize} 
    \item \textbf{Postcondizioni}: Numero totale di classi verificate visualizzato.     
    \item \textbf{Flusso Principale}:
        \begin{enumerate}
            \item Il sistema conta tutte le classi documentali sottoposte a verifica
            \item Il sistema mostra il conteggio con l'etichetta "Classi verificate: [N]".
        \end{enumerate}
\end{itemize}

\subusecase{visualizzazioneNumeroClassiIntegre}{Visualizzazione numero classi integre}
\begin{itemize}
    \item \textbf{Attore Primario}: Utente
    \item \textbf{Precondizioni}: 
        \begin{itemize}
            \item L'Utente ha avviato l'applicazione
            \item È stata completata la verifica di integrità del DIP completo
            \item È stato generato un report di verifica del DIP completo.
        \end{itemize} 
    \item \textbf{Postcondizioni}: Numero di classi integre visualizzato.     
    \item \textbf{Flusso Principale}:
        \begin{enumerate}
            \item Il sistema conta le classi con stato "Valido"
            \item Il sistema mostra il conteggio con l'etichetta "Classi integre: [N]" in colore verde.
        \end{enumerate}
\end{itemize}

\subusecase{visualizzazioneNumeroClassiCorrotte}{Visualizzazione numero classi corrotte}
\begin{itemize}
    \item \textbf{Attore Primario}: Utente
    \item \textbf{Precondizioni}: 
        \begin{itemize}
            \item L'Utente ha avviato l'applicazione
            \item È stata completata la verifica di integrità del DIP completo
            \item È stato generato un report di verifica del DIP completo.
        \end{itemize} 
    \item \textbf{Postcondizioni}: Numero di classi corrotte visualizzato.     
    \item \textbf{Flusso Principale}:
        \begin{enumerate}
            \item Il sistema conta le classi con stato "Non Valido"
            \item Il sistema mostra il conteggio con l'etichetta "Classi corrotte: [N]" in colore rosso.
        \end{enumerate}
\end{itemize}

\subusecase{visualizzazioneListaClassiCorrotte}{Visualizzazione lista classi corrotte}
\begin{itemize}
    \item \textbf{Attore Primario}: Utente
    \item \textbf{Precondizioni}: 
        \begin{itemize}
            \item L'Utente ha avviato l'applicazione
            \item È stata completata la verifica di integrità del DIP completo
            \item È stato generato un report di verifica del DIP completo.
            \item Esistono classi corrotte (numero classi corrotte > 0).
        \end{itemize} 
    \item \textbf{Postcondizioni}: Lista completa delle classi corrotte visualizzata.     
    \item \textbf{Flusso Principale}:
        \begin{enumerate}
            \item Il sistema recupera l'elenco di tutte le classi con stato "Non Valido"
            \item Per ogni classe corrotta il sistema mostra il nome della classe e il numero di processi corrotti nella classe.
        \end{enumerate}
\end{itemize}

\subusecase{visualizzazioneDataEOraVerificaDIP}{Visualizzazione data e ora verifica DIP}
\begin{itemize}
    \item \textbf{Attore Primario}: Utente
    \item \textbf{Precondizioni}: 
        \begin{itemize}
            \item L'Utente ha avviato l'applicazione
            \item È stata completata la verifica di integrità del DIP completo
            \item È stato generato un report di verifica del DIP completo.
        \end{itemize} 
    \item \textbf{Postcondizioni}: Timestamp della verifica visualizzato.     
    \item \textbf{Flusso Principale}:
        \begin{enumerate}
            \item Il sistema recupera la data e ora di inizio della verifica del DIP
            \item Il sistema mostra il timestamp nel formato "GG/MM/AAAA HH:MM:SS".
        \end{enumerate}
\end{itemize}

\usecase{visualizzazioneReportIntegritaClasseDocumentale}{Visualizzazione report integrità classe documentale}
\begin{figure}[H]
    \centering
    \includegraphics[width=0.6\textwidth]{../assets/uml/UC38.png}
      \caption{UC38 - Visualizzazione report integrità classe documentale}
    \label{fig:uc_visualizzazioneReportIntegritaClasseDocumentale}
\end{figure}
\begin{itemize}
    \item \textbf{Attore Primario}: Utente
    \item \textbf{Precondizioni}: 
        \begin{itemize}
            \item L'Utente ha avviato l'applicazione
            \item È stata completata la verifica di integrità della classe documentale selezionata.
        \end{itemize} 
    \item \textbf{Postcondizioni}: Report dettagliato della classe visualizzato.     
    \item \textbf{Flusso Principale}:
        \begin{enumerate}
            \item Il sistema mostra il report aggregando i dati di tutti i processi della classe
            \item Il sistema mostra un pannello con le informazioni aggregate della classe.
        \end{enumerate}
    \item \textbf{Inclusioni}: 
        \begin{itemize}
            \item \ref{visualizzazioneNumeroProcessiVerificati} Visualizzazione Numero Processi Verificati
            \item \ref{visualizzazioneNumeroProcessiIntegri} Visualizzazione Numero Processi Integri
            \item \ref{visualizzazioneNumeroProcessiCorrotti} Visualizzazione Numero Processi Corrotti
            \item \ref{visualizzazioneListaProcessiCorrotti} Visualizzazione Lista Processi Corrotti
            \item \ref{visualizzazioneDataEOraVerificaClasse} Visualizzazione Data e Ora Verifica Classe.
        \end{itemize}
\end{itemize}
\begin{figure}[H]
    \centering
    \includegraphics[width=1\textwidth]{../assets/uml/IncUC38.png}
    \caption{Inclusioni UC38 - Visualizzazione Report Integrità Classe Documentale}
    \label{fig:inclusioniVisualizzazioneReportIntegritaClasseDocumentale}
\end{figure}

\subusecase{visualizzazioneNumeroProcessiVerificati}{Visualizzazione Numero Processi Verificati}
\begin{itemize}
    \item \textbf{Attore Primario}: Utente
    \item \textbf{Precondizioni}: 
        \begin{itemize}
            \item L'Utente ha avviato l'applicazione
            \item È stata completata una verifica di integrità della classe documentale
            \item È stato generato un report di verifica della classe.
        \end{itemize} 
    \item \textbf{Postcondizioni}: Numero totale di processi verificati visualizzato.     
    \item \textbf{Flusso Principale}:
        \begin{enumerate}
            \item Il sistema conta tutti i processi della classe sottoposti a verifica
            \item Il sistema mostra il conteggio con l'etichetta "Processi verificati: [N]".
        \end{enumerate}
\end{itemize}

\subusecase{visualizzazioneNumeroProcessiIntegri}{Visualizzazione numero processi integri}
\begin{itemize}
    \item \textbf{Attore Primario}: Utente
    \item \textbf{Precondizioni}: 
        \begin{itemize}
            \item L'Utente ha avviato l'applicazione
            \item È stata completata una verifica di integrità della classe documentale
            \item È stato generato un report di verifica della classe.
        \end{itemize} 
    \item \textbf{Postcondizioni}: Numero di processi integri visualizzato.     
    \item \textbf{Flusso Principale}:
        \begin{enumerate}
            \item Il sistema conta i processi con stato "Valido"
            \item Il sistema mostra il conteggio con l'etichetta "Processi integri: [N]" in colore verde.
        \end{enumerate}
\end{itemize}

\subusecase{visualizzazioneNumeroProcessiCorrotti}{Visualizzazione numero processi corrotti}
\begin{itemize}
    \item \textbf{Attore Primario}: Utente
    \item \textbf{Precondizioni}: 
        \begin{itemize}
            \item L'Utente ha avviato l'applicazione
            \item È stata completata una verifica di integrità della classe documentale
            \item È stato generato un report di verifica della classe.
        \end{itemize} 
    \item \textbf{Postcondizioni}: Numero di processi corrotti visualizzato.     
    \item \textbf{Flusso Principale}:
        \begin{enumerate}
            \item Il sistema conta i processi con stato "Non Valido"
            \item Il sistema mostra il conteggio con l'etichetta "Processi corrotti: [N]" in colore rosso.
        \end{enumerate}
\end{itemize}

\subusecase{visualizzazioneListaProcessiCorrotti}{Visualizzazione lista processi corrotti}
\begin{itemize}
    \item \textbf{Attore Primario}: Utente
    \item \textbf{Precondizioni}: 
        \begin{itemize}
            \item L'Utente ha avviato l'applicazione
            \item È stata completata una verifica di integrità della classe documentale
            \item È stato generato un report di verifica della classe
            \item Esistono processi corrotti (numero processi corrotti > 0).
        \end{itemize}      
    \item \textbf{Postcondizioni}: Lista completa dei processi corrotti visualizzata.     
    \item \textbf{Flusso Principale}:
        \begin{enumerate}
            \item Il sistema recupera l'elenco di tutti i processi con stato "Non Valido"
            \item Per ogni processo corrotto, il sistema mostra il nome del processo e il numero dei documenti corrotti.
        \end{enumerate}
\end{itemize}

\subusecase{visualizzazioneDataEOraVerificaClasse}{Visualizzazione data e ora verifica classe}
\begin{itemize}
    \item \textbf{Attore Primario}: Utente
    \item \textbf{Precondizioni}: 
        \begin{itemize}
            \item L'Utente ha avviato l'applicazione
            \item È stata completata una verifica di integrità della classe documentale
            \item È stato generato un report di verifica della classe.
        \end{itemize}  
    \item \textbf{Postcondizioni}: Timestamp della verifica visualizzato.
    \item \textbf{Flusso Principale}:
        \begin{enumerate}
            \item Il sistema recupera la data e ora di inizio della verifica della classe
            \item Il sistema mostra il timestamp nel formato "GG/MM/AAAA HH:MM:SS".
        \end{enumerate}
\end{itemize}

\usecase{visualizzazioneReportIntegritaProcesso}{Visualizzazione report integrità processo}
\begin{figure}[H]
    \centering
    \includegraphics[width=0.6\textwidth]{../assets/uml/UC39.png}
      \caption{UC39 - Visualizzazione report integrità processo}
    \label{fig:uc_visualizzazioneReportIntegritaProcesso}
\end{figure}
\begin{itemize}
    \item \textbf{Attore Primario}: Utente
    \item \textbf{Precondizioni}: 
        \begin{itemize}
            \item L'Utente ha avviato l'applicazione
            \item È stata completata una verifica di integrità di un processo.
        \end{itemize} 
    \item \textbf{Postcondizioni}: Report dettagliato del processo visualizzato.     
    \item \textbf{Flusso Principale}:
        \begin{enumerate}
            \item Il sistema mostra il report aggregando i dati di tutti i documenti del processo
            \item Il sistema mostra un pannello con le informazioni aggregate del processo.
        \end{enumerate}
    \item \textbf{Inclusioni}: 
        \begin{itemize}
            \item \ref{visualizzazioneNumeroDocumentiVerificati} Visualizzazione numero documenti verificati
            \item \ref{visualizzazioneNumeroDocumentiIntegri} Visualizzazione numero documenti integri
            \item \ref{visualizzazioneNumeroDocumentiCorrotti} Visualizzazione numero documenti corrotti
            \item \ref{visualizzazioneListaDocumentiCorrotti} Visualizzazione lista documenti corrotti
            \item \ref{visualizzazioneDataEOraVerificaProcesso} Visualizzazione data e ora verifica processo
        \end{itemize}
\end{itemize}
\begin{figure}[H]
    \centering
    \includegraphics[width=1\textwidth]{../assets/uml/IncUC39.png}
      \caption{Inclusioni UC39 - Visualizzazione report integrità processo}
    \label{fig:inclusioniVisualizzazioneReportIntegritaProcesso}
\end{figure}

\subusecase{visualizzazioneNumeroDocumentiVerificati}{Visualizzazione numero documenti verificati}
\begin{itemize}
    \item \textbf{Attore Primario}: Utente
    \item \textbf{Precondizioni}: 
        \begin{itemize}
            \item L'Utente ha avviato l'applicazione
            \item È stata completata una verifica di integrità di un processo
            \item È stato generato un report di verifica del processo.
        \end{itemize} 
    \item \textbf{Postcondizioni}: Numero totale di documenti verificati visualizzato.     
    \item \textbf{Flusso Principale}:
        \begin{enumerate}
            \item Il sistema conta tutti i documenti del processo sottoposti a verifica
            \item Il sistema mostra il conteggio con l'etichetta "Documenti verificati: [N]".
        \end{enumerate}
\end{itemize}

\subusecase{visualizzazioneNumeroDocumentiIntegri}{Visualizzazione numero documenti integri}
\begin{itemize}
    \item \textbf{Attore Primario}: Utente
    \item \textbf{Precondizioni}: 
        \begin{itemize}
            \item L'Utente ha avviato l'applicazione
            \item È stata completata una verifica di integrità di un processo
            \item È stato generato un report di verifica del processo.
        \end{itemize} 
    \item \textbf{Postcondizioni}: Numero di documenti integri visualizzato.     
    \item \textbf{Flusso Principale}:
        \begin{enumerate}
            \item Il sistema conta i documenti con stato "Valido"
            \item Il sistema mostra il conteggio con l'etichetta "Documenti integri: [N]" in colore verde.
        \end{enumerate}
\end{itemize}

\subusecase{visualizzazioneNumeroDocumentiCorrotti}{Visualizzazione numero documenti corrotti}
\begin{itemize}
    \item \textbf{Attore Primario}: Utente
    \item \textbf{Precondizioni}:
        \begin{itemize}
            \item L'Utente ha avviato l'applicazione
            \item È stata completata una verifica di integrità di un processo
            \item È stato generato un report di verifica del processo.
        \end{itemize} 
    \item \textbf{Postcondizioni}: Numero di documenti corrotti visualizzato.     
    \item \textbf{Flusso Principale}:
        \begin{enumerate}
            \item Il sistema conta i documenti con stato "Non Valido"
            \item Il sistema mostra il conteggio con l'etichetta "Documenti corrotti: [N]" in colore rosso.
        \end{enumerate}
\end{itemize}

\subusecase{visualizzazioneListaDocumentiCorrotti}{Visualizzazione lista documenti corrotti}
\begin{itemize}
    \item \textbf{Attore Primario}: Utente
    \item \textbf{Precondizioni}: 
        \begin{itemize}
            \item L'Utente ha avviato l'applicazione
            \item È stata completata una verifica di integrità di un processo
            \item È stato generato un report di verifica del processo
            \item Esistono documenti corrotti (numero documenti corrotti > 0).
        \end{itemize}    
    \item \textbf{Postcondizioni}: Lista completa dei documenti corrotti visualizzata.     
    \item \textbf{Flusso Principale}:
        \begin{enumerate}
            \item Il sistema recupera l'elenco di tutti i documenti con stato "Non Valido"
            \item Per ogni documento corrotto, il sistema mostra il nome del documento e l'errore specifico.
        \end{enumerate}
\end{itemize}

\subusecase{visualizzazioneDataEOraVerificaProcesso}{Visualizzazione data e ora verifica processo}
\begin{itemize}
    \item \textbf{Attore Primario}: Utente
    \item \textbf{Precondizioni}: 
        \begin{itemize}
            \item L'Utente ha avviato l'applicazione
            \item È stata completata una verifica di integrità di un processo
            \item È stato generato un report di verifica del processo.
        \end{itemize} 
    \item \textbf{Postcondizioni}: Timestamp della verifica visualizzato.
    \item \textbf{Flusso Principale}:
        \begin{enumerate}
            \item Il sistema recupera la data e ora di inizio della verifica del processo
            \item Il sistema mostra il timestamp nel formato "GG/MM/AAAA HH:MM:SS".
        \end{enumerate}
\end{itemize}

\usecase{visualizzazioneReportIntegritaDocumento}{Visualizzazione report integrità documento}
\begin{figure}[H]
    \centering
    \includegraphics[width=0.6\textwidth]{../assets/uml/UC40.png}
      \caption{UC40 - Visualizzazione report integrità documento}
    \label{fig:uc_visualizzazioneReportIntegritaDocumento}
\end{figure}
\begin{itemize}
    \item \textbf{Attore Primario}: Utente
    \item \textbf{Precondizioni}: 
        \begin{itemize}
            \item L'Utente ha avviato l'applicazione
            \item È stata completata una verifica di integrità di un documento.
        \end{itemize} 
    \item \textbf{Postcondizioni}: Report dettagliato del documento visualizzato.     
    \item \textbf{Flusso Principale}:
        \begin{enumerate}
            \item Il sistema mostra:
                \begin{itemize}
                    \item Nome del documento (\ref{visualizzazioneNomeDocumento})
                    \item Stato della verifica (Valido / Non Valido) (\ref{visualizzazioneStatoVerificaDocumento})
                    \item Data e ora della verifica (\ref{visualizzazioneDataEOraVerificaDocumento})
                    \item Dettagli dell'errore (se presenti) (\ref{visualizzazioneDettagliErroreDocumento})
                \end{itemize}
        \end{enumerate}
    \item \textbf{Inclusioni}: 
        \begin{itemize}
            \item \ref{visualizzazioneNomeDocumento} Visualizzazione nome documento
            \item \ref{visualizzazioneStatoVerificaDocumento} Visualizzazione stato verifica documento
            \item \ref{visualizzazioneDataEOraVerificaDocumento} Visualizzazione data e ora verifica documento
            \item \ref{visualizzazioneDettagliErroreDocumento} Visualizzazione dettaglio errore documento (se presente)
        \end{itemize}
\end{itemize}
\begin{figure}[H]
    \centering
    \includegraphics[width=1\textwidth]{../assets/uml/IncUC40.png}
      \caption{Inclusioni UC40 - Visualizzazione report integrità documento}
    \label{fig:inclusioniVisualizzazioneReportIntegritaDocumento}
\end{figure}



\subusecase{visualizzazioneStatoVerificaDocumento}{Visualizzazione stato verifica documento}
\begin{itemize}
    \item \textbf{Attore Primario}: Utente
    \item \textbf{Precondizioni}: 
        \begin{itemize}
            \item L'Utente ha avviato l'applicazione
            \item È stata completata una verifica di integrità di un documento
            \item È stato generato un report di verifica del documento.
        \end{itemize} 
    \item \textbf{Postcondizioni}: Stato della verifica visualizzato.     
    \item \textbf{Flusso Principale}:
        \begin{enumerate}
            \item Il sistema mostra lo stato della verifica.
        \end{enumerate}
\end{itemize}

\subusecase{visualizzazioneDataEOraVerificaDocumento}{Visualizzazione data e ora verifica documento}
\begin{itemize}
    \item \textbf{Attore Primario}: Utente
    \item \textbf{Precondizioni}: 
        \begin{itemize}
            \item L'Utente ha avviato l'applicazione
            \item È stata completata una verifica di integrità di un documento
            \item È stato generato un report di verifica del documento.
        \end{itemize} 
    \item \textbf{Postcondizioni}: Timestamp della verifica visualizzato.     
    \item \textbf{Flusso Principale}:
        \begin{enumerate}
            \item Il sistema recupera la data e ora di inizio della verifica del documento
            \item Il sistema mostra il timestamp nel formato "GG/MM/AAAA HH:MM:SS".
        \end{enumerate}
\end{itemize}

\subusecase{visualizzazioneDettagliErroreDocumento}{Visualizzazione dettagli errore documento}
\begin{itemize}
    \item \textbf{Attore Primario}: Utente
    \item \textbf{Precondizioni}: 
        \begin{itemize}
            \item L'Utente ha avviato l'applicazione
            \item È stata completata una verifica di integrità di un documento
            \item È stato generato un report di verifica del documento.
            \item Il documento ha stato "Non Valido".
        \end{itemize} 
    \item \textbf{Postcondizioni}: Dettagli dell'errore visualizzati.     
    \item \textbf{Flusso Principale}:
        \begin{enumerate}
            \item Il sistema mostra la descrizione dell'errore (ad esempio: hash calcolato non coincide con hash associato al documento, firma digitale non valida o scaduta).
        \end{enumerate}
\end{itemize}

\usecase{visualizzazioneNomeDocumento}{Visualizzazione nome documento}
\begin{itemize}
    \item \textbf{Attore Primario}: Utente
    \item \textbf{Precondizioni}: 
        \begin{itemize}
            \item L'Utente ha avviato l'applicazione
        \end{itemize}
    \item \textbf{Postcondizioni}: Nome documento visualizzato.     
    \item \textbf{Flusso Principale}:
        \begin{enumerate}
            \item Il sistema mostra il nome del documento.
        \end{enumerate}
\end{itemize}

\usecase{convertiReportVerificaPDF}{Converti report verifica in PDF}
\begin{figure}[H]
    \centering
    \includegraphics[width=0.8\textwidth]{../assets/uml/UC42.png}
      \caption{UC42 - Converti report verifica in PDF}
    \label{fig:uc_convertiReportVerificaPDF}
\end{figure}
\begin{itemize}
    \item \textbf{Attore Primario}: Utente
    \item \textbf{Precondizioni}: 
        \begin{itemize}
            \item L'Utente ha avviato l'applicazione
            \item È stata completata una verifica di integrità (DIP completo / Classe Documentale / Processo / Documento)
            \item È stato generato un report di verifica (DIP completo / Classe Documentale / Processo / Documento).
        \end{itemize} 
    \item \textbf{Postcondizioni}: Il report visualizzato viene convertito in PDF.     
    \item \textbf{Flusso Principale}:
        \begin{enumerate}
            \item L'utente seleziona "Salva Report"
            \item Il sistema converte il report corrente in formato PDF.
        \end{enumerate}
    \item \textbf{Flusso Alternativo}:
    \begin{itemize}
        \item La conversione non va a buon fine: viene visualizzato un messaggio di errore "Impossibile generare il PDF. Riprovare.".
    \end{itemize}
    \item \textbf{Estensioni}: \ref{erroreGenerazionePDF} Errore generazione PDF
\end{itemize}

\usecase{scaricaFile}{Scarica file}
\begin{figure}[H]
    \centering
    \includegraphics[width=0.8\textwidth]{../assets/uml/UC43.png}
    \caption{UC43 - Scarica File}
    \label{fig:uc_scaricaFile}
\end{figure}
\begin{itemize}
    \item \textbf{Attore Primario}: Utente
    \item \textbf{Precondizioni}: 
        \begin{itemize}
            \item L'Utente ha avviato l'applicazione
            \item L'Utente ha selezionato un file da scaricare (Documento o Report).
        \end{itemize} 
    \item \textbf{Postcondizioni}: L'utente ha scaricato il File nella cartella selezionata.     
    \item \textbf{Flusso Principale}:
        \begin{enumerate}
            \item Il sistema apre un dialogo di selezione cartella
            \item L'utente seleziona la cartella di destinazione
            \item Il sistema salva il File nella cartella selezionata
            \item Il sistema conferma il salvataggio con un messaggio "File salvato con successo in [percorso]".
        \end{enumerate}
    \item \textbf{Flusso Alternativo}:
        \begin{itemize}
            \item L'utente tenta di salvare il documento nella cartella del DIP: il sistema blocca l'operazione e mostra il messaggio "Impossibile salvare nel DIP. Selezionare un'altra cartella per preservare l'integrità del DIP."
            \item L'utente annulla l'operazione: il sistema chiude il dialogo senza salvare.
        \end{itemize}
      \item \textbf{Estensioni}: \ref{erroreScaricamentoFile} Errore scaricamento file
\end{itemize}

\usecase{erroreScaricamentoFile}{Errore scaricamento file}
\begin{itemize}
    \item \textbf{Attore Primario}: Utente
    \item \textbf{Precondizioni}: 
        \begin{itemize}
            \item L'Utente ha avviato l'applicazione
            \item L'Utente ha selezionato un file da scaricare (Documento o Report).
        \end{itemize} 
    \item \textbf{Postcondizioni}: L'utente visualizza un messaggio d'errore.
    \item \textbf{Flusso Principale}:
        \begin{enumerate}
            \item Il sistema apre un dialogo di selezione cartella
            \item L'utente seleziona la cartella di destinazione
            \item Il sistema blocca l'operazione e mostra il messaggio "Impossibile salvare nel DIP. Selezionare un'altra cartella per preservare l'integrità del DIP."
            \item L'utente annulla l'operazione: il sistema chiude il dialogo senza salvare.
        \end{enumerate}
\end{itemize}

\usecase{erroreGenerazionePDF}{Errore generazione PDF}
\begin{itemize}
    \item \textbf{Attore Primario}: Utente
    \item \textbf{Precondizioni}: 
        \begin{itemize}
            \item L'Utente ha avviato l'applicazione
            \item È stata completata una verifica di integrità (DIP completo / Classe Documentale / Processo / Documento)
            \item È stato generato un report di verifica (DIP completo / Classe Documentale / Processo / Documento)
            \item È stata effettuata una conversione del report in PDF
            \item Una conversione PDF non è andata a buon fine.
        \end{itemize} 
    \item \textbf{Postcondizioni}: L'utente visualizza un messaggio d'errore.     
    \item \textbf{Flusso Principale}:
        \begin{enumerate}
            \item L'utente ha provato a convertire un Report come PDF
            \item La conversione non va a buon fine
            \item L'utente visualizza un messaggio d'errore.
        \end{enumerate}
\end{itemize}
