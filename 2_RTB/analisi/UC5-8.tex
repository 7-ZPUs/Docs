\usecase{verificaIntegritaDIPCompleto}{Verifica integrità DIP completo}
\begin{figure}[H]
      \centering
      \includegraphics[width=0.6\textwidth]{../assets/uml/UC26.png}
      \caption{\ref{verificaIntegritaDIPCompleto} - Verifica integrità DIP completo}
      \label{fig:uc_verificaIntegritaDIPCompleto}
\end{figure}
\begin{itemize}
    \item \textbf{Attore Primario}: Utente
    \item \textbf{Precondizioni}:
        \begin{itemize}
            \item L'Utente ha avviato l'applicazione
        \end{itemize}
    \item \textbf{Postcondizioni}: 
        \begin{enumerate}
            \item Stato di verifica del DIP e delle sue componenti aggiornato (Non Verificato / Valido / Non Valido).
        \end{enumerate}      
    \item \textbf{Flusso Principale}:
        \begin{enumerate}
            \item L'utente sceglie: "Verifica Integrità dell'intero DIP"
            \item Il sistema aggiorna lo stato della verifica del DIP e delle sue componenti.
        \end{enumerate}
\end{itemize}

\usecase{verificaIntegritaClasseDocumentale}{Verifica integrità classe documentale}
\begin{figure}[H]
      \centering
      \includegraphics[width=0.6\textwidth]{../assets/uml/UC27.png}
      \caption{\ref{verificaIntegritaClasseDocumentale} - Verifica integrità classe documentale}
      \label{fig:uc_verificaIntegritaClasseDocumentale}
\end{figure}
\begin{itemize}
    \item \textbf{Attore Primario}: Utente
    \item \textbf{Precondizioni}: 
        \begin{itemize}
            \item L'Utente ha avviato l'applicazione
            \item L'Utente ha selezionato una classe documentale.
        \end{itemize} 
    \item \textbf{Postcondizioni}: 
        \begin{enumerate}
            \item Stato di verifica della classe documentale e delle sue componenti aggiornato (Non Verificato / Valido / Non Valido).
        \end{enumerate}      
    \item \textbf{Flusso Principale}:
        \begin{enumerate}
            \item L'utente sceglie "Verifica Integrità" sulla classe documentale
            \item Il sistema aggiorna lo stato della verifica della classe documentale e delle sue componenti.
        \end{enumerate}
\end{itemize}

\usecase{verificaIntegritaProcesso}{Verifica integrità processo}
\begin{figure}[H]
      \centering
      \includegraphics[width=0.6\textwidth]{../assets/uml/UC28.png}
      \caption{\ref{verificaIntegritaProcesso} - Verifica integrità processo}
      \label{fig:uc_verificaIntegritaProcesso}
\end{figure}
\begin{itemize}
    \item \textbf{Attore Primario}: Utente
    \item \textbf{Precondizioni}: 
        \begin{itemize}
            \item L'Utente ha avviato l'applicazione
            \item L'Utente ha selezionato un processo.
        \end{itemize}
    \item \textbf{Postcondizioni}: 
        \begin{enumerate}
            \item Stato di verifica del processo e delle sue componenti aggiornato (Non Verificato / Valido / Non Valido).
        \end{enumerate}      
    \item \textbf{Flusso Principale}:
        \begin{enumerate}
            \item L'utente sceglie "Verifica Integrità" sul processo
            \item Il sistema aggiorna lo stato della verifica del processo e delle sue componenti.
        \end{enumerate}
\end{itemize}

\usecase{verificaIntegritaDocumento}{Verifica integrità documento}
\begin{figure}[H]
      \centering
      \includegraphics[width=0.6\textwidth]{../assets/uml/UC29.png}
      \caption{\ref{verificaIntegritaDocumento} - Verifica integrità documento}
      \label{fig:uc_verificaIntegritaDocumento}
\end{figure}
\begin{itemize}
    \item \textbf{Attore Primario}: Utente
    \item \textbf{Precondizioni}: 
        \begin{itemize}
            \item L'Utente ha avviato l'applicazione
            \item L'Utente ha selezionato un documento.
        \end{itemize} 
    \item \textbf{Postcondizioni}: 
        \begin{enumerate}
            \item Stato di verifica documento aggiornato (Non Verificato / Valido / Non Valido).
        \end{enumerate}      
    \item \textbf{Flusso Principale}:
        \begin{enumerate}
            \item L'utente sceglie "Verifica Integrità" sul documento
            \item Il sistema aggiorna lo stato della verifica del documento.
        \end{enumerate}
\end{itemize}

\usecase{visualizzazioneReportIntegritaDIPCompleto}{Visualizzazione report integrità DIP completo}
\begin{figure}[H]
    \centering
    \includegraphics[width=0.6\textwidth]{../assets/uml/UC30.png}
      \caption{\ref{visualizzazioneReportIntegritaDIPCompleto} - Visualizzazione report integrità DIP completo}
    \label{fig:uc_visualizzazioneReportIntegritaDIPCompleto}
\end{figure}
\begin{itemize}
    \item \textbf{Attore Primario}: Utente
    \item \textbf{Precondizioni}: 
        \begin{itemize}
            \item L'Utente ha avviato l'applicazione
            \item È stata completata la verifica di integrità del DIP completo.
        \end{itemize} 
    \item \textbf{Postcondizioni}: Il report dettagliato del DIP viene visualizzato.     
    \item \textbf{Flusso Principale}:
        \begin{enumerate}
            \item L'utente visualizza il report della verifica del DIP completo
        \end{enumerate}
    \item \textbf{Inclusioni}: 
        \begin{itemize}
            \item \ref{visualizzazioneNumeroClassiVerificate} Visualizzazione numero classi verificate
            \item \ref{visualizzazioneNumeroClassiIntegre} Visualizzazione numero classi integre
            \item \ref{visualizzazioneNumeroClassiCorrotte} Visualizzazione numero classi corrotte
            \item \ref{visualizzazioneListaClassiCorrotte} Visualizzazione lista classi corrotte
            \item \ref{visualizzazioneDataEOraVerificaDIP} Visualizzazione data e ora verifica DIP.
        \end{itemize}
\end{itemize}
\begin{figure}[H]
    \centering
    \includegraphics[width=1\textwidth]{../assets/uml/UC30INC.png}
      \caption{Inclusioni \ref{visualizzazioneReportIntegritaDIPCompleto} - Visualizzazione report integrità DIP completo}
    \label{fig:inclusioniVisualizzazioneReportIntegritaDIPCompleto}
\end{figure}

\subusecase{visualizzazioneNumeroClassiVerificate}{Visualizzazione numero classi verificate}
\begin{itemize}
    \item \textbf{Attore Primario}: Utente
    \item \textbf{Precondizioni}: 
        \begin{itemize}
            \item L'Utente ha avviato l'applicazione
            \item È stata completata la verifica di integrità del DIP completo
        \end{itemize} 
    \item \textbf{Postcondizioni}: Numero totale di classi verificate visualizzato.     
    \item \textbf{Flusso Principale}:
        \begin{enumerate}
            \item Il sistema mostra il numero di classi verificate.
        \end{enumerate}
\end{itemize}

\subusecase{visualizzazioneNumeroClassiIntegre}{Visualizzazione numero classi integre}
\begin{itemize}
    \item \textbf{Attore Primario}: Utente
    \item \textbf{Precondizioni}: 
        \begin{itemize}
            \item L'Utente ha avviato l'applicazione
            \item È stata completata la verifica di integrità del DIP completo
        \end{itemize} 
    \item \textbf{Postcondizioni}: Numero di classi integre visualizzato.     
    \item \textbf{Flusso Principale}:
        \begin{enumerate}
            \item Il sistema mostra il numero di classi integre.
        \end{enumerate}
\end{itemize}

\subusecase{visualizzazioneNumeroClassiCorrotte}{Visualizzazione numero classi corrotte}
\begin{itemize}
    \item \textbf{Attore Primario}: Utente
    \item \textbf{Precondizioni}: 
        \begin{itemize}
            \item L'Utente ha avviato l'applicazione
            \item È stata completata la verifica di integrità del DIP completo
        \end{itemize} 
    \item \textbf{Postcondizioni}: Numero di classi corrotte visualizzato.     
    \item \textbf{Flusso Principale}:
        \begin{enumerate}
            \item Il sistema mostra il numero di classi corrotte.
        \end{enumerate}
\end{itemize}

\subusecase{visualizzazioneListaClassiCorrotte}{Visualizzazione lista classi corrotte}
\begin{itemize}
    \item \textbf{Attore Primario}: Utente
    \item \textbf{Precondizioni}: 
        \begin{itemize}
            \item L'Utente ha avviato l'applicazione
            \item È stata completata la verifica di integrità del DIP completo
        \end{itemize}
    \item \textbf{Postcondizioni}: Lista completa delle classi corrotte visualizzata.     
    \item \textbf{Flusso Principale}:
        \begin{enumerate}
            \item Per ogni classe corrotta il sistema mostra il nome della classe e il numero di processi corrotti nella classe.
        \end{enumerate}
    \item \textbf{Flusso Alternativo}:
        \begin{itemize}
            \item Non esistono classi corrotte: viene visualizzato un messaggio "Tutte le classi sono integre".
        \end{itemize}
    \item \textbf{Inclusioni}: \begin{itemize}
        \item \ref{visualizzaElementoListaClassiCorrotte} Visualizzazione classe corrotta
    \end{itemize}
    \item \textbf{Estensioni}: \begin{itemize}
        \item \ref{nessunaClasseCorrotta} Nessuna classe corrotta
    \end{itemize}
\end{itemize}

\subsubusecase{visualizzaElementoListaClassiCorrotte}{Visualizzazione classe corrotta}
\begin{itemize}
    \item \textbf{Attore Primario}: Utente
    \item \textbf{Precondizioni}: 
        \begin{itemize}
            \item L'Utente ha avviato l'applicazione
            \item È stata completata la verifica di integrità del DIP completo
        \end{itemize}
    \item \textbf{Postcondizioni}: Dettagli della classe corrotta visualizzati.
    \item \textbf{Flusso Principale}:
        \begin{enumerate}
            \item Il sistema mostra il nome della classe corrotta
            \item Il sistema mostra il numero di processi corrotti nella classe.
        \end{enumerate}
    \item \textbf{Inclusioni}: \begin{itemize}
        \item \ref{nomeClasseDocumentale} Visualizzazione nome classe documentale
        \item \ref{visualizzazioneNumeroProcessiCorrotti} Visualizzazione numero processi corrotti
    \end{itemize}
\end{itemize}

\subusecase{visualizzazioneDataEOraVerificaDIP}{Visualizzazione data e ora verifica DIP}
\begin{itemize}
    \item \textbf{Attore Primario}: Utente
    \item \textbf{Precondizioni}: 
        \begin{itemize}
            \item L'Utente ha avviato l'applicazione
            \item È stata completata la verifica di integrità del DIP completo
        \end{itemize} 
    \item \textbf{Postcondizioni}: Timestamp della verifica del DIP visualizzato.     
    \item \textbf{Flusso Principale}:
        \begin{enumerate}
            \item Il sistema mostra il timestamp della verifica del DIP nel formato "GG/MM/AAAA HH:MM:SS".
        \end{enumerate}
\end{itemize}

\subusecase{nessunaClasseCorrotta}{Nessuna classe corrotta}
\begin{itemize}
    \item \textbf{Attore Primario}: Utente
    \item \textbf{Precondizioni}: 
        \begin{itemize}
            \item L'Utente ha avviato l'applicazione
            \item È stata completata la verifica di integrità del DIP completo
        \end{itemize}
    \item \textbf{Postcondizioni}: Messaggio "Tutte le classi sono integre" visualizzato.
    \item \textbf{Flusso Principale}:
        \begin{enumerate}
            \item Il sistema mostra un messaggio "Tutte le classi sono integre".
        \end{enumerate}
\end{itemize}

\usecase{visualizzazioneReportIntegritaClasseDocumentale}{Visualizzazione report integrità classe documentale}
\begin{figure}[H]
    \centering
    \includegraphics[width=0.6\textwidth]{../assets/uml/UC31.png}
      \caption{\ref{visualizzazioneReportIntegritaClasseDocumentale} - Visualizzazione report integrità classe documentale}
    \label{fig:uc_visualizzazioneReportIntegritaClasseDocumentale}
\end{figure}
\begin{itemize}
    \item \textbf{Attore Primario}: Utente
    \item \textbf{Precondizioni}: 
        \begin{itemize}
            \item L'Utente ha avviato l'applicazione
            \item È stata completata la verifica di integrità della classe documentale.
        \end{itemize} 
    \item \textbf{Postcondizioni}: Il report dettagliato della classe documentale viene visualizzato.   
    \item \textbf{Flusso Principale}:
        \begin{enumerate}
            \item L'utente visualizza il report della verifica della classe documentale
        \end{enumerate}
    \item \textbf{Inclusioni}: 
        \begin{itemize}
            \item \ref{visualizzazioneNumeroProcessiVerificati} Visualizzazione Numero Processi Verificati
            \item \ref{visualizzazioneNumeroProcessiIntegri} Visualizzazione Numero Processi Integri
            \item \ref{visualizzazioneNumeroProcessiCorrotti} Visualizzazione Numero Processi Corrotti
            \item \ref{visualizzazioneListaProcessiCorrotti} Visualizzazione Lista Processi Corrotti
            \item \ref{visualizzazioneDataEOraVerificaClasse} Visualizzazione Data e Ora Verifica Classe.
        \end{itemize}
\end{itemize}
\begin{figure}[H]
    \centering
    \includegraphics[width=1\textwidth]{../assets/uml/UC31INC.png}
    \caption{Inclusioni \ref{visualizzazioneReportIntegritaClasseDocumentale} - Visualizzazione Report Integrità Classe Documentale}
    \label{fig:inclusioniVisualizzazioneReportIntegritaClasseDocumentale}
\end{figure}

\subusecase{visualizzazioneNumeroProcessiVerificati}{Visualizzazione Numero Processi Verificati}
\begin{itemize}
    \item \textbf{Attore Primario}: Utente
    \item \textbf{Precondizioni}: 
        \begin{itemize}
            \item L'Utente ha avviato l'applicazione
            \item È stata completata una verifica di integrità della classe documentale
        \end{itemize} 
    \item \textbf{Postcondizioni}: Numero totale di processi verificati della classe documentale visualizzato.     
    \item \textbf{Flusso Principale}:
        \begin{enumerate}
            \item Il sistema mostra il numero totale di processi verificati della classe documentale.
        \end{enumerate}
\end{itemize}

\subusecase{visualizzazioneNumeroProcessiIntegri}{Visualizzazione numero processi integri}
\begin{itemize}
    \item \textbf{Attore Primario}: Utente
    \item \textbf{Precondizioni}: 
        \begin{itemize}
            \item L'Utente ha avviato l'applicazione
            \item È stata completata una verifica di integrità della classe documentale
        \end{itemize} 
    \item \textbf{Postcondizioni}: Numero di processi integri della classe documentale visualizzato.     
    \item \textbf{Flusso Principale}:
        \begin{enumerate}
            \item Il sistema mostra il numero di processi integri della classe documentale.
        \end{enumerate}
\end{itemize}

\subusecase{visualizzazioneNumeroProcessiCorrotti}{Visualizzazione numero processi corrotti}
\begin{itemize}
    \item \textbf{Attore Primario}: Utente
    \item \textbf{Precondizioni}: 
        \begin{itemize}
            \item L'Utente ha avviato l'applicazione
            \item È stata completata una verifica di integrità della classe documentale
        \end{itemize} 
    \item \textbf{Postcondizioni}: Numero di processi corrotti della classe documentale visualizzato.     
    \item \textbf{Flusso Principale}:
        \begin{enumerate}
            \item Il sistema mostra il numero di processi corrotti.
        \end{enumerate}
\end{itemize}

\subusecase{visualizzazioneListaProcessiCorrotti}{Visualizzazione lista processi corrotti}
\begin{itemize}
    \item \textbf{Attore Primario}: Utente
    \item \textbf{Precondizioni}: 
        \begin{itemize}
            \item L'Utente ha avviato l'applicazione
            \item È stata completata una verifica di integrità della classe documentale
        \end{itemize}      
    \item \textbf{Postcondizioni}: Lista completa dei processi corrotti della classe documentale visualizzata.     
    \item \textbf{Flusso Principale}:
        \begin{enumerate}
            \item Per ogni processo corrotto, il sistema mostra il nome del processo e il numero dei documenti corrotti.
        \end{enumerate}
    \item \textbf{Flusso Alternativo}:
        \begin{itemize}
            \item Non esistono processi corrotti nella classe documentale: viene visualizzato un messaggio "Tutti i processi sono integri".
        \end{itemize}
    \item \textbf{Inclusioni}: \begin{itemize}
        \item \ref{visualizzaElementoListaProcessiCorrotti} Visualizzazione processo corrotto
        \end{itemize}
    \item \textbf{Estensioni}
    \item \begin{itemize}
        \item \ref{nessunProcessoCorrotto} Nessun processo corrotto
    \end{itemize}
\end{itemize}

\subsubusecase{visualizzaElementoListaProcessiCorrotti}{Visualizzazione processo corrotto}
\begin{itemize}
    \item \textbf{Attore Primario}: Utente
    \item \textbf{Precondizioni}: 
        \begin{itemize}
            \item L'Utente ha avviato l'applicazione
            \item È stata completata una verifica di integrità della classe documentale
        \end{itemize}
    \item \textbf{Postcondizioni}: Dettagli del processo corrotto visualizzati.
    \item \textbf{Flusso Principale}:
        \begin{enumerate}
            \item Il sistema mostra l'id del processo corrotto
            \item Il sistema mostra il numero di documenti corrotti nel processo.
        \end{enumerate}
    \item \textbf{Inclusioni}: \begin{itemize}
        \item \ref{idProcessoClasse} Visualizzazione id processo
        \item \ref{visualizzazioneNumeroDocumentiCorrotti} Visualizzazione numero documenti corrotti
        \end{itemize}
\end{itemize}

\subusecase{visualizzazioneDataEOraVerificaClasse}{Visualizzazione data e ora verifica classe}
\begin{itemize}
    \item \textbf{Attore Primario}: Utente
    \item \textbf{Precondizioni}: 
        \begin{itemize}
            \item L'Utente ha avviato l'applicazione
            \item È stata completata una verifica di integrità della classe documentale
        \end{itemize}  
    \item \textbf{Postcondizioni}: Timestamp della verifica della classe documentale visualizzato.
    \item \textbf{Flusso Principale}:
        \begin{enumerate}
            \item Il sistema mostra il timestamp della verifica della classe documentale nel formato "GG/MM/AAAA HH:MM:SS".
        \end{enumerate}
\end{itemize}

\subusecase{nessunProcessoCorrotto}{Nessun processo corrotto}
\begin{itemize}
    \item \textbf{Attore Primario}: Utente
    \item \textbf{Precondizioni}: 
        \begin{itemize}
            \item L'Utente ha avviato l'applicazione
            \item È stata completata una verifica di integrità della classe documentale
        \end{itemize}  
    \item \textbf{Postcondizioni}: Messaggio "Tutti i processi sono integri" visualizzato.
    \item \textbf{Flusso Principale}:
        \begin{enumerate}
            \item Il sistema mostra un messaggio "Tutti i processi sono integri".
        \end{enumerate}
\end{itemize}

\usecase{visualizzazioneReportIntegritaProcesso}{Visualizzazione report integrità processo}
\begin{figure}[H]
    \centering
    \includegraphics[width=0.6\textwidth]{../assets/uml/UC32.png}
      \caption{\ref{visualizzazioneReportIntegritaProcesso} - Visualizzazione report integrità processo}
    \label{fig:uc_visualizzazioneReportIntegritaProcesso}
\end{figure}
\begin{itemize}
    \item \textbf{Attore Primario}: Utente
    \item \textbf{Precondizioni}: 
        \begin{itemize}
            \item L'Utente ha avviato l'applicazione
            \item È stata completata una verifica di integrità di un processo.
        \end{itemize} 
    \item \textbf{Postcondizioni}: Report dettagliato del processo visualizzato.     
    \item \textbf{Flusso Principale}:
        \begin{enumerate}
            \item L'utente visualizza il report della verifica del processo.
        \end{enumerate}
    \item \textbf{Inclusioni}: 
        \begin{itemize}
            \item \ref{visualizzazioneNumeroDocumentiVerificati} Visualizzazione numero documenti verificati
            \item \ref{visualizzazioneNumeroDocumentiIntegri} Visualizzazione numero documenti integri
            \item \ref{visualizzazioneNumeroDocumentiCorrotti} Visualizzazione numero documenti corrotti
            \item \ref{visualizzazioneListaDocumentiCorrotti} Visualizzazione lista documenti corrotti
            \item \ref{visualizzazioneDataEOraVerificaProcesso} Visualizzazione data e ora verifica processo
        \end{itemize}
\end{itemize}
\begin{figure}[H]
    \centering
    \includegraphics[width=1\textwidth]{../assets/uml/UC32INC.png}
      \caption{Inclusioni \ref{visualizzazioneReportIntegritaProcesso} - Visualizzazione report integrità processo}
    \label{fig:inclusioniVisualizzazioneReportIntegritaProcesso}
\end{figure}

\subusecase{visualizzazioneNumeroDocumentiVerificati}{Visualizzazione numero documenti verificati}
\begin{itemize}
    \item \textbf{Attore Primario}: Utente
    \item \textbf{Precondizioni}: 
        \begin{itemize}
            \item L'Utente ha avviato l'applicazione
            \item È stata completata una verifica di integrità di un processo            
        \end{itemize} 
    \item \textbf{Postcondizioni}: Numero totale di documenti verificati visualizzato.     
    \item \textbf{Flusso Principale}:
        \begin{enumerate}
            \item Il sistema mostra il numero di documenti verificati.
        \end{enumerate}
\end{itemize}

\subusecase{visualizzazioneNumeroDocumentiIntegri}{Visualizzazione numero documenti integri}
\begin{itemize}
    \item \textbf{Attore Primario}: Utente
    \item \textbf{Precondizioni}: 
        \begin{itemize}
            \item L'Utente ha avviato l'applicazione
            \item È stata completata una verifica di integrità di un processo
            
        \end{itemize} 
    \item \textbf{Postcondizioni}: Numero di documenti integri visualizzato.     
    \item \textbf{Flusso Principale}:
        \begin{enumerate}
            \item Il sistema mostra il numero di documenti integri.
        \end{enumerate}
\end{itemize}

\subusecase{visualizzazioneNumeroDocumentiCorrotti}{Visualizzazione numero documenti corrotti}
\begin{itemize}
    \item \textbf{Attore Primario}: Utente
    \item \textbf{Precondizioni}:
        \begin{itemize}
            \item L'Utente ha avviato l'applicazione
            \item È stata completata una verifica di integrità di un processo
            
        \end{itemize} 
    \item \textbf{Postcondizioni}: Numero di documenti corrotti visualizzato.     
    \item \textbf{Flusso Principale}:
        \begin{enumerate}
            \item Il sistema mostra il numero di documenti corrotti.
        \end{enumerate}
\end{itemize}

\subusecase{visualizzazioneListaDocumentiCorrotti}{Visualizzazione lista documenti corrotti}
\begin{itemize}
    \item \textbf{Attore Primario}: Utente
    \item \textbf{Precondizioni}: 
        \begin{itemize}
            \item L'Utente ha avviato l'applicazione
            \item È stata completata una verifica di integrità di un processo
        \end{itemize}    
    \item \textbf{Postcondizioni}: Lista completa dei documenti corrotti visualizzata.     
    \item \textbf{Flusso Principale}:
        \begin{enumerate}
            \item Per ogni documento corrotto, il sistema mostra il nome del documento e l'errore specifico.
        \end{enumerate}
    \item \textbf{Flusso Alternativo}:\begin{enumerate}
        \item Non esistono documenti corrotti nel processo: viene visualizzato un messaggio "Tutti i documenti sono integri".
        \end{enumerate}
    \item \textbf{Inclusioni}: \begin{itemize}
        \item \ref{visualizzaElementoListaDocumentiCorrotti} Visualizzazione documento corrotto
        \end{itemize}
    \item \textbf{Estensioni}: \begin{itemize}
        \item \ref{nessunDocumentoCorrotto} Nessun documento corrotto
    \end{itemize}
\end{itemize}

\subsubusecase{visualizzaElementoListaDocumentiCorrotti}{Visualizzazione documento corrotto}
\begin{itemize}
    \item \textbf{Attore Primario}: Utente
    \item \textbf{Precondizioni}: 
        \begin{itemize}
            \item L'Utente ha avviato l'applicazione
            \item È stata completata una verifica di integrità di un processo
        \end{itemize}
    \item \textbf{Postcondizioni}: Dettagli del documento corrotto visualizzati.
    \item \textbf{Flusso Principale}:
        \begin{enumerate}
            \item Il sistema mostra il nome del documento corrotto
            \item Il sistema mostra l'errore specifico del documento corrotto.
        \end{enumerate}
    \item \textbf{Inclusioni}: \begin{itemize}
        \item \ref{nomeDocumentoProcesso} Visualizzazione nome documento
        \item \ref{visualizzazioneDettagliErroreDocumento} Visualizzazione dettaglio errore documento
        \end{itemize}
\end{itemize}

\subusecase{visualizzazioneDataEOraVerificaProcesso}{Visualizzazione data e ora verifica processo}
\begin{itemize}
    \item \textbf{Attore Primario}: Utente
    \item \textbf{Precondizioni}: 
        \begin{itemize}
            \item L'Utente ha avviato l'applicazione
            \item È stata completata una verifica di integrità di un processo
            
        \end{itemize} 
    \item \textbf{Postcondizioni}: Timestamp della verifica visualizzato.
    \item \textbf{Flusso Principale}:
        \begin{enumerate}
            \item Il sistema mostra il timestamp della verifica del processo nel formato "GG/MM/AAAA HH:MM:SS".
        \end{enumerate}
\end{itemize}

\subusecase{nessunDocumentoCorrotto}{Nessun documento corrotto}
\begin{itemize}
    \item \textbf{Attore Primario}: Utente
    \item \textbf{Precondizioni}: \begin{itemize}
        \item L'Utente ha avviato l'applicazione
        \item È stata completata una verifica di integrità di un processo
        \end{itemize}
    \item \textbf{Postcondizioni}: Messaggio "Tutti i documenti sono integri" visualizzato.
    \item \textbf{Flusso Principale}:
        \begin{enumerate}
            \item Il sistema mostra un messaggio "Tutti i documenti sono integri".
        \end{enumerate}
\end{itemize}

\usecase{visualizzazioneReportIntegritaDocumento}{Visualizzazione report integrità documento}
\begin{figure}[H]
    \centering
    \includegraphics[width=0.6\textwidth]{../assets/uml/UC33.png}
      \caption{\ref{visualizzazioneReportIntegritaDocumento} - Visualizzazione report integrità documento}
    \label{fig:uc_visualizzazioneReportIntegritaDocumento}
\end{figure}
\begin{itemize}
    \item \textbf{Attore Primario}: Utente
    \item \textbf{Precondizioni}: 
        \begin{itemize}
            \item L'Utente ha avviato l'applicazione
            \item È stata completata una verifica di integrità di un documento.
        \end{itemize} 
    \item \textbf{Postcondizioni}: Report dettagliato del documento visualizzato.     
    \item \textbf{Flusso Principale}:
        \begin{enumerate}
            \item Il sistema mostra:
                \begin{itemize}
                    \item Nome del documento (\ref{nomeDocumentoProcesso})
                    \item Stato della verifica (Valido / Non Valido) (\ref{visualizzazioneStatoVerificaDocumento})
                    \item Data e ora della verifica (\ref{visualizzazioneDataEOraVerificaDocumento})
                    \item Dettagli dell'errore (se presenti) (\ref{visualizzazioneDettagliErroreDocumento})
                \end{itemize}
        \end{enumerate}
    \item \textbf{Inclusioni}: 
        \begin{itemize}
            \item \ref{nomeDocumentoProcesso} Visualizzazione nome documento
            \item \ref{visualizzazioneStatoVerificaDocumento} Visualizzazione stato verifica documento
            \item \ref{visualizzazioneDataEOraVerificaDocumento} Visualizzazione data e ora verifica documento
            \item \ref{visualizzazioneDettagliErroreDocumento} Visualizzazione dettaglio errore documento (se presente)
        \end{itemize}
\end{itemize}
\begin{figure}[H]
    \centering
    \includegraphics[width=1\textwidth]{../assets/uml/UC33INC.png}
      \caption{Inclusioni \ref{visualizzazioneReportIntegritaDocumento} - Visualizzazione report integrità documento}
    \label{fig:inclusioniVisualizzazioneReportIntegritaDocumento}
\end{figure}



\subusecase{visualizzazioneStatoVerificaDocumento}{Visualizzazione stato verifica documento}
\begin{itemize}
    \item \textbf{Attore Primario}: Utente
    \item \textbf{Precondizioni}: 
        \begin{itemize}
            \item L'Utente ha avviato l'applicazione
            \item È stata completata una verifica di integrità di un documento
        \end{itemize} 
    \item \textbf{Postcondizioni}: Stato della verifica visualizzato.     
    \item \textbf{Flusso Principale}:
        \begin{enumerate}
            \item Il sistema mostra lo stato della verifica.
        \end{enumerate}
\end{itemize}

\subusecase{visualizzazioneDataEOraVerificaDocumento}{Visualizzazione data e ora verifica documento}
\begin{itemize}
    \item \textbf{Attore Primario}: Utente
    \item \textbf{Precondizioni}: 
        \begin{itemize}
            \item L'Utente ha avviato l'applicazione
            \item È stata completata una verifica di integrità di un documento
        \end{itemize} 
    \item \textbf{Postcondizioni}: Timestamp della verifica visualizzato.     
    \item \textbf{Flusso Principale}:
        \begin{enumerate}
            \item Il sistema mostra il timestamp della verifica del documento nel formato "GG/MM/AAAA HH:MM:SS".
        \end{enumerate}
\end{itemize}

\subusecase{visualizzazioneDettagliErroreDocumento}{Visualizzazione dettaglio errore documento}
\begin{itemize}
    \item \textbf{Attore Primario}: Utente
    \item \textbf{Precondizioni}: 
        \begin{itemize}
            \item L'Utente ha avviato l'applicazione
            \item È stata completata una verifica di integrità di un documento
            \item Il documento ha stato "Non Valido".
        \end{itemize} 
    \item \textbf{Postcondizioni}: Dettaglio dell'errore visualizzato.     
    \item \textbf{Flusso Principale}:
        \begin{enumerate}
            \item Il sistema mostra la descrizione dell'errore (ad esempio: hash calcolato non coincide con hash associato al documento, firma digitale non valida o scaduta).
        \end{enumerate}
\end{itemize}

\usecase{scaricaReport}{Esporta un report di verifica in PDF}
\begin{figure}[H]
    \centering
    \includegraphics[width=0.8\textwidth]{../assets/uml/UC34.png}
    \caption{\ref{scaricaReport} - Esporta un report di verifica in PDF}
    \label{fig:uc_scaricaReport}
\end{figure}
\begin{itemize}
    \item \textbf{Attore Primario}: Utente
    \item \textbf{Precondizioni}:
    \begin{itemize}
        \item L'Utente ha avviato l'applicazione
        \item L'Utente sta visualizzando un report di verifica (DIP completo / Classe Documentale / Processo / Documento)
        \item L'Utente ha scelto l'opzione di esportazione in PDF del report
    \end{itemize}
    \item \textbf{Postcondizioni}: Il report di verifica è stato convertito in PDF e scaricato nella cartella selezionata.
    \item \textbf{Flusso Principale}:
    \begin{enumerate}
        \item Il sistema converte il report di verifica in un file PDF.
        \item Il sistema apre un dialogo di selezione cartella.
        \item L'utente seleziona la cartella di destinazione.
        \item Il sistema salva il file PDF nella cartella selezionata.
        \item Il sistema conferma il salvataggio con un messaggio "Report salvato con successo in [percorso]".
    \end{enumerate}
    \item \textbf{Flusso Alternativo}:
    \begin{itemize}
        \item La conversione in PDF non va a buon fine: il sistema mostra un messaggio d'errore "Errore nella conversione del report in PDF. Riprova."
    \end{itemize}
    \item \textbf{Estensioni}: \begin{itemize}
        \item \ref{erroreGenerazionePDF} Errore generazione PDF
        \item \ref{erroreScaricamentoFile} Errore scaricamento file
        \end{itemize}
\end{itemize}

\usecase{scaricaFile}{Scarica file}
\begin{figure}[H]
    \centering
    \includegraphics[width=0.8\textwidth]{../assets/uml/UC35.png}
    \caption{\ref{scaricaFile} - Scarica File}
    \label{fig:uc_scaricaFile}
\end{figure}
\begin{itemize}
    \item \textbf{Attore Primario}: Utente
    \item \textbf{Precondizioni}: 
        \begin{itemize}
            \item L'Utente ha avviato l'applicazione
            \item L'Utente ha selezionato un file da scaricare (Documento).
        \end{itemize} 
    \item \textbf{Postcondizioni}: L'utente ha scaricato il File nella cartella selezionata.     
    \item \textbf{Flusso Principale}:
        \begin{enumerate}
            \item Il sistema apre un dialogo di selezione cartella
            \item L'utente seleziona la cartella di destinazione
            \item Il sistema salva il File nella cartella selezionata
            \item Il sistema conferma il salvataggio con un messaggio "File salvato con successo in [percorso]".
        \end{enumerate}
    \item \textbf{Flusso Alternativo}:
        \begin{itemize}
            \item L'utente tenta di salvare il documento nella cartella del DIP: il sistema blocca l'operazione.
            \item L'utente annulla l'operazione: il sistema chiude il dialogo senza salvare.
        \end{itemize}
      \item \textbf{Estensioni}: \ref{erroreScaricamentoFile} Errore scaricamento file
\end{itemize}

\usecase{erroreScaricamentoFile}{Errore scaricamento file}
\begin{itemize}
    \item \textbf{Attore Primario}: Utente
    \item \textbf{Precondizioni}: 
        \begin{itemize}
            \item L'Utente ha avviato l'applicazione
            \item L'Utente ha selezionato un file da scaricare (Documento o Report).
        \end{itemize} 
    \item \textbf{Postcondizioni}: L'utente visualizza un messaggio d'errore.
    \item \textbf{Flusso Principale}:
        \begin{enumerate}
            \item Il sistema blocca l'operazione.
            \item L'utente annulla l'operazione: il sistema chiude il dialogo senza salvare.
        \end{enumerate}
\end{itemize}

\usecase{erroreGenerazionePDF}{Errore generazione PDF}
\begin{itemize}
    \item \textbf{Attore Primario}: Utente
    \item \textbf{Precondizioni}: 
        \begin{itemize}
            \item L'Utente ha avviato l'applicazione
            \item È stata completata una verifica di integrità (DIP completo / Classe Documentale / Processo / Documento)
            \item L'utente sta visualizzando un report di verifica (DIP completo / Classe Documentale / Processo / Documento)
            \item È stata scelta l'opzione di conversione in PDF del report
            \item La conversione PDF non è andata a buon fine.
        \end{itemize} 
    \item \textbf{Postcondizioni}: L'utente visualizza un messaggio d'errore.     
    \item \textbf{Flusso Principale}:
        \begin{enumerate}
            \item La conversione non va a buon fine
            \item L'utente visualizza un messaggio d'errore.
        \end{enumerate}
\end{itemize}
