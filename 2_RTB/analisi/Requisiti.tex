\section{Requisiti}

\subsection{Introduzione}
I requisiti vengono classificati secondo le seguenti categorie:
\begin{itemize}
    \item \textbf{Requisiti Funzionali (F)}: descrivono le funzionalità del sistema
    \item \textbf{Requisiti di Qualità (Q)}: descrivono le caratteristiche qualitative del sistema
    \item \textbf{Requisiti di Vincolo (V)}: descrivono i vincoli tecnologici e normativi
\end{itemize}

Ogni requisito è identificato da un codice univoco nella forma:
\begin{center}
\texttt{R-[ID]-[Tipo]-[Priorità]}
\end{center}

Dove:
\begin{itemize}
    \item \textbf{ID}: numero progressivo del requisito
    \item \textbf{Tipo}: F (Funzionale), Q (Qualità), V (Vincolo)
    \item \textbf{Priorità}: Ob (Obbligatorio), De (Desiderabile), Op (Opzionale)
\end{itemize}

\newpage

\subsection{Requisiti Funzionali}

\setlength{\LTleft}{0cm}
\renewcommand{\arraystretch}{1.3}

\begin{longtable}{|p{2.5cm}|p{9.5cm}|p{2.5cm}|}
\hline
\textbf{Codice} & \textbf{Descrizione} & \textbf{Fonti} \\
\hline
R-74x-F-Ob & L'utente deve poter visualizzare il report di integrità del DIP completo con le informazioni aggregate & \ref{visualizzazioneReportIntegritaDIPCompleto} \\
\hline
R-75x-F-Ob & Il sistema deve mostrare il conteggio totale delle classi documentali verificate nel report del DIP & \ref{visualizzazioneNumeroClassiVerificate} \\
\hline
R-76x-F-Ob & Il sistema deve mostrare il numero di classi integre (stato "Valido") in colore verde nel report del DIP & \ref{visualizzazioneNumeroClassiIntegre} \\
\hline
R-77x-F-Ob & Il sistema deve mostrare il numero di classi corrotte (stato "Non Valido") in colore rosso nel report del DIP & \ref{visualizzazioneNumeroClassiCorrotte} \\
\hline
R-78x-F-Ob & Il sistema deve mostrare l'elenco delle classi corrotte indicando nome e numero di processi corrotti per ciascuna & \ref{visualizzazioneListaClassiCorrotte} \\
\hline
R-79x-F-Ob & Il sistema deve mostrare la data e l'ora di inizio della verifica del DIP nel formato "GG/MM/AAAA HH:MM:SS" & \ref{visualizzazioneDataEOraVerificaDIP} \\
\hline
R-80x-F-Ob & L'utente deve poter visualizzare il report di integrità dettagliato per una singola classe documentale & \ref{visualizzazioneReportIntegritaClasseDocumentale} \\
\hline
R-81x-F-Ob & Il sistema deve mostrare il conteggio dei processi verificati all'interno della classe documentale & \ref{visualizzazioneNumeroProcessiVerificati} \\
\hline
R-82x-F-Ob & Il sistema deve mostrare il numero di processi integri (stato "Valido") in colore verde nella classe documentale & \ref{visualizzazioneNumeroProcessiIntegri} \\
\hline
R-83x-F-Ob & Il sistema deve mostrare il numero di processi corrotti (stato "Non Valido") in colore rosso nella classe documentale & \ref{visualizzazioneNumeroProcessiCorrotti} \\
\hline
R-84x-F-Ob & Il sistema deve mostrare la lista dei processi corrotti con il relativo numero di documenti compromessi & \ref{visualizzazioneListaProcessiCorrotti} \\
\hline
R-85x-F-Ob & Il sistema deve mostrare la data e l'ora di inizio della verifica della classe nel formato "GG/MM/AAAA HH:MM:SS" & \ref{visualizzazioneDataEOraVerificaClasse} \\
\hline
R-86x-F-Ob & L'utente deve poter visualizzare il report di integrità dettagliato di un singolo processo & \ref{visualizzazioneReportIntegritaProcesso} \\
\hline
R-87x-F-Ob & Il sistema deve mostrare il conteggio dei documenti verificati all'interno del processo selezionato & \ref{visualizzazioneNumeroDocumentiVerificati} \\
\hline
R-88x-F-Ob & Il sistema deve mostrare il numero di documenti integri (stato "Valido") in colore verde nel processo selezionato & \ref{visualizzazioneNumeroDocumentiIntegri} \\
\hline
R-89x-F-Ob & Il sistema deve mostrare il numero di documenti corrotti (stato "Non Valido") in colore rosso nel processo selezionato & \ref{visualizzazioneNumeroDocumentiCorrotti} \\
\hline
R-90x-F-Ob & Il sistema deve mostrare la lista dei documenti corrotti con indicazione del nome e dell'errore specifico riscontrato & \ref{visualizzazioneListaDocumentiCorrotti} \\
\hline
R-91x-F-Ob & Il sistema deve mostrare la data e l'ora di inizio della verifica del processo nel formato "GG/MM/AAAA HH:MM:SS" & \ref{visualizzazioneDataEOraVerificaProcesso} \\
\hline
R-92x-F-Ob & L'utente deve poter visualizzare il report di integrità di un singolo documento & \ref{visualizzazioneReportIntegritaDocumento} \\
\hline
R-93x-F-Ob & Il sistema deve mostrare il nome del documento all'interno del report di integrità & \ref{visualizzazioneNomeDocumento} \\
\hline
R-94x-F-Ob & Il sistema deve mostrare lo stato della verifica (Valido / Non Valido) per il documento selezionato & \ref{visualizzazioneStatoVerificaDocumento} \\
\hline
R-95x-F-Ob & Il sistema deve mostrare la data e l'ora di inizio della verifica del documento nel formato "GG/MM/AAAA HH:MM:SS" & \ref{visualizzazioneDataEOraVerificaDocumento} \\
\hline
R-96x-F-Ob & Il sistema deve mostrare la descrizione tecnica del dettaglio dell'errore per i documenti con stato "Non Valido" & \ref{visualizzazioneDettagliErroreDocumento} \\
\hline
R-97x-F-Ob & L'utente deve poter avviare la conversione del report di verifica visualizzato in formato PDF & \ref{convertiReportVerificaPDF} \\
\hline
R-98x-F-Ob & Il sistema deve mostrare un messaggio di errore qualora la generazione del file PDF non vada a buon fine & \ref{erroreGenerazionePDF} \\
\hline
R-99x-F-Ob & L'utente deve poter scaricare un file in una cartella locale previa selezione della cartella di destinazione & \ref{scaricaFile} \\
\hline
R-100x-F-Ob & Il sistema deve mostrare un messaggio di conferma indicando il percorso di destinazione al termine del salvataggio & \ref{scaricaFile} \\
\hline
R-101x-F-Ob & Il sistema deve impedire il salvataggio di file all'interno della cartella sorgente del DIP per preservarne l'integrità & \ref{scaricaFile}, \ref{erroreScaricamentoFile} \\
\hline
R-102x-F-Ob & Il sistema deve mostrare un errore specifico se l'utente tenta di scaricare un file nel percorso protetto del DIP & \ref{erroreScaricamentoFile} \\
\hline
R-103x-F-Ob & L'utente deve poter visualizzare le informazioni dell'AiP di provenienza di un documento selezionato & \ref{visualizzaInfoAiP} \\
\hline
R-104x-F-Ob & Il sistema deve mostrare la classe documentale di appartenenza dell'AiP relativo al documento selezionato & \ref{visualizzaClasseDocumentaleAiP} \\
\hline
R-105x-F-Ob & Il sistema deve mostrare lo UUID dell'AiP relativo al documento selezionato & \ref{visualizzaUUIDAiP} \\
\hline

\hline
\caption{Requisiti Funzionali}
\label{tab:req-funzionali}
\end{longtable}

\subsection{Requisiti di Qualità}

\begin{longtable}{|p{2.5cm}|p{8cm}|p{3cm}|}
\hline
\textbf{Codice} & \textbf{Descrizione} & \textbf{Fonti} \\
\hline
\endfirsthead

\hline
\textbf{Codice} & \textbf{Descrizione} & \textbf{Fonti} \\
\hline
\endhead

\hline
\endfoot
\hline
\caption{Requisiti di Qualità}
\label{tab:req-qualita}
\end{longtable}

\subsection{Requisiti di Vincolo}

\begin{longtable}{|p{2.5cm}|p{8cm}|p{3cm}|}
\hline
\textbf{Codice} & \textbf{Descrizione} & \textbf{Fonti} \\
\hline
\endfirsthead

\hline
\textbf{Codice} & \textbf{Descrizione} & \textbf{Fonti} \\
\hline
\endhead

\hline
\endfoot

\caption{Requisiti di Vincolo}
\label{tab:req-vincolo}
\end{longtable}

\subsection{Tracciamento}

\subsubsection{Tracciamento Fonti - Requisiti}

\begin{longtable}{|p{4cm}|p{10cm}|}
\hline
\textbf{Fonte} & \textbf{Requisiti} \\
\hline
\endfirsthead

\hline
\textbf{Fonte} & \textbf{Requisiti} \\
\hline
\endhead

\hline
\endfoot
\ref{visualizzazioneReportIntegritaDIPCompleto} & R-74x-F-Ob \\
\hline
\ref{visualizzazioneNumeroClassiVerificate} & R-75x-F-Ob \\
\hline
\ref{visualizzazioneNumeroClassiIntegre} & R-76x-F-Ob \\
\hline
\ref{visualizzazioneNumeroClassiCorrotte} & R-77x-F-Ob \\
\hline
\ref{visualizzazioneListaClassiCorrotte} & R-78x-F-Ob \\
\hline
\ref{visualizzazioneDataEOraVerificaDIP} & R-79x-F-Ob \\
\hline
\ref{visualizzazioneReportIntegritaClasseDocumentale} & R-80x-F-Ob \\
\hline
\ref{visualizzazioneNumeroProcessiVerificati} & R-81x-F-Ob \\
\hline
\ref{visualizzazioneNumeroProcessiIntegri} & R-82x-F-Ob \\
\hline
\ref{visualizzazioneNumeroProcessiCorrotti} & R-83x-F-Ob \\
\hline
\ref{visualizzazioneListaProcessiCorrotti} & R-84x-F-Ob \\
\hline
\ref{visualizzazioneDataEOraVerificaClasse} & R-85x-F-Ob \\
\hline
\ref{visualizzazioneReportIntegritaProcesso} & R-86x-F-Ob \\
\hline
\ref{visualizzazioneNumeroDocumentiVerificati} & R-87x-F-Ob \\
\hline
\ref{visualizzazioneNumeroDocumentiIntegri} & R-88x-F-Ob \\
\hline
\ref{visualizzazioneNumeroDocumentiCorrotti} & R-89x-F-Ob \\
\hline
\ref{visualizzazioneListaDocumentiCorrotti} & R-90x-F-Ob \\
\hline
\ref{visualizzazioneDataEOraVerificaProcesso} & R-91x-F-Ob \\
\hline
\ref{visualizzazioneReportIntegritaDocumento} & R-92x-F-Ob \\
\hline
\ref{visualizzazioneNomeDocumento} & R-93x-F-Ob \\
\hline
\ref{visualizzazioneStatoVerificaDocumento} & R-94x-F-Ob \\
\hline
\ref{visualizzazioneDataEOraVerificaDocumento} & R-95x-F-Ob \\
\hline
\ref{visualizzazioneDettagliErroreDocumento} & R-96x-F-Ob \\
\hline
\ref{convertiReportVerificaPDF} & R-97x-F-Ob \\
\hline
\ref{erroreGenerazionePDF} & R-98x-F-Ob \\
\hline
\ref{scaricaFile} & R-99x-F-Ob, R-100x-F-Ob, R-101x-F-Ob \\
\hline
\ref{erroreScaricamentoFile} & R-101x-F-Ob, R-102x-F-Ob \\
\hline
\ref{visualizzaInfoAiP} & R-103x-F-Ob \\
\hline
\ref{visualizzaClasseDocumentaleAiP} & R-104x-F-Ob \\
\hline
\ref{visualizzaUUIDAiP} & R-105x-F-Ob \\
\hline
\caption{Tracciamento Fonti - Requisiti}
\label{tab:trace-fonti-req}
\end{longtable}

\subsubsection{Tracciamento Requisiti - Fonti}

\begin{longtable}{|p{3cm}|p{11cm}|}
\hline
\textbf{Requisito} & \textbf{Fonti} \\
\hline
\endfirsthead

\hline
\textbf{Requisito} & \textbf{Fonti} \\
\hline
\endhead

\hline
\endfoot

R-74x-F-Ob & \ref{visualizzazioneReportIntegritaDIPCompleto} \\
\hline
R-75x-F-Ob & \ref{visualizzazioneNumeroClassiVerificate} \\
\hline
R-76x-F-Ob & \ref{visualizzazioneNumeroClassiIntegre} \\
\hline
R-77x-F-Ob & \ref{visualizzazioneNumeroClassiCorrotte} \\
\hline
R-78x-F-Ob & \ref{visualizzazioneListaClassiCorrotte} \\
\hline
R-79x-F-Ob & \ref{visualizzazioneDataEOraVerificaDIP} \\
\hline
R-80x-F-Ob & \ref{visualizzazioneReportIntegritaClasseDocumentale} \\
\hline
R-81x-F-Ob & \ref{visualizzazioneNumeroProcessiVerificati} \\
\hline
R-82x-F-Ob & \ref{visualizzazioneNumeroProcessiIntegri} \\
\hline
R-83x-F-Ob & \ref{visualizzazioneNumeroProcessiCorrotti} \\
\hline
R-84x-F-Ob & \ref{visualizzazioneListaProcessiCorrotti} \\
\hline
R-85x-F-Ob & \ref{visualizzazioneDataEOraVerificaClasse} \\
\hline
R-86x-F-Ob & \ref{visualizzazioneReportIntegritaProcesso} \\
\hline
R-87x-F-Ob & \ref{visualizzazioneNumeroDocumentiVerificati} \\
\hline
R-88x-F-Ob & \ref{visualizzazioneNumeroDocumentiIntegri} \\
\hline
R-89x-F-Ob & \ref{visualizzazioneNumeroDocumentiCorrotti} \\
\hline
R-90x-F-Ob & \ref{visualizzazioneListaDocumentiCorrotti} \\
\hline
R-91x-F-Ob & \ref{visualizzazioneDataEOraVerificaProcesso} \\
\hline
R-92x-F-Ob & \ref{visualizzazioneReportIntegritaDocumento} \\
\hline
R-93x-F-Ob & \ref{visualizzazioneNomeDocumento} \\
\hline
R-94x-F-Ob & \ref{visualizzazioneStatoVerificaDocumento} \\
\hline
R-95x-F-Ob & \ref{visualizzazioneDataEOraVerificaDocumento} \\
\hline
R-96x-F-Ob & \ref{visualizzazioneDettagliErroreDocumento} \\
\hline
R-97x-F-Ob & \ref{convertiReportVerificaPDF} \\
\hline
R-98x-F-Ob & \ref{erroreGenerazionePDF} \\
\hline
R-99x-F-Ob & \ref{scaricaFile} \\
\hline
R-100x-F-Ob & \ref{scaricaFile} \\
\hline
R-101x-F-Ob & \ref{scaricaFile}, \ref{erroreScaricamentoFile} \\
\hline
R-102x-F-Ob & \ref{erroreScaricamentoFile} \\
\hline
R-103x-F-Ob & \ref{visualizzaInfoAiP} \\
\hline
R-104x-F-Ob & \ref{visualizzaClasseDocumentaleAiP} \\
\hline
R-105x-F-Ob & \ref{visualizzaUUIDAiP} \\
\hline

\caption{Tracciamento Requisiti - Fonti}
\label{tab:trace-req-fonti}
\end{longtable}

\subsection{Riepilogo}

\begin{table}[h]
\centering
\begin{tabular}{|l|c|c|c|c|}
\hline
\textbf{Tipologia} & \textbf{Obbligatori} & \textbf{Desiderabili} & \textbf{Opzionali} & \textbf{Totale} \\
\hline
Funzionali & X & Y & Z & N \\
\hline
Qualità & X & Y & Z & N \\
\hline
Vincolo & X & Y & Z & N \\
\hline
\textbf{Totale} & X & Y & Z & \textbf{N} \\
\hline
\end{tabular}
\caption{Riepilogo dei Requisiti}
\label{tab:riepilogo-requisiti}
\end{table}
