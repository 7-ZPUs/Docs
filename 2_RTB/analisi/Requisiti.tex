

\section{Requisiti}

\subsection{Introduzione}
I requisiti vengono classificati secondo le seguenti categorie:
\begin{itemize}
    \item \textbf{Requisiti Funzionali (F)}: descrivono le funzionalità del sistema
    \item \textbf{Requisiti di Qualità (Q)}: descrivono le caratteristiche qualitative del sistema
    \item \textbf{Requisiti di Vincolo (V)}: descrivono i vincoli tecnologici e normativi
\end{itemize}

Ogni requisito è identificato da un codice univoco nella forma:
\begin{center}
\texttt{R-[ID]-[Tipo]-[Priorità]}
\end{center}

Dove:
\begin{itemize}
    \item \textbf{ID}: numero progressivo del requisito
    \item \textbf{Tipo}: F (Funzionale), Q (Qualità), V (Vincolo)
    \item \textbf{Priorità}: Ob (Obbligatorio), De (Desiderabile), Op (Opzionale)
\end{itemize}

\newpage

\subsection{Requisiti Funzionali}

\setlength{\LTleft}{0cm}


\begin{longtable}{|p{2.5cm}|p{8cm}|p{3cm}|}
\hline
\textbf{Codice} & \textbf{Descrizione} & \textbf{Fonti} \\
\hline
R-1-F-Ob & L'utente deve poter visualizzare l'elenco di classi documentali nel DIP & \ref{classiDocumentali} \\
\hline
R-2-F-Ob & In caso non vi siano classi documentali nel DIP, l'utente deve poter visualizzare un messaggio di errore & \ref{elencoVuoto} \\
\hline
R-3-F-Ob & Quando viene selezionata una classe documentale, l'utente deve poter visualizzare ciascuna classe documentale all'interno dell'elenco & \ref{classeDocumentale} \\
\hline
R-4-F-Ob & L'utente deve poter visualizzare il nome della classe documentale & \ref{nomeClasseDocumentale} \\
\hline
R-5-F-Ob & L'utente deve poter visualizzare lo stato di verifica della classe documentale & \ref{statoVerificaElemento} \\
\hline
R-6-F-Ob & L'utente deve poter visualizzare la marcatura temporale della classe documentale & \ref{marcaturaTemporaleElemento} \\
\hline
R-7-F-Ob & L'utente deve poter visualizzare l'elenco dei processi associati alla classe documentale & \ref{processiClasseDocumentale} \\
\hline
R-14-F-Ob & In caso non vi siano processi associati alla classe documentale, l'utente deve poter visualizzare un messaggio di errore & \ref{elencoVuoto} \\
\hline
R-8-F-Ob & Quando viene selezionato un processo, l'utente deve poter visualizzare ciascun processo all'interno dell'elenco & \ref{processoClasseDocumentale} \\
\hline
R-9-F-Ob & L'utente deve poter visualizzare il nome del processo & \ref{idProcessoClasse} \\
\hline
R-10-F-Ob & L'utente deve poter visualizzare l'elenco di documenti associati a un processo & \ref{documentiProcesso} \\
\hline
R-14-F-Ob & In caso non vi siano documenti associati al processo, l'utente deve poter visualizzare un messaggio di errore & \ref{elencoVuoto} & \\
\hline
R-11-F-Ob & Quando viene selezionato un documento, l'utente deve poter visualizzare ciascun documento all'interno dell'elenco & \ref{documentoProcesso} \\
\hline
R-12-F-Ob & L'utente deve poter visualizzare il nome del documento & \ref{nomeDocumentoProcesso} \\
\hline
R-13-F-Ob & L'utente deve poter visualizzare lo stato di verifica del documento & \ref{statoVerificaElemento} \\
\hline
R-14-F-Ob & L'utente deve poter visualizzare la marcatura temporale del documento & \ref{marcaturaTemporaleElemento} \\
\hline
R-15-F-Ob & L'utente deve poter selezionare una classe documentale  & \ref{selezionaClasseDocumentale} \\
\hline
R-16-F-Ob & L'utente deve poter selezionare un processo  & \ref{selezionaProcesso} \\
\hline
R-17-F-Ob & L'utente deve poter visualizzare l'anteprima di un documento selezionato  & \ref{anteprimaDocumento} \\
\hline
R-18-F-Ob & Se il documento selezionato non è visualizzabile in anteprima, l'utente deve poter visualizzare un messaggio di errore  & \ref{formatoDocumentoNonSupportato} \\
\hline
R-18-F-Ob & L'utente deve poter ricercare un documento, un processo, o una classe documentale & \ref{ricercaDIP} \\
\hline
R-19-F-Ob & L'utente può cercare una classe documentale esclusivamente per nome & \begin{tabular}{l} \ref{ricercaClasse} \\ \ref{inserimentoNomeClasseDocumentale} \end{tabular} \\
\hline
R-20-F-Ob & L'utente può cercare un processo esclusivamente per uuid & \begin{tabular}{l} \ref{ricercaProcesso} \\ \ref{inserimentoIdProcesso} \end{tabular} \\
\hline
R-21-F-Op & L'utente deve poter effettuare una ricerca semantica basata sui metadati dei documenti presenti nel DIP & \ref{ricercaDIPSemantica} \\
\hline
R-22-F-Op & Il sistema deve rendere disponibile un comando o opzione per avviare l'indicizzazione semantica dei documenti nel DIP per la ricerca & \ref{indicizzazioneSemantica} \\
\hline
R-23-F-Ob & Quando viene selezionata l'indicizzazione semantica il sistema deve indicizzare i documenti presenti nel DIP  & \ref{indicizzazioneSemantica} \\
\hline
R-24-F-Ob & Se il sistema non riesce ad indicizzare i documenti presenti nel DIP, l'utente deve poter visualizzare un messaggio di errore & \ref{erroreIndicizzazioneSemantica} \\
\hline
R-25-F-Ob & Quando viene selezionata l'indicizzazione semantica il sistema deve rendere disponibile il livello di precisione di ricerca & \ref{PrecisioneSemantica} \\
\hline
R-26-F-Ob & Quando l'utente sta scegliendo il livello di precisione di ricerca, il sistema rende disponibile tre livelli: Basso, Medio e Alto & \ref{PrecisioneSemantica} \\
\hline
R-27-F-Ob & Il sistema deve rendere disponibile un campo di ricerca & \\
\hline
R-28-F-Ob & Il sistema deve rendere disponibile l'opzione di ricerca per documenti & \\
\hline
R-28-F-Ob & Il sistema deve rendere disponibile l'opzione di ricerca per processi & \\
\hline
R-29-F-Ob & Il sistema deve rendere disponibile l'opzione di ricerca per classi documentali & \\
\hline
R-30-F-Ob & Il sistema deve rendere disponibile la ricerca avanzata con filtri & \\
\hline
R-31-F-Ob & Il sistema deve rendere disponibile la sezione di filtri di ricerca comuni & \\
\hline
R-32-F-Ob & Il sistema deve rendere disponibile la sezione di filtri di ricerca specifici per tipo documentale & \\
\hline
R-33-F-Ob & Il sistema deve rendere disponibile la sezione di filtri di ricerca specifici per metadati custom & \\
\hline
R-34-F-Ob & Il sistema deve permettere di applicare più filtri contemporaneamente & \\
\hline
R-35-F-Ob & Il sistema deve permettere di rimuovere i filtri applicati & \\
\hline
R-36-F-Ob & Il sistema deve permettere di resettare tutti i filtri applicati & \\
\hline
R-37-F-Ob & Il sistema deve permettere di selezionare filtri per tipo documentale per un singolo tipo di documento & \\
\hline
R-38-F-Ob & Il sistema deve rendere racchiudere i filtri di ricerca comuni in una sezione espandibile dedicata & \\
\hline
R-39-F-Ob & Il sistema deve rendere racchiudere i filtri di ricerca specifici per tipo documentale in una sezione espandibile dedicata & \\
\hline
R-40-F-Ob & Il sistema deve rendere racchiudere i filtri di ricerca specifici per metadati custom in una sezione espandibile dedicata & \\
\hline
R-41-F-Ob & Il sistema deve permettere di visualizzare i filtri comuni applicabili & \\
\hline
R-42-F-Ob & Il sistema deve rendere disponibile una sezione per il filtro comune "Chiave Descrittiva" & \\
\hline
R-43-F-Ob & L'utente deve poter inserire il valore per il campo "Chiave Descrittiva" & \\
\hline
R-44-F-Ob & L'utente deve poter inserire il valore per il campo "Oggetto" & \\
\hline
R-45-F-Ob & L'utente deve poter inserire il valore per il campo "Parole chiave" & \\
\hline
R-46-F-Ob & Il sistema deve rendere disponibile una sezione per il filtro comune "Classificazione" & \\
\hline
R-47-F-Ob & L'utente deve poter inserire il valore per il campo "Indice di classificazione" & \\
\hline
R-48-F-Ob & L'utente deve poter inserire il valore per il campo "Descrizione" & \\
\hline
R-49-F-Ob & L'utente deve poter inserire il valore per il campo "Piano di fascicolo" & \\
\hline
R-50-F-Ob & Il sistema deve rendere disponibile una sezione per il filtro comune "Tempo di conservazione" & \\
\hline
R-51-F-Ob & L'utente deve poter inserire il valore per il filtro "Tempo di conservazione" & \\
\hline
R-52-F-Ob & L'utente deve poter selezionare l'opzione "Perenne" per il filtro "Tempo di conservazione" & \\
\hline
R-53-F-Ob & Il sistema deve rendere disponibile una sezione per il filtro comune "Note" & \\
\hline
R-54-F-Ob & L'utente deve poter inserire il valore per il filtro "Note" & \\
\hline
R-55-F-Ob & Il sistema deve rendere disponibile una sezione per il filtro comune "Tipo di documento" & \\
\hline
R-56-F-Ob & L'utente deve poter selezionare il valore per il filtro "Tipo di documento" tra "Documento informatico", "Documento amministrativo informatico", "Aggregazione documentale" & \\
\hline
\endfirsthead

\hline
\textbf{Codice} & \textbf{Descrizione} & \textbf{Fonti} \\
\hline
\endhead

\hline
\endfoot
\hline
\caption{Requisiti Funzionali}
\label{tab:req-funzionali}
\end{longtable}

\subsection{Requisiti di Qualità}

\begin{longtable}{|p{2.5cm}|p{8cm}|p{3cm}|}
\hline
\textbf{Codice} & \textbf{Descrizione} & \textbf{Fonti} \\
\hline
\endfirsthead

\hline
\textbf{Codice} & \textbf{Descrizione} & \textbf{Fonti} \\
\hline
\endhead

\hline
\endfoot
\hline
\caption{Requisiti di Qualità}
\label{tab:req-qualita}
\end{longtable}

\subsection{Requisiti di Vincolo}

\begin{longtable}{|p{2.5cm}|p{8cm}|p{3cm}|}
\hline
\textbf{Codice} & \textbf{Descrizione} & \textbf{Fonti} \\
\hline
\endfirsthead

\hline
\textbf{Codice} & \textbf{Descrizione} & \textbf{Fonti} \\
\hline
\endhead

\hline
\endfoot

\caption{Requisiti di Vincolo}
\label{tab:req-vincolo}
\end{longtable}

\subsection{Tracciamento}

\subsubsection{Tracciamento Fonti - Requisiti}

\begin{longtable}{|p{4cm}|p{10cm}|}
\hline
\textbf{Fonte} & \textbf{Requisiti} \\
\hline
\endfirsthead

\hline
\textbf{Fonte} & \textbf{Requisiti} \\
\hline
\endhead

\hline
\endfoot

\hline
\caption{Tracciamento Fonti - Requisiti}
\label{tab:trace-fonti-req}
\end{longtable}

\subsubsection{Tracciamento Requisiti - Fonti}

\begin{longtable}{|p{3cm}|p{11cm}|}
\hline
\textbf{Requisito} & \textbf{Fonti} \\
\hline
\endfirsthead

\hline
\textbf{Requisito} & \textbf{Fonti} \\
\hline
\endhead

\hline
\endfoot

\caption{Tracciamento Requisiti - Fonti}
\label{tab:trace-req-fonti}
\end{longtable}

\subsection{Riepilogo}

\begin{table}[h]
\centering
\begin{tabular}{|l|c|c|c|c|}
\hline
\textbf{Tipologia} & \textbf{Obbligatori} & \textbf{Desiderabili} & \textbf{Opzionali} & \textbf{Totale} \\
\hline
Funzionali & X & Y & Z & N \\
\hline
Qualità & X & Y & Z & N \\
\hline
Vincolo & X & Y & Z & N \\
\hline
\textbf{Totale} & X & Y & Z & \textbf{N} \\
\hline
\end{tabular}
\caption{Riepilogo dei Requisiti}
\label{tab:riepilogo-requisiti}
\end{table}