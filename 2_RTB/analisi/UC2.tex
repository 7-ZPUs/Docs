\subsubsection{UC-2 - Ricerca Nel DIP}

\subsubsection{UC-2.1 - Ricercare Classe Documentale con filtri}
\begin{itemize}
      \item \textbf{Attore Primario}: utente
      \item \textbf{Precondizioni}: L'utente sta visualizzando la lista delle Classi Documentali.
      \item \textbf{Postcondizioni}: Viene visualizzato l'elenco delle Classi Documentali corrispondenti ai filtri applicati.
      \item \textbf{Flusso Principale}:
            \begin{enumerate}
                  \item L'utente inserisce i filtri per la ricerca secondo i metadati delle Classi
                        Documentali.
                  \item Il sistema mostra i risultati della ricerca (UC-2.4).
            \end{enumerate}
      \item \textbf{Flusso Alternativo (Nessun risultato)}: La ricerca non produce risultati e l'utente visualizza un elenco vuoto.
\end{itemize}

\subsubsection{UC-2.2 - Ricercare Processo con filtri}
\begin{itemize}
      \item \textbf{Attore Primario}: utente
      \item \textbf{Precondizioni}: L'utente sta visualizzando la lista dei Processi.
      \item \textbf{Postcondizioni}: Viene visualizzato l'elenco dei Processi corrispondenti ai filtri applicati.
      \item \textbf{Flusso Principale}:
            \begin{enumerate}
                  \item L'utente inserisce i filtri per la ricerca secondo i metadati dei Processi.
                  \item Il sistema mostra i risultati della ricerca (UC-2.4).
            \end{enumerate}
      \item \textbf{Flusso Alternativo (Nessun risultato)}: La ricerca non produce risultati e l'utente visualizza un elenco vuoto.
\end{itemize}

\subsubsection{UC-2.3 - Ricercare Documento con filtri}
\begin{itemize}
      \item \textbf{Attore Primario}: utente
      \item \textbf{Precondizioni}: L'utente ha accesso alla funzionalità di ricerca documentale.
      \item \textbf{Postcondizioni}: Visualizzazione di un elenco di documenti che corrispondono ai criteri inseriti.
      \item \textbf{Flusso Principale}:
            \begin{enumerate}
                  \item L'utente imposta uno o più filtri per la ricerca, scegliendo tra campi comuni
                        (UC-2.3.1) o campi specifici per tipo documentale (UC-2.3.2).
                  \item L'utente avvia la ricerca.
                  \item Il sistema mostra i risultati della ricerca (UC-2.4).
            \end{enumerate}
      \item \textbf{Flusso Alternativo (Nessun risultato)}: La ricerca non produce risultati e il sistema informa l'utente.
\end{itemize}

\subsubsection{UC-2.3.1 - Ricerca per Campi Comuni}
\begin{itemize}
      \item \textbf{Attore Primario}: utente
      \item \textbf{Precondizioni}: L'utente si trova nella schermata di ricerca documentale.
      \item \textbf{Postcondizioni}: I risultati della ricerca vengono filtrati in base ai campi comuni specificati.
      \item \textbf{Flusso Principale}:
            \begin{enumerate}
                  \item L'utente seleziona uno o più campi comuni su cui basare la ricerca (da
                        UC-2.3.1.1 a UC-2.3.1.5).
                  \item L'utente compila i valori per i campi selezionati.
                  \item Il sistema utilizza questi parametri per la ricerca.
            \end{enumerate}
      \item \textbf{Flusso Alternativo (Nessun risultato)}: La ricerca non produce risultati.
\end{itemize}

\subsubsection{UC-2.3.1.1 - Ricerca per Chiave Descrittiva}
\begin{itemize}
      \item \textbf{Attore Primario}: utente
      \item \textbf{Precondizioni}: L'utente sta effettuando una ricerca per campi comuni.
      \item \textbf{Postcondizioni}: I risultati della ricerca vengono filtrati in base alla Chiave Descrittiva, Oggetto o Parole Chiave.
      \item \textbf{Flusso Principale}:
            \begin{enumerate}
                  \item L'utente digita nella casella di ricerca i valori per:
                        \begin{itemize}
                              \item Chiave Descrittiva
                              \item Oggetto
                              \item Parole Chiave.
                        \end{itemize}
                  \item Il sistema restituisce la lista dei documenti che corrispondono ai parametri
                        inseriti (\nameref{risultatiRicerca}).
            \end{enumerate}
\end{itemize}

\subsubsection{UC-2.3.1.2 - Ricerca per Soggetti}
\begin{itemize}
      \item \textbf{Attore Primario}: utente
      \item \textbf{Precondizioni}: L'utente sta effettuando una ricerca per campi comuni.
      \item \textbf{Postcondizioni}: I risultati della ricerca vengono filtrati in base ai soggetti specificati.
      \item \textbf{Flusso Principale}:
            \begin{enumerate}
                  \item L'utente decide di aggiungere uno o più soggetti come filtro di ricerca.
                  \item Per ogni soggetto, l'utente ne specifica i dettagli (Ruolo e altri campi
                        pertinenti, come descritto da UC-2.3.1.2.1 a UC-2.3.1.2.7).
                  \item Il sistema restituisce la lista dei documenti che corrispondono ai parametri
                        inseriti (\nameref{risultatiRicerca}).
            \end{enumerate}
\end{itemize}

\subsubsection{UC-2.3.1.2.1 - Ricerca per Dettagli di un Soggetto: Ruolo = PAI}
\begin{itemize}
      \item \textbf{Attore Primario}: utente
      \item \textbf{Precondizioni}: L'utente ha scelto di filtrare la ricerca per Soggetto e ha selezionato il ruolo PAI.
      \item \textbf{Postcondizioni}: I risultati sono filtrati per i dettagli del soggetto PAI.
      \item \textbf{Flusso Principale}:
            \begin{enumerate}
                  \item L'utente inserisce i valori per i campi:
                        \begin{itemize}
                              \item Denominazione Amministrazione/ Codice IPA
                              \item Denominazione Amministrazione AOO/ Codice IPA AOO
                              \item Denominazione Amministrazione UOR/ Codice IPA UOR
                              \item Indirizzi Digitali di Riferimento
                        \end{itemize}
                  \item Il sistema usa questi valori per filtrare la ricerca.
            \end{enumerate}
\end{itemize}

\subsubsection{UC-2.3.1.2.2 - Ricerca per Dettagli di un Soggetto: Ruolo = PAE}
\begin{itemize}
      \item \textbf{Attore Primario}: utente
      \item \textbf{Precondizioni}: L'utente ha scelto di filtrare la ricerca per Soggetto e ha selezionato il ruolo PAE.
      \item \textbf{Postcondizioni}: I risultati sono filtrati per i dettagli del soggetto PAE.
      \item \textbf{Flusso Principale}:
            \begin{enumerate}
                  \item L'utente inserisce i valori per i campi:
                        \begin{itemize}
                              \item Denominazione Amministrazione
                              \item Denominazione Ufficio
                              \item Indirizzi Digitali di Riferimento
                        \end{itemize}
                  \item Il sistema usa questi valori per filtrare la ricerca.
            \end{enumerate}
\end{itemize}

\subsubsection{UC-2.3.1.2.3 - Ricerca per Dettagli di un Soggetto: Ruolo = AS}
\begin{itemize}
      \item \textbf{Attore Primario}: utente
      \item \textbf{Precondizioni}: L'utente ha scelto di filtrare la ricerca per Soggetto e ha selezionato il ruolo AS.
      \item \textbf{Postcondizioni}: I risultati sono filtrati per i dettagli del soggetto AS.
      \item \textbf{Flusso Principale}:
            \begin{enumerate}
                  \item L'utente inserisce i valori per i campi:
                        \begin{itemize}
                              \item Cognome
                              \item Nome
                              \item Codice Fiscale
                              \item Denominazione Amministrazione/ Codice IPA
                              \item Denominazione Amministrazione AOO/ Codice IPA AOO
                              \item Denominazione Amministrazione UOR/ Codice IPA UOR
                              \item Indirizzi Digitali di Riferimento
                        \end{itemize}
                  \item Il sistema usa questi valori per filtrare la ricerca.
            \end{enumerate}
\end{itemize}

\subsubsection{UC-2.3.1.2.4 - Ricerca per Dettagli di un Soggetto: Ruolo = PG}
\begin{itemize}
      \item \textbf{Attore Primario}: utente
      \item \textbf{Precondizioni}: L'utente ha scelto di filtrare la ricerca per Soggetto e ha selezionato il ruolo PG.
      \item \textbf{Postcondizioni}: I risultati sono filtrati per i dettagli del soggetto PG.
      \item \textbf{Flusso Principale}:
            \begin{enumerate}
                  \item L'utente inserisce i valori per i campi:
                        \begin{itemize}
                              \item Denominazione Organizzazione
                              \item Codice fiscale / Partita Iva
                              \item Denominazione Ufficio
                              \item Indirizzi Digitali di Riferimento
                        \end{itemize}
                  \item Il sistema usa questi valori per filtrare la ricerca.
            \end{enumerate}
\end{itemize}

\subsubsection{UC-2.3.1.2.5 - Ricerca per Dettagli di un Soggetto: Ruolo = PF}
\begin{itemize}
      \item \textbf{Attore Primario}: utente
      \item \textbf{Precondizioni}: L'utente ha scelto di filtrare la ricerca per Soggetto e ha selezionato il ruolo PF.
      \item \textbf{Postcondizioni}: I risultati sono filtrati per i dettagli del soggetto PF.
      \item \textbf{Flusso Principale}:
            \begin{enumerate}
                  \item  L'utente inserisce i valori per i campi:
                        \begin{itemize}
                              \item Cognome
                              \item Nome
                              \item Indirizzi Digitali di Riferimento
                        \end{itemize}
                  \item Il sistema usa questi valori per filtrare la ricerca.
            \end{enumerate}
\end{itemize}

\subsubsection{UC-2.3.1.2.6 - Ricerca per Dettagli di un Soggetto: Ruolo = RUP}
\begin{itemize}
      \item \textbf{Attore Primario}: utente
      \item \textbf{Precondizioni}: L'utente ha scelto di filtrare la ricerca per Soggetto e ha selezionato il ruolo RUP.
      \item \textbf{Postcondizioni}: I risultati sono filtrati per i dettagli del soggetto RUP.
      \item \textbf{Flusso Principale}:
            \begin{enumerate}
                  \item L'utente inserisce i valori per i campi:
                        \begin{itemize}
                              \item Cognome
                              \item Nome
                              \item Denominazione Amministrazione/ Codice IPA
                              \item Denominazione Amministrazione AOO/ Codice IPA AOO
                              \item Denominazione Amministrazione UOR/ Codice IPA UOR
                              \item Indirizzi Digitali di Riferimento
                        \end{itemize}
                  \item Il sistema usa questi valori per filtrare la ricerca.
            \end{enumerate}
\end{itemize}

\subsubsection{UC-2.3.1.2.7 - Ricerca per Dettagli di un Soggetto: Ruolo = SW}
\begin{itemize}
      \item \textbf{Attore Primario}: utente
      \item \textbf{Precondizioni}: L'utente ha scelto di filtrare la ricerca per Soggetto e ha selezionato il ruolo SW.
      \item \textbf{Postcondizioni}: I risultati sono filtrati per i dettagli del soggetto SW.
      \item \textbf{Flusso Principale}:
            \begin{enumerate}
                  \item L'utente inserisce la Denominazione Sistema
                  \item Il sistema usa questi valori per filtrare la ricerca.
            \end{enumerate}
\end{itemize}

\subsubsection{UC-2.3.1.3 - Ricerca per Classificazione}
\begin{itemize}
      \item \textbf{Attore Primario}: utente
      \item \textbf{Precondizioni}: L'utente sta effettuando una ricerca per campi comuni.
      \item \textbf{Postcondizioni}: I risultati della ricerca vengono filtrati in base alla Classificazione.
      \item \textbf{Flusso Principale}:
            \begin{enumerate}
                  \item L'utente compila i parametri di Classificazione tra:
                        \begin{itemize}
                              \item Indice di Classificazione
                              \item Descrizione
                              \item Piano di Fascicolo
                        \end{itemize}.
                  \item Viene restituita la lista dei documenti nei quali la Classificazione, per
                        qualcuno dei parametri cercati, corrisponde al parametro ricercato
                        (\nameref{risultatiRicerca}).
            \end{enumerate}
\end{itemize}

\subsubsection{UC-2.3.1.4 - Ricerca per Tempo di Conservazione}
\begin{itemize}
      \item \textbf{Attore Primario}: utente
      \item \textbf{Precondizioni}: L'utente sta effettuando una ricerca per campi comuni.
      \item \textbf{Postcondizioni}: I risultati della ricerca vengono filtrati in base al Tempo di Conservazione.
      \item \textbf{Flusso Principale}:
            \begin{enumerate}
                  \item L'utente digita nella casella di ricerca il tempo di conservazione o seleziona
                        "Perenne".
                  \item Il sistema restituisce la lista dei documenti che corrispondono al parametro
                        inserito (\nameref{risultatiRicerca}).
            \end{enumerate}
\end{itemize}

\subsubsection{UC-2.3.1.5 - Ricerca per Note}
\begin{itemize}
      \item \textbf{Attore Primario}: utente
      \item \textbf{Precondizioni}: L'utente sta effettuando una ricerca per campi comuni.
      \item \textbf{Postcondizioni}: I risultati della ricerca vengono filtrati in base al contenuto delle Note.
      \item \textbf{Flusso Principale}:
            \begin{enumerate}
                  \item L'utente digita nella casella di ricerca il testo della Nota.
                  \item Il sistema restituisce la lista dei documenti che corrispondono al parametro
                        inserito (\nameref{risultatiRicerca}).
            \end{enumerate}
\end{itemize}

\subsubsection{UC-2.3.2 - Ricerca per Tipo Documentale}
\begin{itemize}
      \item \textbf{Attore Primario}: utente
      \item \textbf{Precondizioni}: L'utente si trova nella schermata di ricerca documentale.
      \item \textbf{Postcondizioni}: Vengono mostrati campi di ricerca specifici per il tipo documentale scelto.
      \item \textbf{Flusso Principale}:
            \begin{enumerate}
                  \item L'utente imposta la ricerca per tipo e seleziona tra: Aggregazione Documentale,
                        Documento Informatico, Documento Amministrativo Informatico.
                  \item Il sistema presenta i filtri specifici per il tipo selezionato (descritto negli
                        UC da 2.3.2.1 a 2.3.2.3).
            \end{enumerate}
      \item \textbf{Flusso Alternativo}: Il tipo documentale cercato non è valido o non è presente.
\end{itemize}

\subsubsection{UC-2.3.2.1 - Ricerca per Campi di Documento Informatico e Amministrativo Informatico}
\begin{itemize}
      \item \textbf{Descrizione}: Questo caso d'uso raggruppa i filtri applicabili sia al Documento Informatico sia al Documento Amministrativo Informatico.
\end{itemize}

\subsubsection{UC-2.3.2.1.1 - Ricerca per Dati di Registrazione}
\begin{itemize}
      \item \textbf{Attore Primario}: utente
      \item \textbf{Precondizioni}: L'utente ha selezionato "Documento Informatico" o "Documento Amministrativo Informatico" come tipo documentale.
      \item \textbf{Postcondizioni}: I risultati sono filtrati per Dati di Registrazione.
      \item \textbf{Flusso Principale}:
            \begin{enumerate}
                  \item L'utente sceglie la Tipologia di Flusso tra: Uscita, Entrata, Interno.
                  \item L'utente sceglie il Tipo di Registro tra: Nessuno, Protocollo
                        Ordinario/Protocollo di Emergenza, Repertorio/Registro.
                  \item L'utente inserisce la Data/Ora di Registrazione (nel caso di un documento
                        protocollato tali parametri fanno riferimento alla protocollazione).
                  \item L'utente inserisce il codice identificativo del Registro.
                  \item Il sistema restituisce la lista dei documenti corrispondenti
                        (\nameref{risultatiRicerca}).
            \end{enumerate}
\end{itemize}

\subsubsection{UC-2.3.2.1.2 - Ricerca per Tipologia Documentale}
\begin{itemize}
      \item \textbf{Attore Primario}: utente
      \item \textbf{Precondizioni}: L'utente ha selezionato "Documento Informatico" o "Documento Amministrativo Informatico".
      \item \textbf{Postcondizioni}: I risultati sono filtrati per Tipologia Documentale.
      \item \textbf{Flusso Principale}:
            \begin{enumerate}
                  \item L'utente inserisce la Tipologia Documentale (es. fatture, delibere, determine).
                  \item Il sistema restituisce la lista dei documenti corrispondenti
                        (\nameref{risultatiRicerca}).
            \end{enumerate}
\end{itemize}

\subsubsection{UC-2.3.2.1.3 - Ricerca per Modalità di Formazione}
\begin{itemize}
      \item \textbf{Attore Primario}: utente
      \item \textbf{Precondizioni}: L'utente ha selezionato "Documento Informatico" o "Documento Amministrativo Informatico".
      \item \textbf{Postcondizioni}: I risultati sono filtrati per Modalità di Formazione.
      \item \textbf{Flusso Principale}:
            \begin{enumerate}
                  \item L'utente sceglie la Modalità di Formazione tra:
                        \begin{itemize}
                              \item creazione tramite l'utilizzo di strumenti software che assicurino la produzione
                                    di documenti nei formati previsti nell’Allegato 2 delle Linee Guida
                              \item acquisizione di un documento informatico per via telematica o su supporto
                                    informatico, acquisizione della copia per immagine su supporto informatico di
                                    un documento analogico, acquisizione della copia informatica di un documento
                                    analogico
                              \item memorizzazione su supporto informatico in formato digitale delle informazioni
                                    risultanti da transazioni o processi informatici o dalla presentazione
                                    telematica di dati attraverso moduli o formulari resi disponibili all’utente;
                              \item generazione o raggruppamento anche in via automatica di un insieme di dati o
                                    registrazioni, provenienti da una o più banche dati, anche appartenenti a più
                                    soggetti interoperanti, secondo una struttura logica predeterminata e
                                    memorizzata in forma statica.
                        \end{itemize}
                  \item Il sistema restituisce la lista dei documenti corrispondenti
                        (\nameref{risultatiRicerca}).
            \end{enumerate}
\end{itemize}

\subsubsection{UC-2.3.2.1.4 - Ricerca per campo Riservato}
\begin{itemize}
      \item \textbf{Attore Primario}: utente
      \item \textbf{Precondizioni}: L'utente ha selezionato "Documento Informatico" o "Documento Amministrativo Informatico".
      \item \textbf{Postcondizioni}: I risultati sono filtrati in base al metadato "Riservato".
      \item \textbf{Flusso Principale}:
            \begin{enumerate}
                  \item L'utente sceglie se il file è Riservato o meno.
                  \item Il sistema restituisce la lista dei documenti corrispondenti
                        (\nameref{risultatiRicerca}).
            \end{enumerate}
\end{itemize}

\subsubsection{UC-2.3.2.1.5 - Ricerca per Identificativo di Formato}
\begin{itemize}
      \item \textbf{Attore Primario}: utente
      \item \textbf{Precondizioni}: L'utente ha selezionato "Documento Informatico" o "Documento Amministrativo Informatico".
      \item \textbf{Postcondizioni}: I risultati sono filtrati per Identificativo di Formato.
      \item \textbf{Flusso Principale}:
            \begin{enumerate}
                  \item L'utente sceglie la Tipologia di Formato all'interno di quelli Previsti dalle
                        linee Guida.
                  \item L'utente inserisce il Nome del Prodotto utilizzato per la creazione del
                        Documento.
                  \item L'utente inserisce la Versione del Prodotto utilizzato per la creazione del
                        Documento
                  \item L'utente inserisce il Produttore del Prodotto utilizzato per la creazione del
                        Documento.
                  \item Il sistema restituisce la lista dei documenti corrispondenti
                        (\nameref{risultatiRicerca}).
            \end{enumerate}
\end{itemize}

\subsubsection{UC-2.3.2.1.6 - Ricerca per Dati di Verifica}
\begin{itemize}
      \item \textbf{Attore Primario}: utente
      \item \textbf{Precondizioni}: L'utente ha selezionato "Documento Informatico" o "Documento Amministrativo Informatico".
      \item \textbf{Postcondizioni}: I risultati sono filtrati in base ai dati di verifica.
      \item \textbf{Flusso Principale}:
            \begin{enumerate}
                  \item L'utente sceglie se il file è Firmato Digitalmente o meno.
                  \item L'utente sceglie se il file è Sigillato Elettronicamente o meno.
                  \item L'utente sceglie se il file ha una Marcatura Temporale o meno.
                  \item L'utente sceglie se vi è conformità copie immagine su supporto informatico o
                        meno.
                  \item Il sistema restituisce la lista dei documenti corrispondenti
                        (\nameref{risultatiRicerca}).
            \end{enumerate}
\end{itemize}

\subsubsection{UC-2.3.2.1.7 - Ricerca per Nome del Documento}
\begin{itemize}
      \item \textbf{Attore Primario}: utente
      \item \textbf{Precondizioni}: L'utente ha selezionato "Documento Informatico" o "Documento Amministrativo Informatico".
      \item \textbf{Postcondizioni}: I risultati sono filtrati per Nome del Documento.
      \item \textbf{Flusso Principale}:
            \begin{enumerate}
                  \item L'utente inserisce il Nome del Documento.
                  \item Il sistema restituisce la lista dei documenti corrispondenti
                        (\nameref{risultatiRicerca}).
            \end{enumerate}
\end{itemize}

\subsubsection{UC-2.3.2.1.8 - Ricerca per Versione del Documento}
\begin{itemize}
      \item \textbf{Attore Primario}: utente
      \item \textbf{Precondizioni}: L'utente ha selezionato "Documento Informatico" o "Documento Amministrativo Informatico".
      \item \textbf{Postcondizioni}: I risultati sono filtrati per Versione del Documento.
      \item \textbf{Flusso Principale}:
            \begin{enumerate}
                  \item L'utente inserisce la Versione del Documento.
                  \item Il sistema restituisce la lista dei documenti corrispondenti
                        (\nameref{risultatiRicerca}).
            \end{enumerate}
\end{itemize}

\subsubsection{UC-2.3.2.1.9 - Ricerca per Identificativo del Documento Primario}
\begin{itemize}
      \item \textbf{Attore Primario}: utente
      \item \textbf{Precondizioni}: L'utente ha selezionato "Documento Informatico" o "Documento Amministrativo Informatico".
      \item \textbf{Postcondizioni}: I risultati sono filtrati per Identificativo del Documento Primario.
      \item \textbf{Flusso Principale}:
            \begin{enumerate}
                  \item L'utente inserisce l'Identificativo del Documento Primario.
                  \item Il sistema restituisce la lista dei documenti corrispondenti
                        (\nameref{risultatiRicerca}).
            \end{enumerate}
\end{itemize}

\subsubsection{UC-2.3.2.1.10 - Ricerca per Tracciature Modifiche di Documento}
\begin{itemize}
      \item \textbf{Attore Primario}: utente
      \item \textbf{Precondizioni}: L'utente ha selezionato "Documento Informatico" o "Documento Amministrativo Informatico".
      \item \textbf{Postcondizioni}: I risultati sono filtrati per le tracciature di modifica.
      \item \textbf{Flusso Principale}:
            \begin{enumerate}
                  \item L'utente sceglie il Tipo di modifica tra:
                        \begin{itemize}
                              \item Annullamento
                              \item Registrazione
                              \item Integrazione
                              \item Annotazione
                        \end{itemize}.
                  \item L'Utente sceglie il Soggetto che ha effettuato la modifica.
                  \item L'Utente inserisce la Data/Ora della Modifica.
                  \item L'Utente inserisce l'identificativo documento versione precedente.
                  \item Il sistema restituisce la lista dei documenti corrispondenti
                        (\nameref{risultatiRicerca}).
            \end{enumerate}
\end{itemize}

\subsubsection{UC-2.3.2.2 - Ricerca per Campi di Documento Informatico}
\begin{itemize}
      \item \textbf{Descrizione}: Questo caso d'uso raggruppa i filtri applicabili esclusivamente al Documento Informatico.
\end{itemize}

\subsubsection{UC-2.3.2.2.1 - Ricerca per Soggetti del Documento Informatico}
\begin{itemize}
      \item \textbf{Attore Primario}: utente
      \item \textbf{Precondizioni}: L'utente ha scelto come tipo di documento il Documento Informatico.
      \item \textbf{Postcondizioni}: I risultati sono filtrati in base ai soggetti specifici.
      \item \textbf{Flusso Principale}:
            \begin{enumerate}
                  \item L'Utente sceglie il Ruolo e i sottocampi specifici tra:
                        \begin{itemize}
                              \item Assegnatario (AS)
                              \item Autore (PF, PG, PAI valido solo nei flussi in entrata)
                              \item Destinatario (PF, PG, PAI valido solo come mittente nei flussi in entrata, come
                                    destinatario nei flussi in uscita, PAE valido solo come mittente nei flussi in
                                    entrata, come destinatario nei flussi in uscita)
                              \item Mittente (PF, PG, PAI valido solo come mittente nei flussi in entrata, come
                                    destinatario nei flussi in uscita, PAE valido solo come mittente nei flussi in
                                    entrata, come destinatario nei flussi in uscita)
                              \item Operatore (PF)
                              \item Produttore (SW)
                              \item RGD (PF)
                              \item RSP (PF)
                              \item Soggetto che effettua la registrazione (PF, PG)
                              \item Altro(PF, PG, PAI valido solo come mittente nei flussi in entrata, come
                                    destinatario nei flussi in uscita, PAE valido solo come mittente nei flussi in
                                    entrata, come destinatario nei flussi in uscita)
                        \end{itemize}
                  \item Per ogni ruolo, l'utente può specificare ulteriori dettagli (UC-2.3.1.2).
                  \item Il sistema restituisce la lista dei documenti corrispondenti
                        (\nameref{risultatiRicerca}).
            \end{enumerate}
\end{itemize}

\subsubsection{UC-2.3.2.3 - Ricerca per Campi di Documento Amministrativo Informatico}
\begin{itemize}
      \item \textbf{Descrizione}: Questo caso d'uso raggruppa i filtri applicabili esclusivamente al Documento Amministrativo Informatico.
\end{itemize}

\subsubsection{UC-2.3.2.3.1 - Ricerca per Soggetti del Documento Amministrativo Informatico}
\begin{itemize}
      \item \textbf{Attore Primario}: utente
      \item \textbf{Precondizioni}: L'utente ha scelto come tipo di documento il Documento Amministrativo Informatico.
      \item \textbf{Postcondizioni}: I risultati sono filtrati in base ai soggetti specifici.
      \item \textbf{Flusso Principale}:
            \begin{enumerate}
                  \item L'utente sceglie il Ruolo tra quelli previsti per il Documento Amministrativo
                        Informatico (Amministratore, Assegnatario, etc.).
                  \item Per ogni ruolo, l'utente può specificare ulteriori dettagli (UC-2.3.1.2).
                  \item Il sistema restituisce la lista dei documenti corrispondenti
                        (\nameref{risultatiRicerca}).
            \end{enumerate}
\end{itemize}

\subsubsection{UC-2.3.2.4 - Ricerca per Campi di Aggregazione Documentale}
\begin{itemize}
      \item \textbf{Descrizione}: Questo caso d'uso raggruppa i filtri applicabili esclusivamente all'Aggregazione Documentale.
\end{itemize}

\subsubsection{UC-2.3.2.4.1 - Ricerca per Tipo di Aggregazione}
\begin{itemize}
      \item \textbf{Attore Primario}: utente
      \item \textbf{Precondizioni}: L'utente ha scelto come tipo di documento l'Aggregazione Documentale.
      \item \textbf{Postcondizioni}: I risultati sono filtrati per Tipo di Aggregazione.
      \item \textbf{Flusso Principale}:
            \begin{enumerate}
                  \item L'utente sceglie tra Fascicolo, Serie Documentale, Serie di Fascicoli.
                  \item Il sistema restituisce la lista delle aggregazioni corrispondenti
                        (\nameref{risultatiRicerca}).
            \end{enumerate}
\end{itemize}

\subsubsection{UC-2.3.2.4.2 - Ricerca per Id dell'Aggregazione}
\begin{itemize}
      \item \textbf{Attore Primario}: utente
      \item \textbf{Precondizioni}: L'utente ha scelto come tipo di documento l'Aggregazione Documentale.
      \item \textbf{Postcondizioni}: I risultati sono filtrati per Id dell'Aggregazione.
      \item \textbf{Flusso Principale}:
            \begin{enumerate}
                  \item L'utente inserisce l'Id dell'Aggregazione.
                  \item Il sistema restituisce la lista delle aggregazioni corrispondenti
                        (\nameref{risultatiRicerca}).
            \end{enumerate}
\end{itemize}

\subsubsection{UC-2.3.2.4.3 - Ricerca per Tipologia di Fascicolo}
\begin{itemize}
      \item \textbf{Attore Primario}: utente
      \item \textbf{Precondizioni}: L'utente sta cercando un'Aggregazione Documentale di tipo "Fascicolo".
      \item \textbf{Postcondizioni}: I risultati sono filtrati per Tipologia di Fascicolo.
      \item \textbf{Flusso Principale}:
            \begin{enumerate}
                  \item L'utente sceglie tra: Affare, Attività, Persona Fisica, Persona Giuridica,
                        Procedimento Amministrativo.
                  \item Il sistema restituisce la lista dei fascicoli corrispondenti
                        (\nameref{risultatiRicerca}).
            \end{enumerate}
\end{itemize}

\subsubsection{UC-2.3.2.4.4 - Ricerca per Id Aggregazione Primario}
\begin{itemize}
      \item \textbf{Attore Primario}: utente
      \item \textbf{Precondizioni}: L'utente ha scelto come tipo di documento l'Aggregazione Documentale.
      \item \textbf{Postcondizioni}: I risultati sono filtrati per Id dell'Aggregazione Primaria.
      \item \textbf{Flusso Principale}:
            \begin{enumerate}
                  \item L'utente inserisce l'Id dell'Aggregazione "padre".
                  \item Il sistema restituisce la lista delle aggregazioni corrispondenti
                        (\nameref{risultatiRicerca}).
            \end{enumerate}
\end{itemize}

\subsubsection{UC-2.3.2.4.5 - Ricerca per Soggetti dell'Aggregazione Documentale}
\begin{itemize}
      \item \textbf{Attore Primario}: utente
      \item \textbf{Precondizioni}: L'utente ha scelto come tipo di documento l'Aggregazione Documentale.
      \item \textbf{Postcondizioni}: I risultati sono filtrati per Soggetti.
      \item \textbf{Flusso Principale}:
            \begin{enumerate}
                  \item L'Utente sceglie il Ruolo e i sottocampi specifici tra:
                        \begin{itemize}
                              \item Amministrazione titolare (PAI)
                              \item Amministrazioni partecipanti (PAI o PAE)
                              \item Assegnatario (AS)
                              \item Soggetto intestatario persona fisica (PF)
                              \item Soggetto intestatario persona giuridica (PG, PAI o PAE)
                              \item RUP
                        \end{itemize}
                  \item Per ogni ruolo, l'utente può specificare ulteriori dettagli (UC-2.3.1.2).
                  \item Il sistema restituisce la lista delle aggregazioni corrispondenti
                        (\nameref{risultatiRicerca}).
            \end{enumerate}
\end{itemize}

\subsubsection{UC-2.3.2.4.6 - Ricerca per Data Apertura}
\begin{itemize}
      \item \textbf{Attore Primario}: utente
      \item \textbf{Precondizioni}: L'utente ha scelto come tipo di documento l'Aggregazione Documentale.
      \item \textbf{Postcondizioni}: I risultati sono filtrati per Data di Apertura.
      \item \textbf{Flusso Principale}:
            \begin{enumerate}
                  \item L'utente inserisce la Data di Apertura.
                  \item Il sistema restituisce la lista delle aggregazioni corrispondenti
                        (\nameref{risultatiRicerca}).
            \end{enumerate}
\end{itemize}

\subsubsection{UC-2.3.2.4.7 - Ricerca per Data Chiusura}
\begin{itemize}
      \item \textbf{Attore Primario}: utente
      \item \textbf{Precondizioni}: L'utente ha scelto come tipo di documento l'Aggregazione Documentale.
      \item \textbf{Postcondizioni}: I risultati sono filtrati per Data di Chiusura.
      \item \textbf{Flusso Principale}:
            \begin{enumerate}
                  \item L'utente inserisce la Data di Chiusura.
                  \item Il sistema restituisce la lista delle aggregazioni corrispondenti
                        (\nameref{risultatiRicerca}).
            \end{enumerate}
\end{itemize}

\subsubsection{UC-2.3.2.4.8 - Ricerca per Procedimento Amministrativo}
\begin{itemize}
      \item \textbf{Attore Primario}: utente
      \item \textbf{Precondizioni}: L'utente ha scelto come tipo di documento l'Aggregazione Documentale.
      \item \textbf{Postcondizioni}: I risultati sono filtrati per i dati del Procedimento Amministrativo.
      \item \textbf{Flusso Principale}:
            \begin{enumerate}
                  \item L'utente inserisce la Materia, Argomento e Struttura per i procedimenti.
                  \item L'utente inserisce il Procedimento come denominazione.
                  \item L'utente inserisce il Catalogo dei Procedimenti come URI di pubblicazione del
                        catalogo.
                  \item  L'utente può aggiungere una o più Fasi come filtro (UC-2.3.2.4.8.1).
                  \item Viene restituita la lista dei documenti nei quali i parametri di un
                        Procedimento amministrativo corrispondono ai parametri ricercati
                        (\nameref{risultatiRicerca}).
            \end{enumerate}
\end{itemize}

\subsubsection{UC-2.3.2.4.8.1 - Ricerca per Fasi del Procedimento}
\begin{itemize}
      \item \textbf{Attore Primario}: utente
      \item \textbf{Precondizioni}: L'utente sta cercando per Procedimento Amministrativo.
      \item \textbf{Postcondizioni}: La ricerca viene filtrata per i dati di una o più fasi.
      \item \textbf{Flusso Principale}:
            \begin{enumerate}
                  \item L'utente aggiunge un filtro per "Fase".
                  \item L'utente inserisce i dati della fase:
                        \begin{itemize}
                              \item Tipo Fase (Preparatoria, Istruttoria, Consultiva, Decisoria o deliberativa,
                                    Integrazione dell'efficacia)
                              \item Data di inizio
                              \item Data di fine
                        \end{itemize}
                  \item L'utente può aggiungere altre fasi.
                  \item Il sistema usa questi parametri per la ricerca.
            \end{enumerate}
\end{itemize}

\subsubsection{UC-2.3.2.4.8.1 - Ricerca per Fasi del Procedimento}
\begin{itemize}
      \item \textbf{Attore Primario}: utente
      \item \textbf{Precondizioni}: L'utente sta cercando per Procedimento Amministrativo.
      \item \textbf{Postcondizioni}: La ricerca viene filtrata per i dati di una o più fasi.
      \item \textbf{Flusso Principale}:
            \begin{enumerate}
                  \item L'utente aggiunge un filtro per "Fase".
                  \item L'utente sceglie il Tipo Fase tra: Preparatoria, Istruttoria, Consultiva,
                        Decisoria o deliberativa, Integrazione dell'efficacia.
                  \item L'utente inserisce la Data di inizio della fase.
                  \item L'utente inserisce la Data di fine della fase, se terminata.
                  \item L'utente può aggiungere altre fasi.
                  \item Il sistema usa questi parametri per la ricerca.

                  \item Viene restituita la lista dei documenti nei quali i parametri di Fase
                        corrispondono ai parametri ricercati (\nameref{risultatiRicerca}).
            \end{enumerate}
\end{itemize}

\subsubsection{UC-2.3.2.4.9 - Ricerca per Assegnazione}
\begin{itemize}
      \item \textbf{Attore Primario}: utente
      \item \textbf{Precondizioni}: L'utente ha scelto come tipo di documento l'Aggregazione Documentale.
      \item \textbf{Postcondizioni}: La ricerca è filtrata per i dati di Assegnazione.
      \item \textbf{Flusso Principale}:
            \begin{enumerate}
                  \item L'utente aggiunge un filtro per "Assegnazione".
                  \item L'utente inserisce i dati:
                        \begin{itemize}
                              \item Tipo Assegnazione
                              \item Soggetto Assegnatario (UC-2.3.)
                              \item Data di Inizio
                              \item Data di Fine
                        \end{itemize}
                  \item L'utente può aggiungere altre assegnazioni.
                  \item Il sistema restituisce la lista delle aggregazioni corrispondenti
                        (\nameref{risultatiRicerca}).
            \end{enumerate}
\end{itemize}

\subsubsection{UC-2.3.2.4.10 - Ricerca per Progressivo}
\begin{itemize}
      \item \textbf{Attore Primario}: utente
      \item \textbf{Precondizioni}: L'utente ha scelto come tipo di documento l'Aggregazione Documentale.
      \item \textbf{Postcondizioni}: I risultati sono filtrati per numero Progressivo.
      \item \textbf{Flusso Principale}:
            \begin{enumerate}
                  \item L'utente inserisce il numero Progressivo dell'Aggregazione.
                  \item Il sistema restituisce la lista delle aggregazioni corrispondenti
                        (\nameref{risultatiRicerca}).
            \end{enumerate}
\end{itemize}

\subsubsection{UC-2.4 - Visualizzazione Risultati di Ricerca} \label{risultatiRicerca}
\begin{itemize}
      \item \textbf{Attore Primario}: utente
      \item \textbf{Precondizioni}: L'utente ha eseguito una ricerca (tramite UC-2.1, UC-2.2, UC-2.3 o UC-2.5).
      \item \textbf{Postcondizioni}: I risultati della ricerca sono presentati all'utente.
      \item \textbf{Flusso Principale}:
            \begin{enumerate}
                  \item Il sistema visualizza un elenco di documenti o aggregazioni che corrispondono
                        ai criteri di ricerca.
                  \item Per ogni elemento, vengono mostrate le informazioni principali (Nome, Data,
                        Tipo).
            \end{enumerate}
      \item \textbf{Flusso Alternativo (Nessun risultato)}: Il sistema comunica che la ricerca non ha prodotto alcun risultato.
\end{itemize}

\subsubsection{UC-2.5 - Ricerca Semantica}
\begin{itemize}
      \item \textbf{Attore Primario}: utente
      \item \textbf{Attore Secondario}: LLM in remoto
      \item \textbf{Precondizioni}: L'utente ha accesso alla funzionalità di ricerca semantica e dispone di una connessione internet.
      \item \textbf{Postcondizioni}: Visualizzazione di un elenco di documenti o aggregazioni pertinenti alla richiesta in linguaggio naturale.
      \item \textbf{Flusso Principale}:
            \begin{enumerate}
                  \item L'utente descrive in linguaggio naturale cosa sta cercando nell'apposita barra
                        di ricerca.
                  \item Il sistema invia la richiesta al servizio LLM.
                  \item Il sistema riceve i risultati e li visualizza (UC-2.4).
            \end{enumerate}
      \item \textbf{Flusso Alternativo}:
            \begin{itemize}
                  \item La ricerca non riporta alcun risultato pertinente.
                  \item La connessione internet viene a mancare, impedendo di completare la richiesta.
            \end{itemize}
\end{itemize}
