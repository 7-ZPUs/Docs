\usecase{ricercaDIP}{Ricerca nel DIP}
\begin{itemize}
      \item \textbf{Attore Primario}: Utente
      \item \textbf{Precondizioni}: 
      \begin{itemize} 
            \item L'Utente ha avviato l'applicazione
            \item Il sistema ha indicizzato i documenti presenti nel DIP
      \end{itemize}
      \item \textbf{Postcondizioni}:
            \begin{itemize}
                  \item Il sistema produce i risultati di ricerca
            \end{itemize}
      \item \textbf{Flusso Principale}
      \begin{enumerate}
            \item 
      \end{enumerate}
\end{itemize}

\varusecase{ricercaDIPConFiltri}{Ricerca nel DIP con filtri}
\begin{itemize}
      \item \textbf{Attore Primario}: Utente
      \item \textbf{Precondizioni}: 
      \begin{itemize} 
            \item L'Utente ha avviato l'applicazione
            \item Il sistema ha indicizzato i documenti presenti nel DIP
      \end{itemize}
      \item \textbf{Postcondizioni}:
            \begin{itemize}
                  \item Il sistema produce i risultati di ricerca in base ai filtri specificati
            \end{itemize}
      \item \textbf{Flusso Principale}:
            \begin{enumerate}
                  \item L'Utente aggiunge filtri come parametri per la ricerca $\rightarrow$ Vedi \hyperref[aggiuntaFiltriRicercaDocumento]{[UC-\ref{aggiuntaFiltriRicercaDocumento}]}
                  \item Vengono controllati i valori dei filtri di ricerca
                  \item Il sistema produce i risultati di ricerca in base ai filtri specificati
            \end{enumerate}
      \item \textbf{Flusso alternativo}: 
            \begin{itemize}
                  \item Uno o più campi non rispettano i formati previsti $\rightarrow$ Vedi \hyperref[campoNonValido]{[UC-\ref{campoNonValido}]}
            \end{itemize}
      \item \textbf{Inclusioni}: 
            \begin{itemize}[label=$\rhd$]
                  \item \hyperref[aggiuntaFiltriRicercaDocumento]{[UC-\ref{aggiuntaFiltriRicercaDocumento}]}
                  \item \hyperref[validazioneCampi]{[UC-\ref{validazioneCampi}]}
            \end{itemize}
\end{itemize}
\usecase{campoNonValido}{Uno o più valori dei filtri di ricerca non validi}
\begin{itemize}
    \item \textbf{Attore Primario}: Utente
    \item \textbf{Precondizioni}: 
    \begin{itemize}
        \item L'utente ha avviato l'applicazione
        \item L'Utente sta effettuando una ricerca nel DIP con filtri
    \end{itemize}
    \item \textbf{Postcondizioni}: Il sistema annulla la ricerca mostrando un messaggio di errore che indica i campi non validi.
    \item \textbf{Flusso Principale}:
          \begin{enumerate}
              \item Il sistema annulla l'operazione di ricerca.
              \item Il sistema mostra a video un messaggio di errore che indica quali campi non rispettano i formati previsti.
          \end{enumerate}
\end{itemize}

\varusecase{ricercaDIPSemantica}{Ricerca nel DIP semantica}
\begin{itemize}
    \item \textbf{Attore Primario}: Utente
    \item \textbf{Precondizioni}:
    \begin{itemize}
      \item L'Utente ha avviato l'applicazione
      \item Il sistema ha indicizzato i documenti presenti nel DIP
    \end{itemize}
    \item \textbf{Postcondizioni}:
    \begin{itemize}
        \item Il sistema produce i risultati di ricerca semantica
    \end{itemize}
    \item \textbf{Flusso principale}:
    \begin{itemize}
        \item L'Utente specifica la query di ricerca
        \item Il sistema produce i risultati di ricerca per la query specificata
    \end{itemize}
    \item \textbf{Flusso alternativo}:
    \begin{itemize}
        \item I documenti non sono stati indicizzati dall'engine semantico
    \end{itemize}
\end{itemize}

\usecase{documentiNonIndicizzati}{Documenti non indicizzati dall'engine semantico}
\begin{itemize}
    \item \textbf{Attore Primario}: Utente
    \item \textbf{Precondizioni}: 
    \begin{itemize}
        \item L'utente ha avviato l'applicazione
        \item L'Utente sta effettuando una ricerca nel DIP semantica
    \end{itemize}
    \item \textbf{Postcondizioni}: Il sistema annulla la ricerca mostrando un messaggio di errore che dice all'Utente che i documenti non sono stati indicizzati dall'engine semantico.
    \item \textbf{Flusso Principale}:
          \begin{enumerate}
      \item Il sistema annulla l'operazione di ricerca.
              \item Il sistema mostra a video un messaggio di errore che indica che i documenti non sono stati indicizzati dall'engine semantico.
          \end{enumerate}
\end{itemize}

\subusecase{PrecisioneSemantica}{Imposta precisione di ricerca semantica}
\begin{itemize}
      \item \textbf{Attore Primario}: Utente
      \item \textbf{Precondizioni}:
      \begin{itemize}
            \item L'Utente ha avviato l'applicazione
            \item Il sistema ha indicizzato i documenti presenti nel DIP
      \end{itemize}
      \item \textbf{Postcondizioni}:
      \begin{itemize}
            \item L'Utente ha impostato il valore di precisione per la ricerca semantica
      \end{itemize}
      \item \textbf{Flusso principale}:
      \begin{itemize}
            \item L'Utente specifica il valore di precisione per la ricerca semantica scegliendo tra: \textbf{Basso}, \textbf{Medio} o \textbf{Alto}.
            \item Il sistema applica il livello di precisione alla ricerca semantica.
      \end{itemize}
\end{itemize}

\varusecase{ricercaClasse}{Ricercare Classe Documentale per Nome}
\begin{itemize}
      \item \textbf{Attore Primario}: Utente
      \item \textbf{Precondizioni}: 
      \begin{itemize} 
            \item L'Utente ha avviato l'applicazione
            \item Il sistema ha indicizzato i documenti presenti nel DIP
      \end{itemize}
      \item \textbf{Postcondizioni}:
            \begin{itemize}
                  \item Il sistema produce i risultati di ricerca in base al Nome inserito
            \end{itemize}
      \item \textbf{Flusso Principale}:
            \begin{enumerate}
                  \item L'Utente inserisce il Nome della Classe Documentale
                  \item Il sistema valida il formato del Nome
                  \item Il sistema produce i risultati di ricerca in base al Nome inserito
            \end{enumerate}
     \item \textbf{Flusso Alternativo}: La ricerca non produce risultati e il sistema informa l'Utente $\rightarrow$ Vedi \hyperref[nessunRisultato]{[UC-\ref{nessunRisultato}]}
      \item \textbf{Estensioni}: 
      \begin{itemize}[label=$\rhd$]
            \item \hyperref[nessunRisultato]{[UC-\ref{nessunRisultato}]}
      \end{itemize}

      \item \textbf{Specializza}: 
      \begin{itemize}[label=$\rhd$]
            \item \hyperref[ricercaDIP]{[UC-\ref{ricercaDIP}]}
      \end{itemize}
\end{itemize}

% continuare verifica da qui.

\varusecase{ricecrcaProcesso}{Ricercare Processo per Id}
\begin{itemize}
      \item \textbf{Attore Primario}: Utente
      \item \textbf{Precondizioni}: \begin{itemize}
            \item L'Utente ha avviato l'applicazione
            \item Il sistema ha indicizzato i documenti presenti nel DIP
      \end{itemize}
      \item \textbf{Postcondizioni}: Il sistema usa il filtro per la ricerca del Processo
      \item \textbf{Flusso Principale}:
            \begin{enumerate}
    \item L'Utente inserisce l'Id del Processo.
    \item Il sistema valida il formato del Nome $\rightarrow$ Vedi \hyperref[validazioneTestoLibero]{[UC-10.\ref{validazioneTestoLibero}]}
    \item Il sistema mostra i risultati della ricerca $\rightarrow$ Vedi \hyperref[visualizzaRisultati]{[UC-\ref{visualizzaRisultati}]}
      \end{enumerate}
      \item \textbf{Flusso Alternativo (Nessun risultato)}: La ricerca non produce risultati e il sistema informa l'Utente $\rightarrow$ Vedi \hyperref[nessunRisultato]{[UC-\ref{nessunRisultato}]}

\item \textbf{Inclusioni}: 
    \begin{itemize}[label=$\rhd$]
        \item \hyperref[visualizzaRisultati]{[UC-\ref{visualizzaRisultati}]}
    \end{itemize}

\item \textbf{Estensioni}: 
    \begin{itemize}[label=$\rhd$]
        \item \hyperref[nessunRisultato]{[UC-\ref{nessunRisultato}]}
    \end{itemize}

\item \textbf{Specializza}: 
    \begin{itemize}[label=$\rhd$]
        \item \hyperref[ricercaDIP]{[UC-\ref{ricercaDIP}]}
    \end{itemize}
\end{itemize}

\usecase{aggiuntaFiltriRicercaDocumento}{Aggiunta filtri per Ricerca Documento}
\begin{itemize}
      \item \textbf{Attore Primario}: Utente
      \item \textbf{Precondizioni}: 
      \begin{itemize}
            \item L'Utente ha avviato l'applicazione
            \item Il sistema ha indicizzato i documenti presenti nel DIP
      \end{itemize}
      \item \textbf{Postcondizioni}:  Il filtro aggiunto dall'Utente è parametro per la ricerca e per la generazione del risultato
      \item \textbf{Flusso Principale}:
        \begin{enumerate}
            \item L'Utente imposta uno o più filtri per la ricerca tra campi comuni $\rightarrow$ Vedi \hyperref[aggiuntaFiltriComuni]{[UC-7.\ref{aggiuntaFiltriComuni}]}
            \item L'Utente imposta uno o più filtri per la ricerca tra campi specifici per tipo documentale $\rightarrow$ Vedi \hyperref[addFiltriTipoDocumento]{[UC-7.\ref{addFiltriTipoDocumento}]}
            \item L'Utente imposta uno o più filtri Custom $\rightarrow$ Vedi \hyperref[addCustomMetadata]{[UC-7.\ref{addCustomMetadata}]}
    \end{enumerate}
\end{itemize}

\subusecase{aggiuntaFiltriComuni}{Aggiunta filtri Comuni}
\begin{itemize}
      \item \textbf{Attore Primario}: Utente
      \item \textbf{Precondizioni}:\begin{itemize}
            \item L'Utente ha avviato l'applicazione
            \item Il sistema ha indicizzato i documenti presenti nel DIP
      \end{itemize}
      \item \textbf{Postcondizioni}:  Il filtro aggiunto dall'Utente è parametro per la ricerca e per la generazione del risultato
      \item \textbf{Flusso Principale}:
            \begin{enumerate}
                \item L'Utente seleziona uno o più campi comuni su cui basare la ricerca $\rightarrow$ Vedi da \hyperref[addChiaveDescrittiva]{[UC-7.1.\ref{addChiaveDescrittiva}]} a \hyperref[addSoggetti]{[UC-7.1.\ref{addSoggetti}]}
                \item L'Utente compila i valori per i campi selezionati.
                \item Validazione del formato per i campi che lo richiedano $\rightarrow$ Vedi \hyperref[validazioneCampi]{[UC-\ref{validazioneCampi}]}
                \item Il sistema utilizza questi parametri per la ricerca.
        \end{enumerate}
\end{itemize}

\subsubusecase{addChiaveDescrittiva}{Aggiunta filtro Chiave Descrittiva}
\begin{itemize}
      \item \textbf{Attore Primario}: Utente
      \item \textbf{Precondizioni}: \begin{itemize}
            \item L'Utente ha avviato l'applicazione
            \item Il sistema ha indicizzato i documenti presenti nel DIP
            \item L'Utente ha selezionato il filtro Chiave Descrittiva tra i campi comuni
      \end{itemize}
      \item \textbf{Postcondizioni}:  Il filtro aggiunto dall'Utente è parametro per la ricerca e per la generazione del risultato
      \item \textbf{Flusso Principale}:
            \begin{enumerate}
                  \item L'Utente compila i valori per:
                        \begin{itemize}
                              \item Chiave Descrittiva
                              \item Oggetto
                              \item Parole Chiave.
                        \end{itemize}
            \end{enumerate}
\end{itemize}

\subsubusecase{addClassificazione}{Aggiunta filtro Classificazione}
\begin{itemize}
      \item \textbf{Attore Primario}: Utente
      \item \textbf{Precondizioni}:  \begin{itemize}
            \item L'Utente ha avviato l'applicazione
            \item Il sistema ha indicizzato i documenti presenti nel DIP
            \item L'Utente ha selezionato il filtro Classificazione tra i campi comuni
      \end{itemize}
      \item \textbf{Postcondizioni}:  Il filtro aggiunto dall'Utente è parametro per la ricerca e per la generazione del risultato.
      \item \textbf{Flusso Principale}:
            \begin{enumerate}
                  \item L'Utente compila i parametri di Classificazione tra:
                        \begin{itemize}
                              \item Indice di Classificazione
                              \item Descrizione
                              \item Piano di Fascicolo
                        \end{itemize}
            \end{enumerate}
\end{itemize}

\subsubusecase{addTempoConserva}{Aggiunta filtro Tempo di Conservazione}
\begin{itemize}
      \item \textbf{Attore Primario}: Utente
      \item \textbf{Precondizioni}: \begin{itemize}
            \item L'Utente ha avviato l'applicazione
            \item Il sistema ha indicizzato i documenti presenti nel DIP
            \item L'Utente ha selezionato il filtro Tempo di Conservazione tra i campi comuni
      \end{itemize}
      \item \textbf{Postcondizioni}:  Il filtro aggiunto dall'Utente è parametro per la ricerca e per la generazione del risultato.
      \item \textbf{Flusso Principale}:
            \begin{enumerate}
                  \item L'Utente inserisce il valore numerico del il tempo di conservazione o seleziona
                        "Perenne".
            \end{enumerate}
\end{itemize}

\subsubusecase{addNote}{Aggiunta filtro Note}
\begin{itemize}
      \item \textbf{Attore Primario}: Utente
      \item \textbf{Precondizioni}: \begin{itemize}
            \item L'Utente ha avviato l'applicazione
            \item Il sistema ha indicizzato i documenti presenti nel DIP
            \item L'Utente ha selezionato il filtro Note tra i campi comuni
      \end{itemize}
      \item \textbf{Postcondizioni}:  Il filtro aggiunto dall'Utente è parametro per la ricerca e per la generazione del risultato.
      \item \textbf{Flusso Principale}:
            \begin{enumerate}
                  \item L'Utente inserisce il testo della Nota.
            \end{enumerate}
\end{itemize}

\subsubusecase{addTipoDocumento}{Aggiunta filtro Tipo documento}
\begin{itemize}
      \item \textbf{Attore Primario}: Utente
      \item \textbf{Precondizioni}: \begin{itemize}
            \item L'Utente ha avviato l'applicazione
            \item Il sistema ha indicizzato i documenti presenti nel DIP
            \item L'Utente ha selezionato il filtro Tipo documento tra i campi comuni
      \end{itemize}
      \item \textbf{Postcondizioni}:  Il filtro aggiunto dall'Utente è parametro per la ricerca e per la generazione del risultato.
      \item \textbf{Flusso Principale}:
            \begin{enumerate}
                    \item L'Utente sceglie il tipo di documento per l'aggiunta del filtro di soggetti tra: \begin{itemize}
                        \item Documento Informatico
                        \item Documento Amministrativo Informatico
                        \item Aggregazione Documentale
                    \end{itemize}.
            \end{enumerate}
\end{itemize}

\subsubusecase{addSoggetti}{Aggiunta filtro Soggetti}
\begin{itemize}
      \item \textbf{Attore Primario}: Utente
      \item \textbf{Precondizioni}: \begin{itemize}
            \item L'Utente ha avviato l'applicazione
            \item Il sistema ha indicizzato i documenti presenti nel DIP
            \item L'Utente ha selezionato il filtro Soggetti tra i campi comuni
      \end{itemize}
      \item \textbf{Postcondizioni}:  Il filtro aggiunto dall'Utente è parametro per la ricerca e per la generazione del risultato.
      \item \textbf{Flusso Principale}:
            \begin{enumerate}
                  \item L'Utente sceglie il Tipo di Documento $\rightarrow$ Vedi \hyperref[addTipoDocumento]{[UC-7.1.\ref{addTipoDocumento}]}
                  \item L'Utente sceglie uno o più tipi di soggetti in base al tipo di documento e ne specifica i campi.
            \end{enumerate}
\end{itemize}

\deepusecase{addSoggettiDI}{Aggiunta filtro soggetti per Documento Informatico}
\begin{itemize}
      \item \textbf{Attore Primario}: Utente
      \item \textbf{Precondizioni}: 
      \begin{itemize}
            \item L'Utente ha avviato l'applicazione
            \item Il sistema ha indicizzato i documenti presenti nel DIP
            \item L'Utente ha selezionato il filtro Soggetti tra i campi comuni
            \item L'Utente ha scelto come tipo di documento il Documento Informatico
      \end{itemize}
      \item \textbf{Postcondizioni}:  Il filtro aggiunto dall'Utente è parametro per la ricerca e per la generazione del risultato.
      \item \textbf{Flusso Principale}:
            \begin{enumerate}
                  \item L'Utente sceglie il Ruolo e i sottocampi specifici tra:
                        \begin{itemize}
                              \item Assegnatario (AS)
                              \item Autore (PF, PG, PAI valido solo nei flussi in entrata)
                              \item Destinatario (PF, PG, PAI valido solo come mittente nei flussi in entrata, come
                                    destinatario nei flussi in uscita, PAE valido solo come mittente nei flussi in
                                    entrata, come destinatario nei flussi in uscita)
                              \item Mittente (PF, PG, PAI valido solo come mittente nei flussi in entrata, come
                                    destinatario nei flussi in uscita, PAE valido solo come mittente nei flussi in
                                    entrata, come destinatario nei flussi in uscita)
                              \item Operatore (PF)
                              \item Produttore (SW)
                              \item RGD (PF)
                              \item RSP (PF)
                              \item Soggetto che effettua la registrazione (PF, PG)
                              \item Altro(PF, PG, PAI valido solo come mittente nei flussi in entrata, come
                                    destinatario nei flussi in uscita, PAE valido solo come mittente nei flussi in
                                    entrata, come destinatario nei flussi in uscita)
                        \end{itemize}
                  \item Per ogni ruolo, l'Utente può specificare ulteriori dettagli $\rightarrow$ Vedi \hyperref[addDettagliSoggetto]{[UC-7.\ref{addDettagliSoggetto}]}
            \end{enumerate}
\end{itemize}

\deepusecase{addSoggettiDAI}{Aggiunta filtro soggetti per Documento Amministrativo informatico}
\begin{itemize}
      \item \textbf{Attore Primario}: Utente
      \item \textbf{Precondizioni}: 
      \begin{itemize}
            \item L'Utente ha avviato l'applicazione
            \item Il sistema ha indicizzato i documenti presenti nel DIP
            \item L'Utente ha selezionato il filtro Soggetti tra i campi comuni
            \item L'Utente ha scelto come tipo di documento il Documento Amministrativo Informatico
      \end{itemize}
      \item \textbf{Postcondizioni}:  Il filtro aggiunto dall'Utente è parametro per la ricerca e per la generazione del risultato.
      \item \textbf{Flusso Principale}:
            \begin{enumerate}
                  \item L'Utente sceglie il Ruolo e i sottocampi specifici tra:
                        \begin{itemize}
                              \item Amministrazione che effettua la registrazione (PAI)
                              \item Assegnatario (AS)
                              \item Autore (PF, PG valido nei flussi di entrata, PAI, PAE valido nel flussi in entrata)
                              \item Destinatario (PF, PG, PAI, PAE)
                              \item Mittente (PF, PG, PAI, PAE)
                              \item Operatore (PF)
                              \item Produttore (SW)
                              \item RGD (PF)
                              \item RSP (PF)
                              \item RUP (RUP)
                        \end{itemize}
                  \item Per ogni ruolo, l'Utente può specificare ulteriori dettagli $\rightarrow$ Vedi \hyperref[addDettagliSoggetto]{[UC-7.\ref{addDettagliSoggetto}]}
            \end{enumerate}
\end{itemize}

\deepusecase{addSoggettiAgg}{Aggiunta filtro soggetti per Aggregazione}
\begin{itemize}
      \item \textbf{Attore Primario}: Utente
      \item \textbf{Precondizioni}: 
      \begin{itemize}
            \item L'Utente ha avviato l'applicazione
            \item Il sistema ha indicizzato i documenti presenti nel DIP
            \item L'Utente ha selezionato il filtro Soggetti tra i campi comuni
            \item L'Utente ha scelto come tipo di documento l'Aggregazione Documentale
      \end{itemize}
      \item \textbf{Postcondizioni}:  Il filtro aggiunto dall'Utente è parametro per la ricerca e per la generazione del risultato.
      \item \textbf{Flusso Principale}:
            \begin{enumerate}
                  \item L'Utente sceglie il Ruolo e i sottocampi specifici tra:
                        \begin{itemize}
                              \item Amministrazione titolare (PAI).
                              \item Amministrazioni partecipanti (PAI o PAE).
                              \item Assegnatario (AS).
                              \item Soggetto intestatario persona fisica (PF).
                              \item Soggetto intestatario persona giuridica (PG, PAI o PAE).
                              \item RUP.
                        \end{itemize}
                  \item Per ogni ruolo, l'Utente può specificare ulteriori dettagli $\rightarrow$ Vedi \hyperref[addDettagliSoggetto]{[UC-7.\ref{addDettagliSoggetto}]}
            \end{enumerate}
\end{itemize}

\subusecase{addFiltriTipoDocumento}{Aggiunta filtri per Tipo Documentale}
\begin{itemize}
      \item \textbf{Attore Primario}: Utente
      \item \textbf{Precondizioni}: 
      \begin{itemize}
            \item L'Utente ha avviato l'applicazione
            \item Il sistema ha indicizzato i documenti presenti nel DIP
            \item L'Utente ha selezionato il filtro il filtro Tipo documentale
      \end{itemize}
      \item \textbf{Postcondizioni}:  Il filtro aggiunto dall'Utente è parametro per la ricerca e per la generazione del risultato.
      \item \textbf{Flusso Principale}:
            \begin{enumerate}
                  \item L'Utente può aggiungere i filtri specifici per il tipo selezionato $\rightarrow$ Vedi negli UC da \hyperref[aggiuntaFiltriDI-DAI]{[UC-7.2.\ref{aggiuntaFiltriDI-DAI}]} a \hyperref[aggiuntaFiltriAggregazione]{[UC-7.2.\ref{aggiuntaFiltriAggregazione}]}
            \end{enumerate}
\end{itemize}

\subsubusecase{aggiuntaFiltriDI-DAI}{Aggiunta filtri per Documento Informatico e Amministrativo Informatico}
\begin{itemize}
      \item \textbf{Attore Primario}: Utente
      \item \textbf{Precondizioni}: 
      \begin{itemize}
            \item L'Utente ha avviato l'applicazione
            \item Il sistema ha indicizzato i documenti presenti nel DIP
            \item L'Utente ha scelto come tipo Documento Informatico o Amministrativo Informatico
      \end{itemize}
      \item \textbf{Postcondizioni}:  Il filtro aggiunto dall'Utente è parametro per la ricerca e per la generazione del risultato.
      \item \textbf{Flusso Principale}:
            \begin{enumerate}
                  \item L'Utente può aggiungere i filtri specifici per il tipo Documento Informatico e Amministrativo Informatico $\rightarrow$ Vedi negli UC da \hyperref[addDatiRegistrazione]{[UC-7.2.1.\ref{addDatiRegistrazione}]} a \hyperref[addTracciatureModificheDocumento]{[UC-7.2.1.\ref{addTracciatureModificheDocumento}]}
            \end{enumerate}
\end{itemize}

\deepusecase{addDatiRegistrazione}{Aggiunta filtro Dati di Registrazione}
\begin{itemize}
      \item \textbf{Attore Primario}: Utente
      \item \textbf{Precondizioni}: 
           \begin{itemize}
            \item L'Utente ha avviato l'applicazione
            \item Il sistema ha indicizzato i documenti presenti nel DIP
            \item L'Utente ha scelto come tipo Documento Informatico o Amministrativo Informatico
      \end{itemize}
      \item \textbf{Postcondizioni}:  Il filtro aggiunto dall'Utente è parametro per la ricerca e per la generazione del risultato.
      \item \textbf{Flusso Principale}:
            \begin{enumerate}
                  \item L'Utente sceglie la Tipologia di Flusso tra: \begin{itemize}
                              \item Uscita
                              \item Entrata
                              \item Interno.
                        \end{itemize}
                  \item L'Utente sceglie il Tipo di Registro tra: \begin{itemize}
                              \item Nessuno,
                              \item Protocollo Ordinario/Protocollo di Emergenza
                              \item Repertorio/Registro.
                        \end{itemize}
                  \item L'Utente inserisce la Data/Ora di Registrazione (nel caso di un documento
                        protocollato tali parametri fanno riferimento alla protocollazione).
                  \item L'Utente inserisce il codice identificativo del Registro.
            \end{enumerate}
\end{itemize}

\deepusecase{addTipologiaDocumentale}{Aggiunta filtro Tipologia Documentale}
\begin{itemize}
      \item \textbf{Attore Primario}: Utente
      \item \textbf{Precondizioni}: 
      \begin{itemize}
            \item L'Utente ha avviato l'applicazione
            \item Il sistema ha indicizzato i documenti presenti nel DIP
            \item L'Utente ha scelto come tipo Documento Informatico o Amministrativo Informatico
      \end{itemize}
      \item \textbf{Postcondizioni}:  Il filtro aggiunto dall'Utente è parametro per la ricerca e per la generazione del risultato.
      \item \textbf{Flusso Principale}:
            \begin{enumerate}
                  \item L'Utente inserisce il nome della Tipologia Documentale (fatture, delibere,
                        determine).
            \end{enumerate}
\end{itemize}

\deepusecase{addModalitaFormazione}{Aggiunta filtro Modalità di Formazione}
\begin{itemize}
      \item \textbf{Attore Primario}: Utente
      \item \textbf{Precondizioni}: 
      \begin{itemize}
            \item L'Utente ha avviato l'applicazione
            \item Il sistema ha indicizzato i documenti presenti nel DIP
            \item L'Utente ha scelto come tipo Documento Informatico o Amministrativo Informatico
      \end{itemize}
      \item \textbf{Postcondizioni}:  Il filtro aggiunto dall'Utente è parametro per la ricerca e per la generazione del risultato.
      \item \textbf{Flusso Principale}:
            \begin{enumerate}
                  \item L'Utente sceglie la Modalità di Formazione tra:
                        \begin{itemize}
                              \item Creazione di un documento informatico ex novo
                              \item Acquisizione di un documento informatico da altro documento informatico o da
                                    supporto informatico
                              \item Memorizzazione su supporto informatico in formato digitale
                              \item Generazione o raggruppamento in forma statica
                        \end{itemize}
            \end{enumerate}
\end{itemize}

\deepusecase{addRiservato}{Aggiunta filtro campo Riservato}
\begin{itemize}
      \item \textbf{Attore Primario}: Utente
      \item \textbf{Precondizioni}: 
      \begin{itemize}
            \item L'Utente ha avviato l'applicazione
            \item Il sistema ha indicizzato i documenti presenti nel DIP
            \item L'Utente ha scelto come tipo Documento Informatico o Amministrativo Informatico
      \end{itemize}
      \item \textbf{Postcondizioni}:  Il filtro aggiunto dall'Utente è parametro per la ricerca e per la generazione del risultato.
      \item \textbf{Flusso Principale}:
            \begin{enumerate}
                  \item L'Utente sceglie se il file è Riservato o meno.
            \end{enumerate}
\end{itemize}

\deepusecase{addIdentificativoFormato}{Aggiunta filtro Identificativo di Formato}
\begin{itemize}
      \item \textbf{Attore Primario}: Utente
      \item \textbf{Precondizioni}: 
      \begin{itemize}
            \item L'Utente ha avviato l'applicazione
            \item Il sistema ha indicizzato i documenti presenti nel DIP
            \item L'Utente ha scelto come tipo Documento Informatico o Amministrativo Informatico
      \end{itemize}
      \item \textbf{Postcondizioni}:  Il filtro aggiunto dall'Utente è parametro per la ricerca e per la generazione del risultato.
      \item \textbf{Flusso Principale}:
            \begin{enumerate}
                  \item L'Utente sceglie la Tipologia di Formato all'interno di quelli Previsti dalle
                        linee Guida (.pdf, .xml, .docx, ecc.).
                  \item L'Utente inserisce il Nome del Prodotto utilizzato per la creazione del
                        Documento.
                  \item L'Utente inserisce il numero della Versione del Prodotto utilizzato per la
                        creazione del Documento
                  \item L'Utente inserisce il nome il Produttore del Prodotto utilizzato per la
                        creazione del Documento.
            \end{enumerate}
\end{itemize}

\deepusecase{addDatiVerifica}{Aggiunta filtro Dati di Verifica}
\begin{itemize}
      \item \textbf{Attore Primario}: Utente
      \item \textbf{Precondizioni}: 
      \begin{itemize}
            \item L'Utente ha avviato l'applicazione
            \item Il sistema ha indicizzato i documenti presenti nel DIP
            \item L'Utente ha scelto come tipo Documento Informatico o Amministrativo Informatico
      \end{itemize}
      \item \textbf{Postcondizioni}:  Il filtro aggiunto dall'Utente è parametro per la ricerca e per la generazione del risultato.
      \item \textbf{Flusso Principale}:
            \begin{enumerate}
                  \item L'Utente sceglie se il file è Firmato Digitalmente o meno.
                  \item L'Utente sceglie se il file è Sigillato Elettronicamente o meno.
                  \item L'Utente sceglie se il file ha una Marcatura Temporale o meno.
                  \item L'Utente sceglie se vi è conformità copie immagine su supporto informatico o
                        meno.
            \end{enumerate}
\end{itemize}

\deepusecase{addNomeDocumento}{Aggiunta filtro Nome del Documento}
\begin{itemize}
      \item \textbf{Attore Primario}: Utente
      \item \textbf{Precondizioni}: 
      \begin{itemize}
            \item L'Utente ha avviato l'applicazione
            \item Il sistema ha indicizzato i documenti presenti nel DIP
            \item L'Utente ha scelto come tipo Documento Informatico o Amministrativo Informatico
      \end{itemize}
      \item \textbf{Postcondizioni}:  Il filtro aggiunto dall'Utente è parametro per la ricerca e per la generazione del risultato.
      \item \textbf{Flusso Principale}:
            \begin{enumerate}
                  \item L'Utente inserisce il Nome del Documento.
            \end{enumerate}
\end{itemize}

\deepusecase{addVersioneDocumento}{Aggiunta filtro Versione del Documento}
\begin{itemize}
      \item \textbf{Attore Primario}: Utente
      \item \textbf{Precondizioni}: 
      \begin{itemize}
            \item L'Utente ha avviato l'applicazione
            \item Il sistema ha indicizzato i documenti presenti nel DIP
            \item L'Utente ha scelto come tipo Documento Informatico o Amministrativo Informatico
      \end{itemize}
      \item \textbf{Postcondizioni}: I risultati sono filtrati per Versione del Documento.
      \item \textbf{Flusso Principale}:
            \begin{enumerate}
                  \item L'Utente inserisce il numero la Versione del Documento.
            \end{enumerate}
\end{itemize}

\deepusecase{addIdentificativoDocumentoPrimario}{Aggiunta filtro Identificativo del Documento Primario}
\begin{itemize}
      \item \textbf{Attore Primario}: Utente
      \item \textbf{Precondizioni}:
      \begin{itemize}
            \item L'Utente ha avviato l'applicazione
            \item Il sistema ha indicizzato i documenti presenti nel DIP
            \item L'Utente ha scelto come tipo Documento Informatico o Amministrativo Informatico
      \end{itemize}
      \item \textbf{Postcondizioni}:  Il filtro aggiunto dall'Utente è parametro per la ricerca e per la generazione del risultato.
      \item \textbf{Flusso Principale}:
            \begin{enumerate}
                  \item L'Utente inserisce l'Identificativo del Documento Primario.
            \end{enumerate}
\end{itemize}

\deepusecase{addTracciatureModificheDocumento}{Aggiunta filtro Tracciature Modifiche di Documento}
\begin{itemize}
      \item \textbf{Attore Primario}: Utente
      \item \textbf{Precondizioni}: 
      \begin{itemize}
            \item L'Utente ha avviato l'applicazione
            \item Il sistema ha indicizzato i documenti presenti nel DIP
            \item L'Utente ha scelto come tipo Documento Informatico o Amministrativo Informatico
      \end{itemize}
      \item \textbf{Postcondizioni}:  Il filtro aggiunto dall'Utente è parametro per la ricerca e per la generazione del risultato.
      \item \textbf{Flusso Principale}:
            \begin{enumerate}
                  \item L'Utente sceglie il Tipo di modifica tra:
                        \begin{itemize}
                              \item Annullamento
                              \item Registrazione
                              \item Integrazione
                              \item Annotazione
                        \end{itemize}
                  \item L'Utente sceglie il Soggetto che ha effettuato la modifica.
                  \item L'Utente inserisce la Data/Ora della Modifica.
                  \item L'Utente inserisce l'identificativo documento versione precedente.
            \end{enumerate}
\end{itemize}

\subsubusecase{aggiuntaFiltriAggregazione}{Aggiunta filtri per Aggregazione Documentale Informatica}
\begin{itemize}
      \item \textbf{Attore Primario}: Utente
      \item \textbf{Precondizioni}: 
      \begin{itemize}
            \item L'Utente ha avviato l'applicazione
            \item Il sistema ha indicizzato i documenti presenti nel DIP
            \item L'Utente sta effettuando una ricerca per campi specifici al tipo Aggregazione
      \end{itemize}
      \item \textbf{Postcondizioni}:  Il filtro aggiunto dall'Utente è parametro per la ricerca e per la generazione del risultato.
      \item \textbf{Flusso Principale}:
            \begin{enumerate}
                  \item L'Utente può aggiungere i filtri specifici per il tipo Aggregazione $\rightarrow$ Vedi negli UC da \hyperref[addTipoAggregazione]{[UC-7.2.2.\ref{addTipoAggregazione}]} a \hyperref[addProgressivo]{[UC-7.2.2.\ref{addProgressivo}]}
            \end{enumerate}
\end{itemize}

\deepusecase{addTipoAggregazione}{Aggiunta filtro Tipo di Aggregazione}
\begin{itemize}
      \item \textbf{Attore Primario}: Utente
      \item \textbf{Precondizioni}: 
      \begin{itemize}
            \item L'Utente ha avviato l'applicazione
            \item Il sistema ha indicizzato i documenti presenti nel DIP
            \item L'Utente sta effettuando una ricerca per campi specifici al tipo Aggregazione
      \end{itemize}
      \item \textbf{Postcondizioni}:  Il filtro aggiunto dall'Utente è parametro per la ricerca e per la generazione del risultato.
      \item \textbf{Flusso Principale}:
            \begin{enumerate}
                  \item L'Utente sceglie tra:\begin{itemize}
                              \item Fascicolo,
                              \item Serie Documentale,
                              \item Serie di Fascicoli.
                        \end{itemize}
            \end{enumerate}
\end{itemize}

\deepusecase{addIdAggregazione}{Aggiunta filtro Id dell'Aggregazione}
\begin{itemize}
      \item \textbf{Attore Primario}: Utente
      \item \textbf{Precondizioni}: 
      \begin{itemize}
            \item L'Utente ha avviato l'applicazione
            \item Il sistema ha indicizzato i documenti presenti nel DIP
            \item L'Utente sta effettuando una ricerca per campi specifici al tipo Aggregazione
      \end{itemize}
      \item \textbf{Postcondizioni}:  Il filtro aggiunto dall'Utente è parametro per la ricerca e per la generazione del risultato.
      \item \textbf{Flusso Principale}:
            \begin{enumerate}
                  \item L'Utente inserisce l'Id dell'Aggregazione.
            \end{enumerate}
\end{itemize}

\deepusecase{addTipologiaFascicolo}{Aggiunta filtro Tipologia di Fascicolo}
\begin{itemize}
      \item \textbf{Attore Primario}: Utente
      \item \textbf{Precondizioni}: 
      \begin{itemize}
            \item L'Utente ha avviato l'applicazione
            \item Il sistema ha indicizzato i documenti presenti nel DIP
            \item L'Utente sta effettuando una ricerca per campi specifici al tipo Aggregazione
      \end{itemize}
      \item \textbf{Postcondizioni}:  Il filtro aggiunto dall'Utente è parametro per la ricerca e per la generazione del risultato.
      \item \textbf{Flusso Principale}:
            \begin{enumerate}
                  \item L'Utente sceglie tra: \begin{itemize}
                              \item Affare,
                              \item Attività,
                              \item Persona Fisica,
                              \item Persona Giuridica,
                              \item  Procedimento Amministrativo.
                        \end{itemize}
            \end{enumerate}
\end{itemize}

\deepusecase{addIdAggregazionePrimario}{Aggiunta filtro Id Aggregazione Primario}
\begin{itemize}
      \item \textbf{Attore Primario}: Utente
      \item \textbf{Precondizioni}: 
      \begin{itemize}
            \item L'Utente ha avviato l'applicazione
            \item Il sistema ha indicizzato i documenti presenti nel DIP
            \item L'Utente sta effettuando una ricerca per campi specifici al tipo Aggregazione
      \end{itemize}
      \item \textbf{Postcondizioni}:  Il filtro aggiunto dall'Utente è parametro per la ricerca e per la generazione del risultato.
      \item \textbf{Flusso Principale}:
            \begin{enumerate}
                  \item L'Utente inserisce l'Id dell'Aggregazione Primario.
            \end{enumerate}
\end{itemize}

\deepusecase{addDataApertura}{Aggiunta filtro Data Apertura}
\begin{itemize}
      \item \textbf{Attore Primario}: Utente
      \item \textbf{Precondizioni}: 
      \begin{itemize}
            \item L'Utente ha avviato l'applicazione
            \item Il sistema ha indicizzato i documenti presenti nel DIP
            \item L'Utente sta effettuando una ricerca per campi specifici al tipo Aggregazione
      \end{itemize}
      \item \textbf{Postcondizioni}:  Il filtro aggiunto dall'Utente è parametro per la ricerca e per la generazione del risultato.
      \item \textbf{Flusso Principale}:
            \begin{enumerate}
                  \item L'Utente inserisce la Data di Apertura.
            \end{enumerate}
\end{itemize}

\deepusecase{addDataChiusura}{Aggiunta filtro Data Chiusura}
\begin{itemize}
      \item \textbf{Attore Primario}: Utente
      \item \textbf{Precondizioni}: 
      \begin{itemize}
            \item L'Utente ha avviato l'applicazione
            \item Il sistema ha indicizzato i documenti presenti nel DIP
            \item L'Utente sta effettuando una ricerca per campi specifici al tipo Aggregazione
      \end{itemize}
      \item \textbf{Postcondizioni}:  Il filtro aggiunto dall'Utente è parametro per la ricerca e per la generazione del risultato.
      \item \textbf{Flusso Principale}:
            \begin{enumerate}
                  \item L'Utente inserisce la Data di Chiusura.
            \end{enumerate}
\end{itemize}

\deepusecase{addProcedimentoAmministrativo}{Aggiunta filtro Procedimento Amministrativo}
\begin{itemize}
      \item \textbf{Attore Primario}: Utente
      \item \textbf{Precondizioni}: 
      \begin{itemize}
            \item L'Utente ha avviato l'applicazione
            \item Il sistema ha indicizzato i documenti presenti nel DIP
            \item L'Utente sta effettuando una ricerca per campi specifici al tipo Aggregazione
      \end{itemize}
      \item \textbf{Postcondizioni}:  Il filtro aggiunto dall'Utente è parametro per la ricerca e per la generazione del risultato.
      \item \textbf{Flusso Principale}:
            \begin{enumerate}
                  \item L'Utente inserisce la Materia, Argomento e Struttura per i procedimenti.
                  \item L'Utente inserisce il Procedimento come denominazione.
                  \item L'Utente inserisce il Catalogo dei Procedimenti come URI di pubblicazione del
                        catalogo.
                  \item L'Utente può aggiungere una o più Fasi come filtro $\rightarrow$ Vedi \hyperref[addFasiProcedimentoAmministrativo]{[UC-7.2.2.7.\ref{addFasiProcedimentoAmministrativo}]}\end{enumerate}
\end{itemize}

\subdeepusecase{addFasiProcedimentoAmministrativo}{Aggiunta filtro Fasi del Procedimento Amministrativo}
\begin{itemize}
      \item \textbf{Attore Primario}: Utente
      \item \textbf{Precondizioni}: 
      \begin{itemize}
            \item L'Utente ha avviato l'applicazione
            \item Il sistema ha indicizzato i documenti presenti nel DIP
            \item L'Utente sta effettuando una ricerca per campi specifici al tipo Aggregazione
            \item L'Utente ha aggiunto un Procedimento come parametro per la ricerca
      \end{itemize}
      \item \textbf{Postcondizioni}:  Il filtro aggiunto dall'Utente è parametro per la ricerca e per la generazione del risultato.
      \item \textbf{Flusso Principale}:
            \begin{enumerate}
                  \item L'Utente aggiunge un filtro per "Fase".
                  \item L'Utente inserisce i dati della fase:
                        \begin{itemize}
                              \item Tipo Fase (Preparatoria, Istruttoria, Consultiva, Decisoria o deliberativa,
                                    Integrazione dell'efficacia).
                              \item Data di inizio.
                              \item Data di fine.
                        \end{itemize}
            \end{enumerate}
\end{itemize}

\deepusecase{addAssegnazione}{Aggiunta filtro Assegnazione}
\begin{itemize}
      \item \textbf{Attore Primario}: Utente
      \item \textbf{Precondizioni}: 
      \begin{itemize}
            \item L'Utente ha avviato l'applicazione
            \item Il sistema ha indicizzato i documenti presenti nel DIP
            \item L'Utente sta effettuando una ricerca per campi specifici al tipo Aggregazione
      \end{itemize}
      \item \textbf{Postcondizioni}:  Il filtro aggiunto dall'Utente è parametro per la ricerca e per la generazione del risultato.
      \item \textbf{Flusso Principale}:
            \begin{enumerate}
                  \item L'Utente aggiunge uno o più filtri per "Assegnazione".
                  \item L'Utente inserisce i dati:
                        \begin{itemize}
                              \item Tipo Assegnazione.
                              \item Soggetto Assegnatario $\rightarrow$ Vedi \hyperref[addSoggetti]{[UC-7.1.\ref{addSoggetti}]}
                              \item Data di Inizio.
                              \item Data di Fine.
                        \end{itemize}
            \end{enumerate}
\end{itemize}

\deepusecase{addProgressivo}{Aggiunta filtro Progressivo}
\begin{itemize}
      \item \textbf{Attore Primario}: Utente
      \item \textbf{Precondizioni}: 
      \begin{itemize}
            \item L'Utente ha avviato l'applicazione
            \item Il sistema ha indicizzato i documenti presenti nel DIP
            \item L'Utente sta effettuando una ricerca per campi specifici al tipo Aggregazione
      \end{itemize}
      \item \textbf{Postcondizioni}:  Il filtro aggiunto dall'Utente è parametro per la ricerca e per la generazione del risultato.
      \item \textbf{Flusso Principale}:
            \begin{enumerate}
                  \item L'Utente inserisce il numero Progressivo dell'Aggregazione.
            \end{enumerate}
\end{itemize}

\subusecase{addDettagliSoggetto}{Aggiunta dettagli per il ruolo di un soggetto}
\begin{itemize}
      \item \textbf{Attore Primario}: Utente
      \item \textbf{Precondizioni}: 
      \begin{itemize}
            \item L'Utente ha avviato l'applicazione
            \item Il sistema ha indicizzato i documenti presenti nel DIP
            \item L'Utente ha selezionato il filtro Soggetti tra i campi comuni
            \item L'Utente ha scelto tipo di documento
            \item L'Utente ha scelto un ruolo come campo di ricerca $\rightarrow$ Vedi negli UC da \hyperref[addSoggettiDocInfo]{[UC-\ref{addSoggettiDocInfo}]} a \hyperref[addSoggettiAgg]{[UC-7.1.6.\ref{addSoggettiAgg}]}
      \end{itemize}
      \item \textbf{Postcondizioni}:  Il filtro aggiunto dall'Utente è parametro per la ricerca e per la generazione del risultato.
      \item \textbf{Flusso Principale}:
            \begin{enumerate}
                  \item L'Utente inserisce i dettagli del ruolo.
            \end{enumerate}
\end{itemize}

\subsubusecase{addDettagliPAI}{Aggiunta dettagli Ruolo= PAI}
\begin{itemize}
      \item \textbf{Attore Primario}: Utente
      \item \textbf{Precondizioni}: 
      \begin{itemize}
            \item L'Utente ha avviato l'applicazione
            \item Il sistema ha indicizzato i documenti presenti nel DIP
            \item L'Utente ha selezionato il filtro Soggetti tra i campi comuni
            \item L'Utente ha scelto tipo di documento
            \item L'Utente ha scelto un ruolo come campo di ricerca $\rightarrow$ Vedi negli UC da \hyperref[addSoggettiDocInfo]{[UC-\ref{addSoggettiDocInfo}]} a \hyperref[addSoggettiAgg]{[UC-7.1.6.\ref{addSoggettiAgg}]}
            \item L'Utente ha selezionato il ruolo PAI
      \end{itemize}
      \item \textbf{Postcondizioni}:  Il filtro aggiunto dall'Utente è parametro per la ricerca e per la generazione del risultato.
      \item \textbf{Flusso Principale}:
            \begin{enumerate}
                  \item L'Utente inserisce i valori per i campi:
                        \begin{itemize}
                              \item Denominazione Amministrazione/ Codice IPA
                              \item Denominazione Amministrazione AOO/ Codice IPA AOO
                              \item Denominazione Amministrazione UOR/ Codice IPA UOR
                              \item Indirizzi Digitali di Riferimento
                        \end{itemize}
            \end{enumerate}
\end{itemize}

\subsubusecase{addDettagliPAE}{Aggiunta dettagli Ruolo= PAE}
\begin{itemize}
      \item \textbf{Attore Primario}: Utente
      \item \textbf{Precondizioni}: \begin{itemize}
            \item L'Utente ha avviato l'applicazione
            \item Il sistema ha indicizzato i documenti presenti nel DIP
            \item L'Utente ha selezionato il filtro Soggetti tra i campi comuni
            \item L'Utente ha scelto tipo di documento
            \item L'Utente ha scelto un ruolo come campo di ricerca $\rightarrow$ Vedi negli UC da \hyperref[addSoggettiDocInfo]{[UC-\ref{addSoggettiDocInfo}]} a \hyperref[addSoggettiAgg]{[UC-7.1.6.\ref{addSoggettiAgg}]}
            \item L'Utente ha selezionato il ruolo PAE
      \end{itemize}
      \item \textbf{Postcondizioni}:  Il filtro aggiunto dall'Utente è parametro per la ricerca e per la generazione del risultato.
      \item \textbf{Flusso Principale}:
            \begin{enumerate}
                  \item L'Utente inserisce i valori per i campi:
                        \begin{itemize}
                              \item Denominazione Amministrazione
                              \item Denominazione Ufficio
                              \item Indirizzi Digitali di Riferimento
                        \end{itemize}
            \end{enumerate}
\end{itemize}

\subsubusecase{addDettagliAS}{Aggiunta dettagli Ruolo= AS}
\begin{itemize}
      \item \textbf{Attore Primario}: Utente
      \item \textbf{Precondizioni}: \begin{itemize}
            \item L'Utente ha avviato l'applicazione
            \item Il sistema ha indicizzato i documenti presenti nel DIP
            \item L'Utente ha selezionato il filtro Soggetti tra i campi comuni
            \item L'Utente ha scelto tipo di documento
            \item L'Utente ha scelto un ruolo come campo di ricerca $\rightarrow$ Vedi negli UC da \hyperref[addSoggettiDocInfo]{[UC-\ref{addSoggettiDocInfo}]} a \hyperref[addSoggettiAgg]{[UC-7.1.6.\ref{addSoggettiAgg}]}
            \item L'Utente ha selezionato il ruolo AS
      \end{itemize}
      \item \textbf{Postcondizioni}:  Il filtro aggiunto dall'Utente è parametro per la ricerca e per la generazione del risultato.
      \item \textbf{Flusso Principale}:
            \begin{enumerate}
                  \item L'Utente inserisce i valori per i campi:
                        \begin{itemize}
                              \item Cognome
                              \item Nome
                              \item Codice Fiscale
                              \item Denominazione Amministrazione/ Codice IPA
                              \item Denominazione Amministrazione AOO/ Codice IPA AOO
                              \item Denominazione Amministrazione UOR/ Codice IPA UOR
                              \item Indirizzi Digitali di Riferimento
                        \end{itemize}
            \end{enumerate}
\end{itemize}

\subsubusecase{addDettagliPG}{Aggiunta dettagli Ruolo= PG}
\begin{itemize}
      \item \textbf{Attore Primario}: Utente
      \item \textbf{Precondizioni}: \begin{itemize}
            \item L'Utente ha avviato l'applicazione
            \item Il sistema ha indicizzato i documenti presenti nel DIP
            \item L'Utente ha selezionato il filtro Soggetti tra i campi comuni
            \item L'Utente ha scelto tipo di documento
            \item L'Utente ha scelto un ruolo come campo di ricerca $\rightarrow$ Vedi negli UC da \hyperref[addSoggettiDocInfo]{[UC-\ref{addSoggettiDocInfo}]} a \hyperref[addSoggettiAgg]{[UC-7.1.6.\ref{addSoggettiAgg}]}
            \item L'Utente ha selezionato il ruolo PG
      \end{itemize}
      \item \textbf{Postcondizioni}:  Il filtro aggiunto dall'Utente è parametro per la ricerca e per la generazione del risultato.
      \item \textbf{Flusso Principale}:
            \begin{enumerate}
                  \item L'Utente inserisce i valori per i campi:
                        \begin{itemize}
                              \item Denominazione Organizzazione
                              \item Codice fiscale / Partita Iva
                              \item Denominazione Ufficio
                              \item Indirizzi Digitali di Riferimento
                        \end{itemize}
            \end{enumerate}
\end{itemize}

\subsubusecase{addDettagliPF}{Aggiunta dettagli Ruolo= PF}
\begin{itemize}
      \item \textbf{Attore Primario}: Utente
      \item \textbf{Precondizioni}: \begin{itemize}
            \item L'Utente ha avviato l'applicazione
            \item Il sistema ha indicizzato i documenti presenti nel DIP
            \item L'Utente ha selezionato il filtro Soggetti tra i campi comuni
            \item L'Utente ha scelto tipo di documento
            \item L'Utente ha scelto un ruolo come campo di ricerca $\rightarrow$ Vedi negli UC da \hyperref[addSoggettiDocInfo]{[UC-\ref{addSoggettiDocInfo}]} a \hyperref[addSoggettiAgg]{[UC-7.1.6.\ref{addSoggettiAgg}]}
            \item L'Utente ha selezionato il ruolo PF
      \end{itemize}
      \item \textbf{Postcondizioni}:  Il filtro aggiunto dall'Utente è parametro per la ricerca e per la generazione del risultato.
      \item \textbf{Flusso Principale}:
            \begin{enumerate}
                  \item  L'Utente inserisce i valori per i campi:
                        \begin{itemize}
                              \item Cognome
                              \item Nome
                              \item Indirizzi Digitali di Riferimento
                        \end{itemize}
            \end{enumerate}
\end{itemize}

\subsubusecase{addDettagliRUP}{Aggiunta dettagli Ruolo= RUP}
\begin{itemize}
      \item \textbf{Attore Primario}: Utente
      \item \textbf{Precondizioni}: \begin{itemize}
            \item L'Utente ha avviato l'applicazione
            \item Il sistema ha indicizzato i documenti presenti nel DIP
            \item L'Utente ha selezionato il filtro Soggetti tra i campi comuni
            \item L'Utente ha scelto tipo di documento
            \item L'Utente ha scelto un ruolo come campo di ricerca $\rightarrow$ Vedi negli UC da \hyperref[addSoggettiDocInfo]{[UC-\ref{addSoggettiDocInfo}]} a \hyperref[addSoggettiAgg]{[UC-7.1.6.\ref{addSoggettiAgg}]}
            \item L'Utente ha selezionato il ruolo RUP
      \end{itemize}
      \item \textbf{Postcondizioni}:  Il filtro aggiunto dall'Utente è parametro per la ricerca e per la generazione del risultato.
      \item \textbf{Flusso Principale}:
            \begin{enumerate}
                  \item L'Utente inserisce i valori per i campi:
                        \begin{itemize}
                              \item Cognome
                              \item Nome
                              \item Denominazione Amministrazione/ Codice IPA
                              \item Denominazione Amministrazione AOO/Denominazione Amministrazione UOR
                              \item  Codice IPA AOO/Codice IPA UOR
                              \item Indirizzi Digitali di Riferimento
                        \end{itemize}
            \end{enumerate}
\end{itemize}

\subsubusecase{addDettagliSW}{Aggiunta dettagli Ruolo= SW}
\begin{itemize}
      \item \textbf{Attore Primario}: Utente
      \item \textbf{Precondizioni}: \begin{itemize}
            \item L'Utente ha avviato l'applicazione
            \item Il sistema ha indicizzato i documenti presenti nel DIP
            \item L'Utente ha selezionato il filtro Soggetti tra i campi comuni
            \item L'Utente ha scelto tipo di documento
            \item L'Utente ha scelto un ruolo come campo di ricerca $\rightarrow$ Vedi negli UC da \hyperref[addSoggettiDocInfo]{[UC-\ref{addSoggettiDocInfo}]} a \hyperref[addSoggettiAgg]{[UC-7.1.6.\ref{addSoggettiAgg}]}
            \item L'Utente ha selezionato il ruolo SW
      \end{itemize}
      \item \textbf{Postcondizioni}:  Il filtro aggiunto dall'Utente è parametro per la ricerca e per la generazione del risultato.
      \item \textbf{Flusso Principale}:
            \begin{enumerate}
                  \item L'Utente inserisce la Denominazione Sistema
                  \item Il sistema usa questi valori per filtrare la ricerca.
            \end{enumerate}
\end{itemize}

\subusecase{addCustomMetadata}{Aggiunta filtri per Custom Metadata}
\begin{itemize}
      \item \textbf{Attore Primario}: Utente
      \item \textbf{Precondizioni}: \begin{itemize} \item L'Utente ha avviato l'applicazione
      \item Il sistema ha indicizzato i documenti presenti nel DIP
      \item Sono presenti custom metadata descritti nei file schemas
      \end{itemize}
      \item \textbf{Postcondizioni}:  Il filtro aggiunto dall'Utente è parametro per la ricerca e per la generazione del risultato.
      \item \textbf{Flusso Principale}:
            \begin{enumerate}
                  \item L'Utente per ogni metadato custom, inserisce il valore per cui ricerca.
                  \item Il sistema usa questi valori per filtrare la ricerca.
            \end{enumerate}
      \item \textbf{Nota}: Per costruzione aziendale i custom metadata sono tutti stringhe e quindi l'inserimento e la validazione seguono le regole dei campi di testo liberi.
\end{itemize}

\usecase{visualizzaRisultati}{Visualizzazione Risultati di Ricerca} \label{risultatiRicerca}
\begin{itemize}
      \item \textbf{Attore Primario}: Utente
      \item \textbf{Precondizioni}: \begin{itemize}
            \item L'Utente ha avviato l'applicazione
            \item Il sistema ha indicizzato i documenti presenti nel DIP
            \item L'Utente ha eseguito una ricerca $\rightarrow$ Vedi \hyperref[ricercaDIP]{[UC-\ref{ricercaDIP}]} e sue generalizzazioni
      \end{itemize}
      \item \textbf{Postcondizioni}: I risultati della ricerca sono presentati all'Utente come lista di elementi.
      \item \textbf{Flusso Principale}:
            \begin{enumerate}
                  \item L'Utente visualizza un elenco di documenti o aggregazioni che corrispondono
                        ai criteri di ricerca.
                  \item Per ogni elemento, vengono mostrate le informazioni principali (Nome, Data,
                        Tipo).
                  \item L'Utente può visualizzare ulteriori informazioni e funzioni selezionando uno dei documenti $\rightarrow$ Vedi \hyperref[informazioniElemento]{[UC-1.\ref{informazioniElemento}]}
            \end{enumerate}
      \item \textbf{Flussi Alternativi}:
            \begin{itemize}
                  \item Il sistema comunica che la ricerca non ha prodotto alcun risultato $\rightarrow$ Vedi \hyperref[nessunRisultato]{[UC-\ref{nessunRisultato}]}
            \end{itemize}
\end{itemize}

\usecase{nessunRisultato}{Nessun Risultato}
\begin{itemize}
      \item \textbf{Attore Primario}: Utente
      \item \textbf{Precondizioni}: \begin{itemize}
            \item L'Utente ha avviato l'applicazione
            \item Il sistema ha indicizzato i documenti presenti nel DIP
            \item L'Utente ha eseguito una ricerca $\rightarrow$ Vedi \hyperref[ricercaDIP]{[UC-\ref{ricercaDIP}]} e sue specializzazioni
      \end{itemize}
      \item \textbf{Postcondizioni}: Il sistema informa l'Utente che non sono stati trovati risultati.
      \item \textbf{Flusso Principale}:
            \begin{enumerate}
                  \item Il sistema visualizza un messaggio che indica che la ricerca non ha prodotto
                        alcun risultato.
                  \item Viene mostrato l'insieme dei filtri che erano stati applicati, in modo da
                        permettere all'Utente di comprendere eventuali errori di inserimento o
                        incoerenze.
            \end{enumerate}
\end{itemize}

\usecase{validazioneCampi}{Validazione campi}
\begin{itemize}
      \item \textbf{Attore Primario}: Sistema
      \item \textbf{Precondizioni}: 
      \begin{itemize}
            \item L'Utente ha avviato l'applicazione
            \item Il sistema ha indicizzato i documenti presenti nel DIP
            \item L'Utente ha compilato dei campi di ricerca che richiedono un controllo di validità.
      \end{itemize}
      \item \textbf{Postcondizioni}: Il campo considerato valido viene aggiunto ai parametri di ricerca.
      \item \textbf{Flusso Principale}:
            \begin{enumerate}
                  \item Il sistema valida i campi che necessitano di controlli (UC-)
                  \item Se rispettano le convenzioni il filtro viene aggiunto ai parametri di ricerca.
            \end{enumerate}
      \item \textbf{Flusso Alternativo}:
            \begin{itemize}
                  \item Il sistema identifica dei problemi con i valori inseriti per i campi.
                  \item Il sistema segnala il problema e richiede la correzione da parte dell'Utente, non inserendo il campo considerato tra i parametri di ricerca (UC-)
            \end{itemize}
\end{itemize}

\subusecase{validazioneTestoLibero}{Validazione campi di testo libero}
\begin{itemize}
      \item \textbf{Attore Primario}: Sistema
      \item \textbf{Precondizioni}: \begin{itemize}
            \item L'Utente ha avviato l'applicazione
            \item Il sistema ha indicizzato i documenti presenti nel DIP
            \item L'Utente ha compilato dei campi di ricerca testuale.
      \end{itemize}
      \item \textbf{Postcondizioni}: Il campo considerato valido viene aggiunto ai parametri di ricerca.
      \item \textbf{Flusso Principale}:
            \begin{enumerate}
                  \item Il sistema controlla la presenza di tentativi di Code Injection e Sql Injection.
            \end{enumerate}
      \item \textbf{Flusso Alternativo}:
            \begin{itemize}
                  \item Il sistema identifica dei problemi con i valori inseriti per i campi.
                  \item Il sistema segnala il problema e richiede la correzione da parte dell'Utente, non inserendo il campo considerato tra i parametri di ricerca (UC-)
            \end{itemize}
\end{itemize}


\subusecase{validazioneIdentificativiIPA}{Validazione campi di identificativi IPA}
\begin{itemize}
      \item \textbf{Attore Primario}: Sistema
      \item \textbf{Precondizioni}: 
      \begin{itemize}
            \item L'Utente ha avviato l'applicazione
            \item Il sistema ha indicizzato i documenti presenti nel DIP
            \item L'Utente ha compilato dei campi di codici identificativi IPA che richiedono un controllo di validità.
      \end{itemize}
      \item \textbf{Postcondizioni}: Il campo considerato valido viene aggiunto ai parametri di ricerca.
      \item \textbf{Flusso Principale}:
            \begin{enumerate}
                  \item Il sistema controlla che il codice rispetti il formato standard di 6 Cifre per Codici IPA
            \end{enumerate}
      \item \textbf{Flusso Alternativo}:
            \begin{itemize}
                  \item Il sistema identifica dei problemi con i valori inseriti per i campi.
                  \item Il sistema segnala il problema e richiede la correzione da parte dell'Utente, non inserendo il campo considerato tra i parametri di ricerca (UC-)
            \end{itemize}
\end{itemize}

\subusecase{validazioneCodiciFiscali}{Validazione campi di codici fiscali}
\begin{itemize}
      \item \textbf{Attore Primario}: Sistema
      \item \textbf{Precondizioni}: 
      \begin{itemize}
            \item L'Utente ha avviato l'applicazione
            \item Il sistema ha indicizzato i documenti presenti nel DIP
            \item L'Utente ha compilato dei campi di codici fiscali che richiedono un controllo di validità.
      \end{itemize}
      \item \textbf{Postcondizioni}: Il campo considerato valido viene aggiunto ai parametri di ricerca.
      \item \textbf{Flusso Principale}:
            \begin{enumerate}
                  \item Il sistema controlla che il codice rispetti il formato standard per il tipo di codice fiscale:
                        \begin{verbatim}^[A-Z]{6}[0-9LMNPQRSTUV]{2}[ABCDEHLMPRST][0-9LMNPQRSTUV]{2}[A-Z][0-9LMNPQRSTUV]{3}[A-Z]$\end{verbatim}
            \end{enumerate}
      \item \textbf{Flusso Alternativo}:
            \begin{itemize}
                  \item Il sistema identifica dei problemi con i valori inseriti per i campi.
                  \item Il sistema segnala il problema e richiede la correzione da parte dell'Utente, non inserendo il campo considerato tra i parametri di ricerca (UC-)
            \end{itemize}
\end{itemize}

\subusecase{validazioneSoggetti}{Validazione campi di soggetti}
\begin{itemize}
      \item \textbf{Attore Primario}: Sistema
      \item \textbf{Precondizioni}: 
      \begin{itemize}
            \item L'Utente ha avviato l'applicazione
            \item Il sistema ha indicizzato i documenti presenti nel DIP
            \item L'Utente ha selezionato il filtro Soggetti tra i campi comuni
            \item L'Utente ha compilato i campi di Soggetti che richiedono un controllo di validità
      \end{itemize}
      \item \textbf{Postcondizioni}: Il campo considerato valido viene aggiunto ai parametri di ricerca.
      \item \textbf{Flusso Principale}:
            \begin{enumerate}
                  \item Il sistema controlla che non vengano inseriti più soggetti con lo stesso ruolo (da chiedere a sm)
            \end{enumerate}
      \item \textbf{Flusso Alternativo}:
            \begin{itemize}
                  \item Il sistema identifica dei problemi con i valori inseriti per i campi.
                  \item Il sistema segnala il problema e richiede la correzione da parte dell'Utente, non inserendo il campo considerato tra i parametri di ricerca (UC-)
            \end{itemize}
\end{itemize}

\subusecase{validazioneEmail}{Validazione campi di email}
\begin{itemize}
      \item \textbf{Attore Primario}: Sistema
      \item \textbf{Precondizioni}: 
       \begin{itemize}
            \item L'Utente ha avviato l'applicazione
            \item Il sistema ha indicizzato i documenti presenti nel DIP
            \item L'Utente ha compilato dei campi di indirizzi elettronici che richiedono un controllo di validità
      \end{itemize}
      \item \textbf{Postcondizioni}: Il campo considerato valido viene aggiunto ai parametri di ricerca.
      \item \textbf{Flusso Principale}:
            \begin{enumerate}
                  \item Il sistema controlla che il formato mail sia valido \begin{verbatim} /^[\w\-\.]*[\w\.]\@[\w\.]*[\w\-\.]+[\w\-]+[\w]\.+[\w]+[\w $]/ \end{verbatim}
            \end{enumerate}
      \item \textbf{Flusso Alternativo}:
            \begin{itemize}
                  \item Il sistema identifica dei problemi con i valori inseriti per i campi.
                  \item Il sistema segnala il problema e richiede la correzione da parte dell'Utente, non inserendo il campo considerato tra i parametri di ricerca (UC-)
            \end{itemize}
\end{itemize}

\subusecase{validazionePartitaIVA}{Validazione campi di partita IVA}
\begin{itemize}
      \item \textbf{Attore Primario}: Sistema
      \item \textbf{Precondizioni}: 
      \begin{itemize}
            \item L'Utente ha avviato l'applicazione
            \item Il sistema ha indicizzato i documenti presenti nel DIP
            \item L'Utente ha compilato dei campi di partita IVA che richiedono un controllo di validità
      \end{itemize}
      \item \textbf{Postcondizioni}: Il campo considerato valido viene aggiunto ai parametri di ricerca.
      \item \textbf{Flusso Principale}:
            \begin{enumerate}
                  \item Il sistema controlla che il formato delle partite IVA sia valido \begin{verbatim}/^([1-7]\d{10})$/\end{verbatim}
            \end{enumerate}
      \item \textbf{Flusso Alternativo}:
            \begin{itemize}
                  \item Il sistema identifica dei problemi con i valori inseriti per i campi.
                  \item Il sistema segnala il problema e richiede la correzione da parte dell'Utente, non inserendo il campo considerato tra i parametri di ricerca (UC-)
            \end{itemize}
\end{itemize}