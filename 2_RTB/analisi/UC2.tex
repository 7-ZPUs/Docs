\usecase{ricercaDIP}{Ricerca nel DIP}
\begin{itemize}
      \item \textbf{Attore Primario}: Utente
      \item \textbf{Precondizioni}: 
      \begin{itemize} 
            \item L'Utente ha avviato l'applicazione
            \item Il sistema ha indicizzato i documenti presenti nel DIP
      \end{itemize}
      \item \textbf{Postcondizioni}:
            \begin{itemize}
                  \item Il sistema produce i risultati di ricerca in base ai criteri specificati
            \end{itemize}
      \item \textbf{Flusso Principale}
      \begin{enumerate}
            \item L'utente specifica i criteri di ricerca
            \item Il sistema produce i risultati di ricerca in base ai criteri specificati
            \item L'Utente visualizza i risultati di ricerca (\ref{visualizzaRisultati})
      \end{enumerate}
      \item \textbf{Flusso Alternativo}:
      \begin{itemize}
            \item La ricerca non produce risultati e il sistema informa l'Utente (\ref{nessunRisultato})
      \end{itemize}
      \item \textbf{Inclusioni}: \ref{visualizzaRisultati} Visualizzazione Risultati di Ricerca
      \item \textbf{Estensioni}: \ref{nessunRisultato} Nessun risultato
\end{itemize}

\subusecase{ricercaDIPConFiltri}{Ricerca nel DIP con filtri}
\begin{itemize}
      \item \textbf{Attore Primario}: Utente
      \item \textbf{Precondizioni}: 
      \begin{itemize} 
            \item L'Utente ha avviato l'applicazione
            \item Il sistema ha indicizzato i documenti presenti nel DIP
      \end{itemize}
      \item \textbf{Postcondizioni}:
            \begin{itemize}
                  \item Il sistema produce i risultati di ricerca in base ai filtri specificati
            \end{itemize}
      \item \textbf{Flusso Principale}:
            \begin{enumerate}
                  \item L'Utente aggiunge filtri come parametri per la ricerca  (\ref{specificaFiltriRicercaDocumento})
                  \item Vengono controllati i valori dei filtri di ricerca
                  \item Il sistema produce i risultati di ricerca in base ai filtri specificati
            \end{enumerate}
      \item \textbf{Flusso alternativo}: 
            \begin{enumerate}
                  \item Uno o più campi non rispettano i formati previsti (\ref{campoNonValido})
            \end{enumerate}
      \item \textbf{Inclusioni}: \ref{specificaFiltriRicercaDocumento} Specifica filtri per Ricerca Documento 
      \item \textbf{Estensioni}: \ref{campoNonValido} Uno o più valori dei filtri di ricerca non validi
      \item \textbf{Specializza}: \ref{ricercaDIP} Ricerca nel DIP
\end{itemize}

\subusecase{ricercaDIPSemantica}{Ricerca nel DIP semantica}
\begin{itemize}
    \item \textbf{Attore Primario}: Utente
    \item \textbf{Precondizioni}:
    \begin{itemize}
      \item L'Utente ha avviato l'applicazione
      \item Il sistema ha indicizzato i documenti presenti nel DIP con l'engine semantico
    \end{itemize}
    \item \textbf{Postcondizioni}:
    \begin{itemize}
        \item Il sistema produce i risultati di ricerca semantica per la query specificata.
    \end{itemize}
    \item \textbf{Flusso principale}:
    \begin{enumerate}
        \item L'Utente specifica la query di ricerca
        \item Il sistema produce i risultati di ricerca per la query specificata
    \end{enumerate}
    \item \textbf{Specializza}: \ref{ricercaDIP} Ricerca nel DIP
    
\end{itemize}

\subusecase{ricercaClasse}{Ricerca Classe Documentale per Nome}
\begin{itemize}
      \item \textbf{Attore Primario}: Utente
      \item \textbf{Precondizioni}: 
      \begin{itemize} 
            \item L'Utente ha avviato l'applicazione
            \item Il sistema ha indicizzato i documenti presenti nel DIP
      \end{itemize}
      \item \textbf{Postcondizioni}:
            \begin{itemize}
                  \item Il sistema produce i risultati di ricerca in base al nome inserito
            \end{itemize}
      \item \textbf{Flusso Principale}:
            \begin{enumerate}
                  \item L'Utente inserisce il nome della Classe Documentale da ricercare (\ref{inserimentoNomeClasseDocumentale}).
                  \item Il sistema produce i risultati di ricerca in base al nome inserito
            \end{enumerate}

      \item \textbf{Inclusioni}: \ref{inserimentoNomeClasseDocumentale} Inserimento Nome Classe Documentale
      \item \textbf{Specializza}: \ref{ricercaDIP} Ricerca nel DIP
\end{itemize}

\subsubusecase{inserimentoNomeClasseDocumentale}{Inserimento Nome Classe Documentale}
\begin{itemize}
      \item \textbf{Attore Primario}: Utente
      \item \textbf{Precondizioni}:
      \begin{itemize}
            \item L'Utente ha avviato l'applicazione
            \item L'Utente sta effettuando una ricerca nel DIP per Classe Documentale
      \end{itemize}
      \item \textbf{Postcondizioni}: Il sistema riceve il nome della Classe Documentale che l'Utente vuole ricercare.
      \item \textbf{Flusso Principale}:
      \begin{enumerate}
            \item L'Utente inserisce il nome della Classe Documentale da ricercare (\ref{compilaValoreFiltro})
            \item Il sistema riceve il nome della Classe Documentale
      \end{enumerate}
      \item \textbf{Inclusioni}: \ref{compilaValoreFiltro} Compila Valore Filtro
\end{itemize}

\subusecase{ricercaProcesso}{Ricerca Processo per Id}
\begin{itemize}
      \item \textbf{Attore Primario}: Utente
      \item \textbf{Precondizioni}: \begin{itemize}
            \item L'Utente ha avviato l'applicazione
            \item Il sistema ha indicizzato i documenti presenti nel DIP
      \end{itemize}
      \item \textbf{Postcondizioni}: Il sistema produce i risultati di ricerca in base all'Id inserito
      \item \textbf{Flusso Principale}:
      \begin{enumerate}
            \item L'Utente inserisce l'Id del Processo (\ref{inserimentoIdProcesso})
            \item Il sistema produce i risultati di ricerca in base all'Id inserito
      \end{enumerate}
      \item \textbf{Inclusioni}: \ref{inserimentoIdProcesso} Inserimento Id Processo
      \item \textbf{Specializza}: \ref{ricercaDIP} Ricerca nel DIP
\end{itemize}

\subsubusecase{inserimentoIdProcesso}{Inserimento Id Processo}
\begin{itemize}
      \item \textbf{Attore Primario}: Utente
      \item \textbf{Precondizioni}:
      \begin{itemize}
            \item L'Utente ha avviato l'applicazione
            \item L'Utente sta effettuando una ricerca nel DIP per Processo
      \end{itemize}
      \item \textbf{Postcondizioni}:
      \begin{itemize}
            \item Il sistema riceve l'Id del Processo che l'Utente vuole ricercare
      \end{itemize}
      \item \textbf{Flusso Principale}:
      \begin{enumerate}
            \item L'Utente inserisce l'Id del Processo da ricercare (\ref{compilaValoreFiltro})
            \item Il sistema riceve l'Id del Processo
      \end{enumerate}
      \item \textbf{Inclusioni}: \ref{compilaValoreFiltro} Compila Valore Filtro
\end{itemize}

\usecase{campoNonValido}{Uno o più valori dei filtri di ricerca non validi}
\begin{itemize}
      \item \textbf{Attore Primario}: Utente
      \item \textbf{Precondizioni}: 
      \begin{itemize}
            \item L'utente ha avviato l'applicazione
            \item L'Utente sta effettuando una ricerca nel DIP con filtri
      \end{itemize}
      \item \textbf{Postcondizioni}: Il sistema annulla la ricerca mostrando un messaggio di errore che indica i campi non validi
      \item \textbf{Flusso Principale}:
      \begin{enumerate}
            \item Il sistema annulla l'operazione di ricerca
            \item Il sistema mostra a video un messaggio di errore che indica quali campi non rispettano i formati previsti
      \end{enumerate}
\end{itemize}

\usecase{indicizzazioneSemantica}{Indicizza DIP con engine semantico}
\begin{itemize}
    \item \textbf{Attore Primario}: Utente
    \item \textbf{Precondizioni}:
    \begin{itemize}
      \item L'Utente ha avviato l'applicazione
    \end{itemize}
    \item \textbf{Postcondizioni}:
    \begin{itemize}
        \item Il sistema ha indicizzato i documenti presenti nel DIP con l'engine semantico
    \end{itemize}
    \item \textbf{Flusso principale}:
    \begin{enumerate}
        \item L'Utente seleziona l'opzione di indicizzazione semantica
        \item Il sistema indicizza i documenti presenti nel DIP con l'engine semantico
    \end{enumerate}
    \item \textbf{Flusso alternativo}:
    \begin{itemize}
        \item L'Utente tenta di indicizzare il DIP con l'engine semantico
        \item L'indicizzazione semantica del DIP fallisce mostrando un messaggio di errore (\ref{erroreIndicizzazioneSemantica})
    \end{itemize}
    \item \textbf{Estensioni}: \ref{erroreIndicizzazioneSemantica} Errore Indicizzazione Semantica
\end{itemize}

\usecase{erroreIndicizzazioneSemantica}{Indicizzazione semantica fallita}
\begin{itemize}
      \item \textbf{Attore Primario}: Utente
      \item \textbf{Precondizioni}:
      \begin{itemize}
        \item L'Utente ha avviato l'applicazione
        \item L'Utente ha selezionato l'opzione di indicizzazione semantica
      \end{itemize}
      \item \textbf{Postcondizioni}:
      \begin{itemize}
          \item Il sistema annulla l'indicizzazione semantica del DIP, mostrando un messaggio di errore
      \end{itemize}
      \item \textbf{Flusso Principale}:
      \begin{enumerate}
          \item Il sistema annulla l'operazione di indicizzazione semantica
          \item Il sistema mostra a video un messaggio di errore che indica il motivo del fallimento
      \end{enumerate}
\end{itemize}

\usecase{StatoIndicizzazione}{Visualizza stato indicizzazione semantica}
\begin{itemize}
    \item \textbf{Attore Primario}: Utente
    \item \textbf{Precondizioni}:
    \begin{itemize}
      \item L'Utente ha avviato l'applicazione
      \item L'Utente ha selezionato l'opzione di indicizzazione semantica
    \end{itemize}
    \item \textbf{Postcondizioni}:
    \begin{itemize}
        \item L'Utente vede lo stato di indicizzazione per la ricerca semantica
    \end{itemize}
    \item \textbf{Flusso principale}:
    \begin{enumerate}
        \item L'Utente visualizza lo stato "In corso" o "Completata" dell'indicizzazione semantica
    \end{enumerate}
\end{itemize}

\usecase{specificaFiltriRicercaDocumento}{Specifica filtri per Ricerca Documento}
\begin{itemize}
      \item \textbf{Attore Primario}: Utente
      \item \textbf{Precondizioni}: 
      \begin{itemize}
            \item L'Utente ha avviato l'applicazione
            \item L'Utente sta effettuando una ricerca nel DIP con filtri
      \end{itemize}
      \item \textbf{Postcondizioni}: Vengono specificati i filtri per la ricerca dei documenti.
      \item \textbf{Flusso Principale}:
      \begin{enumerate}
            \item L'Utente specifica i filtri per la ricerca tra campi comuni (\ref{specificaFiltriComuni})
            \item L'Utente specifica i filtri per la ricerca tra campi specifici per tipo documentale (\ref{addFiltriTipoDocumento})
            \item L'Utente specifica i filtri Custom (\ref{addCustomMetadata})
      \end{enumerate}
      \item \textbf{Inclusioni}: 
      \begin{itemize}
            \item \ref{specificaFiltriComuni} Specifica filtri per Campi Comuni
            \item \ref{addFiltriTipoDocumento} Specifica filtri per Tipo di Documento
            \item \ref{addCustomMetadata} Specifica filtri per Custom Metadata
      \end{itemize}
\end{itemize}

\subusecase{specificaFiltriComuni}{Specifica filtri per Campi Comuni}
\begin{itemize}
      \item \textbf{Attore Primario}: Utente
      \item \textbf{Precondizioni}:\begin{itemize}
            \item L'Utente ha avviato l'applicazione
            \item L'Utente sta effettuando una ricerca nel DIP con filtri
      \end{itemize}
      \item \textbf{Postcondizioni}: Vengono specificati i filtri per i campi comuni per la ricerca dei documenti e utilizzati per la generazione del risultato.
      \item \textbf{Flusso Principale}:
      \begin{enumerate}
            \item L'Utente seleziona i campi comuni su cui basare la ricerca. (\ref{selezionaCampiRicercaComuni})
            \item L'Utente specifica i filtri per i campi selezionati (da \ref{addChiaveDescrittiva} a \ref{addSoggetti})
      \end{enumerate}
      \item \textbf{Inclusioni}: 
      \begin{itemize}
            \item \ref{selezionaCampiRicercaComuni} Seleziona campi di ricerca comuni
            \item \ref{addChiaveDescrittiva} Specifica filtro Chiave Descrittiva
            \item \ref{addClassificazione} Specifica filtro Classificazione
            \item \ref{addTempoConserva} Specifica filtro Tempo di Conservazione
            \item \ref{addNote} Specifica filtro Note
            \item \ref{addTipoDocumento} Specifica filtro Tipo documento
            \item \ref{addSoggetti} Specifica filtro soggetti
      \end{itemize}
\end{itemize}

\subsubusecase{selezionaCampiRicercaComuni}{Selezione campi di ricerca comuni}
\begin{itemize}
      \item \textbf{Attore Primario}: Utente
      \item \textbf{Precondizioni}: \begin{itemize}
            \item L'Utente ha avviato l'applicazione
            \item L'Utente sta effettuando una ricerca nel DIP con filtri
      \end{itemize}
      \item \textbf{Postcondizioni}:  I campi selezionati dall'Utente sono aggiunti ai campi selezionati per la ricerca.
      \item \textbf{Flusso Principale}:
      \begin{enumerate}
            \item L'Utente visualizza l'elenco dei campi comuni disponibili per la ricerca (\ref{visualizzaListaCampiComuni})
            \item L'Utente sceglie uno o più campi tra quelli disponibili
            \item Il sistema aggiunge gli elementi selezionati ai campi selezionati per la ricerca
      \end{enumerate}
      \item \textbf{Inclusioni}: \ref{visualizzaListaCampiComuni} Visualizza lista campi comuni
\end{itemize}

\deepusecase{visualizzaListaCampiComuni}{Visualizza lista campi comuni}
\begin{itemize}
      \item \textbf{Attore Primario}: Utente
      \item \textbf{Precondizioni}: 
      \begin{itemize}
            \item L'Utente ha avviato l'applicazione
            \item L'Utente sta effettuando una ricerca nel DIP con filtri
      \end{itemize}
      \item \textbf{Postcondizioni}:  L'Utente visualizza l'elenco dei campi comuni disponibili per la ricerca
      \item \textbf{Flusso Principale}:
      \begin{enumerate}
            \item Il sistema mostra a video l'elenco dei campi comuni disponibili per la ricerca e il nome corrispondente (\ref{visualizzaNomeCampo})
      \end{enumerate}
      \item \textbf{Inclusioni}: \ref{visualizzaNomeCampo} Visualizza nome del campo
\end{itemize}

\subsubusecase{addChiaveDescrittiva}{Specifica filtro Chiave Descrittiva}
\begin{itemize}
      \item \textbf{Attore Primario}: Utente
      \item \textbf{Precondizioni}: \begin{itemize}
            \item L'Utente ha avviato l'applicazione
            \item L'Utente sta effettuando una ricerca nel DIP con filtri
            \item L'Utente ha selezionato il filtro Chiave Descrittiva tra i campi comuni
      \end{itemize}
      \item \textbf{Postcondizioni}: \begin{itemize}
            \item L'Utente ha specificato i valori del filtro Chiave Descrittiva
            \item Il filtro Chiave Descrittiva è parametro per la ricerca con filtri e per la generazione del risultato
      \end{itemize}
      \item \textbf{Flusso Principale}:
            \begin{enumerate}
                  \item L'Utente compila i valori di Chiave Descrittiva quali:
                        \begin{itemize}
                              \item Chiave Descrittiva (\ref{compilaValoreFiltro})
                              \item Oggetto (\ref{compilaValoreFiltro})
                              \item Parole Chiave (\ref{compilaValoreFiltro})
                        \end{itemize}
                  \item Il sistema imposta i valori del filtro come parametri di ricerca.
            \end{enumerate}
      \item \textbf{Inclusioni}: \ref{compilaValoreFiltro} Compilazione Valore Filtro
\end{itemize}

\subsubusecase{addClassificazione}{Specifica filtro Classificazione}
\begin{itemize}
      \item \textbf{Attore Primario}: Utente
      \item \textbf{Precondizioni}:  \begin{itemize}
            \item L'Utente ha avviato l'applicazione
            \item L'Utente sta effettuando una ricerca nel DIP con filtri
            \item L'Utente ha selezionato il filtro Classificazione tra i campi comuni
      \end{itemize}
      \item \textbf{Postcondizioni}: \begin{itemize}
            \item L'Utente ha specificato i valori del filtro Classificazione
            \item Il filtro Classificazione è parametro per la ricerca con filtri e per la generazione del risultato
      \end{itemize}
      \item \textbf{Flusso Principale}:
            \begin{enumerate}
                  \item L'Utente compila i valori di Classificazione (\ref{compilaValoreFiltro}) tra:
                        \begin{itemize}
                              \item Indice di Classificazione
                              \item Descrizione
                              \item Piano di Fascicolo
                        \end{itemize}
                  \item Il sistema imposta i valori del filtro come parametri di ricerca.
            \end{enumerate}
      \item \textbf{Inclusioni}: \ref{compilaValoreFiltro} Compilazione Valore Filtro
\end{itemize}

\subsubusecase{addTempoConserva}{Specifica filtro Tempo di Conservazione}
\begin{itemize}
      \item \textbf{Attore Primario}: Utente
      \item \textbf{Precondizioni}: \begin{itemize}
            \item L'Utente ha avviato l'applicazione
            \item L'Utente sta effettuando una ricerca nel DIP con filtri
            \item L'Utente ha selezionato il filtro Tempo di Conservazione tra i campi comuni
      \end{itemize}
      \item \textbf{Postcondizioni}: \begin{itemize}
            \item L'Utente ha specificato i valori del filtro Tempo di Conservazione
            \item Il filtro Tempo di Conservazione è parametro per la ricerca con filtri e per la generazione del risultato
      \end{itemize}
      \item \textbf{Flusso Principale}:
            \begin{enumerate}
                  \item L'Utente inserisce il valore numerico del tempo di conservazione o seleziona
                        "Perenne" (\ref{compilaValoreFiltro})
                  \item Il sistema imposta il valore come parametro per la ricerca.
            \end{enumerate}
      \item \textbf{Inclusioni}: \ref{compilaValoreFiltro} Compilazione Valore Filtro
\end{itemize}

\subsubusecase{addNote}{Specifica filtro Note}
\begin{itemize}
      \item \textbf{Attore Primario}: Utente
      \item \textbf{Precondizioni}: \begin{itemize}
            \item L'Utente ha avviato l'applicazione
            \item L'Utente sta effettuando una ricerca nel DIP con filtri
            \item L'Utente ha selezionato il filtro Note tra i campi comuni
      \end{itemize}
      \item \textbf{Postcondizioni}: \begin{itemize}
            \item L'Utente ha specificato i valori del filtro Note
            \item Il filtro Note è parametro per la ricerca con filtri e per la generazione del risultato
      \end{itemize}
      \item \textbf{Flusso Principale}:
            \begin{enumerate}
                  \item L'Utente inserisce il testo della Nota (\ref{compilaValoreFiltro})
                  \item Il sistema imposta il valore come parametro per la ricerca
            \end{enumerate}
      \item \textbf{Inclusioni}: \ref{compilaValoreFiltro} Compilazione Valore Filtro
\end{itemize}

\subsubusecase{addTipoDocumento}{Specifica filtro Tipo documento}
\begin{itemize}
      \item \textbf{Attore Primario}: Utente
      \item \textbf{Precondizioni}: \begin{itemize}
            \item L'Utente ha avviato l'applicazione
            \item L'Utente sta effettuando una ricerca nel DIP con filtri
            \item L'Utente ha selezionato il filtro Tipo documento tra i campi comuni
      \end{itemize}
      \item \textbf{Postcondizioni}: \begin{itemize}
            \item L'Utente ha specificato i valori del filtro Tipo documento
            \item Il filtro Tipo documento è parametro per la ricerca con filtri e per la generazione del risultato
      \end{itemize}
      \item \textbf{Flusso Principale}:
      \begin{enumerate}
            \item L'Utente sceglie il tipo di documento tra: \begin{itemize}
                \item Documento Informatico
                \item Documento Amministrativo Informatico
                \item Aggregazione Documentale
            \end{itemize}
            \item Il sistema imposta il valore come parametro per la ricerca (\ref{compilaValoreFiltro})
      \end{enumerate}
      \item \textbf{Inclusioni}: \ref{compilaValoreFiltro} Compilazione Valore Filtro
\end{itemize}

\subsubusecase{addSoggetti}{Specifica filtro soggetti}
\begin{itemize}
      \item \textbf{Attore Primario}: Utente
      \item \textbf{Precondizioni}: 
      \begin{itemize}
            \item L'Utente ha avviato l'applicazione
            \item L'Utente sta effettuando una ricerca nel DIP con filtri
            \item L'Utente ha selezionato il filtro Soggetti tra i campi comuni
            \item L'Utente ha specificato il filtro Tipo di Documento
      \end{itemize}
      \item \textbf{Postcondizioni}: \begin{itemize}
            \item L'Utente ha aggiunto uno o più soggetti al filtro Soggetti
            \item I valori dei soggetti aggiunti sono parametri per la ricerca
      \end{itemize}
      \item \textbf{Flusso Principale}:
            \begin{enumerate}
                  \item L'utente aggiunge uno o più soggetti al filtro Soggetti (\ref{addSoggetto}) $(0...n)$.
                  \item Il sistema imposta i valori dei soggetti aggiunti come parametri per la ricerca
            \end{enumerate}
      \item \textbf{Inclusioni}: \ref{addSoggetto} Aggiunta Soggetto ad un filtro
\end{itemize}



\subusecase{addFiltriTipoDocumento}{Specifica filtri per Tipo di Documento}
\begin{itemize}
      \item \textbf{Attore Primario}: Utente
      \item \textbf{Precondizioni}: 
      \begin{itemize}
            \item L'Utente ha avviato l'applicazione
            \item L'Utente sta effettuando una ricerca nel DIP con filtri
            \item L'Utente ha specificato il filtro Tipo di Documento
      \end{itemize}
      \item \textbf{Postcondizioni}: Vengono specificati i filtri per il tipo di documento selezionato e utilizzati per la generazione del risultato.
      \item \textbf{Flusso Principale}:
      \begin{enumerate}
            \item L'utente aggiunge i filtri specifici per il tipo di documento selezionato (descritti negli UC \ref{specificaFiltriDI-DAI} e \ref{specificaFiltriAggregazione})
      \end{enumerate}
\end{itemize}

\subsubusecase{specificaFiltriDI-DAI}{Specifica filtri per Documento Informatico e Amministrativo Informatico}
\begin{itemize}
      \item \textbf{Attore Primario}: Utente
      \item \textbf{Precondizioni}: 
      \begin{itemize}
            \item L'Utente ha avviato l'applicazione
            \item L'Utente sta effettuando una ricerca nel DIP con filtri
            \item L'Utente ha specificato come Tipo di Documento: Documento Informatico o Documento Amministrativo Informatico
      \end{itemize}
      \item \textbf{Postcondizioni}: Vengono specificati i filtri per i campi del Documento Informatico e Amministrativo Informatico e utilizzati per la generazione del risultato.
      \item \textbf{Flusso Principale}:
      \begin{enumerate}
            \item L'Utente seleziona i campi specifici per il tipo documento su cui basare la ricerca (\ref{selezionaCampiDI-DAI})
            \item L'Utente specifica i filtri per i campi selezionati per il tipo Documento Informatico e Amministrativo Informatico (descritti negli UC da \ref{addDatiRegistrazione} a \ref{addTracciatureModificheDocumento})
      \end{enumerate}
      \item \textbf{Inclusioni}:
      \begin{itemize}
            \item \ref{selezionaCampiDI-DAI} Seleziona campi di ricerca per DI/DAI
            \item \ref{addDatiRegistrazione} Specifica filtro Dati di Registrazione
            \item \ref{addTipologiaDocumentale} Specifica filtro Tipologia Documentale
            \item \ref{addModalitaFormazione} Specifica filtro Modalità di Formazione
            \item \ref{addRiservato} Specifica filtro campo Riservato
            \item \ref{addIdentificativoFormato} Specifica filtro Identificativo di Formato
            \item \ref{addDatiVerifica} Specifica filtro Dati di Verifica
            \item \ref{addNomeDocumento} Specifica filtro Nome del Documento
            \item \ref{addVersioneDocumento} Specifica filtro Versione del Documento
            \item \ref{addIdentificativoDocumentoPrimario} Specifica filtr Identificativo del Documento Primario
            \item \ref{addTracciatureModificheDocumento} Specifica filtro Tracciature di Modifiche del Documento
      \end{itemize}
      \item \textbf{Specializza}: \ref{addFiltriTipoDocumento} Specifica filtri per Tipo di Documento
\end{itemize}

\deepusecase{selezionaCampiDI-DAI}{Selezione campi di ricerca per DI/DAI}
\begin{itemize}
      \item \textbf{Attore Primario}: Utente
      \item \textbf{Precondizioni}: \begin{itemize}
            \item L'Utente ha avviato l'applicazione
            \item L'Utente sta effettuando una ricerca nel DIP con filtri
            \item L'Utente ha selezionato il filtro per il Tipo di Documento come DI o DAI
      \end{itemize}
      \item \textbf{Postcondizioni}:  I campi selezionati dall'Utente sono aggiunti ai campi selezionati per la ricerca.
      \item \textbf{Flusso Principale}:
            \begin{enumerate}
                  \item L'Utente visualizza l'elenco dei campi specifici per DI/DAI disponibili per la ricerca (\ref{visualizzaNomeCampo})
                  \item L'Utente sceglie uno o più campi tra quelli disponibili.
                  \item Il sistema aggiunge gli elementi selezionati ai campi selezionati per la ricerca
            \end{enumerate}
      \item \textbf{Inclusioni}: \ref{visualizzaListaCampiDIDAI} Visualizza lista campi DI/DAI
\end{itemize}

\subdeepusecase{visualizzaListaCampiDIDAI}{Visualizza lista campi DI/DAI}
\begin{itemize}
      \item \textbf{Attore Primario}: Utente
      \item \textbf{Precondizioni}: \begin{itemize}
            \item L'Utente ha avviato l'applicazione
            \item L'Utente sta effettuando una ricerca nel DIP con filtri
            \item L'Utente ha selezionato il filtro per il Tipo di Documento come DI o DAI
      \end{itemize}
      \item \textbf{Postcondizioni}:  L'Utente visualizza l'elenco dei campi specifici per DI/DAI disponibili per la ricerca.
      \item \textbf{Flusso Principale}:
            \begin{enumerate}
                  \item Il sistema mostra a video l'elenco dei campi specifici per DI/DAI disponibili per la ricerca e il nome corrispondente (\ref{visualizzaNomeCampo})
            \end{enumerate}
      \item \textbf{Inclusioni}: \ref{visualizzaNomeCampo} Visualizza nome del campo
\end{itemize}

\deepusecase{addDatiRegistrazione}{Specifica filtro Dati di Registrazione}
\begin{itemize}
      \item \textbf{Attore Primario}: Utente
      \item \textbf{Precondizioni}: 
      \begin{itemize}
            \item L'Utente ha avviato l'applicazione
            \item L'Utente sta effettuando una ricerca nel DIP con filtri
            \item L'Utente ha specificato come Tipo di Documento: Documento Informatico o Documento Amministrativo Informatico
            \item L'Utente ha selezionato il filtro Dati di Registrazione tra i campi specifici per tipo documentale
      \end{itemize}
      \item \textbf{Postcondizioni}: 
      \begin{itemize}
            \item L'Utente ha specificato i valori del filtro Dati di Registrazione
            \item Il filtro Dati di Registrazione è parametro per la ricerca con filtri e per la generazione del risultato
      \end{itemize}
      \item \textbf{Flusso Principale}:
      \begin{enumerate}
            \item L'Utente specifica la Tipologia di Flusso (\ref{compilaValoreFiltro}) tra: \begin{itemize}
                  \item Uscita
                  \item Entrata
                  \item Interno
            \end{itemize}
            \item Nessuno,
            \item L'Utente specifica il Tipo di Registro (\ref{compilaValoreFiltro}) tra: 
            \begin{itemize}
                  \item Protocollo Ordinario/Protocollo di Emergenza
                  \item Repertorio/Registro
            \end{itemize}
            \item L'Utente specifica la Data/Ora di Registrazione (nel caso di un documento
            protocollato tali parametri fanno riferimento alla protocollazione) (\ref{compilaValoreFiltro})
            \item L'Utente specifica il codice identificativo del Registro (\ref{compilaValoreFiltro})
            \item Il sistema imposta i valori del filtro come parametri per la ricerca
      \end{enumerate}
      \item \textbf{Inclusioni}: \ref{compilaValoreFiltro} Compila valore filtro
\end{itemize}

\deepusecase{addTipologiaDocumentale}{Specifica filtro Tipologia Documentale}
\begin{itemize}
      \item \textbf{Attore Primario}: Utente
      \item \textbf{Precondizioni}: 
      \begin{itemize}
            \item L'Utente ha avviato l'applicazione
            \item L'Utente sta effettuando una ricerca nel DIP con filtri
            \item L'Utente ha specificato come tipo di documento: Documento Informatico o Documento Amministrativo Informatico
            \item L'utente ha selezionato il filtro Tipologia Documentale tra i campi specifici per tipo documentale
      \end{itemize}
      \item \textbf{Postcondizioni}: 
      \begin{itemize}
            \item L'Utente ha specificato i valori del filtro Tipologia Documentale
            \item Il filtro Tipologia Documentale è parametro per la ricerca con filtri e per la generazione del risultato
      \end{itemize}
      \item \textbf{Flusso Principale}:
      \begin{enumerate}
            \item L'Utente specifica il nome della Tipologia Documentale (fatture, delibere,
            determine) (\ref{compilaValoreFiltro})
            \item Il sistema imposta questo valore come parametro per la ricerca
      \end{enumerate}
      \item \textbf{Inclusioni}: \ref{compilaValoreFiltro} Compila valore filtro
\end{itemize}

\deepusecase{addModalitaFormazione}{Specifica filtro Modalità di Formazione}
\begin{itemize}
      \item \textbf{Attore Primario}: Utente
      \item \textbf{Precondizioni}: 
      \begin{itemize}
            \item L'Utente ha avviato l'applicazione
            \item L'Utente sta effettuando una ricerca nel DIP con filtri
            \item L'Utente ha specificato come tipo di documento: Documento Informatico o Documento Amministrativo Informatico
            \item L'utente ha selezionato il filtro Modalità di Formazione tra i campi specifici per tipo documentale
      \end{itemize}
      \item \textbf{Postcondizioni}: 
      \begin{itemize}
            \item L'Utente ha specificato i valori del filtro Modalità di Formazione
            \item Il filtro Modalità di Formazione è parametro per la ricerca con filtri e per la generazione del risultato
      \end{itemize}
      \item \textbf{Flusso Principale}:
      \begin{enumerate}
            \item L'Utente specifica la Modalità di Formazione (\ref{compilaValoreFiltro}) tra:
            \begin{itemize}
                  \item Creazione di un documento informatico ex novo
                  \item Acquisizione di un documento informatico da altro documento informatico o da supporto informatico
                  \item Memorizzazione su supporto informatico in formato digitale
                  \item Generazione o raggruppamento in forma statica
            \end{itemize}
            \item Il sistema imposta questo valore come parametro per la ricerca
      \end{enumerate}
      \item \textbf{Inclusioni}: \ref{compilaValoreFiltro} Compila valore filtro
\end{itemize}

\deepusecase{addRiservato}{Specifica filtro campo Riservato}
\begin{itemize}
      \item \textbf{Attore Primario}: Utente
      \item \textbf{Precondizioni}: 
      \begin{itemize}
            \item L'Utente ha avviato l'applicazione
            \item L'Utente sta effettuando una ricerca nel DIP con filtri
            \item L'Utente ha specificato come tipo di documento: Documento Informatico o Documento Amministrativo Informatico
            \item L'utente ha selezionato il filtro Riservato tra i campi specifici per tipo documentale
      \end{itemize}
      \item \textbf{Postcondizioni}: 
      \begin{itemize}
            \item L'Utente ha specificato i valori del filtro Riservato
            \item Il filtro Riservato è parametro per la ricerca con filtri e per la generazione del risultato
      \end{itemize}
      \item \textbf{Flusso Principale}:
      \begin{enumerate}
            \item L'Utente specifica se il file è Riservato o meno (\ref{compilaValoreFiltro})
            \item Il sistema imposta questo valore come parametro per la ricerca
      \end{enumerate}
      \item \textbf{Inclusioni}: \ref{compilaValoreFiltro} Compila valore filtro
\end{itemize}

\deepusecase{addIdentificativoFormato}{Specifica filtro Identificativo di Formato}
\begin{itemize}
      \item \textbf{Attore Primario}: Utente
      \item \textbf{Precondizioni}: 
      \begin{itemize}
            \item L'Utente ha avviato l'applicazione
            \item L'Utente sta effettuando una ricerca nel DIP con filtri
            \item L'Utente ha specificato come tipo di documento: Documento Informatico o Documento Amministrativo Informatico
            \item L'utente ha selezionato il filtro Identificativo di Formato tra i campi specifici per tipo documentale
      \end{itemize}
      \item \textbf{Postcondizioni}: 
      \begin{itemize}
            \item L'Utente ha specificato i valori del filtro Identificativo di Formato
            \item Il filtro Identificativo di Formato è parametro per la ricerca con filtri e per la generazione del risultato
      \end{itemize}
      \item \textbf{Flusso Principale}:
            \begin{enumerate}
                  \item L'Utente specifica la Tipologia di Formato all'interno di quelli Previsti dalle
                        linee Guida (.pdf, .xml, .docx, ecc.) (\ref{compilaValoreFiltro})
                  \item L'Utente specifica il Nome del Prodotto utilizzato per la creazione del
                        Documento (\ref{compilaValoreFiltro})
                  \item L'Utente specifica il numero della Versione del Prodotto utilizzato per la
                        creazione del Documento (\ref{compilaValoreFiltro})
                  \item L'Utente specifica il nome il Produttore del Prodotto utilizzato per la
                        creazione del Documento (\ref{compilaValoreFiltro})
                  \item Il sistema imposta i valori del filtro come parametri di ricerca.
            \end{enumerate}
      \item \textbf{Inclusioni}: \ref{compilaValoreFiltro} Compila valore filtro
\end{itemize}

\deepusecase{addDatiVerifica}{Specifica filtro Dati di Verifica}
\begin{itemize}
      \item \textbf{Attore Primario}: Utente
      \item \textbf{Precondizioni}: 
      \begin{itemize}
            \item L'Utente ha avviato l'applicazione
            \item L'Utente sta effettuando una ricerca nel DIP con filtri
            \item L'Utente ha specificato come tipo di documento: Documento Informatico o Documento Amministrativo Informatico
            \item L'utente ha selezionato il filtro Dati di Verifica tra i campi specifici per tipo documentale
      \end{itemize}
      \item \textbf{Postcondizioni}: \begin{itemize}
            \item L'Utente ha specificato i valori del filtro Dati di Verifica
            \item Il filtro Dati di Verifica è parametro per la ricerca con filtri e per la generazione del risultato
      \end{itemize}
      \item \textbf{Flusso Principale}:
      \begin{enumerate}
            \item L'Utente specifica se il file è Firmato Digitalmente o meno (\ref{compilaValoreFiltro})
            \item L'Utente specifica se il file è Sigillato Elettronicamente o meno (\ref{compilaValoreFiltro})
            \item L'Utente specifica se il file ha una Marcatura Temporale o meno (\ref{compilaValoreFiltro})
            \item L'Utente specifica se vi è conformità copie immagine su supporto informatico o
                  meno (\ref{compilaValoreFiltro})
            \item Il sistema imposta i valori del filtro come parametri di ricerca
      \end{enumerate}
      \item \textbf{Inclusioni}: \ref{compilaValoreFiltro} Compila valore filtro
\end{itemize}

\deepusecase{addNomeDocumento}{Specifica filtro Nome del Documento}
\begin{itemize}
      \item \textbf{Attore Primario}: Utente
      \item \textbf{Precondizioni}: 
      \begin{itemize}
            \item L'Utente ha avviato l'applicazione
            \item L'Utente sta effettuando una ricerca nel DIP con filtri
            \item L'Utente ha specificato come tipo di documento: Documento Informatico o Documento Amministrativo Informatico
            \item L'utente ha selezionato il filtro Nome del Documento tra i campi specifici per tipo documentale
      \end{itemize}
      \item \textbf{Postcondizioni}: \begin{itemize}
            \item L'Utente ha specificato i valori del filtro Nome del Documento
            \item Il filtro Nome del Documento è parametro per la ricerca con filtri e per la generazione del risultato
      \end{itemize}
      \item \textbf{Flusso Principale}:
            \begin{enumerate}
                  \item L'Utente specifica il Nome del Documento (\ref{compilaValoreFiltro})
                  \item Il sistema imposta questo valore come parametro per la ricerca
            \end{enumerate}
      \item \textbf{Inclusioni}: \ref{compilaValoreFiltro} Compila valore filtro
\end{itemize}

\deepusecase{addVersioneDocumento}{Specifica filtro Versione del Documento}
\begin{itemize}
      \item \textbf{Attore Primario}: Utente
      \item \textbf{Precondizioni}: 
      \begin{itemize}
            \item L'Utente ha avviato l'applicazione
            \item L'Utente sta effettuando una ricerca nel DIP con filtri
            \item L'Utente ha specificato come tipo di documento: Documento Informatico o Documento Amministrativo Informatico
            \item L'utente ha selezionato il filtro Versione del Documento tra i campi specifici per tipo documentale
      \end{itemize}
      \item \textbf{Postcondizioni}: \begin{itemize}
            \item L'Utente ha specificato i valori del filtro Versione del Documento
            \item Il filtro Versione del Documento è parametro per la ricerca con filtri e per la generazione del risultato
      \end{itemize}
      \item \textbf{Flusso Principale}:
            \begin{enumerate}
                  \item L'Utente specifica il numero che rappresenta la Versione del Documento. (\ref{compilaValoreFiltro})
                  \item Il sistema imposta questo valore come parametro per la ricerca.
            \end{enumerate}
      \item \textbf{Inclusioni}: \ref{compilaValoreFiltro} Compila valore filtro
\end{itemize}

\deepusecase{addIdentificativoDocumentoPrimario}{Specifica filtro Identificativo del Documento Primario}
\begin{itemize}
      \item \textbf{Attore Primario}: Utente
      \item \textbf{Precondizioni}:
      \begin{itemize}
            \item L'Utente ha avviato l'applicazione
            \item L'Utente sta effettuando una ricerca nel DIP con filtri
            \item L'Utente ha specificato come tipo di documento: Documento Informatico o Documento Amministrativo Informatico
            \item L'utente ha selezionato il filtro Identificativo del Documento Primario tra i campi specifici per tipo documentale
      \end{itemize}
      \item \textbf{Postcondizioni}: \begin{itemize}
            \item L'Utente ha specificato i valori del filtro Identificativo del Documento Primario
            \item Il filtro Identificativo del Documento Primario è parametro per la ricerca con filtri e per la generazione del risultato
      \end{itemize}
      \item \textbf{Flusso Principale}:
            \begin{enumerate}
                  \item L'Utente specifica l'Identificativo del Documento Primario. (\ref{compilaValoreFiltro})
                  \item Il sistema imposta questo valore come parametro per la ricerca.
            \end{enumerate}
      \item \textbf{Inclusioni}: \ref{compilaValoreFiltro} Compila valore filtro
\end{itemize}

\deepusecase{addTracciatureModificheDocumento}{Specifica filtro Tracciature Modifiche di Documento}
\begin{itemize}
      \item \textbf{Attore Primario}: Utente
      \item \textbf{Precondizioni}: 
      \begin{itemize}
            \item L'Utente ha avviato l'applicazione
            \item L'Utente sta effettuando una ricerca nel DIP con filtri
            \item L'Utente ha specificato come tipo di documento: Documento Informatico o Documento Amministrativo Informatico
            \item L'utente ha selezionato il filtro Tracciature Modifiche di Documento tra i campi specifici per tipo documentale
      \end{itemize}
      \item \textbf{Postcondizioni}: \begin{itemize}
            \item L'Utente ha specificato i valori del filtro Tracciature Modifiche di Documento
            \item Il filtro Tracciature Modifiche di Documento è parametro per la ricerca con filtri e per la generazione del risultato
      \end{itemize}
      \item \textbf{Flusso Principale}:
            \begin{enumerate}
                  \item L'Utente sceglie il Tipo di modifica (\ref{compilaValoreFiltro}) tra:
                        \begin{itemize}
                              \item Annullamento
                              \item Registrazione
                              \item Integrazione
                              \item Annotazione
                        \end{itemize}
                  \item L'Utente specifica il Soggetto che ha effettuato la modifica. (\ref{compilaValoreFiltro})
                  \item L'Utente specifica la Data/Ora della Modifica. (\ref{compilaValoreFiltro})
                  \item L'Utente specifica l'identificativo documento versione precedente. (\ref{compilaValoreFiltro})
                  \item Il sistema imposta i valori del filtro come parametri di ricerca.
            \end{enumerate}
      \item \textbf{Inclusioni}: \ref{compilaValoreFiltro} Compila valore filtro
\end{itemize}

\subsubusecase{specificaFiltriAggregazione}{Specifica filtri per Aggregazione Documentale Informatica}
\begin{itemize}
      \item \textbf{Attore Primario}: Utente
      \item \textbf{Precondizioni}: 
      \begin{itemize}
            \item L'Utente ha avviato l'applicazione
            \item L'Utente sta effettuando una ricerca nel DIP con filtri
            \item L'Utente ha specificato come tipo di documento: Aggregazione Documentale
      \end{itemize}
      \item \textbf{Postcondizioni}: Vengono specificati i filtri per i campi dell'Aggregazione Documentale e utilizzati per la generazione del risultato.
      \item \textbf{Flusso Principale}:
      \begin{enumerate}
            \item L'Utente seleziona i campi specifici per l'Aggregazione Documentale su cui basare la ricerca (\ref{selezionaCampiAggregazioneDocumentale}).
            \item L'Utente aggiunge i filtri specifici per il tipo Aggregazione (descritto negli UC da \ref{addTipoAggregazione} a \ref{addProgressivoAggregazione}).
      \end{enumerate}
      \item \textbf{Inclusioni}:
      \begin{itemize}
            \item \ref{selezionaCampiAggregazioneDocumentale} Seleziona campi di ricerca per Aggregazione Documentale
            \item \ref{addTipoAggregazione} Specifica filtro Tipo di Aggregazione
            \item \ref{addIdAggregazione} Specifica filtro Id dell'Aggregazione
            \item \ref{addTipologiaFascicolo} Specifica filtro Tipologia di Fascicolo
            \item \ref{addIdAggregazionePrimario} Specifica filtro Id Aggregazione Primario
            \item \ref{addDataApertura} Specifica filtro Data di Apertura
            \item \ref{addDataChiusura} Specifica filtro Data di Chiusura
            \item \ref{addProcedimentoAmministrativo} Specifica filtro Procedimento Amministrativo
            \item \ref{addAssegnazione} Specifica filtro Assegnazione
            \item \ref{addProgressivoAggregazione} Specifica filtro Progressivo di Aggregazione
      \end{itemize}
      \item \textbf{Specializza}: \ref{addFiltriTipoDocumento} Specifiche filtri per Tipo di Documento
\end{itemize}

\deepusecase{selezionaCampiAggregazioneDocumentale}{Selezione campi di ricerca per Aggregazione Documentale}
\begin{itemize}
      \item \textbf{Attore Primario}: Utente
      \item \textbf{Precondizioni}: \begin{itemize}
            \item L'Utente ha avviato l'applicazione
            \item L'Utente sta effettuando una ricerca nel DIP con filtri
            \item L'Utente ha selezionato il filtro per il Tipo di Documento come Aggregazione Documentale
      \end{itemize}
      \item \textbf{Postcondizioni}:  I campi selezionati dall'Utente sono aggiunti ai campi selezionati per la ricerca.
      \item \textbf{Flusso Principale}:
      \begin{enumerate}
            \item L'Utente visualizza l'elenco dei campi specifici per Aggregazione Documentale disponibili per la ricerca. (\ref{visualizzaListaCampiAggregazioneDocumentale})
            \item L'Utente sceglie uno o più campi tra quelli disponibili.
            \item Il sistema aggiunge gli elementi selezionati ai campi selezionati per la ricerca.
      \end{enumerate}
      \item \textbf{Inclusioni}: \ref{visualizzaListaCampiAggregazioneDocumentale} Visualizza lista campi Aggregazione Documentale
\end{itemize}

\subdeepusecase{visualizzaListaCampiAggregazioneDocumentale}{Visualizza lista campi Aggregazione Documentale}
\begin{itemize}
      \item \textbf{Attore Primario}: Utente
      \item \textbf{Precondizioni}: 
      \begin{itemize}
            \item L'Utente ha avviato l'applicazione
            \item L'Utente sta effettuando una ricerca nel DIP con filtri
            \item L'Utente ha selezionato il filtro per il Tipo di Documento come Aggregazione Documentale
      \end{itemize}
      \item \textbf{Postcondizioni}:  L'Utente visualizza l'elenco dei campi specifici per Aggregazione Documentale disponibili per la ricerca.
      \item \textbf{Flusso Principale}:
      \begin{enumerate}
            \item Il sistema mostra a video l'elenco dei campi specifici per Aggregazione Documentale disponibili per la ricerca e il nome corrispondente. (\ref{visualizzaNomeCampo})
      \end{enumerate}
      \item \textbf{Inclusioni}: \ref{visualizzaNomeCampo} Visualizza nome campo
\end{itemize}

\deepusecase{addTipoAggregazione}{Specifica filtro Tipo di Aggregazione}
\begin{itemize}
      \item \textbf{Attore Primario}: Utente
      \item \textbf{Precondizioni}: 
      \begin{itemize}
            \item L'Utente ha avviato l'applicazione
            \item L'Utente sta effettuando una ricerca nel DIP con filtri
            \item L'Utente ha specificato come tipo di documento: Aggregazione Documentale
            \item L'utente ha selezionato il filtro Tipo di Aggregazione tra i campi specifici per tipo documentale
      \end{itemize}
      \item \textbf{Postcondizioni}: 
      \begin{itemize}
            \item L'Utente ha specificato i valori del filtro Tipo di Aggregazione
            \item Il filtro Tipo di Aggregazione è parametro per la ricerca con filtri e per la generazione del risultato
      \end{itemize}
      \item \textbf{Flusso Principale}:
      \begin{enumerate}
            \item L'Utente specifica il tipo di aggregazione (\ref{compilaValoreFiltro}) tra:
            \begin{itemize}
                  \item Fascicolo,
                  \item Serie Documentale,
                  \item Serie di Fascicoli.
            \end{itemize}
            \item Il sistema imposta questo valore come parametro per la ricerca.
      \end{enumerate}
      \item \textbf{Inclusioni}: \ref{compilaValoreFiltro} Compila valore filtro
\end{itemize}

\deepusecase{addIdAggregazione}{Specifica filtro Id dell'Aggregazione}
\begin{itemize}
      \item \textbf{Attore Primario}: Utente
      \item \textbf{Precondizioni}: 
      \begin{itemize}
            \item L'Utente ha avviato l'applicazione
            \item L'Utente sta effettuando una ricerca nel DIP con filtri
            \item L'Utente ha specificato come tipo di documento: Aggregazione Documentale
            \item L'utente ha selezionato il filtro Id dell'Aggregazione tra i campi specifici per tipo documentale
      \end{itemize}
      \item \textbf{Postcondizioni}: \begin{itemize}
            \item L'Utente ha specificato i valori del filtro Id dell'Aggregazione
            \item Il filtro Id dell'Aggregazione è parametro per la ricerca con filtri e per la generazione del risultato
      \end{itemize}
      \item \textbf{Flusso Principale}:
            \begin{enumerate}
                  \item L'Utente specifica l'Id dell'Aggregazione. (\ref{compilaValoreFiltro})
                  \item Il sistema imposta questo valore come parametro per la ricerca.
            \end{enumerate}
      \item \textbf{Inclusioni}: \ref{compilaValoreFiltro} Compila valore filtro
\end{itemize}

\deepusecase{addTipologiaFascicolo}{Specifica filtro Tipologia di Fascicolo}
\begin{itemize}
      \item \textbf{Attore Primario}: Utente
      \item \textbf{Precondizioni}: 
      \begin{itemize}
            \item L'Utente ha avviato l'applicazione
            \item L'Utente sta effettuando una ricerca nel DIP con filtri
            \item L'Utente ha specificato come tipo di documento: Aggregazione Documentale
            \item L'utente ha selezionato il filtro Tipologia di Fascicolo tra i campi specifici per tipo documentale
      \end{itemize}
      \item \textbf{Postcondizioni}: \begin{itemize}
            \item L'Utente ha specificato i valori del filtro Tipologia di Fascicolo
            \item Il filtro Tipologia di Fascicolo è parametro per la ricerca con filtri e per la generazione del risultato
      \end{itemize}
      \item \textbf{Flusso Principale}:
            \begin{enumerate}
                  \item L'Utente specifica il tipo di fascicolo (\ref{compilaValoreFiltro}) tra: \begin{itemize}
                              \item Affare,
                              \item Attività,
                              \item Persona Fisica,
                              \item Persona Giuridica,
                              \item Procedimento Amministrativo.
                        \end{itemize}
                  \item Il sistema imposta questo valore come parametro per la ricerca.
            \end{enumerate}
      \item \textbf{Inclusioni}: \ref{compilaValoreFiltro} Compila valore filtro
\end{itemize}

\deepusecase{addIdAggregazionePrimario}{Specifica filtro Id Aggregazione Primario}
\begin{itemize}
      \item \textbf{Attore Primario}: Utente
      \item \textbf{Precondizioni}: 
      \begin{itemize}
            \item L'Utente ha avviato l'applicazione
            \item L'Utente sta effettuando una ricerca nel DIP con filtri
            \item L'Utente ha specificato come tipo di documento: Aggregazione Documentale
            \item L'utente ha selezionato il filtro Id dell'Aggregazione Primario tra i campi specifici per tipo documentale
      \end{itemize}
      \item \textbf{Postcondizioni}: \begin{itemize}
            \item L'Utente ha specificato i valori del filtro Id dell'Aggregazione Primario
            \item Il filtro Id dell'Aggregazione Primario è parametro per la ricerca con filtri e per la generazione del risultato
      \end{itemize}
      \item \textbf{Flusso Principale}:
            \begin{enumerate}
                  \item L'Utente specifica l'Id Primario dell'Aggregazione. (\ref{compilaValoreFiltro})
                  \item Il sistema imposta questo valore come parametro per la ricerca.
            \end{enumerate}
      \item \textbf{Inclusioni}: \ref{compilaValoreFiltro} Compila valore filtro
\end{itemize}

\deepusecase{addDataApertura}{Specifica filtro Data Apertura}
\begin{itemize}
      \item \textbf{Attore Primario}: Utente
      \item \textbf{Precondizioni}: 
      \begin{itemize}
            \item L'Utente ha avviato l'applicazione
            \item L'Utente sta effettuando una ricerca nel DIP con filtri
            \item L'Utente ha specificato come tipo di documento: Aggregazione Documentale
            \item L'utente ha selezionato il filtro Data di Apertura tra i campi specifici per tipo documentale
      \end{itemize}
      \item \textbf{Postcondizioni}: \begin{itemize}
            \item L'Utente ha specificato i valori del filtro Data di Apertura
            \item Il filtro Data di Apertura è parametro per la ricerca con filtri e per la generazione del risultato
      \end{itemize}
      \item \textbf{Flusso Principale}:
            \begin{enumerate}
                  \item L'Utente specifica la Data di Apertura. (\ref{compilaValoreFiltro})
                  \item Il sistema imposta questo valore come parametro per la ricerca.
            \end{enumerate}
      \item \textbf{Inclusioni}: \ref{compilaValoreFiltro} Compila valore filtro
\end{itemize}

\deepusecase{addDataChiusura}{Specifica filtro Data Chiusura}
\begin{itemize}
      \item \textbf{Attore Primario}: Utente
      \item \textbf{Precondizioni}: 
      \begin{itemize}
            \item L'Utente ha avviato l'applicazione
            \item L'Utente sta effettuando una ricerca nel DIP con filtri
            \item L'Utente ha specificato come tipo di documento: Aggregazione Documentale
            \item L'utente ha selezionato il filtro Data di Chiusura tra i campi specifici per tipo documentale
      \end{itemize}
      \item \textbf{Postcondizioni}: \begin{itemize}
            \item L'Utente ha specificato i valori del filtro Data di Chiusura
            \item Il filtro Data di Chiusura è parametro per la ricerca con filtri e per la generazione del risultato
      \end{itemize}
      \item \textbf{Flusso Principale}:
            \begin{enumerate}
                  \item L'Utente specifica la Data di Chiusura. (\ref{compilaValoreFiltro})
                  \item Il sistema imposta questo valore come parametro per la ricerca.
            \end{enumerate}
      \item \textbf{Inclusioni}: \ref{compilaValoreFiltro} Compila valore filtro
\end{itemize}

\deepusecase{addProcedimentoAmministrativo}{Specifica filtro Procedimento Amministrativo}
\begin{itemize}
      \item \textbf{Attore Primario}: Utente
      \item \textbf{Precondizioni}: 
      \begin{itemize}
            \item L'Utente ha avviato l'applicazione
            \item L'Utente sta effettuando una ricerca nel DIP con filtri
            \item L'Utente ha specificato come tipo di documento: Aggregazione Documentale
            \item L'utente ha selezionato il filtro Procedimento Amministrativo tra i campi specifici per tipo documentale
      \end{itemize}
      \item \textbf{Postcondizioni}: 
      \begin{itemize}
            \item L'Utente ha specificato i valori del filtro Procedimento Amministrativo
            \item Il filtro Procedimento Amministrativo è parametro per la ricerca con filtri e per la generazione del risultato
      \end{itemize}
      \item \textbf{Flusso Principale}:
      \begin{enumerate}
            \item L'Utente specifica la Materia, Argomento e Struttura per i procedimenti. (\ref{compilaValoreFiltro})
            \item L'Utente specifica il Procedimento come denominazione. (\ref{compilaValoreFiltro})
            \item L'Utente specifica il Catalogo dei Procedimenti come URI di pubblicazione del catalogo. (\ref{compilaValoreFiltro})
            \item L'Utente aggiunge una o più Fasi (\ref{addFasiProcedimentoAmministrativo})(0...n).
            \item Il sistema imposta i valori del filtro come parametri di ricerca.
      \end{enumerate}
      \item \textbf{Inclusioni}: 
      \begin{itemize}
            \item \ref{addFasiProcedimentoAmministrativo} Aggiunta Fase del Procedimento Amministrativo
            \item \ref{compilaValoreFiltro} Compila valore filtro
      \end{itemize}
\end{itemize}
      
\subdeepusecase{addFasiProcedimentoAmministrativo}{Aggiunta Fase del Procedimento Amministrativo}
\begin{itemize}
      \item \textbf{Attore Primario}: Utente
      \item \textbf{Precondizioni}: 
      \begin{itemize}
            \item L'Utente ha avviato l'applicazione
            \item L'Utente sta effettuando una ricerca nel DIP con filtri
            \item L'Utente ha specificato come tipo di documento: Aggregazione Documentale
            \item L'Utente sta specificando il filtro Procedimento Amministrativo
      \end{itemize}
      \item \textbf{Postcondizioni}: Viene specificata una fase e aggiunta al filtro Procedimento Amministrativo.
      \item \textbf{Flusso Principale}:
      \begin{enumerate}
            \item L'utente specifica, per ogni fase da aggiungere, i dati specifici:
            \begin{itemize}
                  \item Tipo Fase (Preparatoria, Istruttoria, Consultiva, Decisoria o deliberativa,
                  Integrazione dell'efficacia). (\ref{compilaValoreFiltro})
                  \item Data di inizio. (\ref{compilaValoreFiltro})
                  \item Data di fine. (\ref{compilaValoreFiltro})
            \end{itemize}
            \item Il sistema salva la fase come parte del filtro Procedimento Amministrativo.
      \end{enumerate}
      \item \textbf{Inclusioni}: \ref{compilaValoreFiltro} Compila valore filtro
\end{itemize}

\subdeepusecase{addAssegnazione}{Specifica filtro Assegnazione}
\begin{itemize}
      \item \textbf{Attore Primario}: Utente
      \item \textbf{Precondizioni}: 
      \begin{itemize}
            \item L'Utente ha avviato l'applicazione
            \item L'Utente sta effettuando una ricerca nel DIP con filtri
            \item L'Utente ha specificato come tipo di documento: Aggregazione Documentale
            \item L'utente ha selezionato il filtro Assegnazione tra i campi specifici per tipo documentale
      \end{itemize}
      \item \textbf{Postcondizioni}: 
      \begin{itemize}
            \item L'Utente ha specificato i valori del filtro Assegnazione
            \item Il filtro Assegnazione è parametro per la ricerca con filtri e per la generazione del risultato
      \end{itemize}
      \item \textbf{Flusso Principale}:
      \begin{enumerate}
            \item L'Utente aggiunge uno o più valori per "Assegnazione". (\ref{addAssegnazione})(0...n).
            \item Il sistema imposta i valori del filtro come parametri per la ricerca.
      \end{enumerate}
      \item \textbf{Inclusioni}: \ref{addAssegnazione} Aggiunta Valore Assegnazione
\end{itemize}
      
\subdeepusecase{addAssegnazione}{Aggiunta Valore Assegnazione}
\begin{itemize}
      \item \textbf{Attore Primario}: Utente
      \item \textbf{Precondizioni}: 
      \begin{itemize}
            \item L'Utente ha avviato l'applicazione
            \item L'Utente sta effettuando una ricerca nel DIP con filtri
            \item L'Utente ha specificato come tipo di documento: Aggregazione Documentale
            \item L'Utente sta specificando un filtro che richiede un valore di Assegnazione
      \end{itemize}
      \item \textbf{Postcondizioni}: Viene specificato un valore di Assegnazione e aggiunto al filtro che l'Utente sta specificando.
      \item \textbf{Flusso Principale}:
      \begin{enumerate}
            \item L'utente specifica:
            \begin{itemize}
                  \item Tipo Assegnazione. (\ref{compilaValoreFiltro})
                  \item Soggetto Assegnatario. (\ref{compilaValoreFiltro})
                  \item Data di Inizio. (\ref{compilaValoreFiltro})
                  \item Data di Fine. (\ref{compilaValoreFiltro})
            \end{itemize}
            \item Il sistema salva il valore di assegnazione come parte del filtro Assegnazione.
      \end{enumerate}
      \item \textbf{Inclusioni}: \ref{compilaValoreFiltro} Compila valore filtro
\end{itemize}
      
\deepusecase{addProgressivoAggregazione}{Specifica filtro Progressivo Aggregazione}
\begin{itemize}
      \item \textbf{Attore Primario}: Utente
      \item \textbf{Precondizioni}: 
      \begin{itemize}
            \item L'Utente ha avviato l'applicazione
            \item L'Utente sta effettuando una ricerca nel DIP con filtri
            \item L'Utente ha specificato come tipo di documento: Aggregazione Documentale
            \item L'utente ha selezionato il filtro Progressivo dell'Aggregazione tra i campi specifici per tipo documentale
      \end{itemize}
      \item \textbf{Postcondizioni}: 
      \begin{itemize}
            \item L'Utente ha specificato i valori del filtro Progressivo dell'Aggregazione (\ref{compilaValoreFiltro})
            \item Il filtro Progressivo dell'Aggregazione è parametro per la ricerca con filtri e per la generazione del risultato
      \end{itemize}
      \item \textbf{Flusso Principale}:
      \begin{enumerate}
            \item L'Utente specifica il numero Progressivo dell'Aggregazione.
            \item Il sistema imposta questo valore come parametro per la ricerca.
      \end{enumerate}
      \item \textbf{Inclusioni}: \ref{compilaValoreFiltro} Compila valore filtro
\end{itemize}



\subusecase{addCustomMetadata}{Specifica filtri per Custom Metadata}
\begin{itemize}
      \item \textbf{Attore Primario}: Utente
      \item \textbf{Precondizioni}: \begin{itemize}
            \item L'Utente ha avviato l'applicazione
            \item L'Utente sta effettuando una ricerca nel DIP con filtri
      \end{itemize}
      \item \textbf{Postcondizioni}: Vengono specificati i filtri per i custom metadata e utilizzati per la generazione del risultato.
      \item \textbf{Flusso Principale}:
      \begin{enumerate}
            \item L'Utente, specifica uno o più filtri per i metadati custom (\ref{addFiltroMetadatoCustom}).
      \end{enumerate}
      \item \textbf{Inclusioni}: \ref{addFiltroMetadatoCustom} Specifica filtro Metadato Custom
      \item \textbf{Nota}: Per costruzione aziendale i custom metadata sono tutti stringhe e quindi l'inserimento e la validazione seguono le regole dei campi di testo liberi.
\end{itemize}

\subsubusecase{addFiltroMetadatoCustom}{Specifica filtro Metadato Custom}
\begin{itemize}
      \item \textbf{Attore Primario}: Utente
      \item \textbf{Precondizioni}: \begin{itemize}
            \item L'Utente ha avviato l'applicazione
            \item L'Utente sta effettuando una ricerca nel DIP con filtri
            \end{itemize}
      \item \textbf{Postcondizioni}: Il filtro aggiunto dall'Utente è parametro per la ricerca con filtri e per la generazione del risultato.
      \item \textbf{Flusso Principale}:
      \begin{enumerate}
            \item L'Utente specifica il nome del metadato custom. (\ref{compilaValoreFiltro}).
            \item L'Utente specifica il valore del metadato custom per cui ricercare. (\ref{compilaValoreFiltro}).
            \item Il sistema imposta questi valori come parametro per la ricerca.
      \end{enumerate}
      \item \textbf{Inclusioni}: \ref{compilaValoreFiltro} Compila valore filtro
\end{itemize}

\usecase{addSoggetto}{Aggiunta Soggetto ad un filtro}
\begin{itemize}
      \item \textbf{Attore Primario}: Utente
      \item \textbf{Precondizioni}: 
      \begin{itemize}
            \item L'Utente ha avviato l'applicazione
            \item L'Utente sta effettuando una ricerca nel DIP con filtri
            \item L'Utente sta specificando un filtro che prevede l'aggiunta di soggetti
      \end{itemize}
      \item \textbf{Postcondizioni}: Viene specificato un soggetto e aggiunto al filtro che l'Utente sta specificando.
      \item \textbf{Flusso Principale}:
      \begin{enumerate}
            \item L'utente specifica il ruolo del soggetto tra quelli disponibili per il tipo di documento scelto. (\ref{specificaRuoloSoggetto}).
            \item L'utente specifica il tipo di soggetto tra quelli disponibili per il ruolo selezionato. (\ref{specificaTipoSoggetto}).
            \item L'utente specifica i valori dei campi per cui ricercare, relativi al tipo di soggetto selezionato (\ref{addDettagliSoggetto}).
            \item Viene aggiunto un soggetto con le specifiche indicate al filtro che l'Utente sta specificando.
      \end{enumerate}
      \item \textbf{Inclusioni}: 
      \begin{itemize}
            \item \ref{specificaRuoloSoggetto} Specifica ruolo del soggetto
            \item \ref{specificaTipoSoggetto} Specifica tipo di soggetto
            \item \ref{addDettagliSoggetto} Specifica dettagli soggetto
      \end{itemize}
\end{itemize}

\subusecase{specificaRuoloSoggetto}{Specifica ruolo del soggetto}
\begin{itemize}
      \item \textbf{Attore Primario}: Utente
      \item \textbf{Precondizioni}: 
      \begin{itemize}
            \item L'Utente ha avviato l'applicazione
            \item L'Utente sta effettuando una ricerca nel DIP con filtri
            \item L'Utente ha specificato il filtro Tipo di Documento
            \item L'Utente sta aggiungendo un soggetto ad un filtro
      \end{itemize}
      \item \textbf{Postcondizioni}: Viene specificato il ruolo del soggetto che l'utente sta aggiungendo, in base alla specifica del filtro Tipo di Documento.
      \item \textbf{Flusso Principale}:
            \begin{enumerate}
                  \item L'Utente visualizza i ruoli disponibili in base alla specifica del filtro Tipo di Documento. (\ref{visualizzaRuoliPerTipoDocumento}).
                  \item L'Utente seleziona il Ruolo tra quelli disponibili per il tipo di documento selezionato.
                  \item Il sistema salva il ruolo selezionato per il soggetto che l'Utente sta aggiungendo.
            \end{enumerate}
      \item \textbf{Inclusioni}: \ref{visualizzaRuoliPerTipoDocumento} Visualizza ruoli disponibili per tipo di documento
\end{itemize}

\subsubusecase{visualizzaRuoliPerTipoDocumento}{Visualizza ruoli disponibili per tipo di documento}
\begin{itemize}
      \item \textbf{Attore Primario}: Utente
      \item \textbf{Precondizioni}: 
      \begin{itemize}
            \item L'Utente ha avviato l'applicazione
            \item L'Utente ha specificato il filtro Tipo di Documento
      \end{itemize}
      \item \textbf{Postcondizioni}: Vengono mostrati i ruoli disponibili in base alla specifica del filtro Tipo di Documento.
      \item \textbf{Flusso Principale}:
            \begin{enumerate}
                  \item Il sistema mostra per ogni ruolo disponibile in base alla specifica del filtro Tipo di Documento:
                        \begin{itemize}
                              \item Nome del Ruolo (\ref{visualizzaNomeRuolo})
                        \end{itemize}
            \end{enumerate}
      \item \textbf{Inclusioni}: \ref{visualizzaNomeRuolo} Visualizza nome del ruolo
\end{itemize}

\deepusecase{visualizzaNomeRuolo}{Visualizza nome del ruolo}
\begin{itemize}
      \item \textbf{Attore Primario}: Utente
      \item \textbf{Precondizioni}:
      \item \begin{itemize}
            \item L'Utente ha avviato l'applicazione
            \item L'Utente sta visualizzando una lista di ruoli.
      \end{itemize}
      \item \textbf{Postcondizioni}: L'Utente visualizza il nome del ruolo.
      \item \textbf{Flusso Principale}:
      \item \begin{enumerate}
            \item Il sistema mostra a video il nome del ruolo.
      \end{enumerate}
\end{itemize}

\subusecase{specificaTipoSoggetto}{Specifica tipo di soggetto}
\begin{itemize}
      \item \textbf{Attore Primario}: Utente
      \item \textbf{Precondizioni}: 
      \begin{itemize}
            \item L'Utente ha avviato l'applicazione
            \item L'Utente sta effettuando una ricerca nel DIP con filtri
            \item L'Utente sta aggiungendo un soggetto ad un filtro
            \item L'Utente ha specificato il ruolo del soggetto
      \end{itemize}
      \item \textbf{Postcondizioni}: Viene specificato il tipo del soggetto che l'utente sta aggiungendo, in base al ruolo selezionato per il soggetto che l'Utente sta aggiungendo.
      \item \textbf{Flusso Principale}:
            \begin{enumerate}
                  \item L'Utente visualizza i tipi di soggetto disponibili in base al ruolo selezionato per il soggetto che l'Utente sta aggiungendo (\ref{visualizzaTipiSoggettoPerRuolo}).
                  \item L'Utente seleziona il Tipo di Soggetto tra quelli disponibili per il ruolo selezionato (per il soggetto che l'Utente sta aggiungendo).
                  \item Il sistema salva il tipo di soggetto selezionato per il soggetto che l'Utente sta aggiungendo.
            \end{enumerate}
      \item \textbf{Inclusioni}: \ref{visualizzaTipiSoggettoPerRuolo} Visualizza tipi di soggetto disponibili per ruolo
\end{itemize}

\subsubusecase{visualizzaTipiSoggettoPerRuolo}{Visualizza tipi di soggetto disponibili per ruolo}
\begin{itemize}
      \item \textbf{Attore Primario}: Utente
      \item \textbf{Precondizioni}: 
      \begin{itemize}
            \item L'Utente ha avviato l'applicazione
            \item L'Utente sta aggiungendo un soggetto ad un filtro, di cui ne ha specificato il ruolo (\ref{specificaRuoloSoggetto})
      \end{itemize}
      \item \textbf{Postcondizioni}: Vengono mostrati i tipi di soggetto disponibili per il ruolo selezionato.
      \item \textbf{Flusso Principale}:
            \begin{enumerate}
                  \item Il sistema mostra, per ogni tipo di soggetto disponibile per il ruolo selezionato, il Nome del Tipo di Soggetto (\ref{visualizzaNomeTipoSoggetto})
            \end{enumerate}
      \item \textbf{Inclusioni}: \ref{visualizzaNomeTipoSoggetto} Visualizza nome del tipo di soggetto
\end{itemize}

\deepusecase{visualizzaNomeTipoSoggetto}{Visualizza nome del tipo di soggetto}
\begin{itemize}
      \item \textbf{Attore Primario}: Utente
      \item \textbf{Precondizioni}:
      \item \begin{itemize}
            \item L'Utente ha avviato l'applicazione
            \item L'Utente sta visualizzando una lista di tipi di soggetto.
      \end{itemize}
      \item \textbf{Postcondizioni}: L'Utente visualizza il nome del tipo di soggetto.
      \item \textbf{Flusso Principale}:
      \item \begin{enumerate}
            \item Il sistema mostra a video il nome del tipo di soggetto.
      \end{enumerate}
\end{itemize}

\subusecase{addDettagliSoggetto}{Specifica dettagli soggetto}
\begin{itemize}
      \item \textbf{Attore Primario}: Utente
      \item \textbf{Precondizioni}: 
      \begin{itemize}
            \item L'Utente ha avviato l'applicazione
            \item L'Utente sta effettuando una ricerca nel DIP con filtri
            \item L'Utente sta aggiungendo un soggetto, di cui ha specificato il tipo, ad un filtro. (\ref{specificaTipoSoggetto})
      \end{itemize}
      \item \textbf{Postcondizioni}: Le specifiche aggiunte dall'Utente sono salvate per il soggetto che sta aggiungendo.
      \item \textbf{Flusso Principale}:
      \begin{enumerate}
            \item L'Utente inserisce i dettagli del tipo di soggetto specificato. (da \ref{addDettagliPAI} a \ref{addDettagliSW})
            \item Il sistema salva i dettagli inseriti per il soggetto che l'utente sta aggiungendo.
      \end{enumerate}
\end{itemize}

\subsubusecase{addDettagliPAI}{Specifica dettagli soggetto tipo: PAI}
\begin{itemize}
      \item \textbf{Attore Primario}: Utente
      \item \textbf{Precondizioni}: 
      \begin{itemize}
            \item L'Utente ha avviato l'applicazione
            \item L'Utente sta effettuando una ricerca nel DIP con filtri
            \item L'Utente sta aggiungendo un soggetto di tipo: PAI ad un filtro.
      \end{itemize}
      \item \textbf{Postcondizioni}:  Le specifiche aggiunte dall'Utente sono salvate per il soggetto che sta aggiungendo.
      \item \textbf{Flusso Principale}:
            \begin{enumerate}
                  \item L'Utente inserisce i valori per i campi:
                        \begin{itemize}
                              \item Denominazione Amministrazione/ Codice IPA (\ref{compilaValoreFiltro})
                              \item Denominazione Amministrazione AOO/ Codice IPA AOO (\ref{compilaValoreFiltro})
                              \item Denominazione Amministrazione UOR/ Codice IPA UOR (\ref{compilaValoreFiltro})
                              \item Indirizzi Digitali di Riferimento (\ref{compilaValoreFiltro})
                        \end{itemize}
                  \item Il sistema salva i dettagli inseriti per il soggetto che l'utente sta aggiungendo.
            \end{enumerate}
      \item \textbf{Inclusioni}: \ref{compilaValoreFiltro} Compila valore filtro
      \item \textbf{Specializza}: \ref{addDettagliSoggetto} Specifica dettagli soggetto
\end{itemize}

\subsubusecase{addDettagliPAE}{Specifica dettagli soggetto tipo: PAE}
\begin{itemize}
      \item \textbf{Attore Primario}: Utente
      \item \textbf{Precondizioni}: \begin{itemize}
            \item L'Utente ha avviato l'applicazione
            \item L'Utente sta effettuando una ricerca nel DIP con filtri
            \item L'Utente sta aggiungendo un soggetto di tipo: PAE ad un filtro.
      \end{itemize}
      \item \textbf{Postcondizioni}:  Le specifiche aggiunte dall'Utente sono salvate per il soggetto che sta aggiungendo.
      \item \textbf{Flusso Principale}:
            \begin{enumerate}
                  \item L'Utente inserisce i valori per i campi:
                        \begin{itemize}
                              \item Denominazione Amministrazione (\ref{compilaValoreFiltro})
                              \item Denominazione Ufficio (\ref{compilaValoreFiltro})
                              \item Indirizzi Digitali di Riferimento (\ref{compilaValoreFiltro})
                        \end{itemize}
                  \item Il sistema salva i dettagli inseriti per il soggetto che l'utente sta aggiungendo.
            \end{enumerate}
      \item \textbf{Inclusioni}: \ref{compilaValoreFiltro} Compila valore filtro
      \item \textbf{Specializza}: \ref{addDettagliSoggetto} Specifica dettagli soggetto
\end{itemize}

\subsubusecase{addDettagliAS}{Specifica dettagli soggetto tipo: AS}
\begin{itemize}
      \item \textbf{Attore Primario}: Utente
      \item \textbf{Precondizioni}:\begin{itemize}
            \item L'Utente ha avviato l'applicazione
            \item L'Utente sta effettuando una ricerca nel DIP con filtri
            \item L'Utente sta aggiungendo un soggetto di tipo: AS ad un filtro.
            \end{itemize}
      \item \textbf{Postcondizioni}:  Le specifiche aggiunte dall'Utente sono salvate per il soggetto che sta aggiungendo.
      \item \textbf{Flusso Principale}:
            \begin{enumerate}
                  \item L'Utente inserisce i valori per i campi:
                        \begin{itemize}
                              \item Cognome (\ref{compilaValoreFiltro})
                              \item Nome (\ref{compilaValoreFiltro})
                              \item Codice Fiscale (\ref{compilaValoreFiltro})
                              \item Denominazione Amministrazione/ Codice IPA (\ref{compilaValoreFiltro})
                              \item Denominazione Amministrazione AOO/ Codice IPA AOO (\ref{compilaValoreFiltro})
                              \item Denominazione Amministrazione UOR/ Codice IPA UOR (\ref{compilaValoreFiltro})
                              \item Indirizzi Digitali di Riferimento (\ref{compilaValoreFiltro})
                        \end{itemize}
                  \item Il sistema salva i dettagli inseriti per il soggetto che l'utente sta aggiungendo.
            \end{enumerate}
      \item \textbf{Inclusioni}: \ref{compilaValoreFiltro} Compila valore filtro
      \item \textbf{Specializza}: \ref{addDettagliSoggetto} Specifica dettagli soggetto
\end{itemize}

\subsubusecase{addDettagliPG}{Specifica dettagli soggetto tipo: PG}
\begin{itemize}
      \item \textbf{Attore Primario}: Utente
      \item \textbf{Precondizioni}: \begin{itemize}
            \item L'Utente ha avviato l'applicazione
            \item L'Utente sta effettuando una ricerca nel DIP con filtri
            \item L'Utente sta aggiungendo un soggetto di tipo: PG ad un filtro.
      \end{itemize}
      \item \textbf{Postcondizioni}:  Le specifiche aggiunte dall'Utente sono salvate per il soggetto che sta aggiungendo.
      \item \textbf{Flusso Principale}:
            \begin{enumerate}
                  \item L'Utente inserisce i valori per i campi:
                        \begin{itemize}
                              \item Denominazione Organizzazione (\ref{compilaValoreFiltro})
                              \item Codice fiscale / Partita Iva (\ref{compilaValoreFiltro})
                              \item Denominazione Ufficio (\ref{compilaValoreFiltro})
                              \item Indirizzi Digitali di Riferimento (\ref{compilaValoreFiltro})
                        \end{itemize}
                  \item Il sistema salva i dettagli inseriti per il soggetto che l'utente sta aggiungendo.
            \end{enumerate}
      \item \textbf{Inclusioni}: \ref{compilaValoreFiltro} Compila valore filtro
      \item \textbf{Specializza}: \ref{addDettagliSoggetto} Specifica dettagli soggetto
\end{itemize}

\subsubusecase{addDettagliPF}{Specifica dettagli soggetto tipo: PF}
\begin{itemize}
      \item \textbf{Attore Primario}: Utente
      \item \textbf{Precondizioni}: \begin{itemize}
            \item L'Utente ha avviato l'applicazione
            \item L'Utente sta effettuando una ricerca nel DIP con filtri
            \item L'Utente sta aggiungendo un soggetto di tipo: PF ad un filtro.
      \end{itemize}
      \item \textbf{Postcondizioni}:  Le specifiche aggiunte dall'Utente sono salvate per il soggetto che sta aggiungendo.
      \item \textbf{Flusso Principale}:
            \begin{enumerate}
                  \item  L'Utente inserisce i valori per i campi:
                        \begin{itemize}
                              \item Cognome (\ref{compilaValoreFiltro})
                              \item Nome (\ref{compilaValoreFiltro})
                              \item Indirizzi Digitali di Riferimento (\ref{compilaValoreFiltro})
                        \end{itemize}
                  \item Il sistema salva i dettagli inseriti per il soggetto che l'utente sta aggiungendo.
            \end{enumerate}
      \item \textbf{Inclusioni}: \ref{compilaValoreFiltro} Compila valore filtro
      \item \textbf{Specializza}: \ref{addDettagliSoggetto} Specifica dettagli soggetto
\end{itemize}

\subsubusecase{addDettagliRUP}{Specifica dettagli soggetto tipo: RUP}
\begin{itemize}
      \item \textbf{Attore Primario}: Utente
      \item \textbf{Precondizioni}: \begin{itemize}
            \item L'Utente ha avviato l'applicazione
            \item L'Utente sta effettuando una ricerca nel DIP con filtri
            \item L'Utente sta aggiungendo un soggetto di tipo: RUP ad un filtro.
      \end{itemize}
      \item \textbf{Postcondizioni}:  Le specifiche aggiunte dall'Utente sono salvate per il soggetto che sta aggiungendo.
      \item \textbf{Flusso Principale}:
            \begin{enumerate}
                  \item L'Utente inserisce i valori per i campi:
                        \begin{itemize}
                              \item Cognome (\ref{compilaValoreFiltro})
                              \item Nome (\ref{compilaValoreFiltro})
                              \item Denominazione Amministrazione/ Codice IPA (\ref{compilaValoreFiltro})
                              \item Denominazione Amministrazione AOO/Denominazione Amministrazione UOR (\ref{compilaValoreFiltro})
                              \item  Codice IPA AOO/Codice IPA UOR (\ref{compilaValoreFiltro})
                              \item Indirizzi Digitali di Riferimento (\ref{compilaValoreFiltro})
                        \end{itemize}
                  \item Il sistema salva i dettagli inseriti per il soggetto che l'utente sta aggiungendo.
            \end{enumerate}
      \item \textbf{Inclusioni}: \ref{compilaValoreFiltro} Compila valore filtro
      \item \textbf{Specializza}: \ref{addDettagliSoggetto} Specifica dettagli soggetto
\end{itemize}

\subsubusecase{addDettagliSW}{Specifica dettagli soggetto tipo: SW}
\begin{itemize}
      \item \textbf{Attore Primario}: Utente
      \item \textbf{Precondizioni}: \begin{itemize}
            \item L'Utente ha avviato l'applicazione
            \item L'Utente sta effettuando una ricerca nel DIP con filtri
            \item L'Utente sta aggiungendo un soggetto di tipo: SW ad un filtro.
      \end{itemize}
      \item \textbf{Postcondizioni}:  Le specifiche aggiunte dall'Utente sono salvate per il soggetto che sta aggiungendo.
      \item \textbf{Flusso Principale}:
            \begin{enumerate}
                  \item L'Utente inserisce la Denominazione Sistema. (\ref{compilaValoreFiltro})
                  \item Il sistema salva i dettagli inseriti per il soggetto che l'utente sta aggiungendo.
            \end{enumerate}
      \item \textbf{Inclusioni}: \ref{compilaValoreFiltro} Compila valore filtro
      \item \textbf{Specializza}: \ref{addDettagliSoggetto} Specifica dettagli soggetto
\end{itemize}

\usecase{visualizzaNomeCampo}{Visualizza nome del campo}
\begin{itemize}
      \item \textbf{Attore Primario}: Utente
      \item \textbf{Precondizioni}: 
      \begin{itemize}
            \item L'Utente ha avviato l'applicazione
            \item L'Utente sta effettuando una ricerca nel DIP con filtri
            \item L'utente sta visualizzando una lista di campi
      \end{itemize}
      \item \textbf{Postcondizioni}:  L'Utente visualizza il nome del campo
      \item \textbf{Flusso Principale}:
      \begin{enumerate}
            \item Il sistema mostra a video il nome del campo
      \end{enumerate}
\end{itemize}

\usecase{visualizzaRisultati}{Visualizzazione Risultati di Ricerca} \label{risultatiRicerca}
\begin{itemize}
      \item \textbf{Attore Primario}: Utente
      \item \textbf{Precondizioni}: \begin{itemize}
            \item L'Utente ha avviato l'applicazione
            \item Il sistema ha indicizzato i documenti presenti nel DIP
            \item L'Utente ha eseguito una ricerca (tramite \ref{ricercaDIP} e sue specializzazioni).
      \end{itemize}
      \item \textbf{Postcondizioni}: I risultati della ricerca sono presentati all'Utente come lista di elementi.
      \item \textbf{Flusso Principale}:
      \begin{enumerate}
            \item L'Utente visualizza un elenco di documenti o aggregazioni che corrispondono ai criteri di ricerca.
            \item Per ogni elemento, vengono mostrate le informazioni principali (Nome, Data, Tipo) (\ref{infoRisultatiRicerca})
      \end{enumerate}
      \item \textbf{Inclusioni}: \ref{infoRisultatiRicerca} Informazioni Elenco Elementi
\end{itemize}

\subusecase{infoRisultatiRicerca}{Informazioni Elenco Elementi}
\begin{itemize}
      \item \textbf{Attore Primario}: Utente
      \item \textbf{Precondizioni}: 
      \begin{itemize}
            \item L'Utente ha avviato l'applicazione
            \item Il sistema ha indicizzato i documenti presenti nel DIP
            \item L'Utente ha eseguito una ricerca (tramite \ref{ricercaDIP} e sue specializzazioni).
            \item La ricerca ha prodotto dei risultati.
      \end{itemize}
      \item \textbf{Postcondizioni}: I risultati della ricerca sono presentati all'Utente come lista di elementi.
      \item \textbf{Flusso Principale}:
      \begin{enumerate}
            \item L'Utente visualizza il nome del documento o dell'aggregazione (\ref{visualizzazioneNomeDocumento}).
            \item L'Utente visualizza la data di creazione o di registrazione del documento o dell'aggregazione (\ref{visualizzaDataRegistrazioneDocumento}).
            \item L'Utente visualizza il tipo di elemento tra Documento, Aggregazione, Processo e Classe Documentale.
      \end{enumerate}
      \item \textbf{Inclusioni}:
      \begin{itemize}
            \item \ref{visualizzazioneNomeDocumento} Visualizzazione Nome Documento
            \item \ref{visualizzaDataRegistrazioneDocumento} Visualizzazione Data Registrazione Documento
      \end{itemize}
\end{itemize}

\usecase{nessunRisultato}{Nessun Risultato}
\begin{itemize}
      \item \textbf{Attore Primario}: Utente
      \item \textbf{Precondizioni}: \begin{itemize}
            \item L'Utente ha avviato l'applicazione
            \item Il sistema ha indicizzato i documenti presenti nel DIP
            \item L'Utente ha eseguito una ricerca (tramite \ref{ricercaDIP} e sue specializzazioni).
      \end{itemize}
      \item \textbf{Postcondizioni}: Il sistema informa l'Utente che non sono stati trovati risultati.
      \item \textbf{Flusso Principale}:
      \begin{enumerate}
            \item Il sistema visualizza un messaggio che indica che la ricerca non ha prodotto alcun risultato.
      \end{enumerate}
\end{itemize}
      
\usecase{compilaValoreFiltro}{Compila Valore Filtro}
\begin{figure}[H]
      \centering
      \includegraphics[width=0.6\textwidth]{../assets/uml/UC24.png}
      \caption{UC24 - Compila valore filtro}
      \label{fig:uc_compilaValoreFiltro}
\end{figure}
\begin{itemize}
      \item \textbf{Attore Primario}: Utente
      \item \textbf{Precondizioni}: 
      \begin{itemize}
            \item L'Utente ha avviato l'applicazione
            \item L'Utente sta effettuando una ricerca nel DIP con filtri
            \item L'Utente ha selezionato un filtro da compilare tra quelli disponibili.
            \item L'Utente sta specificando il valore per un metadato del filtro selezionato.
      \end{itemize}
      \item \textbf{Postcondizioni}: Il valore inserito dall'Utente viene salvato come parte del filtro selezionato.
      \item \textbf{Flusso Principale}:
      \begin{enumerate}
            \item L'Utente inserisce il valore per il metadato di cui vuole specificare il valore.
            \item Il sistema salva il valore inserito come parte del filtro selezionato.
      \end{enumerate}
      \item \textbf{Flusso Alternativo}:
      \begin{itemize}
            \item Il valore non rispetta il formato corretto (\ref{formatoNonCorretto}).
      \end{itemize}
      \item \textbf{Estensioni}: \ref{formatoNonCorretto} Formato Non Corretto
\end{itemize}

\usecase{formatoNonCorretto}{Formato Non Corretto}
\begin{itemize}
      \item \textbf{Attore Primario}: Utente
      \item \textbf{Precondizioni}: \begin{itemize}
            \item L'Utente ha avviato l'applicazione
            \item L'Utente sta effettuando una ricerca nel DIP con filtri
            \item L'Utente ha selezionato un filtro da compilare tra quelli disponibili.
            \item L'Utente sta specificando il valore per un metadato del filtro selezionato.
      \end{itemize}
      \item \textbf{Postcondizioni}: Il sistema informa l'Utente che il valore inserito non rispetta il formato corretto.
      \item \textbf{Flusso Principale}:
      \begin{enumerate}
            \item Il sistema ritorna un messaggio che indica che il valore inserito non rispetta il formato corretto.
      \end{enumerate}
\end{itemize}