\subsubsection{UC-2 Ricerca nel DIP}
\begin{itemize}
      \item \textbf{Attore Primario}: Utente
      \item \textbf{Precondizioni}:
            \begin{itemize}
                  \item L'utente ha avviato l'applicazione
            \end{itemize}
      \item \textbf{Postcondizioni}:
            \begin{itemize}
                  \item Vengono prodotti i risultati di ricerca
            \end{itemize}
\end{itemize}

\subsubsection{UC-2a Ricerca nel DIP con filtri}
\begin{itemize}
      \item \textbf{Attore Primario}: Utente
      \item \textbf{Precondizioni}:
            \begin{itemize}
                  \item L'utente ha avviato l'applicazione
            \end{itemize}
      \item \textbf{Postcondizioni}:
            \begin{itemize}
                  \item Vengono prodotti i risultati di ricerca
            \end{itemize}
      \item \textbf{Flusso Principale}:
            \begin{itemize}
                  \item L'utente aggiunge filtri come parametri per la ricerca(UC-3)
            \end{itemize}
      \item \textbf{Flusso alternativo}:
\end{itemize}

\subsubsection{UC-2b Ricerca semantica}
\begin{itemize}
      \item \textbf{Attore Primario}: Utente
      \item \textbf{Precondizioni}:
            \begin{itemize}
                  \item L'utente ha avviato l'applicazione
                  \item Il sistema dispone di una connessione internet
            \end{itemize}
      \item \textbf{Postcondizioni}:
            \begin{itemize}
                  \item Vengono prodotti i risultati di ricerca
            \end{itemize}
\end{itemize}

\subsubsection{UC-2 - Ricerca Nel DIP}
\begin{itemize}
      \item \textbf{Attore Primario}: Utente
      \item \textbf{Precondizioni}: L'Utente ha avviato il sistema DIP-Reader.
      \item \textbf{Postcondizioni}: Viene visualizzato l'elenco degli elementi (Classi Documentali, Processi, Documenti) corrispondenti ai filtri applicati.
      \item \textbf{Flusso Principale}:
            \begin{enumerate}
                  \item L'Utente inserisce i filtri per la ricerca:
                        \begin{itemize}
                              \item Ricercare Classe Documentale con filtri (UC-2.1)
                              \item Ricercare Processo con filtri (UC-2.2)
                              \item Ricercare Documento con filtri (UC-2.3)
                        \end{itemize}
                  \item Il sistema mostra i risultati della ricerca (UC-2.4).
            \end{enumerate}
      \item \textbf{Flusso Alternativo (Nessun risultato)}: La ricerca non produce risultati e il sistema informa l'Utente (UC-2.4.1).
\end{itemize}

\subsubsection{UC-2.1 - Ricercare Classe Documentale con filtri}
\begin{itemize}
      \item \textbf{Attore Primario}: Utente
      \item \textbf{Precondizioni}: L'Utente sta visualizzando la lista delle Classi Documentali.
      \item \textbf{Postcondizioni}: Viene visualizzato l'elenco delle Classi Documentali corrispondenti ai filtri applicati.
      \item \textbf{Flusso Principale}:
            \begin{enumerate}
                  \item L'Utente inserisce i filtri per la ricerca secondo i metadati delle Classi
                        Documentali.
                  \item Il sistema mostra i risultati della ricerca (UC-2.4).
            \end{enumerate}
      \item \textbf{Flusso Alternativo (Nessun risultato)}: La ricerca non produce risultati e il sistema informa l'Utente (UC-2.4.1).
\end{itemize}

\subsubsection{UC-2.2 - Ricercare Processo con filtri}
\begin{itemize}
      \item \textbf{Attore Primario}: Utente
      \item \textbf{Precondizioni}: L'Utente sta visualizzando la lista dei Processi.
      \item \textbf{Postcondizioni}: Viene visualizzato l'elenco dei Processi corrispondenti ai filtri applicati.
      \item \textbf{Flusso Principale}:
            \begin{enumerate}
                  \item L'Utente inserisce i filtri per la ricerca secondo i metadati dei Processi.
                  \item Il sistema mostra i risultati della ricerca (UC-2.4).
            \end{enumerate}
      \item \textbf{Flusso Alternativo (Nessun risultato)}: La ricerca non produce risultati e il sistema informa l'Utente (UC-2.4.1).
\end{itemize}

\subsubsection{UC-3 Aggiunta filtri per Ricerca Documento}
\begin{itemize}
      \item \textbf{Attore Primario}: Utente
      \item \textbf{Precondizioni}: L'Utente ha avviato l'applicazione.
      \item \textbf{Postcondizioni}:  Il filtro aggiunto dall'utente è parametro per la ricerca e per la generazione del risultato
      \item \textbf{Flusso Principale}:
            \begin{enumerate}
                  \item L'Utente imposta uno o più filtri per la ricerca tra campi comuni (UC-3.1)
                  \item L'Utente imposta uno o più filtri per la ricerca tra campi specifici per tipo
                        documentale (UC-3.2).
            \end{enumerate}
\end{itemize}

\subsubsection{UC-3.1 - Aggiunta filtri Comuni}
\begin{itemize}
      \item \textbf{Attore Primario}: Utente
      \item \textbf{Precondizioni}:  L'Utente ha avviato l'applicazione.
      \item \textbf{Postcondizioni}:  Il filtro aggiunto dall'utente è parametro per la ricerca e per la generazione del risultato
      \item \textbf{Flusso Principale}:
            \begin{enumerate}
                  \item L'Utente seleziona uno o più campi comuni su cui basare la ricerca (da UC-3.1.1
                        a UC-3.1.5).
                  \item L'Utente compila i valori per i campi selezionati.
                  \item Il sistema utilizza questi parametri per la ricerca.
            \end{enumerate}
\end{itemize}

\subsubsection{UC-3.1.1 - Aggiunta filtro Chiave Descrittiva}
\begin{itemize}
      \item \textbf{Attore Primario}: Utente
      \item \textbf{Precondizioni}: L'Utente sta aggiungendo campi comuni come parametri alla ricerca.
      \item \textbf{Postcondizioni}:  Il filtro aggiunto dall'utente è parametro per la ricerca e per la generazione del risultato
      \item \textbf{Flusso Principale}:
            \begin{enumerate}
                  \item L'Utente digita nella casella di ricerca i valori per:
                        \begin{itemize}
                              \item Chiave Descrittiva
                              \item Oggetto
                              \item Parole Chiave.
                        \end{itemize}
            \end{enumerate}
\end{itemize}

\subsubsection{UC-3.1.2 - Aggiunta filtro Classificazione}
\begin{itemize}
      \item \textbf{Attore Primario}: Utente
      \item \textbf{Precondizioni}: L'Utente sta effettuando una ricerca per campi comuni.
      \item \textbf{Postcondizioni}:  Il filtro aggiunto dall'utente è parametro per la ricerca e per la generazione del risultato.
      \item \textbf{Flusso Principale}:
            \begin{enumerate}
                  \item L'Utente compila i parametri di Classificazione tra:
                        \begin{itemize}
                              \item Indice di Classificazione
                              \item Descrizione
                              \item Piano di Fascicolo
                        \end{itemize}
            \end{enumerate}
\end{itemize}

\subsubsection{UC-3.1.3 - Aggiunta filtro Tempo di Conservazione}
\begin{itemize}
      \item \textbf{Attore Primario}: Utente
      \item \textbf{Precondizioni}: L'Utente sta effettuando una ricerca per campi comuni.
      \item \textbf{Postcondizioni}:  Il filtro aggiunto dall'utente è parametro per la ricerca e per la generazione del risultato.
      \item \textbf{Flusso Principale}:
            \begin{enumerate}
                  \item L'Utente digita nella casella di ricerca il tempo di conservazione o seleziona
                        "Perenne".
            \end{enumerate}
\end{itemize}

\subsubsection{UC-3.1.4 - Aggiunta filtro Note}
\begin{itemize}
      \item \textbf{Attore Primario}: Utente
      \item \textbf{Precondizioni}: L'Utente sta effettuando una ricerca per campi comuni.
      \item \textbf{Postcondizioni}:  Il filtro aggiunto dall'utente è parametro per la ricerca e per la generazione del risultato.
      \item \textbf{Flusso Principale}:
            \begin{enumerate}
                  \item L'Utente digita nella casella di ricerca il testo della Nota.
            \end{enumerate}
\end{itemize}

\subsubsection{UC-3.1.5 - Aggiunta filtro soggetti?}
\begin{itemize}
      \item \textbf{Attore Primario}: Utente
      \item \textbf{Precondizioni}: L'Utente sta effettuando una ricerca per campi comuni.
      \item \textbf{Postcondizioni}:  Il filtro aggiunto dall'utente è parametro per la ricerca e per la generazione del risultato.
      \item \textbf{Flusso Principale}:
            \begin{enumerate}
                  \item L'Utente digita nella casella di ricerca il testo della Nota.
            \end{enumerate}
\end{itemize}

\subsubsection{UC-3.1.5.1 - Aggiunta filtro soggetti per Documento Informatico}
\begin{itemize}
      \item \textbf{Attore Primario}: Utente
      \item \textbf{Precondizioni}: L'Utente ha scelto come tipo di documento il Documento Informatico per l'aggiunta del filtro di soggetti.
      \item \textbf{Postcondizioni}:  Il filtro aggiunto dall'utente è parametro per la ricerca e per la generazione del risultato.
      \item \textbf{Flusso Principale}:
            \begin{enumerate}
                  \item L'Utente sceglie il Ruolo e i sottocampi specifici tra:
                        \begin{itemize}
                              \item Assegnatario (AS)
                              \item Autore (PF, PG, PAI valido solo nei flussi in entrata)
                              \item Destinatario (PF, PG, PAI valido solo come mittente nei flussi in entrata, come
                                    destinatario nei flussi in uscita, PAE valido solo come mittente nei flussi in
                                    entrata, come destinatario nei flussi in uscita)
                              \item Mittente (PF, PG, PAI valido solo come mittente nei flussi in entrata, come
                                    destinatario nei flussi in uscita, PAE valido solo come mittente nei flussi in
                                    entrata, come destinatario nei flussi in uscita)
                              \item Operatore (PF)
                              \item Produttore (SW)
                              \item RGD (PF)
                              \item RSP (PF)
                              \item Soggetto che effettua la registrazione (PF, PG)
                              \item Altro(PF, PG, PAI valido solo come mittente nei flussi in entrata, come
                                    destinatario nei flussi in uscita, PAE valido solo come mittente nei flussi in
                                    entrata, come destinatario nei flussi in uscita)
                        \end{itemize}
                  \item Per ogni ruolo, l'Utente può specificare ulteriori dettagli (UC-3.3).
            \end{enumerate}
\end{itemize}

\subsubsection{UC-3.1.5.2 - Aggiunta filtro soggetti per Documento Amministrativo informatico}
\begin{itemize}
      \item \textbf{Attore Primario}: Utente
      \item \textbf{Precondizioni}: L'Utente ha scelto come tipo di documento il Documento Amministrativo Informatico per l'aggiunta del filtro di soggetti.
      \item \textbf{Postcondizioni}:  Il filtro aggiunto dall'utente è parametro per la ricerca e per la generazione del risultato.
      \item \textbf{Flusso Principale}:
            \begin{enumerate}
                  \item L'Utente sceglie il Ruolo e i sottocampi specifici tra:
                        \begin{itemize}
                              \item Amministrazione che effettua la registrazione (PAI)
                              \item Assegnatario (AS)
                              \item Autore (PF, PG valido nei flussi di entrata, PAI, PAE valido nel flussi in entrata)
                              \item Destinatario (PF, PG, PAI, PAE)
                              \item Mittente (PF, PG, PAI, PAE)
                              \item Operatore (PF)
                              \item Produttore (SW)
                              \item RGD (PF)
                              \item RSP (PF)
                              \item RUP (RUP)
                        \end{itemize}
                  \item Per ogni ruolo, l'Utente può specificare ulteriori dettagli (UC-3.3).
            \end{enumerate}
\end{itemize}

\subsubsection{UC-3.1.5.3 - Aggiunta filtro soggetti per Aggregazione}
\begin{itemize}
      \item \textbf{Attore Primario}: Utente
      \item \textbf{Precondizioni}: L'Utente ha scelto come tipo di documento l'Aggregazione Documentale.
      \item \textbf{Postcondizioni}: I risultati sono filtrati per Soggetti.
      \item \textbf{Flusso Principale}:
            \begin{enumerate}
                  \item L'Utente sceglie il Ruolo e i sottocampi specifici tra:
                        \begin{itemize}
                              \item Amministrazione titolare (PAI).
                              \item Amministrazioni partecipanti (PAI o PAE).
                              \item Assegnatario (AS).
                              \item Soggetto intestatario persona fisica (PF).
                              \item Soggetto intestatario persona giuridica (PG, PAI o PAE).
                              \item RUP.
                        \end{itemize}
                  \item Per ogni ruolo, l'Utente può specificare ulteriori dettagli (UC-3.3).
            \end{enumerate}
\end{itemize}

\subsubsection{UC-3.2 - Aggiunta filtri per Tipo Documentale}
\begin{itemize}
      \item \textbf{Attore Primario}: Utente
      \item \textbf{Precondizioni}: L'Utente ha avviato l'applicazione.
      \item \textbf{Postcondizioni}:  Il filtro aggiunto dall'utente è parametro per la ricerca e per la generazione del risultato.
      \item \textbf{Flusso Principale}:
            \begin{enumerate}
                  \item L'Utente seleziona il tipo di documento da cercare tra: \begin{itemize}
                              \item Documento Informatico o Documento Amministrativo Informatico
                              \item Aggregazione Documentale
                        \end{itemize}   .
                  \item L'Utente può aggiungere i filtri specifici per il tipo selezionato (descritto
                        negli UC da 3.2.1 a 3.2.3).
            \end{enumerate}
\end{itemize}

\subsubsection{UC-3.2.1 - Aggiunta filtri per Documento Informatico e Amministrativo Informatico}
\begin{itemize}
      \item \textbf{Attore Primario}: Utente
      \item \textbf{Precondizioni}: L'Utente sta effettuando una ricerca per campi specifici al tipo I/AI.
      \item \textbf{Postcondizioni}:  Il filtro aggiunto dall'utente è parametro per la ricerca e per la generazione del risultato.
      \item \textbf{Flusso Principale}:
            \begin{enumerate}
                  \item L'Utente può aggiungere i filtri specifici per il tipo Documento Informatico e
                        Amministrativo Informatico (descritto negli UC da 3.2.1.1 a 3.2.1.10).
            \end{enumerate}
\end{itemize}

\paragraph{UC-3.2.1.1 - Aggiunta filtro Dati di Registrazione}
\begin{itemize}
      \item \textbf{Attore Primario}: Utente
      \item \textbf{Precondizioni}: L'Utente sta effettuando una ricerca per campi specifici al tipo I/AI.
      \item \textbf{Postcondizioni}:  Il filtro aggiunto dall'utente è parametro per la ricerca e per la generazione del risultato.
      \item \textbf{Flusso Principale}:
            \begin{enumerate}
                  \item L'Utente sceglie la Tipologia di Flusso tra: \begin{itemize}
                              \item Uscita
                              \item Entrata
                              \item Interno.
                        \end{itemize}
                  \item L'Utente sceglie il Tipo di Registro tra: \begin{itemize}
                              \item Nessuno,
                              \item Protocollo Ordinario/Protocollo di Emergenza
                              \item Repertorio/Registro.
                        \end{itemize}
                  \item L'Utente inserisce la Data/Ora di Registrazione (nel caso di un documento
                        protocollato tali parametri fanno riferimento alla protocollazione).
                  \item L'Utente inserisce il codice identificativo del Registro.
            \end{enumerate}
\end{itemize}

\paragraph{UC-3.2.1.2 - Aggiunta filtro Tipologia Documentale}
\begin{itemize}
      \item \textbf{Attore Primario}: Utente
      \item \textbf{Precondizioni}: L'Utente sta effettuando una ricerca per campi specifici al tipo I/AI.
      \item \textbf{Postcondizioni}:  Il filtro aggiunto dall'utente è parametro per la ricerca e per la generazione del risultato.
      \item \textbf{Flusso Principale}:
            \begin{enumerate}
                  \item L'Utente inserisceil nome della Tipologia Documentale (fatture, delibere,
                        determine).
            \end{enumerate}
\end{itemize}

\paragraph{UC-3.2.1.3 - Aggiunta filtro Modalità di Formazione}
\begin{itemize}
      \item \textbf{Attore Primario}: Utente
      \item \textbf{Precondizioni}: L'Utente sta effettuando una ricerca per campi specifici al tipo I/AI.
      \item \textbf{Postcondizioni}:  Il filtro aggiunto dall'utente è parametro per la ricerca e per la generazione del risultato.
      \item \textbf{Flusso Principale}:
            \begin{enumerate}
                  \item L'Utente sceglie la Modalità di Formazione tra:
                        \begin{itemize}
                              \item Creazione di un documento informatico ex novo
                              \item Acquisizione di un documento informatico da altro documento informatico o da
                                    supporto informatico
                              \item Memorizzazione su supporto informatico in formato digitale
                              \item Generazione o raggruppamento in forma statica
                        \end{itemize}
            \end{enumerate}
\end{itemize}

\paragraph{UC-3.2.1.4 - Aggiunta filtro campo Riservato}
\begin{itemize}
      \item \textbf{Attore Primario}: Utente
      \item \textbf{Precondizioni}: L'Utente sta effettuando una ricerca per campi specifici al tipo I/AI.
      \item \textbf{Postcondizioni}:  Il filtro aggiunto dall'utente è parametro per la ricerca e per la generazione del risultato.
      \item \textbf{Flusso Principale}:
            \begin{enumerate}
                  \item L'Utente sceglie se il file è Riservato o meno.
            \end{enumerate}
\end{itemize}

\paragraph{UC-3.2.1.5 - Aggiunta filtro Identificativo di Formato}
\begin{itemize}
      \item \textbf{Attore Primario}: Utente
      \item \textbf{Precondizioni}: L'Utente sta effettuando una ricerca per campi specifici al tipo I/AI.
      \item \textbf{Postcondizioni}:  Il filtro aggiunto dall'utente è parametro per la ricerca e per la generazione del risultato.
      \item \textbf{Flusso Principale}:
            \begin{enumerate}
                  \item L'Utente sceglie la Tipologia di Formato all'interno di quelli Previsti dalle
                        linee Guida (.pdf, .xml, .docx, ecc.).
                  \item L'Utente inserisce il Nome del Prodotto utilizzato per la creazione del
                        Documento.
                  \item L'Utente inserisce il numero della Versione del Prodotto utilizzato per la
                        creazione del Documento
                  \item L'Utente inserisce il nome il Produttore del Prodotto utilizzato per la
                        creazione del Documento.
            \end{enumerate}
\end{itemize}

\paragraph{UC-3.2.1.6 - Aggiunta filtro Dati di Verifica}
\begin{itemize}
      \item \textbf{Attore Primario}: Utente
      \item \textbf{Precondizioni}: L'Utente sta effettuando una ricerca per campi specifici al tipo I/AI.
      \item \textbf{Postcondizioni}:  Il filtro aggiunto dall'utente è parametro per la ricerca e per la generazione del risultato.
      \item \textbf{Flusso Principale}:
            \begin{enumerate}
                  \item L'Utente sceglie se il file è Firmato Digitalmente o meno.
                  \item L'Utente sceglie se il file è Sigillato Elettronicamente o meno.
                  \item L'Utente sceglie se il file ha una Marcatura Temporale o meno.
                  \item L'Utente sceglie se vi è conformità copie immagine su supporto informatico o
                        meno.
            \end{enumerate}
\end{itemize}

\paragraph{UC-3.2.1.7 - Aggiunta filtro Nome del Documento}
\begin{itemize}
      \item \textbf{Attore Primario}: Utente
      \item \textbf{Precondizioni}: L'Utente sta effettuando una ricerca per campi specifici al tipo I/AI.
      \item \textbf{Postcondizioni}:  Il filtro aggiunto dall'utente è parametro per la ricerca e per la generazione del risultato.
      \item \textbf{Flusso Principale}:
            \begin{enumerate}
                  \item L'Utente inserisce il Nome del Documento.
            \end{enumerate}
\end{itemize}

\paragraph{UC-3.2.1.8 - Aggiunta filtro Versione del Documento}
\begin{itemize}
      \item \textbf{Attore Primario}: Utente
      \item \textbf{Precondizioni}: L'Utente ha selezionato "Documento Informatico" o "Documento Amministrativo Informatico".
      \item \textbf{Postcondizioni}: I risultati sono filtrati per Versione del Documento.
      \item \textbf{Flusso Principale}:
            \begin{enumerate}
                  \item L'Utente inserisce il numero la Versione del Documento.
            \end{enumerate}
\end{itemize}

\paragraph{UC-3.2.1.9 - Aggiunta filtro Identificativo del Documento Primario}
\begin{itemize}
      \item \textbf{Attore Primario}: Utente
      \item \textbf{Precondizioni}: L'Utente sta effettuando una ricerca per campi specifici al tipo I/AI.
      \item \textbf{Postcondizioni}:  Il filtro aggiunto dall'utente è parametro per la ricerca e per la generazione del risultato.
      \item \textbf{Flusso Principale}:
            \begin{enumerate}
                  \item L'Utente inserisce l'Identificativo del Documento Primario.
            \end{enumerate}
\end{itemize}

\paragraph{UC-3.2.1.10 - Aggiunta filtro Tracciature Modifiche di Documento}
\begin{itemize}
      \item \textbf{Attore Primario}: Utente
      \item \textbf{Precondizioni}: L'Utente sta effettuando una ricerca per campi specifici al tipo I/AI.
      \item \textbf{Postcondizioni}:  Il filtro aggiunto dall'utente è parametro per la ricerca e per la generazione del risultato.
      \item \textbf{Flusso Principale}:
            \begin{enumerate}
                  \item L'Utente sceglie il Tipo di modifica tra:
                        \begin{itemize}
                              \item Annullamento
                              \item Registrazione
                              \item Integrazione
                              \item Annotazione
                        \end{itemize}.
                  \item L'Utente sceglie il Soggetto che ha effettuato la modifica.
                  \item L'Utente inserisce la Data/Ora della Modifica.
                  \item L'Utente inserisce l'identificativo documento versione precedente.
            \end{enumerate}
\end{itemize}

\subsubsection{UC-3.2.2 - Aggiunta filtri per Aggregazione Documentale Informatica}
\begin{itemize}
      \item \textbf{Attore Primario}: Utente
      \item \textbf{Precondizioni}: L'Utente sta effettuando una ricerca per campi specifici al tipo Aggregazione.
      \item \textbf{Postcondizioni}:  Il filtro aggiunto dall'utente è parametro per la ricerca e per la generazione del risultato.
      \item \textbf{Flusso Principale}:
            \begin{enumerate}
                  \item L'Utente può aggiungere i filtri specifici per il tipo Documento Informatico e
                        Amministrativo Informatico (descritto negli UC da 3.2.2.1 a 3.2.2.10).
            \end{enumerate}
\end{itemize}

\paragraph{UC-3.2.2.1 - Aggiunta filtro Tipo di Aggregazione}
\begin{itemize}
      \item \textbf{Attore Primario}: Utente
      \item \textbf{Precondizioni}: L'Utente sta effettuando una ricerca per campi specifici al tipo Aggregazione.
      \item \textbf{Postcondizioni}:  Il filtro aggiunto dall'utente è parametro per la ricerca e per la generazione del risultato.
      \item \textbf{Flusso Principale}:
            \begin{enumerate}
                  \item L'Utente sceglie tra:\begin{itemize}
                              \item Fascicolo,
                              \item Serie Documentale,
                              \item Serie di Fascicoli.
                        \end{itemize}
            \end{enumerate}
\end{itemize}

\paragraph{UC-3.2.2.2 - Aggiunta filtro Id dell'Aggregazione}
\begin{itemize}
      \item \textbf{Attore Primario}: Utente
      \item \textbf{Precondizioni}: L'Utente sta effettuando una ricerca per campi specifici al tipo Aggregazione.
      \item \textbf{Postcondizioni}:  Il filtro aggiunto dall'utente è parametro per la ricerca e per la generazione del risultato.
      \item \textbf{Flusso Principale}:
            \begin{enumerate}
                  \item L'Utente inserisce l'Id dell'Aggregazione.
            \end{enumerate}
\end{itemize}

\paragraph{UC-3.2.2.3 - Aggiunta filtro Tipologia di Fascicolo}
\begin{itemize}
      \item \textbf{Attore Primario}: Utente
      \item \textbf{Precondizioni}: L'Utente sta effettuando una ricerca per campi specifici al tipo Aggregazione.
      \item \textbf{Postcondizioni}:  Il filtro aggiunto dall'utente è parametro per la ricerca e per la generazione del risultato.
      \item \textbf{Flusso Principale}:
            \begin{enumerate}
                  \item L'Utente sceglie tra: \begin{itemize}
                              \item Affare,
                              \item Attività,
                              \item Persona Fisica,
                              \item Persona Giuridica,
                              \item  Procedimento Amministrativo.
                        \end{itemize}
            \end{enumerate}
\end{itemize}

\paragraph{UC-3.2.2.4 - Aggiunta filtro Id Aggregazione Primario}
\begin{itemize}
      \item \textbf{Attore Primario}: Utente
      \item \textbf{Precondizioni}: L'Utente sta effettuando una ricerca per campi specifici al tipo Aggregazione.
      \item \textbf{Postcondizioni}:  Il filtro aggiunto dall'utente è parametro per la ricerca e per la generazione del risultato.
      \item \textbf{Flusso Principale}:
            \begin{enumerate}
                  \item L'Utente inserisce l'Id dell'Aggregazione "padre".
            \end{enumerate}
\end{itemize}

\paragraph{UC-3.2.2.5 - Aggiunta filtro Data Apertura}
\begin{itemize}
      \item \textbf{Attore Primario}: Utente
      \item \textbf{Precondizioni}: L'Utente sta effettuando una ricerca per campi specifici al tipo Aggregazione.
      \item \textbf{Postcondizioni}:  Il filtro aggiunto dall'utente è parametro per la ricerca e per la generazione del risultato.
      \item \textbf{Flusso Principale}:
            \begin{enumerate}
                  \item L'Utente inserisce la Data di Apertura.
            \end{enumerate}
\end{itemize}

\paragraph{UC-3.2.2.6 - Aggiunta filtro Data Chiusura}
\begin{itemize}
      \item \textbf{Attore Primario}: Utente
      \item \textbf{Precondizioni}: L'Utente sta effettuando una ricerca per campi specifici al tipo Aggregazione.
      \item \textbf{Postcondizioni}:  Il filtro aggiunto dall'utente è parametro per la ricerca e per la generazione del risultato.
      \item \textbf{Flusso Principale}:
            \begin{enumerate}
                  \item L'Utente inserisce la Data di Chiusura.
            \end{enumerate}
\end{itemize}

\paragraph{UC-3.2.2.7 - Aggiunta filtro Procedimento Amministrativo}
\begin{itemize}
      \item \textbf{Attore Primario}: Utente
      \item \textbf{Precondizioni}: L'Utente sta effettuando una ricerca per campi specifici al tipo Aggregazione.
      \item \textbf{Postcondizioni}:  Il filtro aggiunto dall'utente è parametro per la ricerca e per la generazione del risultato.
      \item \textbf{Flusso Principale}:
            \begin{enumerate}
                  \item L'Utente inserisce la Materia, Argomento e Struttura per i procedimenti.
                  \item L'Utente inserisce il Procedimento come denominazione.
                  \item L'Utente inserisce il Catalogo dei Procedimenti come URI di pubblicazione del
                        catalogo.
                  \item  L'Utente può aggiungere una o più Fasi come filtro (UC-2.3.2.4.8.1).
            \end{enumerate}
\end{itemize}

\paragraph{UC-3.2.2.7.1 - Aggiunta filtro Fasi del Procedimento}
\begin{itemize}
      \item \textbf{Attore Primario}: Utente
      \item \textbf{Precondizioni}: \begin{enumerate}
                  \item L'Utente sta effettuando una ricerca per campi specifici al tipo Aggregazione.
                  \item L'Utente ha aggiunto un Procedimento come parametro per la ricerca.
            \end{enumerate}
      \item \textbf{Postcondizioni}:  Il filtro aggiunto dall'utente è parametro per la ricerca e per la generazione del risultato.
      \item \textbf{Flusso Principale}:
            \begin{enumerate}
                  \item L'Utente aggiunge un filtro per "Fase".
                  \item L'Utente inserisce i dati della fase:
                        \begin{itemize}
                              \item Tipo Fase (Preparatoria, Istruttoria, Consultiva, Decisoria o deliberativa,
                                    Integrazione dell'efficacia).
                              \item Data di inizio.
                              \item Data di fine.
                        \end{itemize}
                  \item L'Utente può aggiungere altre fasi.
            \end{enumerate}
\end{itemize}

\paragraph{UC-3.2.2.8 - Aggiunta filtro Assegnazione}
\begin{itemize}
      \item \textbf{Attore Primario}: Utente
      \item \textbf{Precondizioni}: L'Utente ha scelto come tipo di documento l'Aggregazione Documentale.
      \item \textbf{Postcondizioni}:  Il filtro aggiunto dall'utente è parametro per la ricerca e per la generazione del risultato.
      \item \textbf{Flusso Principale}:
            \begin{enumerate}
                  \item L'Utente aggiunge uno o più filtri per "Assegnazione".
                  \item L'Utente inserisce i dati:
                        \begin{itemize}
                              \item Tipo Assegnazione.
                              \item Soggetto Assegnatario (UC-2.3.1.2).
                              \item Data di Inizio.
                              \item Data di Fine.
                        \end{itemize}
            \end{enumerate}
\end{itemize}

\paragraph{UC-3.2.2.9 - Aggiunta filtro Progressivo}
\begin{itemize}
      \item \textbf{Attore Primario}: Utente
      \item \textbf{Precondizioni}: L'Utente ha scelto come tipo di documento l'Aggregazione Documentale.
      \item \textbf{Postcondizioni}:  Il filtro aggiunto dall'utente è parametro per la ricerca e per la generazione del risultato.
      \item \textbf{Flusso Principale}:
            \begin{enumerate}
                  \item L'Utente inserisce il numero Progressivo dell'Aggregazione.
            \end{enumerate}
\end{itemize}

\paragraph{UC-3.3 - Aggiunta dettagli per il ruolo di un soggetto}
\begin{itemize}
      \item \textbf{Attore Primario}: Utente
      \item \textbf{Precondizioni}: L'Utente ha scelto un ruolo come campo di ricerca.
      \item \textbf{Postcondizioni}:  Il filtro aggiunto dall'utente è parametro per la ricerca e per la generazione del risultato.
      \item \textbf{Flusso Principale}:
            \begin{enumerate}
                  \item L'Utente inserisce i dettagli del ruolo.
            \end{enumerate}
\end{itemize}

\paragraph{UC-3.3.1 - Ricerca per Dettagli Ruolo= PAI}
\begin{itemize}
      \item \textbf{Attore Primario}: Utente
      \item \textbf{Precondizioni}: L'Utente ha scelto di filtrare la ricerca per Soggetto e ha selezionato il ruolo PAI.
      \item \textbf{Postcondizioni}:  Il filtro aggiunto dall'utente è parametro per la ricerca e per la generazione del risultato.
      \item \textbf{Flusso Principale}:
            \begin{enumerate}
                  \item L'Utente inserisce i valori per i campi:
                        \begin{itemize}
                              \item Denominazione Amministrazione/ Codice IPA
                              \item Denominazione Amministrazione AOO/ Codice IPA AOO
                              \item Denominazione Amministrazione UOR/ Codice IPA UOR
                              \item Indirizzi Digitali di Riferimento
                        \end{itemize}
            \end{enumerate}
\end{itemize}

\paragraph{UC-3.3.2 - Ricerca per Dettagli Ruolo= PAE}
\begin{itemize}
      \item \textbf{Attore Primario}: Utente
      \item \textbf{Precondizioni}: L'Utente ha scelto di filtrare la ricerca per Soggetto e ha selezionato il ruolo PAE.
      \item \textbf{Postcondizioni}:  Il filtro aggiunto dall'utente è parametro per la ricerca e per la generazione del risultato.
      \item \textbf{Flusso Principale}:
            \begin{enumerate}
                  \item L'Utente inserisce i valori per i campi:
                        \begin{itemize}
                              \item Denominazione Amministrazione
                              \item Denominazione Ufficio
                              \item Indirizzi Digitali di Riferimento
                        \end{itemize}
            \end{enumerate}
\end{itemize}

\paragraph{UC-3.3.3 - Ricerca per Dettagli Ruolo= AS}
\begin{itemize}
      \item \textbf{Attore Primario}: Utente
      \item \textbf{Precondizioni}: L'Utente ha scelto di filtrare la ricerca per Soggetto e ha selezionato il ruolo AS.
      \item \textbf{Postcondizioni}:  Il filtro aggiunto dall'utente è parametro per la ricerca e per la generazione del risultato.
      \item \textbf{Flusso Principale}:
            \begin{enumerate}
                  \item L'Utente inserisce i valori per i campi:
                        \begin{itemize}
                              \item Cognome
                              \item Nome
                              \item Codice Fiscale
                              \item Denominazione Amministrazione/ Codice IPA
                              \item Denominazione Amministrazione AOO/ Codice IPA AOO
                              \item Denominazione Amministrazione UOR/ Codice IPA UOR
                              \item Indirizzi Digitali di Riferimento
                        \end{itemize}
            \end{enumerate}
\end{itemize}

\paragraph{UC-3.3.4 - Ricerca per Dettagli Ruolo= PG}
\begin{itemize}
      \item \textbf{Attore Primario}: Utente
      \item \textbf{Precondizioni}: L'Utente ha scelto di filtrare la ricerca per Soggetto e ha selezionato il ruolo PG.
      \item \textbf{Postcondizioni}:  Il filtro aggiunto dall'utente è parametro per la ricerca e per la generazione del risultato.
      \item \textbf{Flusso Principale}:
            \begin{enumerate}
                  \item L'Utente inserisce i valori per i campi:
                        \begin{itemize}
                              \item Denominazione Organizzazione
                              \item Codice fiscale / Partita Iva
                              \item Denominazione Ufficio
                              \item Indirizzi Digitali di Riferimento
                        \end{itemize}
            \end{enumerate}
\end{itemize}

\paragraph{UC-3.3.5 - Ricerca per Dettagli Ruolo= PF}
\begin{itemize}
      \item \textbf{Attore Primario}: Utente
      \item \textbf{Precondizioni}: L'Utente ha scelto di filtrare la ricerca per Soggetto e ha selezionato il ruolo PF.
      \item \textbf{Postcondizioni}:  Il filtro aggiunto dall'utente è parametro per la ricerca e per la generazione del risultato.
      \item \textbf{Flusso Principale}:
            \begin{enumerate}
                  \item  L'Utente inserisce i valori per i campi:
                        \begin{itemize}
                              \item Cognome
                              \item Nome
                              \item Indirizzi Digitali di Riferimento
                        \end{itemize}
            \end{enumerate}
\end{itemize}

\paragraph{UC-3.3.6 - Ricerca per Dettagli Ruolo= RUP}
\begin{itemize}
      \item \textbf{Attore Primario}: Utente
      \item \textbf{Precondizioni}: L'Utente ha scelto di filtrare la ricerca per Soggetto e ha selezionato il ruolo RUP.
      \item \textbf{Postcondizioni}:  Il filtro aggiunto dall'utente è parametro per la ricerca e per la generazione del risultato.
      \item \textbf{Flusso Principale}:
            \begin{enumerate}
                  \item L'Utente inserisce i valori per i campi:
                        \begin{itemize}
                              \item Cognome
                              \item Nome
                              \item Denominazione Amministrazione/ Codice IPA
                              \item Denominazione Amministrazione AOO/ Codice IPA AOO
                              \item Denominazione Amministrazione UOR/ Codice IPA UOR
                              \item Indirizzi Digitali di Riferimento
                        \end{itemize}
            \end{enumerate}
\end{itemize}

\paragraph{UC-3.3.7 - Ricerca per Dettagli Ruolo= SW}
\begin{itemize}
      \item \textbf{Attore Primario}: Utente
      \item \textbf{Precondizioni}: L'Utente ha scelto di filtrare la ricerca per Soggetto e ha selezionato il ruolo SW.
      \item \textbf{Postcondizioni}:  Il filtro aggiunto dall'utente è parametro per la ricerca e per la generazione del risultato.
      \item \textbf{Flusso Principale}:
            \begin{enumerate}
                  \item L'Utente inserisce la Denominazione Sistema
                  \item Il sistema usa questi valori per filtrare la ricerca.
            \end{enumerate}
\end{itemize}

\subsubsection{UC-4 - Visualizzazione Risultati di Ricerca} \label{risultatiRicerca}
\begin{itemize}
      \item \textbf{Attore Primario}: Utente
      \item \textbf{Precondizioni}: L'Utente ha eseguito una ricerca (tramite UC-2a, UC-2b).
      \item \textbf{Postcondizioni}: I risultati della ricerca sono presentati all'Utente.
      \item \textbf{Flusso Principale}:
            \begin{enumerate}
                  \item Il sistema visualizza un elenco di documenti o aggregazioni che corrispondono
                        ai criteri di ricerca.
                  \item Per ogni elemento, vengono mostrate le informazioni principali (Nome, Data,
                        Tipo).
            \end{enumerate}
      \item \textbf{Flussi Alternativi}:
            \begin{itemize}
                  \item Il sistema comunica che la ricerca non ha prodotto alcun risultato
                        (\ref{nessunRisultato})
                  \item L'Utente può selezionare un documento o un'aggregazione per visualizzare
                        ulteriori dettagli (UC-1.4).
            \end{itemize}
\end{itemize}

\subsubsection{UC-4.1 - Nessun Risultato} \label{nessunRisultato}
\begin{itemize}
      \item \textbf{Attore Primario}: Utente
      \item \textbf{Precondizioni}: L'Utente ha eseguito una ricerca (tramite UC-2.1, UC-2.2, UC-2.3 o UC-2.5).
      \item \textbf{Postcondizioni}: Il sistema informa l'Utente che non sono stati trovati risultati.
      \item \textbf{Flusso Principale}:
            \begin{enumerate}
                  \item Il sistema visualizza un messaggio che indica che la ricerca non ha prodotto
                        alcun risultato.
                  \item Viene mostrato l'insieme dei filtri che erano stati applicati, in modo da
                        permettere all'Utente di comprendere eventuali errori di inserimento o
                        incoerenze.
                  \item L'Utente può scegliere di modificare i criteri di ricerca e riprovare.
            \end{enumerate}
\end{itemize}

\subsubsection{UC-2.5 - Ricerca Semantica}
\begin{itemize}
      \item \textbf{Attore Primario}: Utente
      \item \textbf{Attore Secondario}: LLM in remoto
      \item \textbf{Precondizioni}: L'Utente ha accesso alla funzionalità di ricerca semantica e dispone di una connessione internet.
      \item \textbf{Postcondizioni}: Visualizzazione di un elenco di documenti o aggregazioni pertinenti alla richiesta in linguaggio naturale.
      \item \textbf{Flusso Principale}:
            \begin{enumerate}
                  \item L'Utente descrive in linguaggio naturale cosa sta cercando nell'apposita barra
                        di ricerca.
                  \item Il sistema invia la richiesta al servizio LLM.
                  \item Il sistema riceve i risultati e li visualizza (UC-2.4).
            \end{enumerate}
      \item \textbf{Flusso Alternativo}:
            \begin{itemize}
                  \item La ricerca non riporta alcun risultato pertinente.
                  \item La connessione internet viene a mancare, impedendo di completare la richiesta.
            \end{itemize}
\end{itemize}
