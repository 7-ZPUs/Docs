\subsubsection{UC-6 - Verifica integrità classe documentale}
\begin{itemize}
    \item \textbf{Attore Primario}: Utente
    \item \textbf{Precondizioni}:
    \begin{enumerate}
        \item L'Utente ha avviato l'applicazione
        \item L'Utente si trova all'interno di una classe documentale
    \end{enumerate}
    \item \textbf{Postcondizioni}: Visualizzazione di un report di integrità della classe documentale
    \item \textbf{Flusso Principale}:
    \begin{enumerate}
        \item L'utente attiva la funzione di verifica di integrità della classe documentale tramite apposito pulsante
        \item Viene effettuata la verifica di integrità dei processi al suo interno
        \item Il sistema restituisce una vista di report sull'integrità di conservazione sulla classe documentale
    \end{enumerate}
    \item \textbf{Flusso Alternativo}: La verifica fallisce per uno o più processi e viene restituito un messaggio di errore con la descrizione dei file non integri
\end{itemize}

\subsubsection{UC-7 - Verifica integrità processo}
\begin{itemize}
    \item \textbf{Attore Primario}: Utente
    \item \textbf{Precondizioni}:
    \begin{enumerate}
        \item L'Utente ha avviato l'applicazione
        \item L'Utente si trova all'interno di una cartella di processo
    \end{enumerate}
    \item \textbf{Postcondizioni}: Visualizzazione di un report di integrità del processo
    \item \textbf{Flusso Principale}:
    \begin{enumerate}
        \item L'utente attiva la funzione di verifica di integrità del processo tramite apposito pulsante
        \item Viene effettuata la verifica di integrità dei file al suo interno
        \item Il sistema restituisce una vista di report sull'integrità di conservazione del processo
    \end{enumerate}
    \item \textbf{Flusso Alternativo}: La verifica fallisce e viene restituito un messaggio di errore
\end{itemize}