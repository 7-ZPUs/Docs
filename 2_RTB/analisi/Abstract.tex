\begin{center}
    \includegraphics[width=9.5cm]{../assets/logo7ZPUs.jpg}\\
    \small\hspace{10cm} 7zpus.swe@gmail.com\\
    \vspace{0.5cm}
    \LARGE \textbf{Analisi dei Requisiti}\\
\end{center}

\vspace{0.3cm}
\hrule
\vspace{0.5cm}

\setlength{\LTleft}{-1cm}
\section*{Tabella di Versionamento}
    \begin{center}
    \begin{longtable}{|c|c|c|c|c|}
        \hline        
        \textbf{Versione} & \textbf{Data} & \textbf{Autore}  & \textbf{Verificatore} & \textbf{Descrizione} \\
        \hline
        0.14.0 & 2026/02/08 & Gingillino Aaron & Rocco Matteo A. & \begin{tabular}[c]{@{}c@{}} Stesura requisiti, \\tracciamento e \\diagrammi UC24-UC36 \end{tabular} \\
        \hline
        0.13.0 & 2026/02/06 & Georgescu Diana & Laoud Zakaria & \begin{tabular}[c]{@{}c@{}} Stesura requisiti, \\tracciamento e \\diagrammi UC37-UC46 \end{tabular} \\
        \hline
        0.12.0 & 2026/02/04 & Laoud Zakaria & Rocco Matteo A. & \begin{tabular}[c]{@{}c@{}} Stesura requisiti, \\tracciamento e \\diagrammi UC47-UC72 \end{tabular} \\
        \hline
        0.11.0 & 2026/02/04 & Rocco Matteo A. & Laoud Zakaria & \begin{tabular}[c]{@{}c@{}} Stesura requisiti, \\tracciamento e \\diagrammi UC1-UC23 \end{tabular} \\
        \hline
        0.10.1 & 2026/01/14 & Rocco Matteo A. & Georgescu Diana & \begin{tabular}[c]{@{}c@{}} Revisione e correzione \\generale del documento \end{tabular} \\
        \hline
        0.10.0 & 2025/12/5 & Soligo Lorenzo & Rocco Matteo A. & \begin{tabular}[c]{@{}c@{}} Aggiunta dettaglio \\ UC-Visualizza risultati\end{tabular}\\
        \hline
        0.9.3 & 2025/12/5 & Vigolo Davide & Rocco Matteo A. & \begin{tabular}[c]{@{}c@{}}Allineamento degli\\ Use Case\end{tabular} \\
        \hline
        0.9.2 & 2025/12/30 & Soligo Lorenzo & Vigolo Davide & \begin{tabular}[c]{@{}c@{}}Automatizzazione \\ codici Use Case\end{tabular} \\
        \hline
        0.9.1 & 2025/12/28 & \begin{tabular}[c]{@{}c@{}}Fattoni Antonio \\ Soligo Lorenzo \\ Vigolo Davide\end{tabular}& \begin{tabular}[c]{@{}c@{}}Fattoni Antonio \\ Soligo Lorenzo \\ Vigolo Davide\end{tabular} & \begin{tabular}[c]{@{}c@{}}Revisione Congiunta \\ degli Use Case\end{tabular} \\
        \hline
        0.9.0 & 2025/12/28 & Soligo Lorenzo & Vigolo Davide & \begin{tabular}[c]{@{}c@{}} Aggiunta UC27 $\rightarrow$ UC42 su \\ branch comune\end{tabular} \\
        \hline
        0.8.0 & 2025/12/28 & Vigolo Davide & Soligo Lorenzo & \begin{tabular}[c]{@{}c@{}} Aggiunta UC14 $\rightarrow$ UC26 su \\ branch comune\end{tabular} \\
        \hline
        0.7.0 & 2025/12/28 & Vigolo Davide & Soligo Lorenzo & \begin{tabular}[c]{@{}c@{}} Aggiunta UC10 $\rightarrow$ UC13 su \\ branch comune\end{tabular} \\
        \hline
        0.6.0 & 2025/12/28 & Fattoni Antonio & Vigolo Davide & \begin{tabular}[c]{@{}c@{}} Aggiunta UC5 $\rightarrow$ UC8 su \\ branch comune\end{tabular} \\
        \hline
        0.5.0 & 2025/12/28 & Soligo Lorenzo & Fattoni Antonio & \begin{tabular}[c]{@{}c@{}} Aggiunta UC1 $\rightarrow$ UC4 su \\ branch comune\end{tabular} \\
        \hline
        0.4.0 & 2025/12/24 & Soligo Lorenzo & Fattoni Antonio & \begin{tabular}[c]{@{}c@{}} Revisione Bozza UC1 $\rightarrow$ UC4\\ dopo incontro con \\ Prof. Cardin \end{tabular} \\
        \hline
        0.3.0 & 2025/12/21 & Soligo Lorenzo & Fattoni Antonio & \begin{tabular}[c]{@{}c@{}} Revisione Bozza UC1 $\rightarrow$ UC4 \\ dopo incontro Riunione \\ Interna\end{tabular} \\
        \hline
        0.2.0 & 12/12/2025 & Soligo Lorenzo & Rocco Matteo A. & \begin{tabular}[c]{@{}c@{}} Modularizzazione\\ file Tex \end{tabular} \\
        \hline
        0.1.0 & 07/11/2025 & Georgescu Diana & Laoud Zakaria & \begin{tabular}[c]{@{}c@{}} Creazione del template\\ e stesura iniziale \end{tabular} \\
        \hline
    \end{longtable}
    \end{center}

\tableofcontents
\newpage
\listoftables
\newpage
\listoffigures
\newpage

\section{Introduzione}

\subsection{Scopo}
Questo documento si pone l'obiettivo di delineare in modo chiaro le caratteristiche del software da realizzare, partendo dall'analisi dei bisogni e delle aspettative della \glossario{proponente}.
L'elaborazione dei requisiti trae origine dallo studio preliminare del \glossario{capitolato}, al fine di individuare gli \glossario{attori} coinvolti e le funzionalità attese.\\
 
\noindent Il presente documento sarà utilizzato come punto di riferimento per tutto l'arco dello sviluppo del prodotto, dalla \glossario{progettazione} alla \glossario{validazione}, e permetterà il tracciamento di ogni decisione progettuale, consentendoci di soddisfare le aspettative della proponente.\\

\noindent Il documento di \glossario{Analisi dei Requisiti} è redatto dagli \glossario{analisti} del team di progetto ed è destinato principalmente a tre categorie di soggetti.
In primo luogo, al \glossario{\textbf{committente}}, che attraverso la sua consultazione può verificare che i \glossario{requisiti} siano stati correttamente compresi e formalizzati in linea con le proprie aspettative.
In secondo luogo, al team di \textbf{\glossario{progettisti} e \glossario{programmatori}}, per i quali il documento rappresenta una guida di riferimento essenziale durante la fase di sviluppo del prodotto.
Infine, al \textbf{team di \glossario{verificatori}}, che si baserà sulle informazioni contenute nel suddetto documento per definire i \glossario{test} e verificare la conformità del prodotto ai requisiti.
Il documento sarà inoltre disponibile per \glossario{amministratori} e \glossario{responsabili di progetto}, offrendo una panoramica completa delle caratteristiche e delle funzionalità previste per il prodotto.
Data la natura incrementale del processo di sviluppo, questo documento verrà aggiornato periodicamente per riflettere eventuali modifiche o integrazioni ai requisiti.

\subsection{Glossario}
Per una corretta comprensione del documento, si rimanda al documento di \href{https://cdn.jsdelivr.net/gh/7-zpus/Docs@main/2_RTB/Glossario.pdf}{\ul{Glossario}\setulcolor{black}} contenente la definizione dei termini contrassegnati dalla \textit{G} a pedice (\glossario{Glossario}).

\subsection{Riferimenti}
Il documento cerca di aderire il più possibile, seppur non in modo vincolante, alla struttura e ai contenuti previsti dallo standard \glossario{IEEE 830:1998} per la specifica dei requisiti software e di rispettare le indicazioni fornite dalle dispense didattiche del corso di Ingegneria del Software dell'Università di Padova.

\subsubsection{Riferimenti Normativi}
\begin{itemize}
    \item \href{https://www.math.unipd.it/~tullio/IS-1/2025/Progetto/C3.pdf}{\ul{Capitolato C3: DIPReader}\setulcolor{black}} \ped{(ultimo accesso: 13/11/2025)}
    \item \href{https://www.math.unipd.it/~tullio/IS-1/2025/Dispense/PD1.pdf}{\ul{Regolamento di Progetto Didattico a.a. 2025/2026}\setulcolor{black}} \ped{(ultimo accesso: 17/11/2025)}
    \item \href{https://cdn.jsdelivr.net/gh/7-zpus/Docs@main/2_RTB/NormeDiProgetto.pdf}{\ul{Norme di Progetto}\setulcolor{black}} \ped{(ultimo accesso: 11/02/2026)}
\end{itemize}

\subsubsection{Riferimenti Informativi}
\begin{itemize}
    \item \href{https://cdn.jsdelivr.net/gh/7-zpus/Docs@main/2_RTB/Glossario.pdf}{\ul{Glossario di progetto}\setulcolor{black}} \ped{(ultimo accesso: 11/02/2026)}
    \item \href{https://ieeexplore.ieee.org/document/720574}{\ul{Standard IEEE 830:1998}\setulcolor{black}} \ped{(ultimo accesso: 24/11/2025)}
    \item Dispense del corso di Ingegneria del Software 2025/2026:
    \begin{itemize}
        \item \href{https://www.math.unipd.it/~tullio/IS-1/2025/Dispense/T05.pdf}{\ul{https://www.math.unipd.it/~tullio/IS-1/2025/Dispense/T05.pdf}\setulcolor{black}} \ped{(ultimo accesso: 24/11/2025)}
        \item \href{https://www.math.unipd.it/~rcardin/swea/2022/Diagrammi%20Use%20Case.pdf}{\ul{https://www.math.unipd.it/~rcardin/swea/2022/Diagrammi\%20Use\%20Case.pdf}\setulcolor{black}} \ped{(ultimo accesso: 24/11/2025)}
        \item \href{https://www.math.unipd.it/~rcardin/swea/2023/Diagrammi%20delle%20Classi.pdf}{\ul{https://www.math.unipd.it/~rcardin/swea/2023/Diagrammi\%20delle\%20Classi.pdf}\setulcolor{black}} \ped{(ultimo accesso: 24/11/2025)}
        \item \href{https://kurzy.kpi.fei.tuke.sk/zsi/resources/CockburnBookDraft.pdf}{\ul{A. Cockburn, Writing Effective Use Cases}\setulcolor{black}} \ped{(ultimo accesso: 30/12/2025)}
        \item \href{https://www.omg.org/spec/UML/2.5.1/PDF}{\ul{OMG UML 2.5.1 Specification}\setulcolor{black}} \ped{(ultimo accesso: 30/12/2025)}
    \end{itemize}
\end{itemize}

\section{Descrizione del Prodotto}

\subsection{Panoramica}
Il software oggetto di sviluppo è \textbf{DIPReader}, un'applicazione per l'accesso e la visualizzazione di documenti informatici all'interno di un Distribution Information Package (\glossario{DIP}), ovvero un archivio compresso distribuito da un sistema di conservazione centralizzato. \\

\noindent Un DIP contiene un insieme di cartelle e documenti tecnici con valore legale, strutturati secondo specifiche normative che ne garantiscono autenticità, integrità, affidabilità e ottimizzazione dello spazio di archiviazione.
I pacchetti DIP sono generalmente distribuiti come archivi compressi in formato .zip e possono includere una grande quantità di documenti eterogenei tra loro.
La loro struttura interna può differire in modo significativo da quella comunemente utilizzata nei sistemi di memorizzazione personali; risulta pertanto complesso, per l'utente, consultarli tramite strumenti generici di decompressione e visualizzazione.
DIPReader nasce per affrontare questa difficoltà, fornendo una soluzione dedicata che consenta di importare, esplorare, ricercare e validare in modo efficiente e intuitivo i contenuti dei pacchetti DIP, supportando al contempo le esigenze operative e normative del contesto in cui vengono utilizzati.

\subsection{Funzionalità}
Il prodotto da realizzare deve offrire le seguenti funzionalità fondamentali:
\begin{itemize}
    \item Possibilità di navigare in modo intuitivo all'interno di un DIP.
    \item Possibilità di visualizzare i contenuti di ciascuna cartella.
    \item Possibilità di visualizzare alcuni formati di documenti tramite anteprima.
    \item Possibilità di ricercare i documenti sia per nome che per attributi significativi (\glossario{metadata}).
    \item Possibilità di filtrare i documenti visualizzati per agevolare la ricerca attraverso i metadata
    \item Possibilità di salvare uno o più documenti nel computer dell'utente.
    \item Possibilità di verificare l'autenticità di un file attraverso il suo \glossario{processo di conservazione}.
    \item Mantenere la portabilità del DIP senza necessitare installazioni su disco.
\end{itemize}
Il prodotto deve inoltre operare in modo fluido durante le attività di navigazione, ricerca e visualizzazione anche in presenza di grandi quantità di dati, scenario molto probabile per natura stessa del contesto in cui opera.
È inoltre apprezzato lo sviluppo delle seguenti, e ulteriori se ritenute utili, funzionalità opzionali:
\begin{itemize}
    \item Possibilità di stampare i documenti selezionati.
    \item \glossario{Ricerca semantica} dei documenti mediante strumenti di intelligenza artificiale.
    \item Possibilità di accedere anche a pacchetti di distribuzione direttamente dal cloud.
    \item Possibilità di verificare l'eventuale firma digitale associata al file.
\end{itemize}

\subsection{Utenti di destinazione}
Il prodotto costituisce una soluzione essenziale per utenti operanti nei settori giudiziario e tecnico, quali l'Agenzia delle Entrate, la Guardia di Finanza e i magistrati che necessitano di consultare documenti digitali complessi presenti nei DIP.


\section{Casi d'Uso}

\subsection{Introduzione}
Per facilitare la comprensione dei casi d'uso, questi saranno descritti da un diagramma ciascuno, che rispetta la sintassi UML (Unified Modeling Language).
La descrizione testuale deve contenere le seguenti informazioni:
\begin{itemize}
    \item \textbf{Attori (principali o secondari)}: rappresentano un ruolo che un'entità esterna al sistema assume quando interagisce con esso per raggiungere un obiettivo.
    \item \textbf{Precondizioni}: condizioni che devono essere vere nello stato del sistema prima che il caso d'uso inizi la sua esecuzione.
    \item \textbf{Postcondizioni}: condizioni che devono essere vere nello stato del sistema dopo che il caso d'uso è terminato.
    \item \textbf{Scenario principale}: interazioni tra attore e sistema che porta al raggiungimento dell'obiettivo del caso d'uso con successo
    \item \textbf{Scenari alternativi}: variazione rispetto al flusso principale.
\end{itemize}

\subsection{Elenco Casi d'Uso}