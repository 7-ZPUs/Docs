\begin{center}
    \includegraphics[width=9.5cm]{../assets/logo7ZPUs.jpg}\\
    \small\hspace{10cm} 7zpus.swe@gmail.com\\
    \vspace{0.5cm}
    \Large \textbf{Analisi dei Requisiti}\\
\end{center}

\vspace{0.3cm}
\hrule
\vspace{0.5cm}


\section*{Tabella di Versionamento}
\begin{table}[H]
    \begin{adjustwidth}{-4cm}{-4cm}
    \centering
    \begin{tabular}{|c|c|c|c|c|}
        \hline
        \textbf{Versione} & \textbf{Data} & \textbf{Autore}  & \textbf{Verificatore} & \textbf{Descrizione} \\
        \hline
        0.6.0 & 2025/12/28 & Vigolo Davide & Soligo Lorenzo & \begin{tabular}[c]{@{}c@{}} Aggiunta UC14 $\rightarrow$ UC26 \end{tabular} \\
        \hline
        0.5.0 & 2025/12/28 & Vigolo Davide & Soligo Lorenzo & \begin{tabular}[c]{@{}c@{}} Aggiunta UC10 $\rightarrow$ UC13 \end{tabular} \\
        \hline
        0.4.0 & 2025/12/28 & Fattoni Antonio & Vigolo Davide & \begin{tabular}[c]{@{}c@{}} Aggiunta UC5 $\rightarrow$ UC8 \end{tabular} \\
        \hline
        0.3.0 & 2025/12/28 & Soligo Lorenzo & Fattoni Antonio & \begin{tabular}[c]{@{}c@{}} Aggiunta UC1 $\rightarrow$ UC4 \end{tabular} \\
        \hline
        0.2.0 & 12/12/2025 & Soligo Lorenzo & Rocco Matteo A. & \begin{tabular}[c]{@{}c@{}} Integrazione dei contenuti \\dal file condiviso grazie LLM, \\divisione dei file per \\lavorazione asincrona. \end{tabular} \\
        \hline
        0.1.0 & 07/11/2025 & Georgescu Diana & Laoud Zakaria & \begin{tabular}[c]{@{}c@{}} Creazione del template\\ 
 e stesura iniziale \end{tabular} \\
        \hline
    \end{tabular}
    \end{adjustwidth}
\end{table}

\tableofcontents
\newpage
\listoftables
\newpage
\listoffigures
\newpage

\section{Introduzione}

\subsection{Scopo}
Questo documento si pone l'obiettivo di delineare in modo chiaro le caratteristiche del software da realizzare, partendo dall'analisi dei bisogni e delle aspettative della proponente.
L'elaborazione dei requisiti trae origine dallo studio preliminare del capitolato, al fine di individuare gli attori coinvolti e le funzionalità attese.\\
 
Il presente documento sarà utilizzato come punto di riferimento per tutto l'arco dello sviluppo del prodotto, dalla progettazione alla validazione, e permetterà il tracciamento di ogni decisione progettuale, consentendoci di soddisfare le aspettative della proponente.\\

Il documento di Analisi dei Requisiti è redatto dagli \textit{Analisti} del team di progetto ed è destinato principalmente a tre categorie di soggetti.
In primo luogo, al \textbf{Committente}, che attraverso la sua consultazione può verificare che i requisiti siano stati correttamente compresi e formalizzati in linea con le proprie aspettative.
In secondo luogo, al Team di \textbf{Progettisti e Programmatori}, per i quali il documento rappresenta una guida di riferimento essenziale durante la fase di sviluppo del Sistema software.
Infine, al \textbf{Team di Verificatori}, che si baserà sulle informazioni contenute nel suddetto documento per definire i casi di test e verificare la conformità del prodotto alle specifiche.
Questo documento sarà, inoltre, a disposizione degli Amministratori e dei Responsabili di Progetto, allo scopo di ottenere una visione chiara e completa delle caratteristiche e delle funzionalità previste per il Sistema.
Data la natura incrementale del processo di sviluppo, questo documento verrà aggiornato periodicamente per riflettere eventuali modifiche o integrazioni ai requisiti.

\subsection{Glossario}
Per una corretta comprensione del documento, si rimanda al documento di \textbf{Glossario} contenente la definizione dei termini contrassegnati dalla \textit{G} a pedice (\glossario{Glossario}).

\subsection{Riferimenti}
Il documento cerca di aderire il più possibile, seppur non in modo vincolante, alla struttura e ai contenuti previsti dallo standard IEEE 830:1998 per la specifica dei requisiti software e di rispettare le indicazioni fornite dalle dispense didattiche del corso di Ingegneria del Software dell'Università di Padova.

\subsubsection{Riferimenti Normativi}
\begin{itemize}
    \item \href{https://www.math.unipd.it/~tullio/IS-1/2025/Progetto/C3.pdf}{\ul{Capitolato C3: DIPReader}\setulcolor{black}} \ped{(ultimo accesso: 13/11/2025)}
    \item \href{https://www.math.unipd.it/~tullio/IS-1/2025/Dispense/PD1.pdf}{\ul{Regolamento di Progetto Didattico a.a. 2025/2026}\setulcolor{black}} \ped{(ultimo accesso: 17/11/2025)}
\end{itemize}

\subsubsection{Riferimenti Informativi}
\begin{itemize}
    \item \href{https://ieeexplore.ieee.org/document/720574}{\ul{Standard IEEE 830:1998}\setulcolor{black}} \ped{(ultimo accesso: 24/11/2025)}
    \item Dispense del corso di Ingegneria del Software 2025/2026:
    \begin{itemize}
        \item \href{https://www.math.unipd.it/~tullio/IS-1/2025/Dispense/T05.pdf}{\ul{https://www.math.unipd.it/~tullio/IS-1/2025/Dispense/T05.pdf}\setulcolor{black}} \ped{(ultimo accesso: 24/11/2025)}
        \item \href{https://www.math.unipd.it/~rcardin/swea/2022/Diagrammi%20Use%20Case.pdf}{\ul{https://www.math.unipd.it/~rcardin/swea/2022/Diagrammi\%20Use\%20Case.pdf}\setulcolor{black}} \ped{(ultimo accesso: 24/11/2025)}
        \item \href{https://www.math.unipd.it/~rcardin/swea/2023/Diagrammi%20delle%20Classi.pdf}{\ul{https://www.math.unipd.it/~rcardin/swea/2023/Diagrammi\%20delle\%20Classi.pdf}\setulcolor{black}} \ped{(ultimo accesso: 24/11/2025)}
    \end{itemize}
\end{itemize}

\section{Descrizione del Prodotto}

\subsection{Panoramica}
Il software oggetto di sviluppo è \textbf{DIPReader}, un'applicazione per l'accesso e la visualizzazione di documenti informatici all'interno di un Distribution Information Package (DIP), ovvero un archivio compresso distribuito da un sistema di conservazione centralizzato. 

Un DIP (Distribution Information Package) contiene un insieme di cartelle e documenti tecnici con valore legale, strutturati secondo specifiche normative che ne garantiscono autenticità, integrità, affidabilità e ottimizzazione dello spazio di archiviazione.
I pacchetti DIP sono generalmente distribuiti come archivi compressi in formato .zip e possono includere una grande quantità di documenti eterogenei tra loro.
La loro struttura interna può differire in modo significativo da quella comunemente utilizzata nei sistemi di memorizzazione personali; risulta pertanto complesso, per l'utente, consultarli tramite strumenti generici di decompressione e visualizzazione.
DIPReader nasce per affrontare questa difficoltà, fornendo una soluzione dedicata che consenta di importare, esplorare, ricercare e validare in modo efficiente e intuitivo i contenuti dei pacchetti DIP, supportando al contempo le esigenze operative e normative del contesto in cui vengono utilizzati.

\subsection{Funzionalità}
Il prodotto da realizzare deve offrire le seguenti funzionalità fondamentali:
\begin{itemize}
    \item Possibilità di navigare in modo intuitivo all'interno di un DIP.
    \item Possibilità di visualizzare i contenuti di ciascuna cartella.
    \item Possibilità di visualizzare alcuni formati di documenti tramite anteprima.
    \item Possibilità di ricercare i documenti sia per nome che per attributi significativi (metadata).
    \item Possibilità di filtrare i documenti visualizzati per agevolare la ricerca attraverso i metadata
    \item Possibilità di salvare uno o più documenti nel computer dell'utente.
    \item Possibilità di verificare l'autenticità di un file attraverso il suo processo di conservazione.
    \item Mantenere la portabilità del DIP senza necessitare installazioni su disco.
\end{itemize}
Il prodotto deve inoltre operare in modo fluido durante le attività di navigazione, ricerca e visualizzazione anche in presenza di grandi quantità di dati, scenario molto probabile per natura stessa del contesto in cui opera.
È inoltre apprezzato lo sviluppo delle seguenti, e ulteriori se ritenute utili, funzionalità opzionali:
\begin{itemize}
    \item Possibilità di stampare i documenti selezionati.
    \item Ricerca semantica dei documenti mediante strumenti di intelligenza artificiale.
    \item Possibilità di accedere anche a pacchetti di distribuzione direttamente dal cloud.
    \item Possibilità di verificare l'eventuale firma digitale associata al file.
\end{itemize}

\subsection{Utenti di destinazione}
Il prodotto costituisce una soluzione essenziale per utenti operanti nei settori giudiziario e tecnico, quali l'Agenzia delle Entrate, la Guardia di Finanza e i magistrati che necessitano di consultare documenti digitali complessi presenti nei DIP.

\section{User Stories}
\begin{longtable}{|p{2cm}|p{12cm}|}
\hline
\textbf{ID} & \textbf{Descrizione} \\ \hline
US01 & Come utente voglio accedere ai contenuti del Distribution Information Package così da consultare i documenti al suo interno. \\ \hline
US02 & Come utente voglio vedere i contenuti del DIP in cartelle o file così da poterlo navigare in modo intuitivo. \\ \hline
US03 & Come utente voglio poter selezionare una cartella così da visualizzare i suoi contenuti o dettagli rilevanti. \\ \hline
US04 & Come utente voglio poter selezionare uno o più file per poterli stampare o salvare. \\ \hline
US05 & Come utente voglio poter visualizzare una anteprima di un file selezionato. \\ \hline
US06 & Come utente voglio poter verificare l'autenticità di un file. \\ \hline 
US07 & Come utente voglio poter visualizzare il contenuto di un file. \\ \hline
US08 & Come utente voglio poter ricercare un file. \\ \hline
US09 & Come utente voglio poter salvare un file. \\ \hline
US10 & Come utente voglio poter stampare un file. \\ \hline
US11 & Come utente voglio poter ordinare gli elementi del DIP. \\ \hline
US12 & Come utente voglio poter filtrare gli elementi del DIP per attributi. \\ \hline
\end{longtable}

\section{Casi d'Uso}

\subsection{Introduzione}
Per facilitare la comprensione dei casi d'uso, questi saranno descritti da un grafico UML e un testo per visualizzare gli obiettivi del prodotto.
La descrizione testuale deve contenere le seguenti informazioni:
\begin{itemize}
    \item \textbf{Attori (principali o secondari)}: rappresentano un ruolo che un'entità esterna al sistema assume quando interagisce con esso per raggiungere un obiettivo.
    \item \textbf{Precondizioni}: condizioni che devono essere vere nello stato del sistema prima che il caso d'uso inizi la sua esecuzione.
    \item \textbf{Postcondizioni}: condizioni che devono essere vere nello stato del sistema dopo che il caso d'uso è terminato.
    \item \textbf{Scenario principale}: interazioni tra attore e sistema che porta al raggiungimento dell'obiettivo del caso d'uso con successo
    \item \textbf{Scenari alternativi}: variazione rispetto al flusso principale.
\end{itemize}

\subsection{Elenco Casi d'Uso}