\usecase{strutturaDIP}{Visualizzazione Struttura DIP}
\begin{itemize}
    \item \textbf{Attore Primario}: Utente
    \item \textbf{Precondizioni}: L'Utente ha avviato il sistema DIP-Reader.
    \item \textbf{Postcondizioni}: L'Utente visualizza la struttura organizzativa del DIP come un albero di \textit{cartelle}.
    \item \textbf{Flusso Principale}:
          \begin{enumerate}
              \item L'Utente visualizza la schermata principale.
              \item L'Utente visualizza con la struttura organizzativa del DIP come un albero navigabile di
                     \textit{cartelle}. L'albero è organizzato gerarchicamente in:
                    \begin{itemize}
                        \item Classi Documentali (UC-\ref{ classiDocumentali})
                        \item Processi (UC-\ref{ processiClasseDocumentale})
                        \item Documenti (UC-\ref{ documentiProcesso})
                    \end{itemize}
          \end{enumerate}
    \item \textbf{Flusso Alternativo}: Se il DIP è vuoto, il sistema visualizza una struttura vuota.
\end{itemize}

\subusecase{classiDocumentali}{Visualizzazione di tutte le classi documentali}
\begin{itemize}
    \item \textbf{Attore Primario}: Utente
    \item \textbf{Precondizioni}: L'Utente ha avviato il sistema DIP-Reader.
    \item \textbf{Postcondizioni}: Il sistema visualizza l'elenco delle Classi Documentali come \textit{cartelle}.
    \item \textbf{Flusso Principale}:
          \begin{enumerate}
              \item L'Utente visualizza la schermata principale.
              \item L'Utente visualizza l'elenco di tutte le Classi Documentali presenti nel DIP.
          \end{enumerate}
    \item \textbf{Flusso Alternativo}: Se non sono presenti Classi Documentali, il sistema mostra un elenco vuoto.
\end{itemize}

\subusecase{processiClasseDocumentale}{Visualizzazione di tutti i processi di una classe documentale}
\begin{itemize}
    \item \textbf{Attore Primario}: Utente
    \item \textbf{Precondizioni}: L'Utente ha selezionato una delle Classi Documentali del DIP.
    \item \textbf{Postcondizioni}: L'Utente visualizza i processi associati alla classe selezionata come \textit{cartelle}.
    \item \textbf{Flusso Principale}: \begin{enumerate}
              \item L'Utente visualizza l'elenco dei processi associati alla classe selezionata.
          \end{enumerate}
    \item \textbf{Flusso Alternativo}: Se la classe documentale non contiene processi, il sistema mostra un elenco vuoto.
\end{itemize}

\subusecase{documentiProcesso}{Visualizzazione di tutti i documenti di un processo}
\begin{itemize}
    \item \textbf{Attore Primario}: Utente
    \item \textbf{Precondizioni}: L'Utente ha selezionato uno dei processi di una Classe Documentale.
    \item \textbf{Postcondizioni}: L'Utente visualizza l'elenco dei documenti associati al processo selezionato.
    \item \textbf{Flusso Principale}:\begin{enumerate}
              \item L'Utente seleziona un processo dall'elenco.
              \item Il sistema mostra l'elenco dei documenti associati a quel processo,
                    distinguendo tra documenti principali e allegati.
          \end{enumerate}
    \item \textbf{Flusso Alternativo}: Se il processo non contiene documenti, il sistema mostra un elenco vuoto.
\end{itemize}

\subusecase{informazioniElemento}{Visualizzazione Informazioni di un elemento}
\begin{itemize}
    \item \textbf{Attore Primario}: Utente
    \item \textbf{Precondizioni}: L'Utente ha selezionato un elemento del DIP.
    \item \textbf{Postcondizioni}: L'Utente visualizza i metadati e le azioni disponibili per l'elemento selezionato tra Documento, Processo, Classe Documentale e DIP.
    \item \textbf{Flusso Principale}:\begin{enumerate}
              \item L'Utente seleziona una cartella (Classe Documentale o Processo) o un documento dalla struttura del DIP (UC-\ref{ strutturaDIP}).
              \item Il sistema mostra i metadati associati, tra cui:
                    \begin{itemize}
                        \item Nome
                        \item Data di creazione
                        \item Stato di Verifica (Non Verificato, Verificato, Non Integro)
                        \item Tipo di documento (se applicabile)
                        \item Formato del documento (se applicabile)
                        \item Dimensione del documento (se applicabile)
                        \item Numero di allegati (se applicabile)
                        \item Numero totale di documenti (per il DIP)
                    \end{itemize}
          \end{enumerate}
    \item \textbf{Flussi Alternativi}: \begin{itemize}
              \item Se l'Utente seleziona un elemento non valido, il sistema non mostra alcuna informazione.
              \item L'Utente esegue un'azione sull'elemento selezionato:
                \begin{itemize}
                    \item Verifica l'integrità (UC-\ref{ verificaIntegritaDIPCompleto}, UC-\ref{ verificaIntegritaProcesso}, UC-\ref{ verificaIntegritaClasseDocumentale})
                    \item Stampa (UC-\ref{ stampaSingoloDoc} e UC-\ref{ stampaInsiemeDoc})
                    \item Salva in locale (UC-\ref{ scaricaFile} e UC-\ref{ salvaPiuDOcs})
                    \item Visualizza l'anteprima (UC-\ref{ anteprimaDocumento})
                    \item Visualizza Informazioni sull'AIP (UC-\ref{ visualizzaInfoAiP})
                \end{itemize}
          \end{itemize}
\end{itemize}

\subusecase{anteprimaDocumento}{Visualizzazione Anteprima Documento}
\begin{itemize}
    \item \textbf{Attore Primario}: Utente
    \item \textbf{Precondizioni}: L'Utente ha selezionato un documento dalla struttura del DIP.
    \item \textbf{Postcondizioni}: Il sistema visualizza un'anteprima del documento selezionato.
    \item \textbf{Flusso Principale}:
          \begin{enumerate}
              \item L'Utente seleziona l'opzione per visualizzare l'anteprima del documento.
              \item Il sistema apre una finestra di anteprima che mostra il contenuto del documento,
                    supportando vari formati (testo, immagini, PDF, ecc.).
          \end{enumerate}
    \item \textbf{Flusso Alternativo}: Se il formato del documento non è supportato per l'anteprima, il sistema mostra un messaggio che specifica perchè il tipo non è supportato.
\end{itemize}