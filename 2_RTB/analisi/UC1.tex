\usecase{strutturaDIP}{Visualizzazione Albero DIP}
\begin{itemize}
    \item \textbf{Attore Primario}: Utente
    \item \textbf{Precondizioni}: L'Utente ha avviato l'applicazione.
    \item \textbf{Postcondizioni}: L'Utente visualizza l'albero del DIP.
    \item \textbf{Flusso Principale}:
          \begin{enumerate}
              \item L'Utente visualizza l'albero del DIP. L'albero è organizzato gerarchicamente in:
                    \begin{itemize}
                        \item Classi Documentali (UC-\ref{classiDocumentali})
                        \item Processi (UC-\ref{processiClasseDocumentale})
                        \item Documenti (UC-\ref{documentiProcesso})
                    \end{itemize}
          \end{enumerate}
\end{itemize}

\subusecase{classiDocumentali}{Visualizzazione elenco classi documentali}
\begin{itemize}
    \item \textbf{Attore Primario}: Utente
    \item \textbf{Precondizioni}: L'Utente ha avviato l'applicazione.
    \item \textbf{Postcondizioni}: L'Utente visualizza l'elenco delle classi documentali.
    \item \textbf{Flusso Principale}:
          \begin{enumerate}
              \item Per ogni classe documentale presente nell'elenco viene mostrato:
                    \begin{itemize}
                        \item Nome della classe documentale (UC-\ref{nomeClasseDocumentale})
                    \end{itemize}
          \end{enumerate}
    \item \textbf{Flusso Alternativo}: Se non sono presenti classi documentali, il sistema mostra un elenco vuoto.
\end{itemize}

\subusecase{processiClasseDocumentale}{Visualizzazione elenco dei processi di una classe documentale}
\begin{itemize}
    \item \textbf{Attore Primario}: Utente
    \item \textbf{Precondizioni}: L'Utente ha selezionato una delle Classi Documentali del DIP.
    \item \textbf{Postcondizioni}: L'Utente visualizza i processi associati alla classe selezionata come \textit{cartelle}.
    \item \textbf{Flusso Principale}:          
          \begin{enumerate}
              \item Per ogni classe processo presente nell'elenco della classe documentale selezionata viene mostrato:
                    \begin{itemize}
                        \item Id del processo
                    \end{itemize}
          \end{enumerate}
    \item \textbf{Flusso Alternativo}: Se la classe documentale non contiene processi, il sistema mostra un elenco vuoto.
\end{itemize}

\subusecase{documentiProcesso}{Visualizzazione elenco dei documenti di un processo}
\begin{itemize}
    \item \textbf{Attore Primario}: Utente
    \item \textbf{Precondizioni}: L'Utente ha selezionato uno dei processi di una Classe Documentale.
    \item \textbf{Postcondizioni}: L'Utente visualizza l'elenco dei documenti associati al processo selezionato.
    \item \textbf{Flusso Principale}:
            \begin{enumerate}
              \item Per ogni documento presente nell'elenco del processo selezionato viene mostrato:
                    \begin{itemize}
                        \item Nome del documento
                    \end{itemize}
            \end{enumerate}
    \item \textbf{Flusso Alternativo}: Se il processo non contiene documenti, il sistema mostra un elenco vuoto.
\end{itemize}

\subusecase{selezioneElemento}{Selezione Elemento del DIP}
\begin{itemize}
    \item \textbf{Attore Primario}: Utente
    \item \textbf{Precondizioni}: 
        \begin{itemize}
            \item L'utente ha avviato l'applicazione
            \item L'utente sta visualizzando una lista di elementi del DIP (Classi Documentali, Processi o Documenti)
        \end{itemize}
    \item \textbf{Postcondizioni}: Viene selezionato l'elemento scelto dall'utente.
    \item \textbf{Flusso Principale}:\begin{enumerate}
              \item Il sistema seleziona l'elemento scelto dall'utente.

\subusecase{informazioniElemento}{Visualizza Informazioni Elemento}
\begin{itemize}
    \item \textbf{Attore Primario}: Utente
    \item \textbf{Precondizioni}: L'Utente ha selezionato un elemento del DIP.
    \item \textbf{Postcondizioni}: L'Utente visualizza il nome, stato di verifica e marcatura temporale dell'elemento selezionato.
    \item \textbf{Flusso Principale}:\begin{enumerate}
              \item Il sistema mostra i metadati associati, tra cui:
                    \begin{itemize}
                        \item Nome
                        \item Stato di Verifica (Non Verificato, Verificato, Non Integro)
                        \item Marcatura temporale (Presente o Meno)
                    \end{itemize}
          \end{enumerate}
    \item \textbf{Inclusioni}: UC-\ref{nomeElemento}, UC-\ref{statoVerificaElemento}, UC-\ref{marcaturaTemporaleElemento}
\end{itemize}

\subusecase{nomeElemento}{Visualizzazione Nome Elemento}
\begin{itemize}
    \item \textbf{Attore Primario}: Utente
    \item \textbf{Precondizioni}: L'Utente ha selezionato un elemento.
    \item \textbf{Postcondizioni}: L'utente visualizza il nome dell' elemento selezionata.
    \item \textbf{Flusso Principale}:
              \item Il sistema mostra a video il nome dell' elemento selezionata
\end{itemize}

\subusecase{statoVerificaElemento}{Visualizzazione Stato di Verifica Elemento}
\begin{itemize}
    \item \textbf{Attore Primario}: Utente
    \item \textbf{Precondizioni}: L'Utente ha selezionato un elemento.
    \item \textbf{Postcondizioni}: L'utente visualizza lo stato di verifica dell' elemento selezionata.
    \item \textbf{Flusso Principale}:
              \item Il sistema mostra a video lo stato di verifica dell' elemento selezionata
\end{itemize}

\subusecase{marcaturaTemporaleElemento}{Visualizzazione Marcatura Temporale Elemento}
\begin{itemize}
    \item \textbf{Attore Primario}: Utente
    \item \textbf{Precondizioni}: L'Utente ha selezionato un elemento.
    \item \textbf{Postcondizioni}: L'utente visualizza la marcatura temporale dell' elemento selezionata.
    \item \textbf{Flusso Principale}:
              \item Il sistema mostra a video la marcatura temporale dell' elemento selezionata
\end{itemize}

% \subusecase{informazioniElemento}{Visualizzazione Informazioni di un elemento}
% \begin{itemize}
%     \item \textbf{Attore Primario}: Utente
%     \item \textbf{Precondizioni}: L'Utente ha selezionato un elemento del DIP.
%     \item \textbf{Postcondizioni}: L'Utente visualizza i metadati e le azioni disponibili per l'elemento selezionato tra Documento, Processo, Classe Documentale e DIP.
%     \item \textbf{Flusso Principale}:\begin{enumerate}
%               \item L'Utente seleziona una cartella (Classe Documentale o Processo) o un documento dalla struttura del DIP (UC-\ref{strutturaDIP}).
%               \item Il sistema mostra i metadati associati, tra cui:
%                     \begin{itemize}
%                         \item Nome
%                         \item Data di creazione
%                         \item Stato di Verifica (Non Verificato, Verificato, Non Integro)
%                         \item Tipo di documento (se applicabile)
%                         \item Formato del documento (se applicabile)
%                         \item Dimensione del documento (se applicabile)
%                         \item Numero di allegati (se applicabile)
%                         \item Numero totale di documenti (per il DIP)
%                     \end{itemize}
%           \end{enumerate}
%     \item \textbf{Flussi Alternativi}: \begin{itemize}
%               \item Se l'Utente seleziona un elemento non valido, il sistema non mostra alcuna informazione.
%               \item L'Utente esegue un'azione sull'elemento selezionato:
%                 \begin{itemize}
%                     \item Verifica l'integrità (UC-\ref{verificaIntegritaDIPCompleto}, UC-\ref{verificaIntegritaProcesso}, UC-\ref{verificaIntegritaClasseDocumentale})
%                     \item Stampa (UC-\ref{stampaSingoloDoc} e UC-\ref{stampaInsiemeDoc})
%                     \item Salva in locale (UC-\ref{scaricaFile} e UC-\ref{salvaPiuDOcs})
%                     \item Visualizza l'anteprima (UC-\ref{anteprimaDocumento})
%                     \item Visualizza Informazioni sull'AIP (UC-\ref{visualizzaInfoAiP})
%                     \item Visualizza tutti i metadati associati (UC-\ref{ })
%                 \end{itemize}
%           \end{itemize}
%     \item \textbf{Inclusioni}: UC-\ref{nomeElemento}, UC-\ref{statoVerificaElemento}, 
%     \item \textbf{Estensioni}: UC-\ref{verificaIntegritaDIPCompleto}, UC-\ref{verificaIntegritaProcesso}, UC-\ref{verificaIntegritaClasseDocumentale}, UC-\ref{stampaSingoloDoc}, UC-\ref{stampaInsiemeDoc}, UC-\ref{scaricaFile}, UC-\ref{salvaPiuDOcs}, UC-\ref{anteprimaDocumento}, UC-\ref{visualizzaInfoAiP}
% \end{itemize}

\subusecase{anteprimaDocumento}{Visualizzazione Anteprima Documento}
\begin{itemize}
    \item \textbf{Attore Primario}: Utente
    \item \textbf{Precondizioni}: \begin{itemize}
        \item L'utente ha avviato l'applicazione
        \item L'Utente ha selezionato un documento dalla struttura del DIP.
    \end{itemize}
    \item \textbf{Postcondizioni}: L'Utente visualizza un'anteprima del documento selezionato.
    \item \textbf{Flusso Principale}:
          \begin{enumerate}
              \item L'Utente seleziona l'opzione per visualizzare l'anteprima del documento.
              \item Il sistema apre una finestra di anteprima che mostra il contenuto del documento.
          \end{enumerate}
    \item \textbf{Flusso Alternativo}: Se il formato del documento non è supportato per l'anteprima, il sistema mostra un messaggio che specifica perché il tipo non è supportato.
    \item \textbf{Estensioni}:
        \begin{itemize}
            \item UC-\ref{formatoDocumentoNonSupportato}: Formato Documento non Supportato dal Sistema
        \end{itemize}
\end{itemize}

\usecase{formatoDocumentoNonSupportato}{Formato Documento non Supportato dal Sistema}
\begin{itemize}
    \item \textbf{Attore Primario}: Utente
    \item \textbf{Precondizioni}: \begin{itemize}
        \item L'utente ha avviato l'applicazione
        \item L'Utente ha selezionato un documento
        \item L'Utente ha selezionato l'opzione di visualizzazione anteprima per il documento selezionato.
    \end{itemize}
    \item \textbf{Postcondizioni}: L'utente viene notificato che il formato del documento non è supportato per l'anteprima.
    \item \textbf{Flusso Principale}:
          \begin{enumerate}
              \item Il sistema mostra a video un messaggio di errore che indica che il formato del documento non è supportato per l'anteprima.
          \end{enumerate}
\end{itemize}

\usecase{elencoVuoto}{Elenco Vuoto}
\begin{itemize}
    \item \textbf{Attore Primario}: Utente
    \item \textbf{Precondizioni}: \begin{itemize}
        \item L'utente ha avviato l'applicazione
        \item L'Utente sta visualizzando un elenco di elementi del DIP, che risulta vuoto.
    \end{itemize}
    \item \textbf{Postcondizioni}: L'utente viene notificato che l'elenco è vuoto.
    \item \textbf{Flusso Principale}: 
          \begin{enumerate}
              \item Il sistema mostra a video un messaggio che indica che l'elenco è vuoto.
          \end{enumerate}
\end{itemize}