\subsubsection{UC-1 - Visualizzazione Struttura DIP}
\begin{itemize}
    \item \textbf{Attore Primario}: Utente
    \item \textbf{Precondizioni}: L'Utente ha avviato il sistema DIP-Reader.
    \item \textbf{Postcondizioni}: Viene visualizzato la struttura organizzativa del DIP come un albero di \textit{folder}
    \item \textbf{Flusso Principale}:
          \begin{enumerate}
              \item L'Utente Avvia DIP-Reader e si trova in Home
              \item Il sistema mostra la struttura organizzativa del DIP come un albero di \textit{folder}. L'albero è organizzato in:
                    \begin{itemize}
                        \item Classi Documentali (UC-1.1)
                        \item Processi (UC-1.2)
                        \item Documenti (UC-1.3)
                    \end{itemize}
          \end{enumerate}
    \item \textbf{Flusso Alternativo}: L'Utente visualizza un elenco vuoto
\end{itemize}

\subsubsection{UC-1.1 - Visualizzazione tutte le Classi Documentali}
\begin{itemize}
    \item \textbf{Attore Primario}: Utente
    \item \textbf{Precondizioni}: L'Utente ha avviato il sistema DIP-Reader.
    \item \textbf{Postcondizioni}: Viene visualizzato l'elenco delle Classi Documentali, presentate come \textit{folder}
    \item \textbf{Flusso Principale}:
          \begin{enumerate}
              \item L'Utente Avvia DIP-Reader e si trova in Home
              \item Il sistema mostra l'elenco delle Classi Documentali presenti nel DIP
          \end{enumerate}
    \item \textbf{Flusso Alternativo}: L'Utente visualizza un elenco vuoto
\end{itemize}

\subsubsection{UC-1.2 - Visualizzazione tutti i processi della classe documentale}
\begin{itemize}
    \item \textbf{Attore Primario}: Utente
    \item \textbf{Precondizioni}: Utente vede la lista delle Classi Documentali.
    \item \textbf{Postcondizioni}: Visualizzazione dei processi associati a quella classe, presentati come \textit{folder}
    \item \textbf{Flusso Principale}: \begin{enumerate}
              \item Utente seleziona una classe dalla lista delle classi di cui vuole vedere i
                    processi
              \item Il sistema mostra l'elenco dei processi associati a quella classe
          \end{enumerate}
    \item \textbf{Flusso Alternativo}: L'Utente visualizza un elenco vuoto
\end{itemize}

\subsubsection{UC-1.3 - Visualizzazione tutti i documenti del processo}
\begin{itemize}
    \item \textbf{Attore Primario}: Utente
    \item \textbf{Precondizioni}: L'Utente visualizza la lista dei processi
    \item \textbf{Postcondizioni}: Viene visualizzato l'elenco dei documenti associati a quel processo
    \item \textbf{Flusso Principale}:\begin{enumerate}
              \item L'Utente seleziona un processo dalla lista dei processi
              \item Il sistema mostra l'elenco dei documenti associati a quel processo,
                    differenziati tra documenti principali e allegati
          \end{enumerate}
    \item \textbf{Flusso Alternativo}: L'Utente visualizza un elenco vuoto
\end{itemize}

\subsubsection{UC-1.5 - Visualizzazione Informazioni Processo}
\begin{itemize}
    \item \textbf{Attore Primario}: Utente
    \item \textbf{Precondizioni}: L'Utente visualizza la lista dei processi
    \item \textbf{Postcondizioni}: Vengono visualizzati i dati del processo selezionato
    \item \textbf{Flusso Principale}:
          \begin{enumerate}
              \item L'Utente seleziona un processo dalla lista
              \item Il sistema mostra i dati del processo selezionato:
                    \begin{itemize}
                        \item Nome del processo
                        \item Data di creazione
                        \item Informazioni sulla Verifica del Processo (UC-)
                        \item Informazioni sul AiP (UC-)
                    \end{itemize}
          \end{enumerate}
    \item \textbf{Flusso Alternativo}: L'Utente visualizza dati vuoti
\end{itemize}

\subsubsection{UC-1.4 - Visualizzazione Informazioni Classe Documentale}
\begin{itemize}
    \item \textbf{Attore Primario}: Utente
    \item \textbf{Precondizioni}: L'Utente visualizza la lista delle classi documentali
    \item \textbf{Postcondizioni}: Vengono visualizzati i dati della classe documentale selezionata
    \item \textbf{Flusso Principale}:
          \begin{enumerate}
              \item L'Utente seleziona una classe documentale dalla lista
              \item Il sistema mostra i dati della classe documentale selezionata:
                    \begin{itemize}
                        \item Nome della classe documentale
                        \item Data di creazione
                        \item Informazioni sulla Verifica del file (UC-)
                        \item Informazioni sul AiP (UC-)
                    \end{itemize}
          \end{enumerate}
    \item \textbf{Flusso Alternativo}: L'Utente visualizza dati vuoti
\end{itemize}


\subsubsection{UC-1.6 - Visualizzazione Informazioni Documento}
\begin{itemize}
    \item \textbf{Attore Primario}: Utente
    \item \textbf{Precondizioni}: L'Utente visualizza la lista dei documenti
    \item \textbf{Postcondizioni}: Vengono visualizzati i dati del documento selezionato
    \item \textbf{Flusso Principale}:
          \begin{enumerate}
              \item L'Utente seleziona un documento dalla lista
              \item Il sistema mostra i dati del documento selezionato:
                    \begin{itemize}
                        \item Nome del documento
                        \item Data di creazione
                        \item Tipo di documento
                        \item Formato del documento
                        \item Dimensione del documento
                        \item Informazioni sulla Verifica del file (UC-5.4, UC-)
                        \item Informazioni sul AiP (UC-)
                    \end{itemize}
          \end{enumerate}
    \item \textbf{Flusso Alternativo}: L'Utente visualizza dati vuoti
\end{itemize}

