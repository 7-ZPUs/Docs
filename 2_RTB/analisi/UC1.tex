\usecase{classiDocumentali}{Visualizza elenco classi documentali}
\begin{figure}[H]
    \centering
    \includegraphics[width=0.9\textwidth]{../assets/uml/UC1.png}
    \caption{UC1 - Visualizza elenco classi documentali}
    \label{fig:uc_classiDocumentali}
\end{figure}
\begin{itemize}
    \item \textbf{Attore Primario}: Utente
    \item \textbf{Precondizioni}: L'Utente ha avviato l'applicazione.
    \item \textbf{Postcondizioni}: L'Utente visualizza l'elenco delle classi documentali.
    \item \textbf{Flusso Principale}:
    \begin{enumerate}
        \item Il sistema mostra a video l'elenco delle classi documentali presenti nel DIP.
    \end{enumerate}
    \item \textbf{Flusso Alternativo}:
    \begin{itemize}
        \item Se non sono presenti classi documentali, il sistema mostra un elenco vuoto. (\ref{elencoVuoto})
    \end{itemize}
    \item \textbf{Estensioni}: \ref{elencoVuoto} Visualizza elenco vuoto
\end{itemize}

% inclusioni: nomeClasseDocumentale
\usecase{classeDocumentale}{Visualizza classe documentale in elenco}
\begin{figure}[H]
    \centering
    \includegraphics[width=0.6\textwidth]{../assets/uml/UC2.png}
    \caption{UC2 - Visualizza classe documentale in elenco}
    \label{fig:uc_classeDocumentale}
\end{figure}
\begin{itemize}
    \item \textbf{Attore Primario}: Utente
    \item \textbf{Precondizioni}: L'Utente ha avviato l'applicazione.
    \item \textbf{Postcondizioni}: L'Utente visualizza l'elenco delle classi documentali.
    \item \textbf{Flusso Principale}:
    \begin{enumerate}
        \item Per ogni singola classe documentale presente nell'elenco viene mostrato:
        \begin{itemize}
            \item Nome della classe documentale (\ref{nomeClasseDocumentale})
            \item Stato di verifica della classe documentale (\ref{statoVerificaElemento})
            \item Marcatura temporale della classe documentale (\ref{marcaturaTemporaleElemento})
        \end{itemize}
    \end{enumerate}
    \item \textbf{Inclusioni}:
    \begin{itemize}
        \item \ref{nomeClasseDocumentale} Visualizza Nome Classe Documentale
        \item \ref{statoVerificaElemento} Visualizza Stato di Verifica Elemento
        \item \ref{marcaturaTemporaleElemento} Visualizza Marcatura Temporale Elemento
    \end{itemize}
\end{itemize}
\begin{figure}[H]
    \centering
    \includegraphics[width=0.5\textwidth]{../assets/uml/UC2INC.png}
    \caption{Inclusioni UC2 - Visualizza elenco processi di una classe documentale}
    \label{fig:inclusioniClasseDocumentale}
\end{figure}

\subusecase{nomeClasseDocumentale}{Visualizza Nome Classe Documentale}
\begin{itemize}
    \item \textbf{Attore Primario}: Utente
    \item \textbf{Precondizioni}: L'Utente ha avviato l'applicazione.
    \item \textbf{Postcondizioni}: L'Utente visualizza il nome della classe documentale.
    \item \textbf{Flusso Principale}:
          \begin{enumerate}
              \item Il sistema mostra a video il nome della classe documentale.
          \end{enumerate}
\end{itemize}

\subusecase{statoVerificaElemento}{Visualizza Stato di Verifica Elemento}
\begin{itemize}
    \item \textbf{Attore Primario}: Utente
    \item \textbf{Precondizioni}: L'Utente ha selezionato un elemento.
    \item \textbf{Postcondizioni}: L'utente visualizza lo stato di verifica dell' elemento selezionato.
    \item \textbf{Flusso Principale}:
    \begin{enumerate}
              \item Il sistema mostra a video lo stato di verifica dell'elemento selezionato
    \end{enumerate}
\end{itemize}

\subusecase{marcaturaTemporaleElemento}{Visualizza Marcatura Temporale Elemento}
\begin{itemize}
    \item \textbf{Attore Primario}: Utente
    \item \textbf{Precondizioni}: L'Utente ha selezionato un elemento.
    \item \textbf{Postcondizioni}: L'utente visualizza la marcatura temporale dell' elemento selezionato.
    \item \textbf{Flusso Principale}:
    \begin{enumerate}
              \item Il sistema mostra a video la marcatura temporale dell' elemento selezionato
    \end{enumerate}
\end{itemize}

\usecase{processiClasseDocumentale}{Visualizza elenco dei processi di una classe documentale}
\begin{figure}[H]
    \centering
    \includegraphics[width=0.9\textwidth]{../assets/uml/UC3.png}
    \caption{UC3 - Visualizza elenco dei processi di una classe documentale}
    \label{fig:uc_processiClasseDocumentale}
\end{figure}
\begin{itemize}
    \item \textbf{Attore Primario}: Utente
    \item \textbf{Precondizioni}: \begin{itemize}
        \item L'utente ha avviato l'applicazione
        \item L'Utente ha selezionato una delle Classi Documentali del DIP
    \end{itemize}
    \item \textbf{Postcondizioni}: L'Utente visualizza i processi associati alla classe selezionata.
    \item \textbf{Flusso Principale}: 
          \begin{enumerate}
              \item Il sistema mostra a video l'elenco dei processi della classe documentale selezionata.
          \end{enumerate}
    \item \textbf{Flusso Alternativo}:
    \begin{itemize}
        \item Se la classe documentale non contiene processi, il sistema mostra un elenco vuoto.
    \end{itemize}
    \item \textbf{Estensioni}: \ref{elencoVuoto} Visualizza elenco vuoto
\end{itemize}

\usecase{processoClasseDocumentale}{Visualizza processi in elenco}
\begin{figure}[H]
    \centering
    \includegraphics[width=0.6\textwidth]{../assets/uml/UC4.png}
    \caption{UC4 - Visualizza processi in elenco}
    \label{fig:uc_processoClasseDocumentale}
\end{figure}
\begin{itemize}
    \item \textbf{Attore Primario}: Utente
    \item \textbf{Precondizioni}:
    \begin{itemize}
        \item L'utente ha avviato l'applicazione
        \item L'Utente ha selezionato una delle Classi Documentali del DIP
    \end{itemize}
    \item \textbf{Postcondizioni}: L'Utente visualizza l'elenco dei processi .
    \item \textbf{Flusso Principale}:
    \begin{enumerate}
        \item Per ogni singolo processo presente nell'elenco viene mostrato:
        \begin{itemize}
            \item Id del processo (\ref{idProcessoClasse})
            \item Stato di verifica del processo (\ref{statoVerificaElemento})
            \item Marcatura temporale del processo (\ref{marcaturaTemporaleElemento})
        \end{itemize}
    \end{enumerate}
    \item \textbf{Inclusioni}:
    \begin{itemize}
        \item \ref{idProcessoClasse} Visualizza Id Processo
        \item \ref{statoVerificaElemento} Visualizza Stato di Verifica Elemento
        \item \ref{marcaturaTemporaleElemento} Visualizza Marcatura Temporale Elemento
    \end{itemize}
\end{itemize}
\begin{figure}[H]
    \centering
    \includegraphics[width=0.5\textwidth]{../assets/uml/UC4INC.png}
    \caption{Inclusioni UC4 - Visualizza elenco dei documenti di un processo}
    \label{fig:inclusioniProcessoClasseDocumentale}
\end{figure}

\subusecase{idProcessoClasse}{Visualizza Id Processo}
\begin{itemize}
    \item \textbf{Attore Primario}: Utente
    \item \textbf{Precondizioni}: \begin{itemize}
        \item L'utente ha avviato l'applicazione
        \item L'Utente ha selezionato una delle Classi Documentali del DIP
    \end{itemize}
    \item \textbf{Postcondizioni}: L'Utente visualizza il nome del processo.
    \item \textbf{Flusso Principale}:
          \begin{enumerate}
              \item Il sistema mostra a video il nome del processo.
          \end{enumerate}
\end{itemize}


\usecase{documentiProcesso}{Visualizza elenco dei documenti di un processo}
\begin{figure}[H]
    \centering
    \includegraphics[width=0.9\textwidth]{../assets/uml/UC5.png}
    \caption{UC5 - Visualizza elenco dei documenti di un processo}
    \label{fig:uc_documentiProcesso}
\end{figure}
\begin{itemize}
    \item \textbf{Attore Primario}: Utente
    \item \textbf{Precondizioni}: 
    \begin{itemize}
        \item L'utente ha avviato l'applicazione
        \item L'Utente ha selezionato una delle Classi Documentali del DIP
        \item L'Utente ha selezionato un processo associato alla classe documentale selezionata
    \end{itemize}
    \item \textbf{Postcondizioni}: L'Utente visualizza l'elenco dei documenti associati al processo selezionato.
    \item \textbf{Flusso Principale}:
    \begin{enumerate}
        \item Il sistema mostra a video l'elenco dei documenti della classe documentale selezionata.
    \end{enumerate}
    \item \textbf{Flusso Alternativo}:
    \begin{itemize}
        \item Se il processo non contiene documenti, il sistema mostra un elenco vuoto.
    \end{itemize}
    \item \textbf{Estensioni}: \ref{elencoVuoto} Visualizza elenco vuoto
\end{itemize}

\usecase{documentoProcesso}{Visualizza documento in elenco}
\begin{figure}[H]
    \centering
    \includegraphics[width=0.6\textwidth]{../assets/uml/UC6.png}
    \caption{UC6 - Visualizza documento in elenco}
    \label{fig:uc_documentoProcesso}
\end{figure}
\begin{itemize}
    \item \textbf{Attore Primario}: Utente
    \item \textbf{Precondizioni}: 
    \begin{itemize}
        \item L'utente ha avviato l'applicazione
        \item L'Utente ha selezionato una delle Classi Documentali del DIP
        \item L'Utente ha selezionato un processo associato alla classe documentale selezionata
    \end{itemize}
    \item \textbf{Postcondizioni}: L'Utente visualizza l'elenco delle classi documentali.
    \item \textbf{Flusso Principale}:
    \item Per ogni documento presente nell'elenco viene mostrato:
        \begin{itemize}
            \item Nome del documento (\ref{nomeDocumentoProcesso})
            \item Stato di verifica del documento (\ref{statoVerificaElemento})
            \item Marcatura temporale del documento (\ref{marcaturaTemporaleElemento})
        \end{itemize}
    \item \textbf{Inclusioni}:
    \begin{itemize}
        \item \ref{nomeDocumentoProcesso} Visualizza Nome Documento
        \item \ref{statoVerificaElemento} Visualizza Stato di Verifica Elemento
        \item \ref{marcaturaTemporaleElemento} Visualizza Marcatura Temporale Elemento
    \end{itemize}
\end{itemize}
\begin{figure}[H]
    \centering
    \includegraphics[width=0.5\textwidth]{../assets/uml/UC6INC.png}
    \caption{Inclusioni UC6 - Visualizza elenco dei documenti di un processo}
    \label{fig:inclusioniDocumentoProcesso}
\end{figure}

\subusecase{nomeDocumentoProcesso}{Visualizza Nome Documento}
\begin{itemize}
    \item \textbf{Attore Primario}: Utente
    \item \textbf{Precondizioni}: L'Utente ha avviato l'applicazione.
    \item \textbf{Postcondizioni}: L'Utente visualizza il nome del documento.
    \item \textbf{Flusso Principale}:
    \begin{enumerate}
        \item Il sistema mostra a video il nome del documento.
    \end{enumerate}
\end{itemize}

\usecase{elencoVuoto}{Elenco Vuoto}
\begin{itemize}
    \item \textbf{Attore Primario}: Utente
    \item \textbf{Precondizioni}: \begin{itemize}
        \item L'utente ha avviato l'applicazione
        \item L'Utente sta visualizzando un elenco di elementi del DIP, che risulta vuoto.
    \end{itemize}
    \item \textbf{Postcondizioni}: L'utente viene notificato che l'elenco è vuoto.
    \item \textbf{Flusso Principale}: 
          \begin{enumerate}
              \item Il sistema mostra a video un messaggio che indica che l'elenco è vuoto.
          \end{enumerate}
\end{itemize}

\usecase{selezioneElemento}{Selezione Elemento del DIP}
\begin{figure}[H]
    \centering
    \includegraphics[width=0.6\textwidth]{../assets/uml/UC8.png}
    \caption{UC8 - Selezione Elemento del DIP}
    \label{fig:uc_selezioneElemento}
\end{figure}
\begin{itemize}
    \item \textbf{Attore Primario}: Utente
    \item \textbf{Precondizioni}: 
        \begin{itemize}
            \item L'utente ha avviato l'applicazione
            \item L'utente sta visualizzando una lista di elementi del DIP (Classi Documentali, Processi o Documenti)
        \end{itemize}
    \item \textbf{Postcondizioni}: Viene selezionato l'elemento scelto dall'utente.
    \item \textbf{Flusso Principale}:
    \begin{enumerate}
              \item Il sistema seleziona l'elemento scelto dall'utente.
    \end{enumerate}
\end{itemize}

\usecase{selezionaClasseDocumentale}{Seleziona classe documentale}
\begin{figure}[H]
    \centering
    \includegraphics[width=0.6\textwidth]{../assets/uml/UC9.png}
    \caption{UC9 - Seleziona classe documentale}
    \label{fig:uc_selezionaClasseDocumentale}
\end{figure}
\begin{itemize}
    \item \textbf{Attore Primario}: Utente
    \item \textbf{Precondizioni}:
    \begin{itemize}
        \item L'utente ha avviato l'applicazione
        \item L'utente sta visualizzando una lista di classe documentale
    \end{itemize}
    \item \textbf{Postcondizioni}:
    \begin{itemize}
        \item Viene selezionata una classe documentale sulla quale effettuare un operazione
    \end{itemize}
    \item \textbf{Flusso principale}:
    \begin{enumerate}
        \item L'utente seleziona una classe documentale dalla lista
    \end{enumerate}
\end{itemize}

\usecase{selezionaProcesso}{Seleziona processo}
\begin{figure}[H]
    \centering
    \includegraphics[width=0.6\textwidth]{../assets/uml/UC10.png}
    \caption{UC10 - Seleziona processo}
    \label{fig:uc_selezionaProcesso}
\end{figure}
\begin{itemize}
    \item \textbf{Attore Primario}: Utente
    \item \textbf{Precondizioni}:
    \begin{itemize}
        \item L'utente ha avviato l'applicazione
        \item L'utente sta visualizzando una lista di processi
    \end{itemize}
    \item \textbf{Postcondizioni}:
    \begin{itemize}
        \item Viene selezionato un processo sul quale effettuare un operazione
    \end{itemize}
    \item \textbf{Flusso principale}:
    \begin{enumerate}
        \item L'utente seleziona un processo dalla lista
    \end{enumerate}
\end{itemize}

\usecase{selezionaDocumento}{Seleziona documento}
\begin{figure}[H]
    \centering
    \includegraphics[width=0.6\textwidth]{../assets/uml/UC11.png}
    \caption{UC11 - Seleziona documento}
    \label{fig:uc_selezionaDocumento}
\end{figure}
\begin{itemize}
    \item \textbf{Attore Primario}: Utente
    \item \textbf{Precondizioni}:
    \begin{itemize}
        \item L'utente ha avviato l'applicazione
        \item L'utente sta visualizzando una lista di documenti
    \end{itemize}
    \item \textbf{Postcondizioni}:
    \begin{itemize}
        \item Viene selezionato un documento sul quale effettuare un operazione
    \end{itemize}
    \item \textbf{Flusso principale}:
    \begin{enumerate}
        \item L'utente seleziona un documento dalla lista
    \end{enumerate}
\end{itemize}

% questo non serve più per quello che ha detto cardin il 9/01,che non c'è use case selezione multipla ma direttamente "download file multipli" o "stampa file multipli"
%\usecase{selezionaPiuDocumenti}{Seleziona più documenti} 
%\begin{itemize}
%    \item \textbf{Attore Primario}: Utente
%    \item \textbf{Precondizioni}:
%    \begin{itemize}
%        \item L'utente ha avviato l'applicazione
%        \item L'utente sta visualizzando una lista di documenti
%    \end{itemize}
%    \item \textbf{Postcondizioni}:
%    \begin{itemize}
%        \item Viene selezionati più documenti sul quale effettuare un operazione
%    \end{itemize}
%    \item \textbf{Flusso principale}:
%    \begin{itemize}
%        \item L'utente seleziona un insieme di documenti dalla lista
%    \end{itemize}
%\end{itemize}



% \subusecase{informazioniElemento}{Visualizza Informazioni di un elemento}
% \begin{itemize}
%     \item \textbf{Attore Primario}: Utente
%     \item \textbf{Precondizioni}: L'Utente ha selezionato un elemento del DIP.
%     \item \textbf{Postcondizioni}: L'Utente visualizza i metadati e le azioni disponibili per l'elemento selezionato tra Documento, Processo, Classe Documentale e DIP.
%     \item \textbf{Flusso Principale}:\begin{enumerate}
%               \item L'Utente seleziona una cartella (Classe Documentale o Processo) o un documento dalla struttura del DIP (\ref{strutturaDIP}).
%               \item Il sistema mostra i metadati associati, tra cui:
%                     \begin{itemize}
%                       \item Verifica l'integrità (\ref{verificaIntegritaDIPCompleto}, \ref{verificaIntegritaProcesso}, \ref{verificaIntegritaClasseDocumentale})
%                       \item Stampa (\ref{stampaSingoloDoc} e \ref{stampaInsiemeDoc})
%                       \item Salva in locale (\ref{scaricaFile} e \ref{salvaPiuDOcs})
%                       \item Visualizza l'anteprima (\ref{anteprimaDocumento})
%                       \item Visualizza Informazioni sull'AIP (\ref{visualizzaInfoAiP})
%                       \item Visualizza tutti i metadati associati (\ref{ })
%                     \end{itemize}
%           \end{enumerate}
%     \item \textbf{Flussi Alternativi}: \begin{itemize}
%               \item Se l'Utente seleziona un elemento non valido, il sistema non mostra alcuna informazione.
%               \item L'Utente esegue un'azione sull'elemento selezionato:
%                 \begin{itemize}
%           
%                 \end{itemize}
%           \end{itemize}
%     \item \textbf{Inclusioni}: \ref{nomeElemento}, \ref{statoVerificaElemento}, 
%     \item \textbf{Estensioni}: \ref{verificaIntegritaDIPCompleto}, \ref{verificaIntegritaProcesso}, \ref{verificaIntegritaClasseDocumentale}, \ref{stampaSingoloDoc}, \ref{stampaInsiemeDoc}, \ref{scaricaFile}, \ref{salvaPiuDOcs}, \ref{anteprimaDocumento}, \ref{visualizzaInfoAiP}
% \end{itemize}

\usecase{anteprimaDocumento}{Visualizza Anteprima Documento}
\begin{figure}[H]
    \centering
    \includegraphics[width=0.9\textwidth]{../assets/uml/UC12.png}
    \caption{UC12 - Visualizza Anteprima Documento}
    \label{fig:uc_anteprimaDocumento}
\end{figure}
\begin{itemize}
    \item \textbf{Attore Primario}: Utente
    \item \textbf{Precondizioni}: 
    \begin{itemize}
        \item L'utente ha avviato l'applicazione
        \item L'Utente ha selezionato un documento dalla struttura del DIP.
    \end{itemize}
    \item \textbf{Postcondizioni}: L'Utente visualizza un'anteprima del documento selezionato.
    \item \textbf{Flusso Principale}:
    \begin{enumerate}
        \item L'Utente seleziona l'opzione per visualizzare l'anteprima del documento.
        \item Il sistema apre una finestra di anteprima che mostra il contenuto del documento.
    \end{enumerate}
    \item \textbf{Flusso Alternativo}:
    \begin{itemize}
        \item Se il formato del documento non è supportato per l'anteprima, il sistema mostra un messaggio che specifica perché il tipo non è supportato.
    \end{itemize}
    \item \textbf{Estensioni}: \ref{formatoDocumentoNonSupportato} Formato Documento non Supportato dal Sistema per l'anteprima
\end{itemize}

\usecase{formatoDocumentoNonSupportato}{Formato Documento non Supportato dal Sistema per l'anteprima}
\begin{itemize}
    \item \textbf{Attore Primario}: Utente
    \item \textbf{Precondizioni}: 
    \begin{itemize}
        \item L'utente ha avviato l'applicazione
        \item L'Utente ha selezionato un documento
        \item L'Utente ha selezionato l'opzione di Visualizza anteprima per il documento selezionato.
        \item Il formato del documento selezionato non è supportato per l'anteprima.
    \end{itemize}
    \item \textbf{Postcondizioni}: 
    \begin{itemize}
        \item Non viene mostrata l'anteprima del documento.
        \item Viene mostrato un messaggio di errore che indica che il formato del documento non è supportato per l'anteprima.
    \end{itemize}
    \item \textbf{Flusso Principale}:
    \begin{enumerate}
        \item Il sistema chiude annulla l'operazione di anteprima.
        \item Il sistema mostra a video un messaggio di errore che indica che il formato del documento non è supportato per l'anteprima.
    \end{enumerate}
\end{itemize}

