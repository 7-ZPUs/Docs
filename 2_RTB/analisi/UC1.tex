\usecase{classiDocumentali}{Visualizza elenco classi documentali}
\begin{figure}[H]
    \centering
    \includegraphics[width=0.9\textwidth]{../assets/uml/UC1.png}
    \caption{\ref{classiDocumentali} - Visualizza elenco classi documentali}
    \label{fig:uc_classiDocumentali}
\end{figure}
\begin{itemize}
    \item \textbf{Attore Primario}: Utente
    \item \textbf{Precondizioni}: L'Utente ha avviato l'applicazione.
    \item \textbf{Postcondizioni}: L'Utente visualizza l'elenco delle classi documentali.
    \item \textbf{Flusso Principale}:
    \begin{enumerate}
        \item Il sistema mostra a video l'elenco delle classi documentali presenti nel DIP.
    \end{enumerate}
    \item \textbf{Flusso Alternativo}:
    \begin{itemize}
        \item Se non sono presenti classi documentali, il sistema mostra un elenco vuoto. (\ref{elencoVuoto})
    \end{itemize}
    \item \textbf{Inclusioni}: \ref{classeDocumentale} Visualizza classe documentale in elenco
    \item \textbf{Estensioni}: \ref{elencoVuoto} Visualizza elenco vuoto
\end{itemize}
\begin{figure}[H]
    \centering
    \includegraphics[width=0.9\textwidth]{../assets/uml/UC1INC.png}
    \caption{Inclusioni \ref{classiDocumentali} - Visualizza elenco classi documentali}
    \label{fig:inclusioniClassiDocumentali}
\end{figure}


% inclusioni: nomeClasseDocumentale
\subusecase{classeDocumentale}{Visualizza classe documentale in elenco}
\begin{itemize}
    \item \textbf{Attore Primario}: Utente
    \item \textbf{Precondizioni}: 
    \begin{itemize}
        \item L'Utente ha avviato l'applicazione.
        \item L'Utente sta visualizzando l'elenco delle classi documentali.
    \end{itemize}
    \item \textbf{Postcondizioni}: L'Utente visualizza una classe documentale.
    \item \textbf{Flusso Principale}:
    \begin{enumerate}
        \item Per ogni classe documentale viene mostrato:
        \begin{itemize}
            \item Nome della classe documentale (\ref{nomeClasseDocumentale})
            \item Stato di verifica della classe documentale (\ref{statoVerificaElemento})
            \item Marcatura temporale della classe documentale (\ref{marcaturaTemporaleElemento})
        \end{itemize}
    \end{enumerate}
    \item \textbf{Inclusioni}:
    \begin{itemize}
        \item \ref{nomeClasseDocumentale} Visualizza Nome Classe Documentale
        \item \ref{statoVerificaElemento} Visualizza Stato di Verifica Elemento
        \item \ref{marcaturaTemporaleElemento} Visualizza Marcatura Temporale Elemento
    \end{itemize}
\end{itemize}

\subsubusecase{nomeClasseDocumentale}{Visualizza nome classe documentale}
\begin{itemize}
    \item \textbf{Attore Primario}: Utente
    \item \textbf{Precondizioni}: L'Utente ha avviato l'applicazione.
    \item \textbf{Postcondizioni}: L'Utente visualizza il nome della classe documentale.
    \item \textbf{Flusso Principale}:
          \begin{enumerate}
              \item Il sistema mostra a video il nome della classe documentale.
          \end{enumerate}
\end{itemize}



\usecase{processiClasseDocumentale}{Visualizza elenco dei processi di una classe documentale}
\begin{figure}[H]
    \centering
    \includegraphics[width=0.9\textwidth]{../assets/uml/UC2.png}
    \caption{\ref{processiClasseDocumentale} - Visualizza elenco dei processi di una classe documentale}
    \label{fig:uc_processiClasseDocumentale}
\end{figure}
\begin{itemize}
    \item \textbf{Attore Primario}: Utente
    \item \textbf{Precondizioni}: \begin{itemize}
        \item L'utente ha avviato l'applicazione
        \item L'Utente ha selezionato una delle Classi Documentali del DIP
    \end{itemize}
    \item \textbf{Postcondizioni}: L'Utente visualizza i processi associati alla classe selezionata.
    \item \textbf{Flusso Principale}: 
          \begin{enumerate}
              \item Il sistema mostra a video l'elenco dei processi della classe documentale selezionata.
          \end{enumerate}
    \item \textbf{Flusso Alternativo}:
    \begin{itemize}
        \item Se la classe documentale non contiene processi, il sistema mostra un elenco vuoto.
    \end{itemize}
    \item \textbf{Inclusioni}: \ref{processoClasseDocumentale} Visualizza processo in elenco
    \item \textbf{Estensioni}: \ref{elencoVuoto} Visualizza elenco vuoto
\end{itemize}
\begin{figure}[H]
    \centering
    \includegraphics[width=0.9\textwidth]{../assets/uml/UC2INC.png}
    \caption{Inclusioni \ref{processiClasseDocumentale} - Visualizza elenco dei processi di una classe documentale}
    \label{fig:inclusioniProcessiClasseDocumentale}
\end{figure}

\subusecase{processoClasseDocumentale}{Visualizza processo in elenco}
\begin{itemize}
    \item \textbf{Attore Primario}: Utente
    \item \textbf{Precondizioni}:
    \begin{itemize}
        \item L'utente ha avviato l'applicazione
        \item L'Utente ha selezionato una delle Classi Documentali del DIP
    \end{itemize}
    \item \textbf{Postcondizioni}: L'Utente visualizza l'elenco dei processi .
    \item \textbf{Flusso Principale}:
    \begin{enumerate}
        \item Per ogni singolo processo presente nell'elenco viene mostrato:
        \begin{itemize}
            \item Id del processo (\ref{idProcessoClasse})
            \item Stato di verifica del processo (\ref{statoVerificaElemento})
            \item Marcatura temporale del processo (\ref{marcaturaTemporaleElemento})
        \end{itemize}
    \end{enumerate}
    \item \textbf{Inclusioni}:
    \begin{itemize}
        \item \ref{idProcessoClasse} Visualizza Id Processo
        \item \ref{statoVerificaElemento} Visualizza Stato di Verifica Elemento
        \item \ref{marcaturaTemporaleElemento} Visualizza Marcatura Temporale Elemento
    \end{itemize}
\end{itemize}


\subsubusecase{idProcessoClasse}{Visualizza Id processo}
\begin{itemize}
    \item \textbf{Attore Primario}: Utente
    \item \textbf{Precondizioni}: \begin{itemize}
        \item L'utente ha avviato l'applicazione
    \end{itemize}
    \item \textbf{Postcondizioni}: L'Utente visualizza l'id del processo.
    \item \textbf{Flusso Principale}:
          \begin{enumerate}
              \item Il sistema mostra a video l'id del processo.
          \end{enumerate}
\end{itemize}


\usecase{documentiProcesso}{Visualizza elenco dei documenti di un processo}
\begin{figure}[H]
    \centering
    \includegraphics[width=0.9\textwidth]{../assets/uml/UC3.png}
    \caption{\ref{documentiProcesso} - Visualizza elenco dei documenti di un processo}
    \label{fig:uc_documentiProcesso}
\end{figure}
\begin{itemize}
    \item \textbf{Attore Primario}: Utente
    \item \textbf{Precondizioni}: 
    \begin{itemize}
        \item L'utente ha avviato l'applicazione
        \item L'Utente ha selezionato una delle Classi Documentali del DIP
        \item L'Utente ha selezionato un processo associato alla classe documentale selezionata
    \end{itemize}
    \item \textbf{Postcondizioni}: L'Utente visualizza l'elenco dei documenti associati al processo selezionato.
    \item \textbf{Flusso Principale}:
    \begin{enumerate}
        \item Il sistema mostra a video l'elenco dei documenti della classe documentale selezionata.
    \end{enumerate}
    \item \textbf{Flusso Alternativo}:
    \begin{itemize}
        \item Se il processo non contiene documenti, il sistema mostra un elenco vuoto.
    \end{itemize}
    \item \textbf{Inclusioni}: \ref{documentoProcesso} Visualizza documento in elenco
    \item \textbf{Estensioni}: \ref{elencoVuoto} Visualizza elenco vuoto
\end{itemize}
\begin{figure}[H]
    \centering
    \includegraphics[width=0.9\textwidth]{../assets/uml/UC3INC.png}
    \caption{Inclusioni \ref{documentiProcesso} - Visualizza elenco dei documenti di un processo}
    \label{fig:inclusioniDocumentiProcesso}
\end{figure}

\subusecase{documentoProcesso}{Visualizza documento in elenco}
\begin{itemize}
    \item \textbf{Attore Primario}: Utente
    \item \textbf{Precondizioni}: 
    \begin{itemize}
        \item L'utente ha avviato l'applicazione
        \item L'Utente ha selezionato una delle Classi Documentali del DIP
        \item L'Utente ha selezionato un processo associato alla classe documentale selezionata
    \end{itemize}
    \item \textbf{Postcondizioni}: L'Utente visualizza l'elenco delle classi documentali.
    \item \textbf{Flusso Principale}:
    \item Per ogni documento presente nell'elenco viene mostrato:
        \begin{itemize}
            \item Nome del documento (\ref{nomeDocumentoProcesso})
            \item Stato di verifica del documento (\ref{statoVerificaElemento})
            \item Marcatura temporale del documento (\ref{marcaturaTemporaleElemento})
        \end{itemize}
    \item \textbf{Inclusioni}:
    \begin{itemize}
        \item \ref{nomeDocumentoProcesso} Visualizza Nome Documento
        \item \ref{statoVerificaElemento} Visualizza Stato di Verifica Elemento
        \item \ref{marcaturaTemporaleElemento} Visualizza Marcatura Temporale Elemento
    \end{itemize}
\end{itemize}


\subsubusecase{nomeDocumentoProcesso}{Visualizza nome documento}
\begin{itemize}
    \item \textbf{Attore Primario}: Utente
    \item \textbf{Precondizioni}: L'Utente ha avviato l'applicazione.
    \item \textbf{Postcondizioni}: L'Utente visualizza il nome del documento.
    \item \textbf{Flusso Principale}:
    \begin{enumerate}
        \item Il sistema mostra a video il nome del documento.
    \end{enumerate}
\end{itemize}

\usecase{statoVerificaElemento}{Visualizza stato di verifica elemento}
\begin{itemize}
    \item \textbf{Attore Primario}: Utente
    \item \textbf{Precondizioni}: L'Utente ha selezionato un elemento.
    \item \textbf{Postcondizioni}: L'utente visualizza lo stato di verifica dell' elemento selezionato.
    \item \textbf{Flusso Principale}:
    \begin{enumerate}
              \item Il sistema mostra a video lo stato di verifica dell'elemento selezionato
    \end{enumerate}
\end{itemize}

\usecase{marcaturaTemporaleElemento}{Visualizza marcatura temporale elemento}
\begin{itemize}
    \item \textbf{Attore Primario}: Utente
    \item \textbf{Precondizioni}: L'Utente ha selezionato un elemento.
    \item \textbf{Postcondizioni}: L'utente visualizza la marcatura temporale dell' elemento selezionato.
    \item \textbf{Flusso Principale}:
    \begin{enumerate}
              \item Il sistema mostra a video la marcatura temporale dell' elemento selezionato
    \end{enumerate}
\end{itemize}


\usecase{elencoVuoto}{Elenco vuoto}
\begin{itemize}
    \item \textbf{Attore Primario}: Utente
    \item \textbf{Precondizioni}: \begin{itemize}
        \item L'utente ha avviato l'applicazione
        \item L'Utente sta visualizzando un elenco di elementi del DIP, che risulta vuoto.
    \end{itemize}
    \item \textbf{Postcondizioni}: L'utente viene notificato che l'elenco è vuoto.
    \item \textbf{Flusso Principale}: 
          \begin{enumerate}
              \item Il sistema mostra a video un messaggio che indica che l'elenco è vuoto.
          \end{enumerate}
\end{itemize}


\usecase{anteprimaDocumento}{Visualizza anteprima documento}
\begin{figure}[H]
    \centering
    \includegraphics[width=0.9\textwidth]{../assets/uml/UC7.png}
    \caption{\ref{anteprimaDocumento} - Visualizza anteprima documento}
    \label{fig:uc_anteprimaDocumento}
\end{figure}
\begin{itemize}
    \item \textbf{Attore Primario}: Utente
    \item \textbf{Precondizioni}: 
    \begin{itemize}
        \item L'utente ha avviato l'applicazione
        \item L'Utente ha selezionato un documento dalla struttura del DIP.
    \end{itemize}
    \item \textbf{Postcondizioni}: L'Utente visualizza un'anteprima del documento selezionato.
    \item \textbf{Flusso Principale}:
    \begin{enumerate}
        \item L'Utente seleziona l'opzione per visualizzare l'anteprima del documento.
        \item Il sistema apre una finestra di anteprima che mostra il contenuto del documento.
    \end{enumerate}
    \item \textbf{Flusso Alternativo}:
    \begin{itemize}
        \item Se il formato del documento non è supportato per l'anteprima, il sistema mostra un messaggio che specifica perché il tipo non è supportato.
    \end{itemize}
    \item \textbf{Estensioni}: \ref{formatoDocumentoNonSupportato} Formato Documento non Supportato dal Sistema per l'anteprima
\end{itemize}

\usecase{formatoDocumentoNonSupportato}{Formato documento non supportato dal sistema per l'anteprima}
\begin{itemize}
    \item \textbf{Attore Primario}: Utente
    \item \textbf{Precondizioni}: 
    \begin{itemize}
        \item L'utente ha avviato l'applicazione
        \item L'Utente ha selezionato un documento
        \item L'Utente ha selezionato l'opzione di Visualizza anteprima per il documento selezionato.
        \item Il formato del documento selezionato non è supportato per l'anteprima.
    \end{itemize}
    \item \textbf{Postcondizioni}: 
    \begin{itemize}
        \item Non viene mostrata l'anteprima del documento.
        \item Viene mostrato un messaggio di errore che indica che il formato del documento non è supportato per l'anteprima.
    \end{itemize}
    \item \textbf{Flusso Principale}:
    \begin{enumerate}
        \item Il sistema chiude annulla l'operazione di anteprima.
        \item Il sistema mostra a video un messaggio di errore che indica che il formato del documento non è supportato per l'anteprima.
    \end{enumerate}
\end{itemize}

