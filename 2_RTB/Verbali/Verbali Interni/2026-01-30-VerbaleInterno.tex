\documentclass[a4paper,12pt]{article} 
\usepackage[utf8]{inputenc} 
\usepackage[T1]{fontenc} 
\usepackage[utf8]{inputenc}
\usepackage[T1]{fontenc} % per i caratteri accentati corretti in PDF
\usepackage[italian]{babel}
\usepackage{lmodern}
\usepackage[sfdefault]{atkinson}
\renewcommand*\familydefault{\sfdefault}
\usepackage{float}
\usepackage{geometry}
\usepackage{xcolor}
\usepackage[most]{tcolorbox}
\usepackage{amssymb}
\usepackage{wasysym}
\usepackage{setspace}
\usepackage{chngpage}
\usepackage{enumitem}
\usepackage{titlesec}
\usepackage{tocloft}
\usepackage{graphicx}
\usepackage{hyperref}
\usepackage{fancyhdr}
\hypersetup{
    colorlinks=true,
    linkcolor=black,
    filecolor=magenta,      
    urlcolor=cyan,
}

% Colori ZPUS - Verde, Nero, Bianco
\definecolor{zpusgreen}{RGB}{4, 138, 55}
\definecolor{zpusdarkgreen}{RGB}{0, 100, 0}
\definecolor{zpusblack}{RGB}{0, 0, 0}
\definecolor{zpuswhite}{RGB}{255, 255, 255}
\definecolor{zpuslightgray}{RGB}{245, 245, 245}

% Stili per i box migliorati
\newtcolorbox{headerbox}{
    colback=zpusgreen,
    colframe=zpusdarkgreen,
    arc=0pt,
    boxrule=0pt,
    left=0pt,
    right=0pt,
    top=8pt,
    bottom=8pt,
    fontupper=\color{zpuswhite}\bfseries\large,
    center
}

\newtcolorbox{infobox}{
    colback=zpuslightgray,
    colframe=zpusgreen,
    arc=4pt,
    boxrule=2pt,
    left=6pt,
    right=6pt,
    top=8pt,
    bottom=8pt,
    fontupper=\color{zpusblack}
}

\newtcolorbox{stepbox}{
    colback=zpuswhite,
    colframe=zpusgreen,
    arc=4pt,
    boxrule=1pt,
    left=6pt,
    right=6pt,
    top=8pt,
    bottom=8pt,
    fontupper=\color{zpusblack}
}

\newtcolorbox{highlightbox}{
    colback=zpusgreen!10,
    colframe=zpusdarkgreen,
    arc=4pt,
    boxrule=2pt,
    left=12pt,
    right=12pt,
    top=12pt,
    bottom=12pt,
    fontupper=\color{zpusblack}\bfseries,
    center
}

\pagestyle{fancy}
\setlength{\headwidth}{\textwidth}
\fancyhfoffset[L,R]{0pt}
\lhead{}
\rhead{7-ZPUs}
\lfoot{}
\rfoot{\thepage}
\cfoot{}
\renewcommand{\headrulewidth}{0.8pt}
\renewcommand{\footrulewidth}{0.8pt}

\renewcommand{\contentsname}{Indice}

\geometry{margin=2.5cm}
\setstretch{1.2}

\titleformat{\section}{\large\bfseries}{\thesection}{1em}{}
\titleformat{\subsection}{\mdseries\bfseries}{\thesubsection}{1em}{}

\definecolor{lightblack}{gray}{0.35}
\newcommand{\glossario}[1]{\textit{#1}\textsubscript{\textbf{\textit{\textcolor{lightblack}{G}}}}}

\begin{document}

\begin{center}
    \includegraphics[width=9.5cm]{../../../assets/logo7zpus.jpg}\\
    \small\hspace{10cm} 7zpus.swe@gmail.com\\
    \Large \textbf{Verbale Interno Gruppo di Progetto}\\
    \vspace{0.5cm}
\end{center}

\noindent
\textbf{Data:} 2026/01/30 \\
\textbf{Durata:} 1 ora\\
\textbf{Luogo:} Incontro online (Discord)

\vspace{0.3cm}
\hrule
\vspace{0.5cm}

\tableofcontents

\newpage


\section*{Tabella di Versionamento}
\begin{table}[H]
    \begin{adjustwidth}{-1cm}{-1cm} % modificare ogni volta in base alla larghezza della tabella per centrarla!!!
    \centering
\begin{tabular}{|c|c|c|c|c|}
    \hline
    \textbf{Versione} & \textbf{Data} & \textbf{Autore}  & \textbf{Verificatore} & \textbf{Descrizione} \\
    \hline
    0.1 & 2026/02/07 & Rocco Matteo A. & Vigolo Davide & \begin{tabular}[c]{@{}l@{}}Creazione del verbale\\ e stesura iniziale\end{tabular} \\
    \hline
\end{tabular}
    \end{adjustwidth}
\end{table}

\section*{Partecipanti}
\begin{itemize}[noitemsep]
    \item Fattoni Antonio 
    \item Georgescu Diana
    \item Laoud Zakaria
    \item Rocco Matteo Alberto
    \item Soligo Lorenzo
    \item Vigolo Davide
\end{itemize}

\section{Ordine del Giorno}
\begin{enumerate}[noitemsep]
    \item Retrospettiva dello sprint e riflessioni
    \item Definizione dei ruoli
    \item Pianificazione delle prossime attività e decisioni
\end{enumerate}


\section{Svolgimento e Discussione}

\subsection{Retrospettiva dello sprint e riflessioni}
Il gruppo ha discusso le difficoltà incontrate durante lo \glossario{sprint} 5, in particolare quelle legate alla rendicontazione e pianificazione del tempo necessario per lo svolgimento delle attività. È emersa l'importanza di una pianificazione più dettagliata e dell'utilizzo più rigoroso di Jira, in particolare la stima del tempo necessario per ogni task e l'inserimento del tempo effettivamente impiegato.\\
Per avere un quadro più chiaro e immediato della rendicontazione del tempo collegata al budget di progetto si è deciso di normare l'utilizzo di un apposito foglio di calcolo condiviso che comprende tabelle aggiornate dinamicamente in base ai tempi inseriti, contenenti i costi previsti ed effettivi per ogni sprint e per ogni membro del gruppo in relazione al ruolo svolto.\\

Si è rivelata inefficace la pianificazione iniziale delle attività dello scorso periodo, caratterizzati da un carico di lavoro eccessivo in rapporto al tempo disponibile, notevolmente ridotto a causa della sessione d'esami e della concomitanza con la consegna del progetto didattico del corso di Tecnologie Web.\\
In particolare nonostante fosse stato previsto il rischio \href{https://cdn.jsdelivr.net/gh/7-zpus/Docs@main/2_RTB/PianoDiProgetto.pdf}{RO05} le azioni intraprese per mitigarlo non sono state sufficienti a garantire il completamento di tutte le attività pianificate, causando il non rispetto delle dipendenze tra task e quindi amplificando ulteriormente i ritardi.\\

Tenendo conto delle osservazioni di cui sopra, il gruppo ha deciso di approfondire la fase di analisi delle attività da svolgere, identificando con maggiore precisione le dipendenze tra task e stimando in modo più accurato il tempo necessario per il loro completamento.\\ 

\subsection{Definizione dei ruoli}
I ruoli principali assegnati a ciascun membro per questo sprint sono i seguenti:
\begin{table}[H]
    \begin{adjustwidth}{-1cm}{-1cm}
    \centering
\begin{tabular}{|c|c|}
    \hline
    \textbf{Membro} & \textbf{Ruolo} \\
    \hline
    Laoud Zakaria & Responsabile \\
    \hline
    Georgescu Diana & Amministratore \\
    \hline
    Gingilino Aaron & Analista \\
    \hline
    Fattoni Antonio & Programmatore \\
    \hline
    Rocco Matteo A. & Analista \\
    \hline
    Soligo Lorenzo & Programmatore \\
    \hline
    Vigolo Davide & Amministratore \\
    \hline
\end{tabular}
    \end{adjustwidth}
\end{table}
\textbf{Tutti} i membri del gruppo faranno anche da \textbf{verificatori}, inoltre alcuni membri si occuperanno della scrittura delle norme di progetto, con ruolo conformemente assegnato a seconda del paragrafo di competenza.

\section{Pianificazione delle prossime attività e decisioni}
Il gruppo ha individuato le attività principali da svolgere durante il prossimo sprint, che includono:
\begin{enumerate}[noitemsep]
    \item Completamento del documento di analisi dei requisiti, in particolare la sezione di tracciamento fonti-requisiti e dei requisiti di qualità e di vincolo.
    \item Completamento del testing sulla Ricerca Semantica per il PoC.
    \item Controllo formale dei documenti prodotti finora e aggiornamento della repository con le versioni finali dei documenti, in vista della baseline RTB.
    \item Aggiornamento dei paragrafi mancanti delle norme di progetto, con assegnazione dei compiti in base alle competenze e ai ruoli dei membri del gruppo.
    \item Invio di una mail a Sanmarco Informatica per posticipare il prossimo incontro di allineamento previsto per il prossimo 5 febbraio per poter avere più materiale da mostrare e discutere durante l'incontro, in particolare per quanto riguarda i risultati dei test sulla Ricerca Semantica.
    \item Completamento del Piano di Qualifica.
    \item Aggiornamento del Glossario con i termini emersi durante l'analisi dei requisiti e lo studio delle tecnologie per il Proof of Concept.
    \item Definizione attività di preparazione alla baseline RTB.
\end{enumerate}


\section*{Tabella delle decisioni}
\begin{table}[H]
    \begin{adjustwidth}{-4cm}{-4cm}
    \centering
\begin{tabular}{|c|c|c|c|c|}
    \hline
    \textbf{Decisione} & \textbf{To Do} & \textbf{Jira Issue} & \textbf{Membro assegnato} & \textbf{Verificatore} \\
    \hline
    \#0 & \begin{tabular}[c]{@{}c@{}} Redazione del verbale\\ interno \end{tabular} & \href{https://7zpus.atlassian.net/browse/DIPR-211}{DIPR-211} & Rocco Matteo A. & Vigolo Davide \\
    \hline
    \#0 & \begin{tabular}[c]{@{}c@{}} Consuntivo Sprint 5 \end{tabular} & \href{https://7zpus.atlassian.net/browse/DIPR-233}{DIPR-233} & Fattoni Antonio & Laoud Zakaria \\
    \hline
    \#0 & \begin{tabular}[c]{@{}c@{}} Preventivo e\\ consuntivo Sprint 6 \end{tabular} & \href{https://7zpus.atlassian.net/browse/DIPR-236}{DIPR-236} & Laoud Zakaria & Georgescu Diana \\
    \hline
    \#1 & \begin{tabular}[c]{@{}c@{}}Stesura requisiti\\ qualità e vincolo \end{tabular} & \href{https://7zpus.atlassian.net/browse/DIPR-215}{DIPR-215} & Rocco Matteo A. & Georgescu Diana \\
    \hline
    \#2 & \begin{tabular}[c]{@{}c@{}} Studio delle tecnologie\\ per Ricerca Semantica  \end{tabular}&  \href{https://7zpus.atlassian.net/browse/DIPR-200}{DIPR-200} & Soligo Lorenzo & Fattoni Antonio \\
    \hline
    \#3 & \begin{tabular}[c]{@{}c@{}} Revisione generale\\ forma documentazione  \end{tabular}& \begin{tabular}[c]{@{}c@{}} \href{https://7zpus.atlassian.net/browse/DIPR-242}{DIPR-242} \end{tabular} & Rocco Matteo A. & Team completo \\
    \hline
    \#3 & Aggiornamento sito web & \href{https://7zpus.atlassian.net/browse/DIPR-243}{DIPR-243} & Fattoni Antonio & Laoud Zakaria \\
    \hline
    \#4 & \begin{tabular}[c]{@{}c@{}}Completamento Paragrafi \\ delle Norme di \\Progetto mancanti  \end{tabular}& \begin{tabular}[c]{@{}c@{}} \href{https://7zpus.atlassian.net/browse/DIPR-225}{DIPR-225} \\ \href{https://7zpus.atlassian.net/browse/DIPR-230}{DIPR-230} \end{tabular} & \begin{tabular}[c]{@{}c@{}} Vigolo Davide \\ Fattoni Antonio \end{tabular} & \begin{tabular}[c]{@{}c@{}} Soligo Lorenzo \\ Vigolo Davide \end{tabular} \\
    \hline
    \#5 & \begin{tabular}[c]{@{}c@{}}Invio mail di \\ posticipa a Sanmarco \end{tabular} & --- & Fattoni Antonio & --- \\
    \hline
    \#6 & \begin{tabular}[c]{@{}c@{}}Completamento sezione \\di cruscotto di qualità\\ e inserimento sezione\\ di test \end{tabular} & \begin{tabular}[c]{@{}c@{}} \href{https://7zpus.atlassian.net/browse/DIPR-225}{DIPR-225} \\ \href{https://7zpus.atlassian.net/browse/DIPR-224}{DIPR-224} \end{tabular}  & \begin{tabular}[c]{@{}c@{}} Davide Vigolo \\ Rocco Matteo A. \end{tabular}  & \begin{tabular}[c]{@{}c@{}} Soligo Lorenzo \\Gingillino Aaron \end{tabular} \\
    \hline
    \#7 & Aggiornamento Glossario & \href{https://7zpus.atlassian.net/browse/DIPR-227}{DIPR-227} & Georgescu Diana & Laoud Zakaria \\
    \hline
    \#8 & \begin{tabular}[c]{@{}c@{}} Stesura lettera di \\ candidatura per RTB \end{tabular} & \href{https://7zpus.atlassian.net/browse/DIPR-239}{DIPR-239} & Laoud Zakaria & Team completo \\
    \hline

\end{tabular}
    \end{adjustwidth}
\end{table}

\vfill
\begin{flushright}
    \textit{7-ZPUs}
\end{flushright}

\end{document}