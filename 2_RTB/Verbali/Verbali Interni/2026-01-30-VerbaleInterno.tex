\documentclass[a4paper,12pt]{article} 
\usepackage[utf8]{inputenc} 
\usepackage[T1]{fontenc} 
\usepackage[utf8]{inputenc}
\usepackage[T1]{fontenc} % per i caratteri accentati corretti in PDF
\usepackage[italian]{babel}
\usepackage{lmodern}
\usepackage[sfdefault]{atkinson}
\renewcommand*\familydefault{\sfdefault}
\usepackage{float}
\usepackage{geometry}
\usepackage{xcolor}
\usepackage[most]{tcolorbox}
\usepackage{amssymb}
\usepackage{wasysym}
\usepackage{setspace}
\usepackage{chngpage}
\usepackage{enumitem}
\usepackage{titlesec}
\usepackage{tocloft}
\usepackage{graphicx}
\usepackage{hyperref}
\usepackage{fancyhdr}
\hypersetup{
    colorlinks=true,
    linkcolor=black,
    filecolor=magenta,      
    urlcolor=cyan,
}

% Colori ZPUS - Verde, Nero, Bianco
\definecolor{zpusgreen}{RGB}{4, 138, 55}
\definecolor{zpusdarkgreen}{RGB}{0, 100, 0}
\definecolor{zpusblack}{RGB}{0, 0, 0}
\definecolor{zpuswhite}{RGB}{255, 255, 255}
\definecolor{zpuslightgray}{RGB}{245, 245, 245}

% Stili per i box migliorati
\newtcolorbox{headerbox}{
    colback=zpusgreen,
    colframe=zpusdarkgreen,
    arc=0pt,
    boxrule=0pt,
    left=0pt,
    right=0pt,
    top=8pt,
    bottom=8pt,
    fontupper=\color{zpuswhite}\bfseries\large,
    center
}

\newtcolorbox{infobox}{
    colback=zpuslightgray,
    colframe=zpusgreen,
    arc=4pt,
    boxrule=2pt,
    left=6pt,
    right=6pt,
    top=8pt,
    bottom=8pt,
    fontupper=\color{zpusblack}
}

\newtcolorbox{stepbox}{
    colback=zpuswhite,
    colframe=zpusgreen,
    arc=4pt,
    boxrule=1pt,
    left=6pt,
    right=6pt,
    top=8pt,
    bottom=8pt,
    fontupper=\color{zpusblack}
}

\newtcolorbox{highlightbox}{
    colback=zpusgreen!10,
    colframe=zpusdarkgreen,
    arc=4pt,
    boxrule=2pt,
    left=12pt,
    right=12pt,
    top=12pt,
    bottom=12pt,
    fontupper=\color{zpusblack}\bfseries,
    center
}

\pagestyle{fancy}
\setlength{\headwidth}{\textwidth}
\fancyhfoffset[L,R]{0pt}
\lhead{}
\rhead{7-ZPUs}
\lfoot{}
\rfoot{\thepage}
\cfoot{}
\renewcommand{\headrulewidth}{0.8pt}
\renewcommand{\footrulewidth}{0.8pt}

\renewcommand{\contentsname}{Indice}

\geometry{margin=2.5cm}
\setstretch{1.2}

\titleformat{\section}{\large\bfseries}{\thesection}{1em}{}
\titleformat{\subsection}{\mdseries\bfseries}{\thesubsection}{1em}{}

\definecolor{lightblack}{gray}{0.35}
\newcommand{\glossario}[1]{\textit{#1}\textsubscript{\textbf{\textit{\textcolor{lightblack}{G}}}}}

\begin{document}

\begin{center}
    \includegraphics[width=9.5cm]{../../../assets/logo7zpus.jpg}\\
    \small\hspace{10cm} 7zpus.swe@gmail.com\\
    \Large \textbf{Verbale Interno Gruppo di Progetto}\\
    \vspace{0.5cm}
\end{center}

\noindent
\textbf{Data:} 2026/01/30 \\
\textbf{Durata:} 1 ora\\
\textbf{Luogo:} Incontro online (Discord)

\vspace{0.3cm}
\hrule
\vspace{0.5cm}

\tableofcontents

\newpage


\section*{Tabella di Versionamento}
\begin{table}[H]
    \begin{adjustwidth}{-1cm}{-1cm} % modificare ogni volta in base alla larghezza della tabella per centrarla!!!
    \centering
\begin{tabular}{|c|c|c|c|c|}
    \hline
    \textbf{Versione} & \textbf{Data} & \textbf{Autore}  & \textbf{Verificatore} & \textbf{Descrizione} \\
    \hline
    0.1 & 2026/02/07 & Rocco Matteo A. & Vigolo Davide & \begin{tabular}[c]{@{}l@{}}Creazione del verbale\\ e stesura iniziale\end{tabular} \\
    \hline
\end{tabular}
    \end{adjustwidth}
\end{table}

\section*{Partecipanti}
\begin{itemize}[noitemsep]
    \item Fattoni Antonio 
    \item Georgescu Diana
    \item Laoud Zakaria
    \item Rocco Matteo Alberto
    \item Soligo Lorenzo
    \item Vigolo Davide
\end{itemize}

\section{Ordine del Giorno}
\begin{enumerate}[noitemsep]
    \item Retrospettiva dello sprint e riflessioni
    \item Definizione dei ruoli
    \item Pianificazione delle prossime attività e decisioni
\end{enumerate}


\section{Svolgimento e Discussione}

\subsection{Retrospettiva dello sprint e riflessioni}
Il gruppo ha discusso le difficoltà dello \glossario{sprint} 5. I problemi principali riguardano la stima del tempo e la sua registrazione. È emerso che serve una migliore pianificazione e bisogna usare Jira in modo più rigoroso. Ogni task deve avere una stima del tempo e va registrato correttamente e in modo veritiero.\\
Perciò il gruppo ha deciso di usare un foglio di calcolo condiviso, questo aiuta a vedere subito il tempo usato e il budget. Il foglio ha tabelle ad aggiornamento automatico e mostra i costi per ogni sprint e per ogni membro. I costi sono divisi per ruolo.\\

La pianificazione iniziale delle attività non è stata efficace. Il carico di lavoro era troppo alto e il tempo disponibile era poco, inoltre la sessione d'esami ha ridotto il tempo ulteriormente. Anche la consegna del progetto di Tecnologie Web ha creato problemi.\\
Il gruppo aveva previsto il rischio \href{https://cdn.jsdelivr.net/gh/7-zpus/Docs@main/2_RTB/PianoDiProgetto.pdf}{RO05}, ma le azioni di mitigazione non sono bastate. Non tutte le attività sono state completate, questo ha causato ritardi e di conseguenza le dipendenze tra task non sono state rispettate. I ritardi si sono così amplificati.\\

Il gruppo ha fatto tesoro di queste osservazioni. Ha deciso di analizzare meglio le attività da fare. Le dipendenze tra task vanno identificate con più cura. La stima del tempo deve essere più precisa.\\ 

\subsection{Definizione dei ruoli}
I ruoli principali assegnati a ciascun membro per questo sprint sono i seguenti:
\begin{table}[H]
    \begin{adjustwidth}{-1cm}{-1cm}
    \centering
\begin{tabular}{|c|c|}
    \hline
    \textbf{Membro} & \textbf{Ruolo} \\
    \hline
    Laoud Zakaria & Responsabile \\
    \hline
    Georgescu Diana & Amministratore \\
    \hline
    Gingilino Aaron & Analista \\
    \hline
    Fattoni Antonio & Programmatore \\
    \hline
    Rocco Matteo A. & Analista \\
    \hline
    Soligo Lorenzo & Programmatore \\
    \hline
    Vigolo Davide & Amministratore \\
    \hline
\end{tabular}
    \end{adjustwidth}
\end{table}
\textbf{Tutti} i membri del gruppo faranno anche da \textbf{verificatori}. Alcuni membri si occuperanno della scrittura delle norme di progetto, con ruolo conformemente assegnato a seconda del paragrafo di competenza.

\section{Pianificazione delle prossime attività e decisioni}
Il gruppo ha individuato le attività per il prossimo sprint:
\begin{enumerate}[noitemsep]
    \item Completare l'analisi dei requisiti. Finire il tracciamento fonti-requisiti. Completare i requisiti di qualità e vincolo.
    \item Finire i test sulla Ricerca Semantica per il PoC.
    \item Fare il controllo formale dei documenti. Caricare le versioni finali sulla repository. Questo serve per la baseline RTB.
    \item Aggiornare le norme di progetto. I compiti vanno assegnati in base ai ruoli e alle competenze.
    \item Inviare una mail a Sanmarco Informatica per chiedere di spostare l'incontro del 5 febbraio. Serve più tempo per avere materiale da mostrare, in particolare i risultati dei test sulla Ricerca Semantica.
    \item Completare il Piano di Qualifica.
    \item Aggiornare il Glossario aggiungendo i termini nuovi. Questi vengono dall'analisi dei requisiti e dallo studio del PoC.
    \item Definire le attività per la baseline RTB.
\end{enumerate}


\section*{Tabella delle decisioni}
\begin{table}[H]
    \begin{adjustwidth}{-4cm}{-4cm}
    \centering
\begin{tabular}{|c|c|c|c|c|}
    \hline
    \textbf{Decisione} & \textbf{To Do} & \textbf{Jira Issue} & \textbf{Membro assegnato} & \textbf{Verificatore} \\
    \hline
    \#0 & \begin{tabular}[c]{@{}c@{}} Redazione del verbale\\ interno \end{tabular} & \href{https://7zpus.atlassian.net/browse/DIPR-211}{DIPR-211} & Rocco Matteo A. & Vigolo Davide \\
    \hline
    \#0 & \begin{tabular}[c]{@{}c@{}} Consuntivo Sprint 5 \end{tabular} & \href{https://7zpus.atlassian.net/browse/DIPR-233}{DIPR-233} & Fattoni Antonio & Laoud Zakaria \\
    \hline
    \#0 & \begin{tabular}[c]{@{}c@{}} Preventivo e\\ consuntivo Sprint 6 \end{tabular} & \href{https://7zpus.atlassian.net/browse/DIPR-236}{DIPR-236} & Laoud Zakaria & Georgescu Diana \\
    \hline
    \#1 & \begin{tabular}[c]{@{}c@{}}Stesura requisiti\\ qualità e vincolo \end{tabular} & \href{https://7zpus.atlassian.net/browse/DIPR-215}{DIPR-215} & Rocco Matteo A. & Georgescu Diana \\
    \hline
    \#2 & \begin{tabular}[c]{@{}c@{}} Studio delle tecnologie\\ per Ricerca Semantica  \end{tabular}&  \href{https://7zpus.atlassian.net/browse/DIPR-200}{DIPR-200} & Soligo Lorenzo & Fattoni Antonio \\
    \hline
    \#3 & \begin{tabular}[c]{@{}c@{}} Revisione generale\\ forma documentazione  \end{tabular}& \begin{tabular}[c]{@{}c@{}} \href{https://7zpus.atlassian.net/browse/DIPR-242}{DIPR-242} \end{tabular} & Rocco Matteo A. & Team completo \\
    \hline
    \#3 & Aggiornamento sito web & \href{https://7zpus.atlassian.net/browse/DIPR-243}{DIPR-243} & Fattoni Antonio & Laoud Zakaria \\
    \hline
    \#4 & \begin{tabular}[c]{@{}c@{}}Completamento Paragrafi \\ delle Norme di \\Progetto mancanti  \end{tabular}& \begin{tabular}[c]{@{}c@{}} \href{https://7zpus.atlassian.net/browse/DIPR-225}{DIPR-225} \\ \href{https://7zpus.atlassian.net/browse/DIPR-230}{DIPR-230} \end{tabular} & \begin{tabular}[c]{@{}c@{}} Vigolo Davide \\ Fattoni Antonio \end{tabular} & \begin{tabular}[c]{@{}c@{}} Soligo Lorenzo \\ Vigolo Davide \end{tabular} \\
    \hline
    \#5 & \begin{tabular}[c]{@{}c@{}}Invio mail di \\ posticipa a Sanmarco \end{tabular} & --- & Fattoni Antonio & --- \\
    \hline
    \#6 & \begin{tabular}[c]{@{}c@{}}Completamento sezione \\di cruscotto di qualità\\ e inserimento sezione\\ di test \end{tabular} & \begin{tabular}[c]{@{}c@{}} \href{https://7zpus.atlassian.net/browse/DIPR-225}{DIPR-225} \\ \href{https://7zpus.atlassian.net/browse/DIPR-224}{DIPR-224} \end{tabular}  & \begin{tabular}[c]{@{}c@{}} Davide Vigolo \\ Rocco Matteo A. \end{tabular}  & \begin{tabular}[c]{@{}c@{}} Soligo Lorenzo \\Gingillino Aaron \end{tabular} \\
    \hline
    \#7 & Aggiornamento Glossario & \href{https://7zpus.atlassian.net/browse/DIPR-227}{DIPR-227} & Georgescu Diana & Laoud Zakaria \\
    \hline
    \#8 & \begin{tabular}[c]{@{}c@{}} Stesura lettera di \\ candidatura per RTB \end{tabular} & \href{https://7zpus.atlassian.net/browse/DIPR-239}{DIPR-239} & Laoud Zakaria & Team completo \\
    \hline

\end{tabular}
    \end{adjustwidth}
\end{table}

\vfill
\begin{flushright}
    \textit{7-ZPUs}
\end{flushright}

\end{document}