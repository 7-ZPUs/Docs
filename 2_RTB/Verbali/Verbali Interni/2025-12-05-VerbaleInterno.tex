\documentclass[a4paper,12pt]{article}
\usepackage[utf8]{inputenc}
\usepackage[T1]{fontenc} % per i caratteri accentati corretti in PDF
\usepackage[italian]{babel}
\usepackage{lmodern}
\renewcommand*\familydefault{\sfdefault}
\usepackage{float}
\usepackage{geometry}
\usepackage{xcolor}
\usepackage[most]{tcolorbox}
\usepackage{amssymb}
\usepackage{wasysym}
\usepackage{setspace}
\usepackage{chngpage}
\usepackage{enumitem}
\usepackage{titlesec}
\usepackage{tocloft}
\usepackage{graphicx}
\usepackage{hyperref}
\usepackage{fancyhdr}
\hypersetup{
    colorlinks=true,
    linkcolor=black,
    filecolor=magenta,      
    urlcolor=cyan,
}

% Colori ZPUS - Verde, Nero, Bianco
\definecolor{zpusgreen}{RGB}{4, 138, 55}
\definecolor{zpusdarkgreen}{RGB}{0, 100, 0}
\definecolor{zpusblack}{RGB}{0, 0, 0}
\definecolor{zpuswhite}{RGB}{255, 255, 255}
\definecolor{zpuslightgray}{RGB}{245, 245, 245}

% Stili per i box migliorati
\newtcolorbox{headerbox}{
    colback=zpusgreen,
    colframe=zpusdarkgreen,
    arc=0pt,
    boxrule=0pt,
    left=0pt,
    right=0pt,
    top=8pt,
    bottom=8pt,
    fontupper=\color{zpuswhite}\bfseries\large,
    center
}

\newtcolorbox{infobox}{
    colback=zpuslightgray,
    colframe=zpusgreen,
    arc=4pt,
    boxrule=2pt,
    left=6pt,
    right=6pt,
    top=8pt,
    bottom=8pt,
    fontupper=\color{zpusblack}
}

\newtcolorbox{stepbox}{
    colback=zpuswhite,
    colframe=zpusgreen,
    arc=4pt,
    boxrule=1pt,
    left=6pt,
    right=6pt,
    top=8pt,
    bottom=8pt,
    fontupper=\color{zpusblack}
}

\newtcolorbox{highlightbox}{
    colback=zpusgreen!10,
    colframe=zpusdarkgreen,
    arc=4pt,
    boxrule=2pt,
    left=12pt,
    right=12pt,
    top=12pt,
    bottom=12pt,
    fontupper=\color{zpusblack}\bfseries,
    center
}

\pagestyle{fancy}
\setlength{\headwidth}{\textwidth}
\fancyhfoffset[L,R]{0pt}
\lhead{}
\rhead{7-ZPUs}
\lfoot{}
\rfoot{\thepage}
\cfoot{}
\renewcommand{\headrulewidth}{0.8pt}
\renewcommand{\footrulewidth}{0.8pt}

\renewcommand{\contentsname}{Indice}

\geometry{margin=2.5cm}
\setstretch{1.2}

\titleformat{\section}{\large\bfseries}{\thesection}{1em}{}
\titleformat{\subsection}{\mdseries\bfseries}{\thesubsection}{1em}{}

\begin{document}

\begin{center}
    \includegraphics[width=9.5cm]{../../../assets/logo7zpus.jpg}\\
    \small\hspace{10cm} 7zpus.swe@gmail.com\\
    \Large \textbf{Verbale Interno Gruppo di Progetto}\\
    \vspace{0.5cm}
\end{center}

\noindent
\textbf{Data:} 05/12/2025 \\
\textbf{Durata:} 1 ora\\
\textbf{Luogo:} Incontro online (Discord)

\vspace{0.3cm}
\hrule
\vspace{0.5cm}

\section*{Tabella di Versionamento}
\begin{table}[H]
    \begin{adjustwidth}{-1cm}{-1cm} % modificare ogni volta in base alla larghezza della tabella per centrarla!!!
    \centering
\begin{tabular}{|c|c|c|c|c|}
    \hline
    \textbf{Versione} & \textbf{Data} & \textbf{Autore}  & \textbf{Verificatore} & \textbf{Descrizione} \\
    \hline
    0.1 & 11/12/2025 & Zakaria Laoud & Matteo Rocco A. & Creazione del verbale e stesura iniziale \\
    \hline
\end{tabular}
    \end{adjustwidth}
\end{table}

\tableofcontents

\newpage

\section*{Partecipanti}
\begin{itemize}[noitemsep]
    \item Fattoni Antonio 
    \item Georgescu Diana
    \item Gingillino Aaron
    \item Laoud Zakaria
    \item Rocco Matteo Alberto
    \item Soligo Lorenzo
    \item Vigolo Davide
\end{itemize}

\section{Ordine del Giorno}
\begin{enumerate}[noitemsep]
    \item Decisione dei ruoli con relativa suddivisione dei compiti 
    \item Retrospettiva settimanale
    \item Resoconto sull'andamento della stesura dell'analisi dei requisiti
\end{enumerate}


\section{Svolgimento e Discussione}

\subsection{Definizione ruoli}
I ruoli necessari in questa sprint sono:
\begin{itemize}[noitemsep]
    \item Responsabile
    \item Amministratore
    \item Analista
    \item Verificatore
\end{itemize}

I ruoli assegnati a ciascun membro per questo sprint sono i seguenti:
\begin{table}[H]
    \begin{adjustwidth}{-1cm}{-1cm}
    \centering
\begin{tabular}{|c|c|}
    \hline
    \textbf{Membro} & \textbf{Ruolo} \\
    \hline
    Fattoni Antonio & Analista \\
    \hline
    Georgescu Diana & Amministratore \\
    \hline
    Gingillino Aaron & Amministratore \\
    \hline
    Laoud Zakaria & Amministratore \\
    \hline
    Rocco Matteo A. & Responsabile \\
    \hline
    Soligo Lorenzo & Analista \\
    \hline
    Vigolo Davide & Amministratore \\
    \hline
\end{tabular}
    \end{adjustwidth}
\end{table}
\textbf{Tutti} i membri avranno anche ruolo di \textbf{verificatore}.

\subsection{Decisione ruoli}
Sono stati suddivisi i ruoli necessari per lo svolgimento delle attività, lasciando la 
prosecuzione dell’analisi dei requisiti agli Analisti, mentre l’aggiornamento del piano di progetto è stato affidato ai Responsabili.

\subsection{Retrospettiva settimanale}
È stata svolta la retrospettiva della seconda settimana dello sprint e si è deciso di segnare il tempo stimato per ogni attività dello sprint.

\subsection{Aggiornamento analisi dei requisiti}
Durante la riunione è stato realizzato un mockup dell’applicazione, con l’obiettivo di individuare eventuali use case mancanti nell’analisi dei requisiti. 
Tale mockup sarà inoltre presentato e discusso con la proponente nell’incontro dell’11/12/2025 per raccogliere feedback e suggerimenti. 
Si è inoltre deciso di ricercare un’alternativa alla visualizzazione ad albero per la rappresentazione delle informazioni, al fine di migliorare l’usabilità dell’interfaccia utente.

\subsubsection{Analisi possibile implementazione di FAISS}
Durante la riunione è stata discussa la possibile implementazione di FAISS ( Facebook AI Similarity Search ), per migliorare le capacità di ricerca e recupero delle informazioni all'interno del nostro sistema.
A tale scopo il gruppo si è preposto di discuterne con la proponente all'incontro del 11/12/2025.

\subsubsection{Ricevimento con il professor Cardin}Per chiarire alcuni dubbi emersi durante la stesura dell’analisi dei requisiti, si è deciso di inviare una mail 
al professor Cardin per fissare un appuntamento in cui discutere tali punti e ottenere chiarimenti utili per il proseguimento del progetto.

\subsection{Altre tematiche discusse}
Oltre alle tematiche principali, sono stati discussi anche altri argomenti rilevanti per il progetto. Infatti si è deciso di:
\begin{itemize}
    \item procedere alla stesura del Piano di Qualifica
    \item aggiornare le norme di progetto ( dalla sezione 3.3 alla 3.8 e la 5)
\end{itemize}

\section{Decisioni}
\begin{enumerate}[noitemsep]
    \item Si è deciso di continuare con la stesura dell'analisi dei requisiti.
    \item Si è deciso di continuare la stesura del piano di progetto.
    \item Si è deciso di procedere alla stesura del consuntivo e preventivo dello sprint 2.
    \item Si è deciso di procedere alla stesura del preventivo dello sprint 3.
    \item Si è deciso di iniziare la stesura del piano di qualifica.
    \item Si è deciso di aggiornare le norme di progetto.
\end{enumerate}

\section*{Tabella delle decisioni}
\begin{table}[H]
    \begin{adjustwidth}{-4cm}{-4cm} % modificare ogni volta in base alla larghezza della tabella per centrarla!!!
    \centering
\begin{tabular}{|c|c|c|c|c|}
    \hline
    \textbf{Decisione} & \textbf{To Do} & \textbf{Jira Issue} & \textbf{Membro assegnato} & \textbf{Verificatore} \\
    \hline
    \#0 & Redazione del verbale & \href{https://7zpus.atlassian.net/browse/DIPR-121}{DIPR-121} & Zakaria Laoud & Matteo Rocco A. \\
    \hline
    \#1 & Analisi dei requisiti & \href{https://7zpus.atlassian.net/browse/DIPR-131}{DIPR-131} & Analisti & Analisti \\
    \hline
    \#2 & Aggiornamento piano di progetto & \href{https://7zpus.atlassian.net/browse/DIPR-126}{DIPR-126} & Rocco Matteo A. &  Responsabili\\
    \hline
    \#3 & Stesura preventivo e consuntivo sprint 2 & \href{https://7zpus.atlassian.net/browse/DIPR-130}{DIPR-130} & Davide Vigolo & Rocco Matteo A.\\
    \hline
    \#4 & Stesura preventivo sprint 3 & \href{https://7zpus.atlassian.net/browse/DIPR-129}{DIPR-129} & Rocco Matteo A. & Davide Vigolo\\
    \hline
    \#5 & Inizio stesura piano di qualifica & \href{https://7zpus.atlassian.net/browse/DIPR-138}{DIPR-138} & Rocco Matteo A. & Davide Vigolo\\
    \hline
    \#6 & Aggiornamento norme di progetto &  & Amministratori & Amministratori\\
    \hline
\end{tabular}
    \end{adjustwidth}
\end{table}

\vfill
\begin{flushright}
    \textit{7-ZPUs}
\end{flushright}

\end{document}
