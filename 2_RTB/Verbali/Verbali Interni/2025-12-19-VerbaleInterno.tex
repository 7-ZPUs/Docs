\documentclass[a4paper,12pt]{article}
\usepackage[utf8]{inputenc}
\usepackage[T1]{fontenc} % per i caratteri accentati corretti in PDF
\usepackage[italian]{babel}
\usepackage{lmodern}
\usepackage[sfdefault]{atkinson}
\renewcommand*\familydefault{\sfdefault}
\usepackage{float}
\usepackage{geometry}
\usepackage{xcolor}
\usepackage[most]{tcolorbox}
\usepackage{amssymb}
\usepackage{wasysym}
\usepackage{setspace}
\usepackage{chngpage}
\usepackage{enumitem}
\usepackage{titlesec}
\usepackage{tocloft}
\usepackage{graphicx}
\usepackage{hyperref}
\usepackage{bookmark}
\usepackage{fancyhdr}
\hypersetup{
    colorlinks=true,
    linkcolor=black,
    filecolor=magenta,      
    urlcolor=cyan,
}

% Colori ZPUS - Verde, Nero, Bianco
\definecolor{zpusgreen}{RGB}{4, 138, 55}
\definecolor{zpusdarkgreen}{RGB}{0, 100, 0}
\definecolor{zpusblack}{RGB}{0, 0, 0}
\definecolor{zpuswhite}{RGB}{255, 255, 255}
\definecolor{zpuslightgray}{RGB}{245, 245, 245}

% Stili per i box migliorati
\newtcolorbox{headerbox}{
    colback=zpusgreen,
    colframe=zpusdarkgreen,
    arc=0pt,
    boxrule=0pt,
    left=0pt,
    right=0pt,
    top=8pt,
    bottom=8pt,
    fontupper=\color{zpuswhite}\bfseries\large,
    center
}

\newtcolorbox{infobox}{
    colback=zpuslightgray,
    colframe=zpusgreen,
    arc=4pt,
    boxrule=2pt,
    left=6pt,
    right=6pt,
    top=8pt,
    bottom=8pt,
    fontupper=\color{zpusblack}
}

\newtcolorbox{stepbox}{
    colback=zpuswhite,
    colframe=zpusgreen,
    arc=4pt,
    boxrule=1pt,
    left=6pt,
    right=6pt,
    top=8pt,
    bottom=8pt,
    fontupper=\color{zpusblack}
}

\newtcolorbox{highlightbox}{
    colback=zpusgreen!10,
    colframe=zpusdarkgreen,
    arc=4pt,
    boxrule=2pt,
    left=12pt,
    right=12pt,
    top=12pt,
    bottom=12pt,
    fontupper=\color{zpusblack}\bfseries,
    center
}

\pagestyle{fancy}
\setlength{\headwidth}{\textwidth}
\fancyhfoffset[L,R]{0pt}
\lhead{}
\rhead{7-ZPUs}
\lfoot{}
\rfoot{\thepage}
\cfoot{}
\renewcommand{\headrulewidth}{0.8pt}
\renewcommand{\footrulewidth}{0.8pt}

\renewcommand{\contentsname}{Indice}

\geometry{margin=2.5cm}
\setlength{\headheight}{14.49998pt}
\setstretch{1.2}

\titleformat{\section}{\large\bfseries}{\thesection}{1em}{}
\titleformat{\subsection}{\mdseries\bfseries}{\thesubsection}{1em}{}

\begin{document}

\begin{center}
    \includegraphics[width=9.5cm]{../../../assets/logo7zpus.jpg}\\
    \small\hspace{10cm} 7zpus.swe@gmail.com\\
    \Large \textbf{Verbale Interno Gruppo di Progetto}\\
    \vspace{0.5cm}
\end{center}

\noindent
\textbf{Data:} 19/12/2025 \\
\textbf{Durata:} 1 ora\\
\textbf{Luogo:} Incontro online (Discord)

\vspace{0.3cm}
\hrule
\vspace{0.5cm}

\section*{Tabella di Versionamento}
\begin{table}[H]
    \begin{adjustwidth}{-1cm}{-1cm} % modificare ogni volta in base alla larghezza della tabella per centrarla!!!
    \centering
\begin{tabular}{|c|c|c|c|c|}
    \hline
    \textbf{Versione} & \textbf{Data} & \textbf{Autore}  & \textbf{Verificatore} & \textbf{Descrizione} \\
    \hline
    0.1 & 07/11/2025 & Vigolo Davide & Verificatore & Creazione del verbale e stesura iniziale \\
    \hline
\end{tabular}
    \end{adjustwidth}
\end{table}


\tableofcontents

\newpage

\section*{Partecipanti}
\begin{itemize}[noitemsep]
    \item Fattoni Antonio 
    \item Georgescu Diana
    \item Gingillino Aaron
    \item Rocco Matteo Alberto
    \item Soligo Lorenzo
    \item Vigolo Davide
\end{itemize}

\section{Ordine del Giorno}
\begin{enumerate}[noitemsep]
    \item Retrospettiva dello sprint
    \item Pianificazione delle attività
    \item Decisioni e prossimi passi
\end{enumerate}


\section{Svolgimento e Discussione}
\subsection{Definizione ruoli}
I ruoli necessari in questa sprint sono:
\begin{itemize}[noitemsep]
    \item Responsabile
    \item Amministratore
    \item Analista
    \item Verificatore
\end{itemize}

I ruoli assegnati a ciascun membro per questo sprint sono i seguenti:
\begin{table}[H]
    \begin{adjustwidth}{-1cm}{-1cm}
    \centering
\begin{tabular}{|c|c|}
    \hline
    \textbf{Membro} & \textbf{Ruolo} \\
    \hline
    Fattoni Antonio & Analista \\
    \hline
    Georgescu Diana & Programmatore \\
    \hline
    Gingillino Aaron & Responsabile \\
    \hline
    Laoud Zakaria & Programmatore \\
    \hline
    Rocco Matteo A. & Analista \\
    \hline
    Soligo Lorenzo & Analista \\
    \hline
    Vigolo Davide & Analista \\
    \hline
\end{tabular}
    \end{adjustwidth}
\end{table}

\subsection{Retrospettiva}
\begin{itemize}[noitemsep]
    \item È stato rilevato un momento di stallo dovuto alla necessità di allineamento sul metodo di redazione degli Use Case.
    \item Si evidenzia la necessità di considerare eventuali indisponibilità dei membri del gruppo (ad es. per motivi di salute).
    \item La sessione congiunta di sviluppo degli Use Case si è rivelata particolarmente efficace per allineare le conoscenze e convergere su un'impostazione condivisa per la redazione degli UC.
    \item Non sono state individuate tempestivamente alcune dipendenze tra attività; il tema verrà analizzato con maggiore attenzione durante la pianificazione.
    \item È risultata efficace la pianificazione redatta dal Responsabile e successivamente trasferita all'Amministratore per la creazione delle attività su Jira.
    \item Il piano di progetto va automatizzato il più possibile, poiché richiede ancora un impegno eccessivo.
    \item Si segnala la necessità di rivedere le norme di branching.
\end{itemize}

\subsection{Attività}
\begin{itemize}[noitemsep]
    \item Attivare la branch protection su \textit{main} e sulle \textit{dev-branches}.
    \item Predisporre GitHub Actions per il calcolo dell'Indice di Gulpease.
    \item Proseguire la redazione delle AdR:
    \begin{itemize}[noitemsep]
        \item Use Case di visualizzazione della struttura DIP.
        \item Use Case di visualizzazione dei metadati del documento.
        \item Use Case di importazione del DIP.
        \item Use case di ricerca semantica
        \item Diagrammi Use Case per ciascun membro, a valle della finalizzazione degli Use Case.
        \item Definizione del contenuto del PoC.
        \item Revisione degli UC5 $\rightarrow$UC8 e UC2.
        \item \textit{Allineamento con la proponente: definizione della vista di default avvisi nella sezione di verifica.}
        \item \textit{Allineamento con la proponente: aspetti relativi al cloud.}
    \end{itemize}
    \item Riunione per S.A.L. 2025/12/27.
\end{itemize}

\section{Decisioni}
\begin{enumerate}[noitemsep]
    \item Nomina del responsabile per la SPRINT 4: Aaron Gingillino.
    \item Si decide di attivare la branch protection sul repository, integrando controlli automatici di qualità sui documenti.
    \item Si pianifica un incontro dedicato alla discussione e all'allineamento sugli Use Case.
    \item Si pianifica un incontro per redarre la mail da inviare al prof. Vardanega per aggiornamenti sullo Stato di Avanzamento dei Lavori (SAL).
    \item Le attività sui paragrafi 3.x degli Use Case vengono rimandate allo sprint 4 per mancata analisi delle dipendenze.
\end{enumerate}


\section*{Tabella delle decisioni}
\begin{table}[H]
    \begin{adjustwidth}{-4cm}{-4cm}
    \centering
    \small
\begin{tabular}{|c|c|c|c|c|}
    \hline
    \textbf{Decisione} & \textbf{To Do} & \textbf{Jira Issue} & \textbf{Membro assegnato} & \textbf{Verificatore} \\
    \hline
    \#0 & Redazione del verbale & \href{https://7zpus.atlassian.net/browse/DIPR-152}{DIPR-152} & Vigolo Davide & Fattoni Antonio \\
    \hline
    \#2 & \begin{tabular}[c]{@{}c@{}} Attivazione branch protection con \\ controlli automatici di qualità \end{tabular} & \href{https://7zpus.atlassian.net/browse/DIPR-163}{DIPR-163} & Vigolo Davide & N/D \\
    \hline
    \#3 & \begin{tabular}[c]{@{}c@{}} Continuazione e verifica finale \\ congiunta degli Use Case \end{tabular} & \href{https://7zpus.atlassian.net/browse/DIPR-131}{DIPR-131} & \begin{tabular}[c]{@{}c@{}} Vigolo Davide, \\ Fattoni Antonio, \\ Soligo Lorenzo \end{tabular} & \begin{tabular}[c]{@{}c@{}} Vigolo Davide, \\ Fattoni Antonio, \\ Soligo Lorenzo \end{tabular} \\
    \hline
    \#4 & Incontro per redazione email SAL & \href{https://7zpus.atlassian.net/browse/DIPR-158}{DIPR-158} & Tutti & N/D \\
    \hline
    \#5 & \begin{tabular}[c]{@{}c@{}} Redazione paragrafi 3.x \\ degli Use Case allo sprint 4 \end{tabular} & \begin{tabular}[c]{@{}c@{}} \href{https://7zpus.atlassian.net/browse/DIPR-141}{DIPR-141} \\ \href{https://7zpus.atlassian.net/browse/DIPR-142}{DIPR-142} \end{tabular} & \begin{tabular}[c]{@{}c@{}} Gingillino Aaron \\ Laoud Zakaria \end{tabular} & \begin{tabular}[c]{@{}c@{}} Soligo Lorenzo \\ Fattoni Antonio \end{tabular}\\
    \hline
\end{tabular}
    \end{adjustwidth}
\end{table}

\vfill
\begin{flushright}
    \textit{7-ZPUs}
\end{flushright}

\end{document}
