\documentclass[a4paper,12pt]{article}
\usepackage[utf8]{inputenc}
\usepackage[T1]{fontenc} % per i caratteri accentati corretti in PDF
\usepackage[italian]{babel}
\usepackage{lmodern}
\renewcommand*\familydefault{\sfdefault}
\usepackage{float}
\usepackage{geometry}
\usepackage{xcolor}
\usepackage[most]{tcolorbox}
\usepackage{amssymb}
\usepackage{wasysym}
\usepackage{setspace}
\usepackage{chngpage}
\usepackage{enumitem}
\usepackage{titlesec}
\usepackage{tocloft}
\usepackage{graphicx}
\usepackage{hyperref}
\usepackage{fancyhdr}
\hypersetup{
    colorlinks=true,
    linkcolor=black,
    filecolor=magenta,      
    urlcolor=cyan,
}

% Colori ZPUS - Verde, Nero, Bianco
\definecolor{zpusgreen}{RGB}{4, 138, 55}
\definecolor{zpusdarkgreen}{RGB}{0, 100, 0}
\definecolor{zpusblack}{RGB}{0, 0, 0}
\definecolor{zpuswhite}{RGB}{255, 255, 255}
\definecolor{zpuslightgray}{RGB}{245, 245, 245}

% Stili per i box migliorati
\newtcolorbox{headerbox}{
    colback=zpusgreen,
    colframe=zpusdarkgreen,
    arc=0pt,
    boxrule=0pt,
    left=0pt,
    right=0pt,
    top=8pt,
    bottom=8pt,
    fontupper=\color{zpuswhite}\bfseries\large,
    center
}

\newtcolorbox{infobox}{
    colback=zpuslightgray,
    colframe=zpusgreen,
    arc=4pt,
    boxrule=2pt,
    left=6pt,
    right=6pt,
    top=8pt,
    bottom=8pt,
    fontupper=\color{zpusblack}
}

\newtcolorbox{stepbox}{
    colback=zpuswhite,
    colframe=zpusgreen,
    arc=4pt,
    boxrule=1pt,
    left=6pt,
    right=6pt,
    top=8pt,
    bottom=8pt,
    fontupper=\color{zpusblack}
}

\newtcolorbox{highlightbox}{
    colback=zpusgreen!10,
    colframe=zpusdarkgreen,
    arc=4pt,
    boxrule=2pt,
    left=12pt,
    right=12pt,
    top=12pt,
    bottom=12pt,
    fontupper=\color{zpusblack}\bfseries,
    center
}

\pagestyle{fancy}
\setlength{\headwidth}{\textwidth}
\fancyhfoffset[L,R]{0pt}
\lhead{}
\rhead{7-ZPUs}
\lfoot{}
\rfoot{\thepage}
\cfoot{}
\renewcommand{\headrulewidth}{0.8pt}
\renewcommand{\footrulewidth}{0.8pt}

\renewcommand{\contentsname}{Indice}

\geometry{margin=2.5cm}
\setstretch{1.2}

\titleformat{\section}{\large\bfseries}{\thesection}{1em}{}
\titleformat{\subsection}{\mdseries\bfseries}{\thesubsection}{1em}{}

\begin{document}

\begin{center}
    \includegraphics[width=9.5cm]{../../../assets/logo7zpus.jpg}\\
    \small\hspace{10cm} 7zpus.swe@gmail.com\\
    \Large \textbf{Verbale Interno Gruppo di Progetto}\\
    \vspace{0.5cm}
\end{center}

\noindent
\textbf{Data:}  28/11/2025 \\
\textbf{Durata:} 1 ora\\
\textbf{Luogo:} Incontro online (Discord)

\vspace{0.3cm}
\hrule
\vspace{0.5cm}

\tableofcontents

\newpage


\section*{Tabella di Versionamento}
\begin{table}[H]
    \begin{adjustwidth}{-1cm}{-1cm} % modificare ogni volta in base alla larghezza della tabella per centrarla!!!
    \centering
\begin{tabular}{|c|c|c|c|c|}
    \hline
    \textbf{Versione} & \textbf{Data} & \textbf{Autore}  & \textbf{Verificatore} & \textbf{Descrizione} \\
    \hline
    0.1 & 29/11/2025 & Fattoni Antonio & Vigolo Davide & Creazione del verbale e stesura iniziale\\
    \hline

\end{tabular}
    \end{adjustwidth}
\end{table}

\section*{Partecipanti}
\begin{itemize}[noitemsep]
    \item Fattoni Antonio 
    \item Georgescu Diana
    \item Gingilino Aaron
    \item Laoud Zakaria
    \item Rocco Matteo Alberto
    \item Soligo Lorenzo
    \item Vigolo Davide
\end{itemize}

\section{Ordine del Giorno}
\begin{enumerate}[noitemsep]
    \item Definizione dei ruoli
    \item Retrospettiva settimanale
    \item Suddivisione lavoro
\end{enumerate}


\section{Svolgimento e Discussione}

\subsection{Definizione ruoli}
I ruoli necessari in questa settimana sono:
\begin{itemize}[noitemsep]
    \item Amministratore
    \item Responsabile
    \item Analisti
    \item Verificatori
\end{itemize}

I ruoli assegnati a ciascun membro per questa settimana sono i seguenti.\textbf{Ognuno} svolgerà anche il ruolo di \textbf{verificatore}.
\begin{table}[H]
    \begin{adjustwidth}{-1cm}{-1cm}
    \centering
\begin{tabular}{|c|c|}
    \hline
    \textbf{Membro} & \textbf{Ruolo} \\
    \hline
    Fattoni Antonio & Analista\\
    \hline
    Georgescu Diana & Analista\\
    \hline
    Gingillino Aaron& Responsabile\\
    \hline
    Laoud Zakaria & Analista\\
    \hline
    Rocco Matteo A. & Analista\\
    \hline
    Soligo Lorenzo & Amministratore\\
    \hline
    Vigolo Davide & Responsabile\\
    \hline
\end{tabular}
    \end{adjustwidth}
\end{table}

\subsection{Retrospettiva}
È stata svolta la retrospettiva relativa alla prima settimana dello sprint, 

\subsection{Suddivisione del lavoro}
È stata trattata la suddivisione del lavoro per la settimana successiva e discusso riguardo il ruolo da assegnare a ciascun membro in base al compito da svolgere.
Il gruppo ha deciso di assengare la stesura dell'analisi dei requisiti agli analisti mentre piano di progetto e norme di progetto verranno assegnate ai responsabili. 

\subsection{Altre tematiche discusse}
È stata presa in considerazione la ridefinizione di alcune milestone e baseline in base all'andamento dei lavori, pensando quindi che sarà necessario rivedere le date prestabilite a fronte di una previsione più realistica.
Sono stati momentaneamente imposti i seguenti obiettivi:
\begin{itemize}[noitemsep]
    \item 2025/12/15 si presuppone di avere terminato gran parte degli use case l'analisi dei requisiti
    \item 2026/01/12 si presuppone di avere una prima versione del POC
\end{itemize}

\section{Decisioni}
\begin{enumerate}[noitemsep]
    \item Si è deciso di continuare con la stesura di analisi dei requisiti
    \item Si è deciso di continuare la stesura delle piano di progetto
    \item Si è deciso di continuare la stesura delle norme di progetto
    \item Si procede alla creazione del diario di bordo del 1/12/2025
    \item Si è deciso di iniziare uno studio sulla stesura del piano di qualifica
    \item Si procede alla stesura del verbale esterno relativo all'incontro con la proponente del 27/11/2025

\end{enumerate}


\section*{Tabella delle decisioni}
\begin{table}[H]
    \begin{adjustwidth}{-6cm}{-6cm}% modificare ogni volta in base alla larghezza della tabella per centrarla!!!
    \centering
\begin{tabular}{|c|c|c|c|c|}
    \hline
    \textbf{Decisione} & \textbf{To Do} & \textbf{Jira Issue} & \textbf{Membro assegnato} & \textbf{Verificatore} \\
    \hline
    \#0 & Redazione del verbale & \href{https://7zpus.atlassian.net/browse/DIPR-108}{DIPR-108}& Fattoni Antoni& Vigolo Davide \\
    \hline
    \#1 & \begin{tabular}[c]{@{}c@{}} Analisi dei requisiti \end{tabular} & \href{https://7zpus.atlassian.net/browse/DIPR-15}{DIPR-15}& Analisti & Analisti \\
    \hline
    \#2 & Scrittura piano di progetto & \href{https://7zpus.atlassian.net/browse/DIPR-100}{DIPR-100}& Rsponsabili & Rsponsabili \\
    \hline
    \#3 & Aggiornamento norme di progetto & \href{}{}& Rsponsabili & Rsponsabili \\
    \hline
    \#4 & \begin{tabular}[c]{@{}c@{}} Creazione slide Diario di bordo 1/12/2025 \end{tabular}& \href{https://7zpus.atlassian.net/browse/DIPR-109}{DIPR-109}& Vigolo Davide & Soligo Lorenzo \\
    \hline
    \#5 & Studio piano di qualifica & \href{https://7zpus.atlassian.net/browse/DIPR-113}{DIPR-113} & Rocco Matteo A. & \\
    \hline
    \#6 & Stesura verbale esterno 27/11/2025 & \href{https://7zpus.atlassian.net/browse/DIPR-111}{DIPR-111} & Soligo Lorenzo & Laoud Zakaria\\
    \hline
\end{tabular}
    \end{adjustwidth}
\end{table}

\vfill
\begin{flushright}
    \textit{7-ZPUs}
\end{flushright}

\end{document}
