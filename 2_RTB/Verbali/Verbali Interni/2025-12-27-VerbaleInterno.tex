\documentclass[a4paper,12pt]{article} 
\usepackage[utf8]{inputenc} 
\usepackage[T1]{fontenc} 
\usepackage[utf8]{inputenc}
\usepackage[T1]{fontenc} % per i caratteri accentati corretti in PDF
\usepackage[italian]{babel}
\usepackage{lmodern}
\usepackage[sfdefault]{atkinson}
\renewcommand*\familydefault{\sfdefault}
\usepackage{float}
\usepackage{geometry}
\usepackage{xcolor}
\usepackage[most]{tcolorbox}
\usepackage{amssymb}
\usepackage{wasysym}
\usepackage{setspace}
\usepackage{chngpage}
\usepackage{enumitem}
\usepackage{titlesec}
\usepackage{tocloft}
\usepackage{graphicx}
\usepackage{hyperref}
\usepackage{fancyhdr}
\hypersetup{
    colorlinks=true,
    linkcolor=black,
    filecolor=magenta,      
    urlcolor=cyan,
}

% Colori ZPUS - Verde, Nero, Bianco
\definecolor{zpusgreen}{RGB}{4, 138, 55}
\definecolor{zpusdarkgreen}{RGB}{0, 100, 0}
\definecolor{zpusblack}{RGB}{0, 0, 0}
\definecolor{zpuswhite}{RGB}{255, 255, 255}
\definecolor{zpuslightgray}{RGB}{245, 245, 245}

% Stili per i box migliorati
\newtcolorbox{headerbox}{
    colback=zpusgreen,
    colframe=zpusdarkgreen,
    arc=0pt,
    boxrule=0pt,
    left=0pt,
    right=0pt,
    top=8pt,
    bottom=8pt,
    fontupper=\color{zpuswhite}\bfseries\large,
    center
}

\newtcolorbox{infobox}{
    colback=zpuslightgray,
    colframe=zpusgreen,
    arc=4pt,
    boxrule=2pt,
    left=6pt,
    right=6pt,
    top=8pt,
    bottom=8pt,
    fontupper=\color{zpusblack}
}

\newtcolorbox{stepbox}{
    colback=zpuswhite,
    colframe=zpusgreen,
    arc=4pt,
    boxrule=1pt,
    left=6pt,
    right=6pt,
    top=8pt,
    bottom=8pt,
    fontupper=\color{zpusblack}
}

\newtcolorbox{highlightbox}{
    colback=zpusgreen!10,
    colframe=zpusdarkgreen,
    arc=4pt,
    boxrule=2pt,
    left=12pt,
    right=12pt,
    top=12pt,
    bottom=12pt,
    fontupper=\color{zpusblack}\bfseries,
    center
}

\pagestyle{fancy}
\setlength{\headwidth}{\textwidth}
\fancyhfoffset[L,R]{0pt}
\lhead{}
\rhead{7-ZPUs}
\lfoot{}
\rfoot{\thepage}
\cfoot{}
\renewcommand{\headrulewidth}{0.8pt}
\renewcommand{\footrulewidth}{0.8pt}

\renewcommand{\contentsname}{Indice}

\geometry{margin=2.5cm}
\setstretch{1.2}

\titleformat{\section}{\large\bfseries}{\thesection}{1em}{}
\titleformat{\subsection}{\mdseries\bfseries}{\thesubsection}{1em}{}

\begin{document}

\begin{center}
    \includegraphics[width=9.5cm]{../../../assets/logo7zpus.jpg}\\
    \small\hspace{10cm} 7zpus.swe@gmail.com\\
    \Large \textbf{Verbale Interno Gruppo di Progetto}\\
    \vspace{0.5cm}
\end{center}

\noindent
\textbf{Data:} 2025/12/27 \\
\textbf{Durata:} 1 ora\\
\textbf{Luogo:} Incontro online (Discord)

\vspace{0.3cm}
\hrule
\vspace{0.5cm}

\tableofcontents

\newpage


\section*{Tabella di Versionamento}
\begin{table}[H]
    \begin{adjustwidth}{-1cm}{-1cm}
    \centering
\begin{tabular}{|c|c|c|c|c|}
    \hline
    \textbf{Versione} & \textbf{Data} & \textbf{Autore}  & \textbf{Verificatore} & \textbf{Descrizione} \\
    \hline
    0.1 & 2025/12/27 & Georgescu Diana & Soligo Lorenzo & \begin{tabular}[c]{@{}c@{}} Creazione del verbale\\ e stesura iniziale \end{tabular} \\
    \hline
\end{tabular}
    \end{adjustwidth}
\end{table}

\section*{Partecipanti}
\begin{itemize}[noitemsep]
    \item Fattoni Antonio 
    \item Georgescu Diana
    \item Gingillino Aaron
    \item Laoud Zakaria
    \item Rocco Matteo Alberto
    \item Soligo Lorenzo
    \item Vigolo Davide
\end{itemize}

\section{Ordine del Giorno}
\begin{enumerate}[noitemsep]
    \item Stato di avanzamento dell'analisi dei requisiti
    \item Preparazione aggiornamento da inviare al docente
    \item Discussione PoC
    \item Difficoltà tecniche e organizzative
    \item Pianificazione attività future
\end{enumerate}

\section{Svolgimento e Discussione}



\subsection{Definizione ruoli}
I ruoli necessari sono:
\begin{itemize}[noitemsep]
    \item Responsabile
    \item Programmatore
    \item Verificatore
    \item Analista
\end{itemize}

I ruoli assegnati a ciascun membro sono i seguenti. Ciascuno svolgerà anche il ruolo di verificatore.
\begin{table}[H]
    \begin{adjustwidth}{-1cm}{-1cm}
    \centering
\begin{tabular}{|c|c|}
    \hline
    \textbf{Membro} & \textbf{Ruolo} \\
    \hline
    Fattoni Antonio & Analista \\
    \hline
    Georgescu Diana & Programmatore \\
    \hline
    Gingillino Aaron & Responsabile \\
    \hline
    Laoud Zakaria & Programmatore \\
    \hline
    Rocco Matteo A. & Analista \\
    \hline
    Soligo Lorenzo & Analista \\
    \hline
    Vigolo Davide & Analista \\
    \hline
\end{tabular}
    \end{adjustwidth}
\end{table}

Per garantire una distribuzione dei ruoli più equa, si è deciso che una volta completata la revisione degli use case, i programmatori assumeranno il ruolo di analisti e viceversa, come indicato nella tabella delle decisioni \ref{decisionTable}.

\subsection{Aggiornamento da inviare al docente}
È stato chiarito che l'aggiornamento da inviare al docente per mail circa lo stato di avanzamento dei lavori dovrà includere:
\begin{itemize}[noitemsep]
    \item sintesi delle attività svolte;
    \item principali difficoltà incontrate;
    \item stato generale del progetto.
\end{itemize}
La comunicazione dovrà essere sintetica e strutturata in modo simile al diario di bordo.


\subsection{Stato di avanzamento dell'analisi dei requisiti}
Il gruppo ha discusso lo stato dell'analisi dei requisiti, evidenziando che:
\begin{itemize}[noitemsep]
    \item sono stati definiti diversi use case, in particolare quelli relativi alla gestione dei metadati;
    \item la ricerca semantica sarà modellata tramite un numero limitato di use case dedicati;
    \item si rende necessario procedere quanto prima al collegamento sistematico tra requisiti e use case.
\end{itemize}

È emersa l'esigenza di effettuare un'ulteriore revisione collettiva dei requisiti per consolidarne la versione corrente.

\subsection{Proof of Concept}
È stato presentato un Proof of Concept preliminare volto a dimostrare la fattibilità tecnica del progetto.  
Il PoC consente l'indicizzazione e l'esplorazione di file presenti in una cartella, mostrando i contenuti e alcune informazioni associate.

Sono stati evidenziati i seguenti aspetti:
\begin{itemize}[noitemsep]
    \item attualmente non è presente un database vettoriale, con conseguente necessità di reindicizzazione;
    \item alcune funzionalità risultano ancora incomplete o migliorabili;
    \item lo sviluppo iniziale è stato realizzato in forma minimale per valutare rapidamente la fattibilità.
\end{itemize}

\subsection{Difficoltà tecniche e organizzative}
Durante la riunione sono state discusse diverse difficoltà:
\begin{itemize}[noitemsep]
    \item complessità nella modellazione dei casi d'uso, in particolare per l'utilizzo corretto delle relazioni \textit{include};
    \item difficoltà iniziali nella stima e rendicontazione delle ore nel piano di progetto;
    \item problematiche tecniche legate all'ambiente di sviluppo Windows;
    \item difficoltà nel reperire documentazione tecnica per alcune tecnologie utilizzate nel PoC.
\end{itemize}

\subsection{Piano di Qualifica}
È stato discusso il Piano di Qualifica, con particolare riferimento agli indicatori di qualità.
Il gruppo ha rilevato che:
\begin{itemize}[noitemsep]
    \item alcuni indicatori risultano difficilmente misurabili nelle fasi iniziali del progetto;
    \item per il momento saranno utilizzate stime e valori indicativi.
\end{itemize}

\subsection{Pianificazione attività future}
Il gruppo ha concordato i seguenti obiettivi a breve termine:
\begin{itemize}[noitemsep]
    \item completare l'analisi dei requisiti entro metà gennaio;
    \item procedere al collegamento requisiti–use case;
    \item dedicare successivamente maggiore attenzione al PoC e al Piano di Qualifica.
\end{itemize}


\section{Decisioni}
\begin{enumerate}[noitemsep]
    \item Proseguire con il completamento e la revisione dell'analisi dei requisiti.
    \item Collegare sistematicamente requisiti e use case.
    \item Completare le inclusioni ed estensioni degli use case, riportandole e collegandole all'interno del documento.
    \item Rifare tutti i diagrammi degli use case.
    \item Continuare la scrittura dei paragrafi mancanti delle Norme di Progetto.
    \item Proseguire lo sviluppo del PoC con finalità dimostrative e di studio delle tecnologie.
    \item Aggiornare il Piano di Progetto inserendo data di inizio e consuntivo delle task avviate dal 27/12/2025.
    \item Avanzare nella stesura del Piano di Qualifica utilizzando indicatori stimati.
\end{enumerate}


\section*{Tabella delle decisioni} \label{decisionTable}
\begin{table}[H]
    \begin{adjustwidth}{-4cm}{-4cm}
    \centering
    \small
\begin{tabular}{|c|c|c|c|c|}
    \hline
    \textbf{Decisione} & \textbf{To Do} & \textbf{Jira Issue} & \textbf{Membro assegnato} & \textbf{Verificatore} \\
    \hline
    \#0 & Redazione del verbale & \href{https://7zpus.atlassian.net/browse/DIPR-159}{DIPR-159} & Georgescu Diana & Soligo Lorenzo \\
    \hline 
    \#1 & \begin{tabular}[c]{@{}c@{}} Revisione e unificazione\\ degli Use Case \end{tabular} & \href{https://7zpus.atlassian.net/browse/DIPR-131}{DIPR-131} & \begin{tabular}[c]{@{}l@{}} Fattoni Antonio \\ Soligo Lorenzo \\ Vigolo Davide\end{tabular} & \begin{tabular}[c]{@{}l@{}} Fattoni Antonio \\ Soligo Lorenzo \\ Vigolo Davide\end{tabular} \\
    \hline
    \#2 & \begin{tabular}[c]{@{}c@{}} Scrittura paragrafi\\ 2.2, 2.3, 4.1 \end{tabular} & \href{https://7zpus.atlassian.net/browse/DIPR-178}{DIPR-178} & Gingillino Aaron & Georgescu Diana \\
    \hline 
    \#2 & \begin{tabular}[c]{@{}c@{}} Scrittura paragrafi\\ 4.2, 4.3 \end{tabular} & \href{https://7zpus.atlassian.net/browse/DIPR-180}{DIPR-180} & Soligo Lorenzo & Laoud Zakaria \\
    \hline 
    \#2 & \begin{tabular}[c]{@{}c@{}} Scrittura paragrafi\\ 3.6, 3.7, 3.8 \end{tabular} & \href{https://7zpus.atlassian.net/browse/DIPR-146}{DIPR-146} & Zakaria Laoud & Soligo Lorenzo \\
    \hline 
    \#3 & \begin{tabular}[c]{@{}c@{}} Aggiunta lista di\\ inclusioni ed estensioni\\ agli Use Case \end{tabular} & \href{https://7zpus.atlassian.net/browse/DIPR-176}{DIPR-176} & Georgescu Diana & \begin{tabular}[c]{@{}l@{}}Gingillino Aaron \\ Laoud Zakaria \\ Rocco Matteo A. \end{tabular}\\
    \hline
    \#4 & \begin{tabular}[c]{@{}c@{}} Rifare tutti i diagrammi \\ degli use case \end{tabular}
    & \begin{tabular}[c]{@{}c@{}} \href{https://7zpus.atlassian.net/browse/DIPR-157}{DIPR-157} \\ \href{https://7zpus.atlassian.net/browse/DIPR-171}{DIPR-171} \\
    \href{https://7zpus.atlassian.net/browse/DIPR-162}{DIPR-162}\\
    \href{https://7zpus.atlassian.net/browse/DIPR-161}{DIPR-161}\end{tabular}
    & \begin{tabular}[c]{@{}l@{}} Georgescu Diana \\ Gingillino Aaron \\ Laoud Zakaria \\ Rocco Matteo A. \end{tabular}
    & \begin{tabular}[c]{@{}l@{}} Georgescu Diana \\ Gingillino Aaron \\ Laoud Zakaria \\ Rocco Matteo A. \end{tabular} \\
    \hline
    \#5 & \begin{tabular}[c]{@{}c@{}} Studio individuale\\ delle tecnologie \end{tabular} & & Tutti & Tutti \\
    \hline
\end{tabular}
    \end{adjustwidth}
\end{table}
\vfill
\begin{flushright}
    \textit{7-ZPUs}
\end{flushright}


\end {document}