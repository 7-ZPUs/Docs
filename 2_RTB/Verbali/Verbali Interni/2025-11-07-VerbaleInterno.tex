\documentclass[a4paper,12pt]{article}
\usepackage[utf8]{inputenc}
\usepackage[T1]{fontenc} % per i caratteri accentati corretti in PDF
\usepackage[italian]{babel}
\usepackage{lmodern}
\renewcommand*\familydefault{\sfdefault}
\usepackage{float}
\usepackage{geometry}
\usepackage{setspace}
\usepackage{chngpage}
\usepackage{enumitem}
\usepackage{titlesec}
\usepackage{tocloft}
\usepackage{graphicx}
\usepackage{hyperref}
\usepackage{fancyhdr}
\hypersetup{
    colorlinks=true,
    linkcolor=black,
    filecolor=magenta,      
    urlcolor=cyan,
}

\pagestyle{fancy}
\setlength{\headwidth}{\textwidth}
\fancyhfoffset[L,R]{0pt}
\lhead{}
\rhead{7-ZPUs}
\lfoot{}
\rfoot{\thepage}
\cfoot{}
\renewcommand{\headrulewidth}{0.8pt}
\renewcommand{\footrulewidth}{0.8pt}

\renewcommand{\contentsname}{Indice}

\geometry{margin=2.5cm}
\setstretch{1.2}

\titleformat{\section}{\large\bfseries}{\thesection}{1em}{}
\titleformat{\subsection}{\mdseries\bfseries}{\thesubsection}{1em}{}

\begin{document}

\begin{center}
    \includegraphics[width=9.5cm]{../../../assets/logo7zpus.jpg}\\
    \small\hspace{10cm} 7zpus.swe@gmail.com\\
    \Large \textbf{Verbale Interno Gruppo di Progetto}\\
    \vspace{0.5cm}
\end{center}

\noindent
\textbf{Data:} 07/11/2025 \\
\textbf{Durata:} 2 ore\\
\textbf{Luogo:} Incontro online (Discord)

\vspace{0.3cm}
\hrule
\vspace{0.5cm}

\tableofcontents

\newpage

\section*{Tabella di Versionamento}
\begin{table}[H]
    \begin{adjustwidth}{-1cm}{-1cm}
    \centering
\begin{tabular}{|c|c|c|c|c|}
    \hline
    \textbf{Versione} & \textbf{Data} & \textbf{Autore}  & \textbf{Verificatore} & \textbf{Descrizione} \\
    \hline
    0.1 & 07/11/2025 & Rocco Matteo A. & Soligo Lorenzo & Creazione del verbale e stesura iniziale \\
    \hline
\end{tabular}
    \end{adjustwidth}
\end{table}

\section*{Partecipanti}
\begin{itemize}[noitemsep]
    \item Fattoni Antonio 
    \item Georgescu Diana
    \item Gingilino Aaron
    \item Laoud Zakaria
    \item Rocco Matteo Alberto
    \item Soligo Lorenzo
    \item Vigolo Davide
\end{itemize}

\section{Ordine del Giorno}
\begin{enumerate}[noitemsep]
    \item Organizzazione incontri
    \item Definizione ruoli
    \item Sito web
    \item Invio email a Sanmarco
    \item Riorganizzazione repository GitHub
    \item Riflessioni su documentazione a seguito della lezione rovesciata del 6/11/2025
    \item Panoramica su Jira come strumento di gestione di progetto
    \item Implementazione tabella decisioni e tracciamento
    \item Analisi dei requisiti
    \item Diario di bordo 10/11/2025
\end{enumerate}

\section{Svolgimento e Discussione}

\subsection{Organizzazione incontri}
Il gruppo ha deciso di regolarizzare gli incontri, seguendo la metodologia Agile Scrum, e di fissare una riunione settimanale 
ogni venerdì mattina, la durata di ogni sprint corrisponderà perciò a una settimana.

\subsection{Definizione ruoli}
Il gruppo ha ragionato sulla suddivisione dei ruoli per la sprint, e ha osservato che i ruoli maggiormente necessari in questa
fase del progetto sono:
\begin{itemize}[noitemsep]
    \item Responsabile
    \item Analista
    \item Amministratore
    \item Verificatore
\end{itemize}
Considerata l'inesperienza nel ricoprire i ruoli di progetto, è stato stabilito di assegnare a ciascun membro lo stesso ruolo 
per la durata di due cicli di sprint, questo anche per permettere poi di condividere maggiori consigli o accortezze al membro che
ricoprirà un determinato ruolo nel prossimo ciclo.
I ruoli assegnati a ciascun membro per questo sprint sono i seguenti:
\begin{table}[H]
    \begin{adjustwidth}{-1cm}{-1cm}
    \centering
\begin{tabular}{|c|c|}
    \hline
    \textbf{Membro} & \textbf{Ruolo} \\
    \hline
    Fattoni Antonio & Programmatore \\
    \hline
    Georgescu Diana & Responsabile/Verificatore \\
    \hline
    Gingilino Aaron & Analista/Verificatore \\
    \hline
    Laoud Zakaria & Analista/Verificatore \\
    \hline
    Rocco Matteo A. & Analista/Verificatore \\
    \hline
    Soligo Lorenzo & Amministratore/Verificatore \\
    \hline
    Vigolo Davide & Analista/Verificatore \\
    \hline
\end{tabular}
    \end{adjustwidth}
\end{table}

\subsection{Sito web}
Per migliorare l'usabilità del sito web, il gruppo ha considerato alcune alternative all'attuale visualizzatore dei documenti presente
al suo interno. La soluzione trovata consiste nel posizionamento del lettore a destra della pagina nel momento in cui un documento viene
aperto, inoltre verrà aggiunto affianco a ciascun documento un link per aprirlo in una nuova scheda del browser.
Seguendo l'indicazione data dal Prof. Vardanega in sede di valutazione della candidatura inoltre si è deciso di riposizionare i documenti
relativi alle "Lettere di Presentazione" in cima alla sezione della documentazione, per evidenziarne l'importanza e il ruolo di
"guida" alla consultazione della restante documentazione.

\subsection{Invio email a Sanmarco}
Si è proceduto all'invio della prima mail a Sanmarco informatica per poter fissare un incontro di kick off.

\subsection{Riorganizzazione repository GitHub}
È stato fatto un ragionamento sull'utilizzo dei feature branch in maniera più studiata e organizzata, con la creazione del branch
dedicato al sito web. Questo ha portato anche a una riflessione sull'importanza di individuare attività da assegnare che siano 
atomiche e ben definite per evitare conflitti quando i membri del gruppo lavorano in parallelo e effettuano operazioni 
di push all'interno della repository.
Per migliorare la navigazione e rispettare il principio di esporre in cima le versioni più recenti e aggiornate si è reso necessario 
modificare l'attuale nomenclatura dei verbali interni ed esterni, anteponendo la data al titolo e ordinandoli in ordine cronologico 
decrescente.

\subsection{Riflessioni su documentazione a seguito della lezione rovesciata del 6/11/2025}
Uno dei macroargomenti trattati a lezione è il glossario, avendo iniziato la consultazione degli standard ISO/IEC/IEEE 12207:2017
e 29148:2018 (per l'analisi dei requisiti e la gestione del ciclo di vita del progetto) alcuni membri avevano già creato un 
glossario individuale a partire dai termini utilizzati all'interno degli standard. È stato deciso di unire i lavori prodotti 
finora e iniziare la redazione del glossario di progetto.

\subsection{Panoramica su Jira come strumento di gestione di progetto}
Il gruppo ha deciso di utilizzare Jira by Atlassian, una piattaforma di gestione di progetti con numerose funzionalità orientate
allo sviluppo software, per tenere traccia delle attività di progetto. L'attrattività di Jira risiede nella possibilità di poter 
impostare l'organizzazione seguendo il framework Agile Scrum, con strumenti integrati di Issue Tracking e Time Tracking,
oltre a fornire una dashboard comoda e intuitiva per ciascuna sprint.
Attualmente la piattaforma è ancora in fase di configurazione e studio, ma il gruppo intende iniziare da questo ciclo di sprint ad 
utilizzarla per la creazione delle Issues e l'assegnamento delle attività.

\subsection{Implementazione tabella decisioni e revisione struttura verbali interni}
In seguito alla valutazione della candidatura è emerso che l'attuale struttura dei verbali interni non permetteva una corretta 
transizione da un verbale al successivo mantenendo una quantità di informazioni sufficiente alla sua interpretazione dall'esterno, 
inoltre era assente il tracciamento delle decisioni e delle attività derivanti dalle decisioni di ciascun verbale. Per risolvere 
questi problemi il gruppo ha deciso di implementare una tabella delle decisioni che abbia riferimento sia al verbale stesso che al 
sistema di Issue Tracking di Jira (individuando quindi task quanto più atomiche e non ambigue).

\subsection{Analisi dei Requisiti}
Si è concordato che sia necessario iniziare la stesura del documento di Analisi dei Requisiti identificandone la struttura e 
i contenuti da includere. Viene usato come riferimento lo standard ISO/IEC/IEEE 12207:2017.

\newpage


\section{Decisioni}
\begin{enumerate}[noitemsep]
    \item Svolgimento primo ciclo di sprint con i ruoli assegnati, con aggiornamento sull'avanzamento dei lavori il 10/11/2025, 
    eventuali nuove attività e decisioni verranno riportate nel verbale successivo. 
    \item Modifica della visualizzazione dei PDF nel sito aggiungendo l'opzione di aprire in una nuova finestra. Studio di un 
    metodo per riordinare i documenti all'interno della repository.
    \item Creazione glossario.
    \item Continuazione setup di Jira (sistema di Issue Tracking e connessione con GitHub, Smart Commit) e inizio stesura delle 
    sezioni di Norme di Progetto riguardanti i processi di supporto. 
    \item Creazione tabella delle decisioni con riferimento al verbale e al sistema di Issue Tracking.
    \item Avvio stesura Analisi dei Requisiti.
    \item Creazione diario di bordo 10/11/2025.
\end{enumerate}

\section*{Tabella delle decisioni}
\begin{table}[H]
    \begin{adjustwidth}{-4cm}{-4cm}
    \centering
\begin{tabular}{|c|c|c|c|c|}
    \hline
    \textbf{Decisione} & \textbf{To Do} & \textbf{Jira Issue} & \textbf{Membro assegnato} & \textbf{Verificatore} \\
    \hline
    \#0 & Redazione del verbale & DIPR-55 & Rocco Matteo A. & Soligo Lorenzo \\
    \hline
    \#2 & Modifiche al sito web & DIPR-61 & Fattoni Antonio & Vigolo Davide \\
    \hline
    \#3 & Stesura Glossario & DIPR-20 & Vigolo Davide & Gingilino Aaron \\
    \hline
    \#4 & Test connessione Jira-GitHub & DIPR-24 & Soligo Lorenzo & \begin{tabular}[c]{@{}c@{}} Rocco Matteo A.\\ Vigolo Davide \end{tabular} \\
    \hline
    \#4 & Jira Smart Commit & DIPR-25 & Soligo Lorenzo & \begin{tabular}[c]{@{}c@{}} Rocco Matteo A.\\ Vigolo Davide \end{tabular} \\
    \hline
    \#4 & \begin{tabular}[c]{@{}c@{}} Documentazione Jira \\nelle Norme di Progetto \end{tabular} & DIPR-26 & Soligo Lorenzo & \begin{tabular}[c]{@{}c@{}} Rocco Matteo A.\\ Vigolo Davide \end{tabular} \\
    \hline
    \#5 & \begin{tabular}[c]{@{}c@{}} Creazione e validazione \\tabella decisioni \end{tabular} & \begin{tabular}[c]{@{}c@{}} DIPR-56 \\DIPR-57 \end{tabular} & Rocco Matteo A. & Soligo Lorenzo \\
    \hline
    \#6 & \begin{tabular}[c]{@{}c@{}} Definizione struttura \\Analisi dei Requisiti  \end{tabular} & DIPR-53 & Georgescu Diana & Laoud Zakaria\\
    \hline
    \#7 & \begin{tabular}[c]{@{}c@{}} Preparazione e validazione\\Diario di Bordo 10/11/2025 \end{tabular} & \begin{tabular}[c]{@{}c@{}} DIPR-51\\DIPR-52 \end{tabular} & Gingilino Aaron & Georgescu Diana \\
    \hline
\end{tabular}
    \end{adjustwidth}
\end{table}

\vfill
\begin{flushright}
    \textit{7-ZPUs}
\end{flushright}

\end{document}
