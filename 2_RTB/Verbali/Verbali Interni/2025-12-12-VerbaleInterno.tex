\documentclass[a4paper,12pt]{article}
\usepackage[utf8]{inputenc}
\usepackage[T1]{fontenc} % per i caratteri accentati corretti in PDF
\usepackage[italian]{babel}
\usepackage{lmodern}
\renewcommand*\familydefault{\sfdefault}
\usepackage{float}
\usepackage{geometry}
\usepackage{xcolor}
\usepackage[most]{tcolorbox}
\usepackage{amssymb}
\usepackage{wasysym}
\usepackage{setspace}
\usepackage{chngpage}
\usepackage{enumitem}
\usepackage{titlesec}
\usepackage{tocloft}
\usepackage{graphicx}
\usepackage{hyperref}
\usepackage{fancyhdr}
\hypersetup{
    colorlinks=true,
    linkcolor=black,
    filecolor=magenta,      
    urlcolor=cyan,
}

% Colori ZPUS - Verde, Nero, Bianco
\definecolor{zpusgreen}{RGB}{4, 138, 55}
\definecolor{zpusdarkgreen}{RGB}{0, 100, 0}
\definecolor{zpusblack}{RGB}{0, 0, 0}
\definecolor{zpuswhite}{RGB}{255, 255, 255}
\definecolor{zpuslightgray}{RGB}{245, 245, 245}

% Stili per i box migliorati
\newtcolorbox{headerbox}{
    colback=zpusgreen,
    colframe=zpusdarkgreen,
    arc=0pt,
    boxrule=0pt,
    left=0pt,
    right=0pt,
    top=8pt,
    bottom=8pt,
    fontupper=\color{zpuswhite}\bfseries\large,
    center
}

\newtcolorbox{infobox}{
    colback=zpuslightgray,
    colframe=zpusgreen,
    arc=4pt,
    boxrule=2pt,
    left=6pt,
    right=6pt,
    top=8pt,
    bottom=8pt,
    fontupper=\color{zpusblack}
}

\newtcolorbox{stepbox}{
    colback=zpuswhite,
    colframe=zpusgreen,
    arc=4pt,
    boxrule=1pt,
    left=6pt,
    right=6pt,
    top=8pt,
    bottom=8pt,
    fontupper=\color{zpusblack}
}

\newtcolorbox{highlightbox}{
    colback=zpusgreen!10,
    colframe=zpusdarkgreen,
    arc=4pt,
    boxrule=2pt,
    left=12pt,
    right=12pt,
    top=12pt,
    bottom=12pt,
    fontupper=\color{zpusblack}\bfseries,
    center
}

\pagestyle{fancy}
\setlength{\headwidth}{\textwidth}
\fancyhfoffset[L,R]{0pt}
\lhead{}
\rhead{7-ZPUs}
\lfoot{}
\rfoot{\thepage}
\cfoot{}
\renewcommand{\headrulewidth}{0.8pt}
\renewcommand{\footrulewidth}{0.8pt}

\renewcommand{\contentsname}{Indice}

\geometry{margin=2.5cm}
\setstretch{1.2}

\titleformat{\section}{\large\bfseries}{\thesection}{1em}{}
\titleformat{\subsection}{\mdseries\bfseries}{\thesubsection}{1em}{}

\begin{document}

\begin{center}
    \includegraphics[width=9.5cm]{../../../assets/logo7zpus.jpg}\\
    \small\hspace{10cm} 7zpus.swe@gmail.com\\
    \Large \textbf{Verbale Interno Gruppo di Progetto}\\
    \vspace{0.5cm}
\end{center}

\noindent
\textbf{Data:} 12/12/2025 \\
\textbf{Durata:} 30 minuti\\
\textbf{Luogo:} Incontro online (Discord e Zoom)

\vspace{0.3cm}
\hrule
\vspace{0.5cm}
\section*{Tabella di Versionamento}
\begin{table}[H]
    \begin{adjustwidth}{-1cm}{-1cm} % modificare ogni volta in base alla larghezza della tabella per centrarla!!!
    \centering
\begin{tabular}{|c|c|c|c|c|}
    \hline
    \textbf{Versione} & \textbf{Data} & \textbf{Autore}  & \textbf{Verificatore} & \textbf{Descrizione} \\
    \hline
    0.1 & 14/12/2025 & Zakaria Laoud & Matteo Rocco A. & Creazione del verbale e stesura iniziale \\
    \hline
\end{tabular}
    \end{adjustwidth}
\end{table}

\tableofcontents

\newpage


\section*{Partecipanti}
\begin{itemize}[noitemsep]
    \item Fattoni Antonio 
    \item Georgescu Diana
    \item Gingilino Aaron
    \item Laoud Zakaria
    \item Rocco Matteo Alberto
    \item Soligo Lorenzo
    \item Vigolo Davide
\end{itemize}

\section{Ordine del Giorno}
\begin{enumerate}[noitemsep]
    \item Decisione dei ruoli con relativa suddivisione dei compiti
    \item Incontro con il professore Cardin
    \item Discussione aggiornamento analisi dei requisiti
    \item Retrospettiva settimanale
\end{enumerate}


\section{Svolgimento e Discussione}

\subsection{Definizione ruoli}
I ruoli necessari in questa sprint sono:
\begin{itemize}[noitemsep]
    \item Responsabile
    \item Amministratore
    \item Verificatore
    \item Analista
\end{itemize}

I ruoli assegnati a ciascun membro per questo sprint sono i seguenti:
\begin{table}[H]
    \begin{adjustwidth}{-1cm}{-1cm}
    \centering
\begin{tabular}{|c|c|}
    \hline
    \textbf{Membro} & \textbf{Ruolo} \\
    \hline
    Fattoni Antonio & Analista \\
    \hline
    Georgescu Diana & Amministratore \\
    \hline
    Gingilino Aaron & Amministratore \\
    \hline
    Laoud Zakaria & Amministratore \\
    \hline
    Rocco Matteo A. & Responsabile \\
    \hline
    Soligo Lorenzo & Analista \\
    \hline
    Vigolo Davide & Amministratore \\
    \hline
\end{tabular}
    \end{adjustwidth}
\end{table}
\textbf{Tutti} i membri del gruppo sono anche \textbf{verificatori}.

\subsection{Incontro con il professore Cardin}

Durante l'incontro con il professore Cardin sono state poste alcune domande riguardanti l'analisi dei requisiti. Di seguito sono riportate le risposte fornite: \\\\
\textbf{Domanda 1:} Quanto in profondità dobbiamo andare con gli use case di ricerca? Dobbiamo farne uno per metadato? \\
\textbf{Risposta:} Poiché gli use case servono a comprendere il sistema in modo più dettagliato, è opportuno spingersi a un livello di approfondimento tale da 
ricavare il maggior numero possibile di requisiti, facilitandone anche il successivo testing. Di conseguenza, sì: è consigliabile definire uno use case per ogni 
metadato su cui è possibile effettuare una ricerca. \\\\
\textbf{Domanda 2:} Come gestiamo gli use case collegati a dei requisiti opzionali? \\
\textbf{Risposta:} Tutti gli use case dovrebbero essere approfonditi per consentire una stima accurata del tempo necessario alla loro realizzazione. Poiché si tratta 
di use case opzionali, andrebbero sviluppati solo se rimane tempo a disposizione; proprio per questo è ancora più importante avere già pronti sia lo use case sia la 
stima dei tempi associata.\\\\
\textbf{Domanda 3:} Quando è che uno use case si stacca dal padre e diventa a sé stante? \\
\textbf{Risposta:} Uno use case diventa indipendente dal padre quando l'attore può completare interamente le funzionalità previste senza dover accedere ad altri use case. \\\\
\textbf{Domanda 4:} Quando è che uno use case ne estende un altro? \\
\textbf{Risposta:} Uno use case estende un altro quando rappresenta uno scenario alternativo o eccezionale del caso d’uso principale. Ad esempio, se lo use case UC-11 estende UC-10,
 significa che UC-11 descrive un percorso alternativo: qualora l’attore segua UC-11, non verranno raggiunte le post-condizioni previste da UC-10. Un esempio tipico, discusso 
 anche a lezione, è quello delle estensioni che modellano possibili errori o eccezioni che possono verificarsi durante l’esecuzione dello use case principale.

\section{Aggiornamento analisi dei requisiti}
Dopo l'incontro con il professore Cardin, il gruppo ha deciso di procedere con l'aggiunta dei casi di errore come estensioni degli use case principali, suddividendo il documento
 in più parti al fine di facilitare le modifiche parallele da parte degli analisti.

\section{Retrospettiva settimanale}
È stata svolta la retrospettiva della seconda settimana dello sprint.

\section{Decisioni}
\begin{enumerate}[noitemsep]
    \item Si è deciso di continuare la stesura delle norme di progetto.
    \item Si è deciso di iniziare la preparazione del diario di bordo.
    \item Si è deciso di aggiornare l'analisi dei requisiti.
    \item Si è deciso di contiuare con la stesura del piano di qualifica.
    \item Si è deciso di iniziare a studiare le nuove tecnologie utili al progetto.
\end{enumerate}

\vspace{2cm}


\section*{Tabella delle decisioni}
\begin{table}[H]
    \begin{adjustwidth}{-4cm}{-4cm} % modificare ogni volta in base alla larghezza della tabella per centrarla!!!
    \centering
\begin{tabular}{|c|c|c|c|c|}
    \hline
    \textbf{Decisione} & \textbf{To Do} & \textbf{Jira Issue} & \textbf{Membro assegnato} & \textbf{Verificatore} \\
    \hline
    \#0 & Redazione del verbale & \href{https://7zpus.atlassian.net/browse/DIPR-122}{DIPR-122} & Zakaria Laoud & Matteo Rocco A. \\
    \hline
    \#1 & Norme di progetto & \href{https://7zpus.atlassian.net/browse/DIPR-139}{DIPR-139} & Amministratori & Amministratori \\
    \hline
    \#2 & Diario di bordo &  & Matteo Rocco A. & Tutti \\
    \hline
    \#3 & Aggiornamento analisi dei requisiti & \href{https://7zpus.atlassian.net/browse/DIPR-131}{DIPR-131} & Analisti & Analisti \\
    \hline
    \#4 & Piano di qualifica & \href{https://7zpus.atlassian.net/browse/DIPR-138}{DIPR-138} & Matteo Rocco A. & Davide Vigolo \\
    \hline
    \#5 & Studio nuove tecnologie &  & Tutti & Tutti \\
    \hline
\end{tabular}
    \end{adjustwidth}
\end{table}

\vfill
\begin{flushright}
    \textit{7-ZPUs}
\end{flushright}

\end{document}
