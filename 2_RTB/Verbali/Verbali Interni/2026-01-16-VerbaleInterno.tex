\documentclass[a4paper,12pt]{article} 
\usepackage[utf8]{inputenc} 
\usepackage[T1]{fontenc} 
\usepackage[utf8]{inputenc}
\usepackage[T1]{fontenc} % per i caratteri accentati corretti in PDF
\usepackage[italian]{babel}
\usepackage{lmodern}
\usepackage[sfdefault]{atkinson}
\renewcommand*\familydefault{\sfdefault}
\usepackage{float}
\usepackage{geometry}
\usepackage{xcolor}
\usepackage[most]{tcolorbox}
\usepackage{amssymb}
\usepackage{wasysym}
\usepackage{setspace}
\usepackage{chngpage}
\usepackage{enumitem}
\usepackage{titlesec}
\usepackage{tocloft}
\usepackage{graphicx}
\usepackage{hyperref}
\usepackage{fancyhdr}
\hypersetup{
    colorlinks=true,
    linkcolor=black,
    filecolor=magenta,      
    urlcolor=cyan,
}

% Colori ZPUS - Verde, Nero, Bianco
\definecolor{zpusgreen}{RGB}{4, 138, 55}
\definecolor{zpusdarkgreen}{RGB}{0, 100, 0}
\definecolor{zpusblack}{RGB}{0, 0, 0}
\definecolor{zpuswhite}{RGB}{255, 255, 255}
\definecolor{zpuslightgray}{RGB}{245, 245, 245}

% Stili per i box migliorati
\newtcolorbox{headerbox}{
    colback=zpusgreen,
    colframe=zpusdarkgreen,
    arc=0pt,
    boxrule=0pt,
    left=0pt,
    right=0pt,
    top=8pt,
    bottom=8pt,
    fontupper=\color{zpuswhite}\bfseries\large,
    center
}

\newtcolorbox{infobox}{
    colback=zpuslightgray,
    colframe=zpusgreen,
    arc=4pt,
    boxrule=2pt,
    left=6pt,
    right=6pt,
    top=8pt,
    bottom=8pt,
    fontupper=\color{zpusblack}
}

\newtcolorbox{stepbox}{
    colback=zpuswhite,
    colframe=zpusgreen,
    arc=4pt,
    boxrule=1pt,
    left=6pt,
    right=6pt,
    top=8pt,
    bottom=8pt,
    fontupper=\color{zpusblack}
}

\newtcolorbox{highlightbox}{
    colback=zpusgreen!10,
    colframe=zpusdarkgreen,
    arc=4pt,
    boxrule=2pt,
    left=12pt,
    right=12pt,
    top=12pt,
    bottom=12pt,
    fontupper=\color{zpusblack}\bfseries,
    center
}

\pagestyle{fancy}
\setlength{\headwidth}{\textwidth}
\fancyhfoffset[L,R]{0pt}
\lhead{}
\rhead{7-ZPUs}
\lfoot{}
\rfoot{\thepage}
\cfoot{}
\renewcommand{\headrulewidth}{0.8pt}
\renewcommand{\footrulewidth}{0.8pt}

\renewcommand{\contentsname}{Indice}

\geometry{margin=2.5cm}
\setstretch{1.2}

\titleformat{\section}{\large\bfseries}{\thesection}{1em}{}
\titleformat{\subsection}{\mdseries\bfseries}{\thesubsection}{1em}{}

\begin{document}

\begin{center}
    \includegraphics[width=9.5cm]{../../../assets/logo7zpus.jpg}\\
    \small\hspace{10cm} 7zpus.swe@gmail.com\\
    \Large \textbf{Verbale Interno Gruppo di Progetto}\\
    \vspace{0.5cm}
\end{center}

\noindent
\textbf{Data:} 2026/01/16 \\
\textbf{Durata:} 30 minuti\\
\textbf{Luogo:} Incontro online (Discord)

\vspace{0.3cm}
\hrule
\vspace{0.5cm}

\tableofcontents

\newpage


\section*{Tabella di Versionamento}
\begin{table}[H]
    \begin{adjustwidth}{-1cm}{-1cm} % modificare ogni volta in base alla larghezza della tabella per centrarla!!!
    \centering
\begin{tabular}{|c|c|c|c|c|}
    \hline
    \textbf{Versione} & \textbf{Data} & \textbf{Autore}  & \textbf{Verificatore} & \textbf{Descrizione} \\
    \hline
    0.1 & 2026/01/16 & Fattoni Antoni & Gingilino Aaron & Creazione del verbale e stesura iniziale \\
    \hline
\end{tabular}
    \end{adjustwidth}
\end{table}

\section*{Partecipanti}
\begin{itemize}[noitemsep]
    \item Fattoni Antonio 
    \item Gingilino Aaron
    \item Laoud Zakaria
    \item Rocco Matteo Alberto
    \item Soligo Lorenzo
    \item Vigolo Davide
\end{itemize}

\section{Ordine del Giorno}
\begin{enumerate}[noitemsep]
    \item Allineamento sullo svolgimento delle task
\end{enumerate}


\section{Svolgimento e Discussione}

\subsection{Allineamento sullo svolgimento delle task}
Durante l'incontro si è discusso dello stato di avanzamento delle task assegnate a ciascun membro del gruppo per lo sprint corrente. Si è proceduto a fare il punto della situazione per la stesura dei diagrammi degli use case e dei requisiti funzionali, assegnati agli analisti, e per lo studio delle nuove funzionalità da implementare nella PoC, assegnate ai programmatori.

\subsection{Definizione ruoli}
I ruoli necessari in questa sprint sono:
\begin{itemize}[noitemsep]
    \item Analista
    \item Verificatore
    \item Programmatore
\end{itemize}

I ruoli assegnati a ciascun membro per questo sprint sono i seguenti:
\begin{table}[H]
    \begin{adjustwidth}{-1cm}{-1cm}
    \centering
\begin{tabular}{|c|c|}
    \hline
    \textbf{Membro} & \textbf{Ruolo} \\
    \hline
    Fattoni Antonio & Programmatore \\
    \hline
    Georgescu Diana & Analista \\
    \hline
    Gingilino Aaron & Analista \\
    \hline
    Laoud Zakaria & Analista \\
    \hline
    Rocco Matteo A. & Programmatore \\
    \hline
    Soligo Lorenzo & Programmatore \\
    \hline
    Vigolo Davide & Programmatore \\
    \hline
\end{tabular}
    \end{adjustwidth}
\end{table}
\textbf{Tutti} i membri del gruppo faranno anche da \textbf{verificatori}.

\section{Decisioni}
\begin{enumerate}[noitemsep]
    \item Si è deciso di procedere con le task precedentemente assegnate, ovvero gli analisti si occuperanno della creazione dei diagrammi degli use case e della stesura dei requisiti funzionali, mentre i programmatori studieranno le nuove funzionalità da implementare nella PoC.
    \item Si è deciso, in particolare, che i Programmatori proveranno ad implementare la ricerca in 3 modi differenti per verificare quale sia il più efficiente.
    \item Si è deciso di mandare una mail all'azienda Sanmarco per richiedere un pacchetto DIP dalle dimensioni maggiori, in modo da poter effettuare test più realistici.
\end{enumerate}


\section*{Tabella delle decisioni}
\begin{table}[H]
    \begin{adjustwidth}{-4cm}{-4cm} % modificare ogni volta in base alla larghezza della tabella per centrarla!!!
    \centering
\begin{tabular}{|c|c|c|c|c|}
    \hline
    \textbf{Decisione} & \textbf{To Do} & \textbf{Jira Issue} & \textbf{Membro assegnato} & \textbf{Verificatore} \\
    \hline
    \#0 & Redazione del verbale interno& \href{https://7zpus.atlassian.net/browse/DIPR-198}{DIPR-198} & Fattoni Antonio & Gingilino Aaron \\
    \hline
    \#1 & Creazione diagrammi UC & \begin{tabular}[c]{@{}c@{}} \href{https://7zpus.atlassian.net/browse/DIPR-157}{DIPR-161} \\ \href{https://7zpus.atlassian.net/browse/DIPR-161}{DIPR-161} \\ \href{https://7zpus.atlassian.net/browse/DIPR-162}{DIPR-162} \\ \href{https://7zpus.atlassian.net/browse/DIPR-171}{DIPR-171}\end{tabular}& Analisti & Analisti \\
    \hline
    \#1 & Stesura requisiti funzionali & \begin{tabular}[c]{@{}c@{}} \href{https://7zpus.atlassian.net/browse/DIPR-172}{DIPR-172} \\ \href{https://7zpus.atlassian.net/browse/DIPR-173}{DIPR-173} \\ \href{https://7zpus.atlassian.net/browse/DIPR-174}{DIPR-174} \\ \href{https://7zpus.atlassian.net/browse/DIPR-175}{DIPR-175} \end{tabular} & Analisti & Analisti \\
    \hline
    \#2 & Studio nuove funzionalità PoC & \href{https://7zpus.atlassian.net/browse/DIPR-12}{DIPR-12} & Programmatori & Programmatori \\
    \hline
    \#2 & Implementazione Ricerca nel PoC & \begin{tabular}[c]{@{}c@{}} \href{https://7zpus.atlassian.net/browse/DIPR-182}{DIPR-182} \\ \href{https://7zpus.atlassian.net/browse/DIPR-183}{DIPR-183} \\ \href{https://7zpus.atlassian.net/browse/DIPR-184}{DIPR-184} \end{tabular} & Programmatori & Programmatori \\
    \hline
\end{tabular}
    \end{adjustwidth}
\end{table}

\vfill
\begin{flushright}
    \textit{7-ZPUs}
\end{flushright}

\end{document}