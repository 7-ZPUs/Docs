\documentclass[a4paper,12pt]{article} 
\usepackage[utf8]{inputenc} 
\usepackage[T1]{fontenc} 
\usepackage[utf8]{inputenc}
\usepackage[T1]{fontenc} % per i caratteri accentati corretti in PDF
\usepackage[italian]{babel}
\usepackage{lmodern}
\usepackage[sfdefault]{atkinson}
\renewcommand*\familydefault{\sfdefault}
\usepackage{float}
\usepackage{geometry}
\usepackage{xcolor}
\usepackage[most]{tcolorbox}
\usepackage{amssymb}
\usepackage{wasysym}
\usepackage{setspace}
\usepackage{chngpage}
\usepackage{enumitem}
\usepackage{titlesec}
\usepackage{tocloft}
\usepackage{graphicx}
\usepackage{hyperref}
\usepackage{fancyhdr}
\hypersetup{
    colorlinks=true,
    linkcolor=black,
    filecolor=magenta,      
    urlcolor=cyan,
}

% Colori ZPUS - Verde, Nero, Bianco
\definecolor{zpusgreen}{RGB}{4, 138, 55}
\definecolor{zpusdarkgreen}{RGB}{0, 100, 0}
\definecolor{zpusblack}{RGB}{0, 0, 0}
\definecolor{zpuswhite}{RGB}{255, 255, 255}
\definecolor{zpuslightgray}{RGB}{245, 245, 245}

% Stili per i box migliorati
\newtcolorbox{headerbox}{
    colback=zpusgreen,
    colframe=zpusdarkgreen,
    arc=0pt,
    boxrule=0pt,
    left=0pt,
    right=0pt,
    top=8pt,
    bottom=8pt,
    fontupper=\color{zpuswhite}\bfseries\large,
    center
}

\newtcolorbox{infobox}{
    colback=zpuslightgray,
    colframe=zpusgreen,
    arc=4pt,
    boxrule=2pt,
    left=6pt,
    right=6pt,
    top=8pt,
    bottom=8pt,
    fontupper=\color{zpusblack}
}

\newtcolorbox{stepbox}{
    colback=zpuswhite,
    colframe=zpusgreen,
    arc=4pt,
    boxrule=1pt,
    left=6pt,
    right=6pt,
    top=8pt,
    bottom=8pt,
    fontupper=\color{zpusblack}
}

\newtcolorbox{highlightbox}{
    colback=zpusgreen!10,
    colframe=zpusdarkgreen,
    arc=4pt,
    boxrule=2pt,
    left=12pt,
    right=12pt,
    top=12pt,
    bottom=12pt,
    fontupper=\color{zpusblack}\bfseries,
    center
}

\pagestyle{fancy}
\setlength{\headwidth}{\textwidth}
\fancyhfoffset[L,R]{0pt}
\lhead{}
\rhead{7-ZPUs}
\lfoot{}
\rfoot{\thepage}
\cfoot{}
\renewcommand{\headrulewidth}{0.8pt}
\renewcommand{\footrulewidth}{0.8pt}

\renewcommand{\contentsname}{Indice}

\geometry{margin=2.5cm}
\setstretch{1.2}

\titleformat{\section}{\large\bfseries}{\thesection}{1em}{}
\titleformat{\subsection}{\mdseries\bfseries}{\thesubsection}{1em}{}

\begin{document}

\begin{center}
    \includegraphics[width=9.5cm]{../../../assets/logo7zpus.jpg}\\
    \small\hspace{10cm} 7zpus.swe@gmail.com\\
    \Large \textbf{Verbale Interno Gruppo di Progetto}\\
    \vspace{0.5cm}
\end{center}

\noindent
\textbf{Data:} 2026/01/09 \\
\textbf{Durata:} 1 ora\\
\textbf{Luogo:} Incontro online (Discord)

\vspace{0.3cm}
\hrule
\vspace{0.5cm}

\tableofcontents

\newpage


\section*{Tabella di Versionamento}
\begin{table}[H]
    \begin{adjustwidth}{-1cm}{-1cm} % modificare ogni volta in base alla larghezza della tabella per centrarla!!!
    \centering
\begin{tabular}{|c|c|c|c|c|}
    \hline
    \textbf{Versione} & \textbf{Data} & \textbf{Autore}  & \textbf{Verificatore} & \textbf{Descrizione} \\
    \hline
    0.1 & 2026/01/12 & Laoud Zakaria & Vigolo Davide & \begin{tabular}[c]{@{}c@{}} Creazione del verbale \\ e stesura iniziale \end{tabular} \\
    \hline
\end{tabular}
    \end{adjustwidth}
\end{table}

\section*{Partecipanti}
\begin{itemize}[noitemsep]
    \item Fattoni Antonio 
    \item Georgescu Diana
    \item Gingillino Aaron
    \item Laoud Zakaria
    \item Rocco Matteo Alberto
    \item Soligo Lorenzo
    \item Vigolo Davide
\end{itemize}

\section{Ordine del Giorno}
\begin{enumerate}[noitemsep]
    \item Retrospettiva Sprint
    \item Analisi e discussione collettiva degli use case
    \item Pianificazione delle prossime attività
    \item Decisione dei ruoli
\end{enumerate}


\section{Svolgimento e Discussione}

\subsection{Retrospettiva Sprint}
\begin{itemize}
    \item Il gruppo ha rilevato la necessità di definire uno standard condiviso per gli use case.
    \item È emerso che, prima di suddividere gli use case tra i vari componenti del gruppo, sarebbe stato opportuno affrontarne una parte in modo congiunto.
    \item Si evidenzia un rallentamento nella stesura dell'analisi dei requisiti.
    \item Tale rallentamento è stato causato da disallineamenti interni e dalla tardiva individuazione delle dipendenze tra gli use case.
    \item Il ritardo nell'analisi dei requisiti ha comportato un conseguente slittamento nella redazione di altri documenti.
    \item È emersa nuovamente la necessità di redigere un piano di qualifica.
    \item In particolare, si è rilevata l'importanza di definire un cruscotto di valutazione e miglioramento.
    \item Gli analisti incaricati della realizzazione dei diagrammi degli use case e dei requisiti funzionali dovranno svolgere una sessione di confronto.
    \item Il confronto avverrà con gli analisti che hanno curato la stesura degli use case.
    \item L'obiettivo della sessione è l'allineamento sulle diverse articolazioni degli use case.
\end{itemize}

\subsection{Analisi e discussione degli use case}
Il gruppo, avendo rilevato alcune discrepanze nella definizione degli use case di visualizzazione, ha proseguito con la loro revisione e ristrutturazione.

\subsection{Pianificazione delle prossime attività}
\begin{itemize}
    \item Il gruppo ha deciso di raccogliere ulteriori dubbi o quesiti da sottoporre al professor Cardin nel prossimo incontro, ancora da programmare.
    \item Il gruppo procederà con la terminazione del documento di analisi dei requisiti.
    \item Il gruppo ha deciso di continuare lo studio per poter arricchire il Proof of Concept con ulteriori funzionalità.
    \item Il gruppo ha deciso di procedere con un aggiornamento alle norme di progetto
\end{itemize}


\subsection{Definizione ruoli}
I ruoli necessari in questa Sprint sono:
\begin{itemize}[noitemsep]
    \item Amministratore
    \item Responsabile
    \item Analista
    \item Verificatore
    \item Programmatore
\end{itemize}

I ruoli assegnati a ciascun membro per questo Sprint sono i seguenti:
\begin{table}[H]
    \begin{adjustwidth}{-1cm}{-1cm}
    \centering
\begin{tabular}{|c|c|}
    \hline
    \textbf{Membro} & \textbf{Ruolo} \\
    \hline
    Fattoni Antonio & Responsabile \\
    \hline
    Georgescu Diana & Analista \\
    \hline
    Gingilino Aaron & Analista \\
    \hline
    Laoud Zakaria & Amministratore \\
    \hline
    Rocco Matteo A. & Analista \\
    \hline
    Soligo Lorenzo & Programmatore \\
    \hline
    Vigolo Davide & Programmatore \\
    \hline
\end{tabular}
    \end{adjustwidth}
\end{table}
\textbf{Tutti} i membri del gruppo faranno anche da \textbf{verificatori}.

\section{Decisioni}
\begin{enumerate}[noitemsep]
    \item Si è deciso di procedere alla creazione dei diagrammi degli use case
    \item Si è deciso di procedere alla stesura dei requisiti prodotti dagli use case
    \item Si è deciso di continuare lo studio del PoC
    \item Si è deciso di procedere alla stesura del consuntivo dello Sprint 4 
    \item Si è deciso di procedere alla stesura del preventivo Sprint 5
    \item Si è deciso di continuare con l'aggiornamento delle norme di progetto
\end{enumerate}


\section*{Tabella delle decisioni}
\begin{table}[H]
    \begin{adjustwidth}{-4cm}{-4cm} % modificare ogni volta in base alla larghezza della tabella per centrarla!!!
    \centering
\begin{tabular}{|c|c|c|c|c|}
    \hline
    \textbf{Decisione} & \textbf{To Do} & \textbf{Jira Issue} & \textbf{Membro assegnato} & \textbf{Verificatore} \\
    \hline
    \#0 & Redazione del verbale & \href{https://7zpus.atlassian.net/browse/DIPR-185}{DIPR-185} & Laoud Zakaria & Vigolo Davide \\
    \hline
    \#0 & \begin{tabular}[c]{@{}c@{}} Redazione del\\ verbale esterno\end{tabular} & \href{https://7zpus.atlassian.net/browse/DIPR-188}{DIPR-188} & Laoud Zakaria & Vigolo Davide \\
    \hline
    \#1 & \begin{tabular}[c]{@{}c@{}} Creazione\\ diagrammi UC \end{tabular} & \begin{tabular}[c]{@{}c@{}} \href{https://7zpus.atlassian.net/browse/DIPR-157}{DIPR-161} \\ \href{https://7zpus.atlassian.net/browse/DIPR-161}{DIPR-161} \\ \href{https://7zpus.atlassian.net/browse/DIPR-162}{DIPR-162} \\ \href{https://7zpus.atlassian.net/browse/DIPR-171}{DIPR-171}\end{tabular}& Analisti & Analisti \\
    \hline
    \#2 & \begin{tabular}[c]{@{}c@{}} Stesura requisiti\\ funzionali \end{tabular} & \begin{tabular}[c]{@{}c@{}} \href{https://7zpus.atlassian.net/browse/DIPR-172}{DIPR-172} \\ \href{https://7zpus.atlassian.net/browse/DIPR-173}{DIPR-173} \\ \href{https://7zpus.atlassian.net/browse/DIPR-174}{DIPR-174} \\ \href{https://7zpus.atlassian.net/browse/DIPR-175}{DIPR-175} \end{tabular} & Analisti & Analisti \\
    \hline
    \#3 & \begin{tabular}[c]{@{}c@{}} Studio nuove\\ funzionalità PoC \end{tabular} & \href{https://7zpus.atlassian.net/browse/DIPR-12}{DIPR-12} & Programmatori & Programmatori \\
    \hline
    \#4 & \begin{tabular}[c]{@{}c@{}} Stesura consuntivo\\ Sprint 4 \end{tabular} & \href{https://7zpus.atlassian.net/browse/DIPR-194}{DIPR-194} & Fattoni Antonio & Laoud Zakaria \\
    \hline
    \#5 & \begin{tabular}[c]{@{}c@{}} Stesura preventivo\\ Sprint 5 \end{tabular} & \href{https://7zpus.atlassian.net/browse/DIPR-196}{DIPR-196} & Fattoni Antonio & Laoud Zakaria \\
    \hline
    \#6 & \begin{tabular}[c]{@{}c@{}} Aggiornamento \\norme di progetto \end{tabular} & \href{https://7zpus.atlassian.net/browse/DIPR-178}{DIPR-178} & Gingillino Aaron & Georgescu Diana \\
    \hline
    \#6 & \begin{tabular}[c]{@{}c@{}} Aggiornamento \\norme di progetto \end{tabular} & \href{https://7zpus.atlassian.net/browse/DIPR-180}{DIPR-180} & Soligo Lorenzo & Gingillino Aaron \\
    \hline
\end{tabular}
    \end{adjustwidth}
\end{table}

\vfill
\begin{flushright}
    \textit{7-ZPUs}
\end{flushright}

\end{document}
