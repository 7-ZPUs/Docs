\documentclass[a4paper,12pt]{article}
\usepackage[utf8]{inputenc}
\usepackage[T1]{fontenc} % per i caratteri accentati corretti in PDF
\usepackage[italian]{babel}
\usepackage{lmodern}
\renewcommand*\familydefault{\sfdefault}
\usepackage{float}
\usepackage{geometry}
\usepackage{xcolor}
\usepackage[most]{tcolorbox}
\usepackage{amssymb}
\usepackage{wasysym}
\usepackage{setspace}
\usepackage{chngpage}
\usepackage{enumitem}
\usepackage{titlesec}
\usepackage{tocloft}
\usepackage{graphicx}
\usepackage{hyperref}
\usepackage{fancyhdr}
\hypersetup{
    colorlinks=true,
    linkcolor=black,
    filecolor=magenta,      
    urlcolor=cyan,
}

% Colori ZPUS - Verde, Nero, Bianco
\definecolor{zpusgreen}{RGB}{4, 138, 55}
\definecolor{zpusdarkgreen}{RGB}{0, 100, 0}
\definecolor{zpusblack}{RGB}{0, 0, 0}
\definecolor{zpuswhite}{RGB}{255, 255, 255}
\definecolor{zpuslightgray}{RGB}{245, 245, 245}

% Stili per i box migliorati
\newtcolorbox{headerbox}{
    colback=zpusgreen,
    colframe=zpusdarkgreen,
    arc=0pt,
    boxrule=0pt,
    left=0pt,
    right=0pt,
    top=8pt,
    bottom=8pt,
    fontupper=\color{zpuswhite}\bfseries\large,
    center
}

\newtcolorbox{infobox}{
    colback=zpuslightgray,
    colframe=zpusgreen,
    arc=4pt,
    boxrule=2pt,
    left=6pt,
    right=6pt,
    top=8pt,
    bottom=8pt,
    fontupper=\color{zpusblack}
}

\newtcolorbox{stepbox}{
    colback=zpuswhite,
    colframe=zpusgreen,
    arc=4pt,
    boxrule=1pt,
    left=6pt,
    right=6pt,
    top=8pt,
    bottom=8pt,
    fontupper=\color{zpusblack}
}

\newtcolorbox{highlightbox}{
    colback=zpusgreen!10,
    colframe=zpusdarkgreen,
    arc=4pt,
    boxrule=2pt,
    left=12pt,
    right=12pt,
    top=12pt,
    bottom=12pt,
    fontupper=\color{zpusblack}\bfseries,
    center
}

\pagestyle{fancy}
\setlength{\headwidth}{\textwidth}
\fancyhfoffset[L,R]{0pt}
\lhead{}
\rhead{7-ZPUs}
\lfoot{}
\rfoot{\thepage}
\cfoot{}
\renewcommand{\headrulewidth}{0.8pt}
\renewcommand{\footrulewidth}{0.8pt}

\renewcommand{\contentsname}{Indice}

\geometry{margin=2.5cm}
\setstretch{1.2}

\titleformat{\section}{\large\bfseries}{\thesection}{1em}{}
\titleformat{\subsection}{\mdseries\bfseries}{\thesubsection}{1em}{}

\begin{document}

\begin{center}
    \includegraphics[width=9.5cm]{../../../assets/logo7zpus.jpg}\\
    \small\hspace{10cm} 7zpus.swe@gmail.com\\
    \Large \textbf{Verbale Interno Gruppo di Progetto}\\
    \vspace{0.5cm}
\end{center}

\noindent
\textbf{Data:}  21/11/2025 \\
\textbf{Durata:} 2 ore\\
\textbf{Luogo:} Incontro online (Discord)

\vspace{0.3cm}
\hrule
\vspace{0.5cm}

\tableofcontents

\newpage


\section*{Tabella di Versionamento}
\begin{table}[H]
    \begin{adjustwidth}{-1cm}{-1cm} % modificare ogni volta in base alla larghezza della tabella per centrarla!!!
    \centering
\begin{tabular}{|c|c|c|c|c|}
    \hline
    \textbf{Versione} & \textbf{Data} & \textbf{Autore}  & \textbf{Verificatore} & \textbf{Descrizione} \\
    \hline
    0.1 & 21/11/2025& Antonio Fattoni & Vigolo Davide & Creazione del verbale e stesura iniziale\\
    \hline
\end{tabular}
    \end{adjustwidth}
\end{table}

\section*{Partecipanti}
\begin{itemize}[noitemsep]
    \item Fattoni Antonio 
    \item Georgescu Diana
    \item Gingilino Aaron
    \item Laoud Zakaria
    \item Rocco Matteo Alberto
    \item Soligo Lorenzo
    \item Vigolo Davide
\end{itemize}

\section{Ordine del Giorno}
\begin{enumerate}[noitemsep]
    \item Definizione ruoli
    \item Organizzazione branch
    \item Suddivisione lavoro
\end{enumerate}


\section{Svolgimento e Discussione}

\subsection{Definizione ruoli}
I ruoli necessari in questa sprint sono:
\begin{itemize}[noitemsep]
    \item Amministratore
    \item Responsabile
    \item Analisti
    \item Verificatori
\end{itemize}

I ruoli assegnati a ciascun membro per questo sprint sono i seguenti, e sono validi anche per il prossimo sprint. All'interno di questo sprint \textbf{ognuno} svolgerà anche il ruolo di \textbf{verificatore}.
\begin{table}[H]
    \begin{adjustwidth}{-1cm}{-1cm}
    \centering
\begin{tabular}{|c|c|}
    \hline
    \textbf{Membro} & \textbf{Ruolo} \\
    \hline
    Fattoni Antonio & Analista\\ 
    \hline
    Georgescu Diana & Analista\\
    \hline
    Gingillino Aaron& Analista\\
    \hline
    Laoud Zakaria & Analista\\
    \hline
    Rocco Matteo A. & Analista\\ 
    \hline
    Soligo Lorenzo & Amministratore\\
    \hline
    Vigolo Davide & Responsabile\\ 
    \hline
\end{tabular}
    \end{adjustwidth}
\end{table}

\subsection{Organizzazione branch}
Il gruppo ha discusso approfonditamente riguardo il metodo da implementare per la gestione dei branch, in particolare si sono presi in considerazione due metodi di branching: feature-branch e issue-branch.
Dopo un'attenta analisi il gruppo ha compreso l'implementazione del metodo issue-branch tramite le automazioni di jira.

\subsection{Suddivisione del lavoro}
Il gruppo si divide in due sottogruppi per iniziare la stesura del piano di progetto e dell'analisi dei requisiti.
Il primo sottogruppo somposto da Vigolo Davide e Gingillino Aaron si prende in carico la stesura del piano di progetto, i restanti membri andranno a formare il secondo sottogruppo per la stesura dell'analisi dei requisiti.

\section{Decisioni}
\begin{enumerate}[noitemsep]
    \item Si è deciso di provare ad implementare il metodo issue-branch per il prossimo sprint per valutarne l'efficacia
    \item Si procede alla creazione del diario di bordo del 24/11/2025
    \item Si comincia la setessura dell'analisi dei requisiti e del piano di progetto
\end{enumerate}


\section*{Tabella delle decisioni}
\begin{table}[H]
    \begin{adjustwidth}{-6cm}{-6cm}% modificare ogni volta in base alla larghezza della tabella per centrarla!!!
    \centering
\begin{tabular}{|c|c|c|c|c|}
    \hline
    \textbf{Decisione} & \textbf{To Do} & \textbf{Jira Issue} & \textbf{Membro assegnato} & \textbf{Verificatore} \\
    \hline
    \#0 & Redazione del verbale & \href{https://7zpus.atlassian.net/browse/DIPR-96}{DIPR-96}& Fattoni Antonio & Vigolo Davide \\
    \#2 & Creazione slide diario di bordo 24/11/2025  & \href{https://7zpus.atlassian.net/browse/DIPR-101}{DIPR-101}& Rocco Matteo A. & Vigolo Davide \\
    \#3 & Stesura analisi dei requisiti e piano di progetto & \href{https://7zpus.atlassian.net/browse/}{}& Tutti &  \\
    \hline
\end{tabular}
    \end{adjustwidth}
\end{table}

\vfill
\begin{flushright}
    \textit{7-ZPUs}
\end{flushright}

\end{document}
