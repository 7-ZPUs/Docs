\documentclass[a4paper,12pt]{article} 
\usepackage[utf8]{inputenc} 
\usepackage[T1]{fontenc} 
\usepackage[utf8]{inputenc}
\usepackage[T1]{fontenc} % per i caratteri accentati corretti in PDF
\usepackage[italian]{babel}
\usepackage{lmodern}
\usepackage[sfdefault]{atkinson}
\renewcommand*\familydefault{\sfdefault}
\usepackage{float}
\usepackage{geometry}
\usepackage{xcolor}
\usepackage[most]{tcolorbox}
\usepackage{amssymb}
\usepackage{wasysym}
\usepackage{setspace}
\usepackage{chngpage}
\usepackage{enumitem}
\usepackage{titlesec}
\usepackage{tocloft}
\usepackage{graphicx}
\usepackage{hyperref}
\usepackage{fancyhdr}
\hypersetup{
    colorlinks=true,
    linkcolor=black,
    filecolor=magenta,      
    urlcolor=cyan,
}

% Colori ZPUS - Verde, Nero, Bianco
\definecolor{zpusgreen}{RGB}{4, 138, 55}
\definecolor{zpusdarkgreen}{RGB}{0, 100, 0}
\definecolor{zpusblack}{RGB}{0, 0, 0}
\definecolor{zpuswhite}{RGB}{255, 255, 255}
\definecolor{zpuslightgray}{RGB}{245, 245, 245}

% Stili per i box migliorati
\newtcolorbox{headerbox}{
    colback=zpusgreen,
    colframe=zpusdarkgreen,
    arc=0pt,
    boxrule=0pt,
    left=0pt,
    right=0pt,
    top=8pt,
    bottom=8pt,
    fontupper=\color{zpuswhite}\bfseries\large,
    center
}

\newtcolorbox{infobox}{
    colback=zpuslightgray,
    colframe=zpusgreen,
    arc=4pt,
    boxrule=2pt,
    left=6pt,
    right=6pt,
    top=8pt,
    bottom=8pt,
    fontupper=\color{zpusblack}
}

\newtcolorbox{stepbox}{
    colback=zpuswhite,
    colframe=zpusgreen,
    arc=4pt,
    boxrule=1pt,
    left=6pt,
    right=6pt,
    top=8pt,
    bottom=8pt,
    fontupper=\color{zpusblack}
}

\newtcolorbox{highlightbox}{
    colback=zpusgreen!10,
    colframe=zpusdarkgreen,
    arc=4pt,
    boxrule=2pt,
    left=12pt,
    right=12pt,
    top=12pt,
    bottom=12pt,
    fontupper=\color{zpusblack}\bfseries,
    center
}

\pagestyle{fancy}
\setlength{\headwidth}{\textwidth}
\fancyhfoffset[L,R]{0pt}
\lhead{}
\rhead{7-ZPUs}
\lfoot{}
\rfoot{\thepage}
\cfoot{}
\renewcommand{\headrulewidth}{0.8pt}
\renewcommand{\footrulewidth}{0.8pt}

\renewcommand{\contentsname}{Indice}

\geometry{margin=2.5cm}
\setstretch{1.2}

\titleformat{\section}{\large\bfseries}{\thesection}{1em}{}
\titleformat{\subsection}{\mdseries\bfseries}{\thesubsection}{1em}{}

\begin{document}

\begin{center}
    \includegraphics[width=9.5cm]{../../../assets/logo7zpus.jpg}\\
    \small\hspace{10cm} 7zpus.swe@gmail.com\\
    \Large \textbf{Verbale Interno Gruppo di Progetto}\\
    \vspace{0.5cm}
\end{center}

\noindent
\textbf{Data:}  2025/11/14 \\
\textbf{Durata:} 2 ore\\
\textbf{Luogo:} Incontro online (Discord)

\vspace{0.3cm}
\hrule
\vspace{0.5cm}

\tableofcontents

\newpage


\section*{Tabella di Versionamento}
\begin{table}[H]
    \begin{adjustwidth}{-1cm}{-1cm} % modificare ogni volta in base alla larghezza della tabella per centrarla!!!
    \centering
\begin{tabular}{|c|c|c|c|c|}
    \hline
    \textbf{Versione} & \textbf{Data} & \textbf{Autore}  & \textbf{Verificatore} & \textbf{Descrizione} \\
    \hline
    0.2 & 2025/11/20 & Rocco Matteo A. & Vigolo Davide & Verifica del verbale\\
    \hline
    0.1 & 2025/11/14 & Aaron Gingillino & Rocco Matteo A. & \begin{tabular}[c]{@{}c@{}} Creazione del verbale\\ e stesura iniziale \end{tabular} \\
    \hline
\end{tabular}
    \end{adjustwidth}
\end{table}

\section*{Partecipanti}
\begin{itemize}[noitemsep]
    \item Fattoni Antonio 
    \item Georgescu Diana
    \item Gingillino Aaron
    \item Laoud Zakaria
    \item Rocco Matteo Alberto
    \item Soligo Lorenzo
    \item Vigolo Davide
\end{itemize}

\section{Ordine del Giorno}
\begin{enumerate}[noitemsep]
    \item Definizione ruoli
    \item Organizzazione incontri con Sanmarco
    \item Suddivisione lavoro
    \item Altre tematiche
\end{enumerate}


\section{Svolgimento e Discussione}

\subsection{Definizione ruoli}
I ruoli necessari in questa sprint sono:
\begin{itemize}[noitemsep]
    \item Amministratore
    \item Responsabile
    \item Analisti
    \item Verificatori
\end{itemize}

\noindent I ruoli assegnati a ciascun membro per questo sprint sono i seguenti, e sono validi anche per il prossimo sprint. All'interno di questo sprint \textbf{ognuno} svolgerà anche il ruolo di \textbf{verificatore}.
\begin{table}[H]
    \begin{adjustwidth}{-1cm}{-1cm}
    \centering
\begin{tabular}{|c|c|}
    \hline
    \textbf{Membro} & \textbf{Ruolo} \\
    \hline
    Fattoni Antonio & Analista\\
    \hline
    Georgescu Diana & Analista\\
    \hline
    Gingillino Aaron& Analista\\
    \hline
    Laoud Zakaria & Analista\\
    \hline
    Rocco Matteo A. & Analista\\
    \hline
    Soligo Lorenzo & Amministratore\\
    \hline
    Vigolo Davide & Responsabile\\
    \hline
\end{tabular}
    \end{adjustwidth}
\end{table}

\subsection{Organizzazione incontri con Sanmarco}
È stata inviata la mail di accordo per gli incontri con la proponente, confermati in seguito per la giornata di giovedì alle ore 17:00 con cadenza bisettimanale. Si è adottato inoltre l'uso del foglio di calcolo Google "Monitoraggio Progetto 7ZPUs" condiviso dalla proponente per mantenerla aggiornata sulle attività che la coinvolgono direttamente e che possono essere fonte di discussione per gli incontri sincroni.

\subsection{Suddivisione del lavoro}
È stata trattata la suddivisione del lavoro per questo sprint. La necessità in questo momento è di comprendere i requisiti e di studiare il materiale fornito dalla proponente, con particolare attenzione al file "allegato\_5\_metadati" e la cartella "\textit{Schema}" contenente gli schemi .xsd per l'interpretazione del DIP. Inoltre è richiesto che ogni membro del gruppo studi anche i file inerenti all'analisi dei requisiti. Siccome tutti i membri del gruppo devono leggere questi file abbiamo discusso su come sarebbe stato possibile confrontarci su questo studio di gruppo. 

\subsection{Altre tematiche discusse}
\begin{itemize}[noitemsep]
    \item È stato discusso se è sensato mantenere due copie dei verbali nella repository, una in formato \LaTeX{} e una in formato pdf derivato dalla versione \LaTeX{}.
    \item È stato discusso un possibile requisito riguardante la funzionalità opzionale di download di pacchetti DIP dal cloud server del sistema informatico: la proponente ha accennato la possibilità che i pacchetti scaricati siano protetti da password, con conseguente necessità di implementare l'inserimento della password per l'accesso ai file. Non è confluito in nessuna decisione in quanto giudicato argomento prematuro rispetto all'obiettivo di questo sprint.
\end{itemize}

\section{Argomenti non affrontati/rimandati}
Discutere se è meglio adottare un branch per ogni issue (Feature Branch Workflow) e come implementare la task di verifica in relazione alla issue.

\section{Decisioni}
\begin{enumerate}[noitemsep]
    \item Si è deciso di mantenere entrambi i formati per i verbali;
    \item Si procede alla creazione del diario di bordo del 2025/11/17
    \item Si è deciso di iniziare la stesura delle norme di progetto secondo il way of working definito fino ad ora.
    \item Si procede alla stesura del verbale esterno relativo all'incontro con la proponente del 2025/11/13 e all'aggiornamento del documento condiviso
    \item Ogni membro del gruppo prenderà appunti durante lo studio del materiale, questo permetterà di confrontare le nozioni acquisite nell'incontro fissato al 2025/11/20;

\end{enumerate}


\section*{Tabella delle decisioni}
\begin{table}[H]
    \begin{adjustwidth}{-6cm}{-6cm}% modificare ogni volta in base alla larghezza della tabella per centrarla!!!
    \centering
\begin{tabular}{|c|c|c|c|c|}
    \hline
    \textbf{Decisione} & \textbf{To Do} & \textbf{Jira Issue} & \textbf{Membro assegnato} & \textbf{Verificatore} \\
    \hline
    \#0 & Redazione del verbale & \href{https://7zpus.atlassian.net/browse/DIPR-34}{DIPR-34}& Gingillino Aaron& Rocco Matteo A. \\
    \hline
    \#2 & \begin{tabular}[c]{@{}c@{}} Creazione slide diario\\ di bordo 17/11/2025 \end{tabular} & \href{https://7zpus.atlassian.net/browse/DIPR-80}{DIPR-80}& Vigolo Davide & Georgescu Diana \\
    \hline
    \#3 & \begin{tabular}[c]{@{}c@{}} Aggiornamento norme\\ di progetto \end{tabular} & \href{https://7zpus.atlassian.net/browse/DIPR-85}{DIPR-85}& Rocco Matteo A. & Soligo Lorenzo \\
    \hline
    \#4 & \begin{tabular}[c]{@{}c@{}} Redazione verbale \\esterno \end{tabular} & \href{https://7zpus.atlassian.net/browse/DIPR-71}{DIPR-71}& Georgescu Diana & Laoud Zakaria \\
    \hline
    \#4 & \begin{tabular}[c]{@{}c@{}} Aggiornamento del Docs \\Post Incontro Sanmarco \end{tabular}& \href{https://7zpus.atlassian.net/browse/DIPR-82}{DIPR-82}& Laoud Zakaria & Fattoni Antonio \\
    \hline
    \#5 & \begin{tabular}[c]{@{}c@{}} Studio materiale \\ aziendale \end{tabular} & \href{https://7zpus.atlassian.net/browse/DIPR-26}{DIPR-26} & Tutti & Tutti\\
    \hline
\end{tabular}
    \end{adjustwidth}
\end{table}

\vfill
\begin{flushright}
    \textit{7-ZPUs}
\end{flushright}

\end{document}
