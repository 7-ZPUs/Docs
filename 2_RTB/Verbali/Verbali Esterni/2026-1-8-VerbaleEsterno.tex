\documentclass[a4paper,12pt]{article}
\usepackage[utf8]{inputenc}
\usepackage[T1]{fontenc} % per i caratteri accentati corretti in PDF
\usepackage[italian, provide=*]{babel}
\usepackage[sfdefault]{atkinson}
\usepackage{float}
\usepackage{geometry}
\usepackage{xcolor}
\usepackage[most]{tcolorbox}
\usepackage{amssymb}
\usepackage{wasysym}
\usepackage{setspace}
\usepackage{chngpage}
\usepackage{enumitem}
\usepackage{titlesec}
\usepackage{tocloft}
\usepackage{graphicx}
\usepackage{hyperref}
\usepackage{fancyhdr}
\hypersetup{
    colorlinks=true,
    linkcolor=black,
    filecolor=magenta,      
    urlcolor=cyan,
}

% Colori ZPUS - Verde, Nero, Bianco
\definecolor{zpusgreen}{RGB}{4, 138, 55}
\definecolor{zpusdarkgreen}{RGB}{0, 100, 0}
\definecolor{zpusblack}{RGB}{0, 0, 0}
\definecolor{zpuswhite}{RGB}{255, 255, 255}
\definecolor{zpuslightgray}{RGB}{245, 245, 245}

% Stili per i box migliorati
\newtcolorbox{headerbox}{
    colback=zpusgreen,
    colframe=zpusdarkgreen,
    arc=0pt,
    boxrule=0pt,
    left=0pt,
    right=0pt,
    top=8pt,
    bottom=8pt,
    fontupper=\color{zpuswhite}\bfseries\large,
    center
}

\newtcolorbox{infobox}{
    colback=zpuslightgray,
    colframe=zpusgreen,
    arc=4pt,
    boxrule=2pt,
    left=6pt,
    right=6pt,
    top=8pt,
    bottom=8pt,
    fontupper=\color{zpusblack}
}

\newtcolorbox{stepbox}{
    colback=zpuswhite,
    colframe=zpusgreen,
    arc=4pt,
    boxrule=1pt,
    left=6pt,
    right=6pt,
    top=8pt,
    bottom=8pt,
    fontupper=\color{zpusblack}
}

\newtcolorbox{highlightbox}{
    colback=zpusgreen!10,
    colframe=zpusdarkgreen,
    arc=4pt,
    boxrule=2pt,
    left=12pt,
    right=12pt,
    top=12pt,
    bottom=12pt,
    fontupper=\color{zpusblack}\bfseries,
    center
}

\pagestyle{fancy}
\setlength{\headwidth}{\textwidth}
\fancyhfoffset[L,R]{0pt}
\lhead{}
\rhead{7-ZPUs}
\lfoot{}
\rfoot{\thepage}
\cfoot{}
\renewcommand{\headrulewidth}{0.8pt}
\renewcommand{\footrulewidth}{0.8pt}

\renewcommand{\contentsname}{Indice}

\geometry{margin=2.5cm}

\titleformat{\section}{\large\bfseries}{\thesection}{1em}{}
\titleformat{\subsection}{\mdseries\bfseries}{\thesubsection}{1em}{}

\begin{document}



\begin{center}
    \includegraphics[width=9.5cm]{../../../assets/logo7zpus.jpg}\\
    \small\hspace{10cm} 7zpus.swe@gmail.com\\
    \Large \textbf{Verbale Esterno Gruppo di Progetto e Sanmarco Informatica}\\
    \vspace{0.5cm}
\end{center}


\noindent
\textbf{Data:} 2026/1/8 \\
\textbf{Durata:} 25 minuti circa\\
\textbf{Luogo:} incontro online ( Meet )

\vspace{0.3cm}
\hrule
\vspace{0.5cm}

\section*{Tabella di Versionamento}
\begin{table}[H]
    \begin{adjustwidth}{-1cm}{-1cm} % modificare ogni volta in base alla larghezza della tabella per centrarla!!!
    \centering
\begin{tabular}{|c|c|c|c|c|}
    \hline
    \textbf{Versione} & \textbf{Data} & \textbf{Autore}  & \textbf{Verificatore} & \textbf{Descrizione} \\
    \hline
    0.1 & 2026/1/8 & Laoud Zakaria & Vigolo Davide & Creazione del verbale e stesura iniziale \\
    \hline
\end{tabular}
    \end{adjustwidth}
\end{table}

\tableofcontents

\newpage





\section*{Partecipanti}
\begin{itemize}[noitemsep]
    \item Fattoni Antonio 
    \item Georgescu Diana
    \item Gingilino Aaron
    \item Laoud Zakaria
    \item Rocco Matteo Alberto
    \item Soligo Lorenzo
    \item Vigolo Davide
    \item Rossi Andrea
\end{itemize}


\section{Ordine del Giorno}
\begin{enumerate}[noitemsep]
    \item Esposizione dello stato di avanzamento dei lavori.
    \item Presentazione di una prima versione di Poc.
    \item Risoluzione di dubbi riguardanti le funzionalità del Proof of Concept.
\end{enumerate}

\vspace{0.5cm}

\section{Svolgimento e Discussione}
Durante l’incontro è stato discusso l’andamento generale del progetto, evidenziando una riduzione delle attività in concomitanza con il periodo d’esami e sottolineando come
 il lavoro si sia concentrato principalmente sulla terminazione dell’analisi dei requisiti e sullo sviluppo di un PoC.
  
 Successivamente è stato presentato il PoC, ponendo l’attenzione sulle sue funzionalità di base.

\subsection{Presentazionde del PoC}
Nel corso della presentazione del PoC sono state illustrate tutte le funzionalità attualmente implementate, tra cui il caricamento e la visualizzazione del DIP. A seguito
 della presentazione, sono emerse alcune criticità relative alla verifica della funzione di hash. In particolare, è stato chiesto alla proponente se, nel DIP di esempio, 
 i codici hash fossero generati casualmente o meno.

Dal confronto con la proponente è emerso che nei DIP i codici hash sono codificati in Base64. Di conseguenza, qualsiasi funzione di verifica deve necessariamente applicare
 tale codifica per poter distinguere correttamente i file corrotti da quelli integri.

\section{Domande e Risposte}

\textbf{Domanda:} È necessario implementare la ricerca semantica anche nel PoC, nonostante sia un requisito opzionale?

\textbf{Risposta:} Sì. Poiché il PoC dovrebbe rappresentare nel modo più fedele possibile il prodotto finale, è preferibile trascurare gli aspetti secondari, come l'interfaccia
 grafica, e concentrarsi invece sulle funzionalità, implementandone il maggior numero possibile. 
\vspace{0.5cm}

\textbf{Domanda:} È possibile lasciare all'utente la scelta del livello di precisione della ricerca semantica (ad esempio alto, medio o basso)?

\textbf{Risposta:} No, poiché ciò rischierebbe di generare confusione nell'utente. È preferibile individuare un livello di precisione che risulti adeguato nella maggior parte
 dei casi. 

\section{Decisioni e Conclusione}
\begin{itemize}
    \item Il Team ha deciso di procedere alla conclusione dell'analisi dei requisiti. 
    \item Il Team ha inoltre deciso di continuare con lo sviluppo del PoC. 
\end{itemize}

\vfill


\begin{flushleft}
\begin{tabular}{@{}p{0.5in}p{2in}@{}}
Data: & \hrulefill \\
\end{tabular}

\vspace{1em}

\begin{tabular}{@{}p{0.5in}p{2in}@{}}
Firma: & \hrulefill \\
& [Sanmarco Informatica SPA] \\
\end{tabular}
\end{flushleft}

\begin{flushright}
    \textit{7-ZPUs}
\end{flushright}

\end{document}
