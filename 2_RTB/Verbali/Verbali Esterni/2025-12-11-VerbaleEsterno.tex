\documentclass[a4paper,12pt]{article}
\usepackage[utf8]{inputenc}
\usepackage[T1]{fontenc} % per i caratteri accentati corretti in PDF
\usepackage[italian, provide=*]{babel}
\usepackage[sfdefault]{atkinson}
\usepackage{float}
\usepackage{geometry}
\usepackage{xcolor}
\usepackage[most]{tcolorbox}
\usepackage{amssymb}
\usepackage{wasysym}
\usepackage{setspace}
\usepackage{chngpage}
\usepackage{enumitem}
\usepackage{titlesec}
\usepackage{tocloft}
\usepackage{graphicx}
\usepackage{hyperref}
\usepackage{fancyhdr}
\hypersetup{
    colorlinks=true,
    linkcolor=black,
    filecolor=magenta,      
    urlcolor=cyan,
}

% Colori ZPUS - Verde, Nero, Bianco
\definecolor{zpusgreen}{RGB}{4, 138, 55}
\definecolor{zpusdarkgreen}{RGB}{0, 100, 0}
\definecolor{zpusblack}{RGB}{0, 0, 0}
\definecolor{zpuswhite}{RGB}{255, 255, 255}
\definecolor{zpuslightgray}{RGB}{245, 245, 245}

% Stili per i box migliorati
\newtcolorbox{headerbox}{
    colback=zpusgreen,
    colframe=zpusdarkgreen,
    arc=0pt,
    boxrule=0pt,
    left=0pt,
    right=0pt,
    top=8pt,
    bottom=8pt,
    fontupper=\color{zpuswhite}\bfseries\large,
    center
}

\newtcolorbox{infobox}{
    colback=zpuslightgray,
    colframe=zpusgreen,
    arc=4pt,
    boxrule=2pt,
    left=6pt,
    right=6pt,
    top=8pt,
    bottom=8pt,
    fontupper=\color{zpusblack}
}

\newtcolorbox{stepbox}{
    colback=zpuswhite,
    colframe=zpusgreen,
    arc=4pt,
    boxrule=1pt,
    left=6pt,
    right=6pt,
    top=8pt,
    bottom=8pt,
    fontupper=\color{zpusblack}
}

\newtcolorbox{highlightbox}{
    colback=zpusgreen!10,
    colframe=zpusdarkgreen,
    arc=4pt,
    boxrule=2pt,
    left=12pt,
    right=12pt,
    top=12pt,
    bottom=12pt,
    fontupper=\color{zpusblack}\bfseries,
    center
}

\pagestyle{fancy}
\setlength{\headwidth}{\textwidth}
\fancyhfoffset[L,R]{0pt}
\lhead{}
\rhead{7-ZPUs}
\lfoot{}
\rfoot{\thepage}
\cfoot{}
\renewcommand{\headrulewidth}{0.8pt}
\renewcommand{\footrulewidth}{0.8pt}

\renewcommand{\contentsname}{Indice}

\geometry{margin=2.5cm}
\setstretch{1.2}

\titleformat{\section}{\large\bfseries}{\thesection}{1em}{}
\titleformat{\subsection}{\mdseries\bfseries}{\thesubsection}{1em}{}

\begin{document}



\begin{center}
    \includegraphics[width=9.5cm]{../../../assets/logo7zpus.jpg}\\
    \small\hspace{10cm} 7zpus.swe@gmail.com\\
    \Large \textbf{Verbale Esterno Gruppo di Progetto e Sanmarco Informatica}\\
    \vspace{0.5cm}
\end{center}


\noindent
\textbf{Data:} 2025/12/11 \\
\textbf{Durata:}  30 minuti\\
\textbf{Luogo:} Incontro online (Google Meet)

\vspace{0.3cm}
\hrule
\vspace{0.5cm}

\tableofcontents

\newpage



\section*{Tabella di Versionamento}
\begin{table}[H]
    \begin{adjustwidth}{-1cm}{-1cm} % modificare ogni volta in base alla larghezza della tabella per centrarla!!!
    \centering
\begin{tabular}{|c|c|c|c|c|}
    \hline
    \textbf{Versione} & \textbf{Data} & \textbf{Autore}  & \textbf{Verificatore} & \textbf{Descrizione} \\
    \hline
    0.1 & 2025/12/13& Aaron Gingillino& Antonio Fattoni& Creazione del verbale e stesura iniziale \\
    \hline
\end{tabular}
    \end{adjustwidth}
\end{table}

\section*{Partecipanti}
\begin{itemize}[noitemsep]
    \item Fattoni Antonio 
    \item Georgescu Diana
    \item Gingillino Aaron
    \item Laoud Zakaria
    \item Rocco Matteo Alberto
    \item Soligo Lorenzo
    \item Vigolo Davide
    \item Rossi Andrea
    \item Sarra Luca
\end{itemize}


\section{Ordine del Giorno} \label{ordineGiorno}
\begin{enumerate}[noitemsep]
    \item Esposizione del mockup dell'interfaccia grafica
    \item Discussione sulla pertinenza dei filtri sui metadati
    \item Ulteriori chiarimenti sul materiale tecnico fornito.
\end{enumerate}

\vspace{0.5cm}

\section{Svolgimento e Discussione}
Il team si è alternato secondo la suddivisione dei compiti, in modo da garantire l'esposizione più puntuale possibile dei dubbi e del materiale prodotto.

\subsection{Mockup di Figma}
Tramite Figma è stato possibile mostrare alla proponente una possibile \href{https://www.figma.com/make/uPoL8sccrgQx38UNpq8X6B/DIP-Reader-UI-Design--Copy-?node-id=0-1&p=f&fullscreen=1}{interfaccia grafica}, con le varie funzionalità che andremmo poi ad implementare. La proponente ha fatto notare che nella sezione "\textbf{Verifica Integrità}", l'interfaccia mescola assieme gli elementi validi, con avvisi e errori; ha suggerito una divisione logica, che potrebbe essere facilitata tramite dei pulsanti che vadano a filtrare tra le tre tipologie di elemento sopra citate. 
\\ Un'altra osservazione da parte della proponente riguardava il termine "\textbf{errore}", che viene utilizzato per etichettare tutti gli elementi che non sono validi. In questo caso il problema è che agli occhi dell'utente il termine "\textbf{errore}" indica un malfunzionamento proibitivo, ma, come esemplificava la proponente, un documento potrebbe essere non valido a causa di un certificato scaduto, che però risultava valido nel momento in cui è stato firmato. In  questi casi è utile utilizzare il termine "\textbf{anomalia}", descrivendo la situazione, e cercando di tranquillizzare l'utente.

\subsection{Proposte UI}
Sono state esposte varie proposte per la parte dell'interfaccia grafica riguardante la gestione del file system, di seguito sono riportate con eventuali critiche:
\begin{itemize}
    \item \textbf{Timeline}: una visualizzazione basata sul tempo, ma poco pratica nel caso vi siano molti documenti.
    \item \textbf{Diagramma a torta}: il file system è visualizzato in un diagramma a torta, in cui è possibile cliccare per accedere ai file e alle cartelle. Questo diagramma è un modo alternativo per visualizzare uno schema ad albero, che però include tutte le sue criticità. Inoltre risulta poco pratico nel caso vi siano molti file in una cartella.
    \item \textbf{Relazionale}: visualizziamo i file tramite delle relazioni (come fa il software Obisidian), il problema è che ci andiamo comunque a ricondurre ad uno schema ad albero.
    \item \textbf{Filtraggio}: lasciamo all'utente una batteria di filtri, tramite i quali è possibile ridurre la portata di ricerca.
\end{itemize}
Quest'ultima opzione è ritenuta come la più viabile da parte della proponente e del Team.

\subsection{Ricerca Tramite Filtri}
La proponente e il Team si trovano d'accordo sul fatto che l'utente difficilmente cerchi in modo puntuale gli allegati, ma che spesso si appoggi a dei metadati comuni. Questo non preclude l'inclusione di una ricerca avanzata che possa permettere all'utente di fare delle ricerche più granulari; bisogna però che questa ricerca tramite metadati avanzati non confonda l'utente tramite sovraccarico cognitivo. 
 \\ La proponente si è soffermata sul campo delle note nello specifico. I campi dei metadati categorizzano molte, ma non tutte, le informazioni che un utente vuole includere nell'allegato. Potrebbe capitare che un utente includa un'informazione rilevante nel campo delle note perché non è inseribile in nessun altro campo dati. Questa informazione potrebbe essere di rilievo durante la ricerca, e quindi va prevista la ricerca delle informazioni anche in questo campo.  Come previsto da normative.

\section{Domande e Risposte} 
\begin{itemize}
    \item \textbf{Domanda 1:} Qual' è la relazione tra il file aipinfo e il processo di conservazione ? Perché condividono lo stesso ID?\\
    	\textbf{Risposta:}  Nel momento in cui andiamo a creare il processo di conservazione lo etichettiamo con un certo ID, per agevolare le interrogazioni al sistema questo ID verrà poi messo anche all'interno del pacchetto di conservazione generato dal processo. Per quanto riguarda il progetto, non sarà necessario compiere ricerche all'interno del file aipinfo, ma potrebbe essere utile reperire le informazioni al suo interno perché vengano mostrare all'utente se richiesto, come ad esempio in una eventuale schermata di dettaglio del processo di conservazione.

    \item \textbf{Domanda 2:} i file preport e sreport sono sempre presenti? Contengono delle informazioni ridondanti?
    \\ \textbf{Risposta:} la loro inclusione è opzionale, viene decisa l'utente nel momento in cui richiede la creazione del pacchetto di distribuzione. Al loro interno troviamo:
    \begin{itemize}
        \item sreport: è un rapporto di versamento, include i documenti accolti dal processo
        \item  preport: è un rapporto di conservazione, ci comunica quali dei documenti accolti sono stati conservati correttamente e quali no
    \end{itemize}
Siccome noi non tratteremo mai con dei rapporti che includono documenti la cui inclusione non è andata a buon fine, questi file contengono delle informazioni ridondati rispetto a aipinfo, pindex e metadata. Questi rapporti possono essere comunque messi a disposizione all'utente.

    \item \textbf{Domanda 3:} Come possiamo manipolare il vettore restituito dalla libreria FAISS?
    \\ \textbf{Risposta:} La libreria FAISS ci permette di creare un database vettoriale sul file system, andremmo a creare un file all'interno del quale includiamo gli indici e la versione vettoriale dei file da memorizzare. 
\\ Quando l'utente pone una domanda in linguaggio naturale, convertiamo la domanda in un vettore tramite la libreria FAISS, includiamo all'interno di questo vettore una quantità di informazioni arbitrarie. Fatto ciò andiamo a confrontare il vettore ottenuto dalla domanda con i vettori presenti del database sopracitato, estraendo un insieme di vettori vicini di lunghezza arbitraria. Così facendo abbiamo le informazioni che più di avvinano alla domanda.
\\ Successivamente passiamo il vettore della domanda e i vettori vicini alla domanda ad un LLM, che permettere di rispondere alla domanda in modo contestualizzato.
\\ Tramite questa pratica possiamo compiere ricerche per sinonimi, e non solo tramite testo contenuto.

\end{itemize}


\section{Decisioni e Conclusione}
In seguito alla discussione sui filtri dei metadati, il Team ha deciso di rivedere alcuni aspetti e di continuare l'analisi per individuare eventuali altri casi d'uso.


\vfill


\begin{flushleft}
\begin{tabular}{@{}p{0.5in}p{2in}@{}}
Data: & \hrulefill \\
\end{tabular}

\vspace{1em}

\begin{tabular}{@{}p{0.5in}p{2in}@{}}
Firma: & \hrulefill \\
& [Sanmarco Informatica SPA] \\
\end{tabular}
\end{flushleft}

\begin{flushright}
    \textit{7-ZPUs}
\end{flushright}

\end{document}