\documentclass[a4paper,12pt]{article}
\usepackage[utf8]{inputenc}
\usepackage[T1]{fontenc} % per i caratteri accentati corretti in PDF
\usepackage[italian, provide=*]{babel}
\usepackage[sfdefault]{atkinson}
\usepackage{float}
\usepackage{geometry}
\usepackage{xcolor}
\usepackage[most]{tcolorbox}
\usepackage{amssymb}
\usepackage{wasysym}
\usepackage{setspace}
\usepackage{chngpage}
\usepackage{enumitem}
\usepackage{titlesec}
\usepackage{tocloft}
\usepackage{graphicx}
\usepackage{hyperref}
\usepackage{fancyhdr}
\hypersetup{
    colorlinks=true,
    linkcolor=black,
    filecolor=magenta,      
    urlcolor=cyan,
}

% Colori ZPUS - Verde, Nero, Bianco
\definecolor{zpusgreen}{RGB}{4, 138, 55}
\definecolor{zpusdarkgreen}{RGB}{0, 100, 0}
\definecolor{zpusblack}{RGB}{0, 0, 0}
\definecolor{zpuswhite}{RGB}{255, 255, 255}
\definecolor{zpuslightgray}{RGB}{245, 245, 245}

% Stili per i box migliorati
\newtcolorbox{headerbox}{
    colback=zpusgreen,
    colframe=zpusdarkgreen,
    arc=0pt,
    boxrule=0pt,
    left=0pt,
    right=0pt,
    top=8pt,
    bottom=8pt,
    fontupper=\color{zpuswhite}\bfseries\large,
    center
}

\newtcolorbox{infobox}{
    colback=zpuslightgray,
    colframe=zpusgreen,
    arc=4pt,
    boxrule=2pt,
    left=6pt,
    right=6pt,
    top=8pt,
    bottom=8pt,
    fontupper=\color{zpusblack}
}

\newtcolorbox{stepbox}{
    colback=zpuswhite,
    colframe=zpusgreen,
    arc=4pt,
    boxrule=1pt,
    left=6pt,
    right=6pt,
    top=8pt,
    bottom=8pt,
    fontupper=\color{zpusblack}
}

\newtcolorbox{highlightbox}{
    colback=zpusgreen!10,
    colframe=zpusdarkgreen,
    arc=4pt,
    boxrule=2pt,
    left=12pt,
    right=12pt,
    top=12pt,
    bottom=12pt,
    fontupper=\color{zpusblack}\bfseries,
    center
}

\pagestyle{fancy}
\setlength{\headwidth}{\textwidth}
\fancyhfoffset[L,R]{0pt}
\lhead{}
\rhead{7-ZPUs}
\lfoot{}
\rfoot{\thepage}
\cfoot{}
\renewcommand{\headrulewidth}{0.8pt}
\renewcommand{\footrulewidth}{0.8pt}

\renewcommand{\contentsname}{Indice}

\geometry{margin=2.5cm}

\titleformat{\section}{\large\bfseries}{\thesection}{1em}{}
\titleformat{\subsection}{\mdseries\bfseries}{\thesubsection}{1em}{}

\begin{document}



\begin{center}
    \includegraphics[width=9.5cm]{../../../assets/logo7zpus.jpg}\\
    \small\hspace{10cm} 7zpus.swe@gmail.com\\
    \Large \textbf{Verbale Esterno Gruppo di Progetto e Sanmarco Informatica}\\
    \vspace{0.5cm}
\end{center}


\noindent
\textbf{Data:} 2025/12/23 \\
\textbf{Durata:}  30 minuti circa\\
\textbf{Luogo:} Incontro online (Google Meet)

\vspace{0.3cm}
\hrule
\vspace{0.5cm}


\section*{Tabella di Versionamento}
\begin{table}[H]
    \begin{adjustwidth}{-1cm}{-1cm} % modificare ogni volta in base alla larghezza della tabella per centrarla!!!
    \centering
\begin{tabular}{|c|c|c|c|c|}
    \hline
    \textbf{Versione} & \textbf{Data} & \textbf{Autore}  & \textbf{Verificatore} & \textbf{Descrizione} \\
    \hline
    0.1 & 2025/12/29 & Rocco Matteo A. & Vigolo Davide & \begin{tabular}[c]{@{}c@{}} Creazione del verbale\\ e stesura iniziale \end{tabular} \\
    \hline
\end{tabular}
    \end{adjustwidth}
\end{table}

\tableofcontents

\newpage

\section*{Partecipanti}
\begin{itemize}[noitemsep]
    \item Fattoni Antonio 
    \item Georgescu Diana
    \item Gingilino Aaron
    \item Laoud Zakaria
    \item Rocco Matteo Alberto
    \item Soligo Lorenzo
    \item Vigolo Davide
    \item Rossi Andrea
    \item Sarra Luca
\end{itemize}


\section{Ordine del Giorno} \label{ordineGiorno}
\begin{enumerate}[noitemsep]
    \item Esposizione da parte dello stato di avanzamento dei lavori.
    \item Domande e risposte su alcuni dubbi di funzionalità del prodotto.
\end{enumerate}

\vspace{0.5cm}

\section{Svolgimento e Discussione}
Il team si è alternato secondo la suddivisione dei compiti, in modo da garantire l'esposizione più puntuale possibile dei dubbi e del materiale prodotto.

\subsection{Stato di avanzamento del lavori}
Il Team ha comunicato il raggiungimento della fase finale di stesura dell'Analisi dei Requisiti e l'intenzione di procedere con lo sviluppo del primo Proof of Concept di prodotto. In seguito ha posto diversi quesiti per chiarire alcuni aspetti non chiari. Le diverse domande sono state raccolte nella sezione \hyperref[qa]{Domande e Risposte}.


\section{Domande e Risposte} \label{qa}
\begin{itemize}
    \item \textbf{Domanda 1:} È rilevante la possibilità di ricercare un processo/classe documentale con il suo identificativo nonostante sia una stringa di caratteri alfanumerici poco familiare per l'utente? \\
    \textbf{Risposta:} La ricerca per identificativo non rientra tra le opzioni di ricerca fondamentali del prodotto, tuttavia è utile nel momento in cui si voglia fare una verifica puntuale di un documento di cui conosco l'identificativo e quindi è necessario avere la possibilità di ricercarlo. Ad esempio se con una ricerca precedente sono giunto a un insieme ridotto di risultati mi aspetto che l'utente si voglia annotare l'identificativo dei documenti trovati per poterli recuperare in seguito se già sa che ne avrà bisogno. Quindi riteniamo la funzionalità rilevante, magari offerta sottoforma di filtro di ricerca avanzato e non tra i filtri di ricerca base.
    
    \item \textbf{Domanda 2:} È necessario implementare la validazione dei campi di ricerca qualora riguardino codici o identificativi di cui si conosce un formato standard (come un codice fiscale/partita iva)? Questo per escludere la possibilità di un errore in input da parte dell'utente e garantire perciò che se non vengono restituiti risultati è perché non sono presenti nel DIP.
    \\ \textbf{Risposta:} Se si ritiene utile fornire questa funzionalità aggiunge sicuramente valore al prodotto per l'utente. Tuttavia suggeriamo di analizzarla anche dal punto di vista implementativo e delle risorse a disposizione del Team, infatti ci si può aspettare che la validazione comporti un costo non indifferente e aumenterebbe significativamente la complessità dello sviluppo del prodotto, perciò la reputiamo una funzionalità interessante ma non fondamentale.

    \item \textbf{Domanda 3:} Relativamente al mockup del prodotto presentato nel precedente incontro (\href{https://cdn.jsdelivr.net/gh/7-ZPUs/Docs@vverbali_in_lavorazione/2_RTB/Verbali/Verbali%20Esterni/2025-12-11-VerbaleEsterno.pdf}{11/12/2025}) abbiamo riscontrato difficoltà a individuare una modalità intuitiva di visualizzare i documenti presenti nella sezione "Verifica Integrità" che rientrano nella categoria dei documenti validati e che non contengono avvisi o anomalie da segnalare. È corretto pensare di raccoglierli in una suddivisione a cartelle corrispondenti alle classi documentali/processi di appartenenza separatamente dai documenti che richiedono segnalazione all'utente?
    \\ \textbf{Risposta:} La struttura proposta è una possibilità che riteniamo adeguata, tuttavia suggeriamo di dare particolare rilevanza ai documenti con anomalie o avvisi ponendoli in cima alla lista dei documenti validati e al di fuori della suddivisione a cartelle, per dare maggiore visibilità a ciascun documento in modo più immediato e chiaro per l'utente.

    \item \textbf{Domanda 4:} In aggiunta alla sezione di "Verifica Integrità" si pensava di fornire informazioni aggiuntive più dettagliate in un report apposito per i documenti con avvisi o anomalie, spiegando il motivo per il quale il documento non rispetta le regole di validazione. È corretto pensare di fornire questa informazione?
    \\ \textbf{Risposta:} Sì, è una buona idea fornire anche all'utente questa informazione che potrà ricercare in seguito e eventualmente segnalare a chi di competenza.

    \item \textbf{Domanda 5:} Dovremmo valutare di fornire all'utente informazioni relative all'interoperabilità dei formati di file presenti nel DIP?
    \\ \textbf{Risposta:} La valutazione di interoperabilità dei formati di file presenti nel DIP rientra nell'ambito di competenza di chi distribuisce il pacchetto di conservazione e l'utente non ha particolare controllo o interesse su questo tipo di informazioni. Sicuramente è necessario invece esporre il metadato \textit{MIME Type}, ovvero il formato del documento, tra le informazioni fornite all'utente.

    \item \textbf{Domanda 6:} Riguardo i metadati aggiuntivi (\textit{Custom Metadata}), dobbiamo prevedere che ci siano dei metadata definiti dai clienti che sono diversi da quelli predisposti dalle normative Agid? 
    \\ \textbf{Risposta:} Sì, la natura di questi metadati prevede che il cliente possa decidere di dare importanza a determinati dettagli per ogni classe documentale e inserirli all'interno della sezione Custom Metadata, ed è possibile trovare la descrizione di questi metadati negli appositi schemi \textit{xsd}. Il formato di rappresentazione per ciascuno di essi sarà la stringa, in quanto non è possibile assumere da parte del Team che il nome di un metadato implichi categoricamente l'appartenenza di quel dato a un determinato formato.
\end{itemize}


\section{Decisioni e Conclusione}
\begin{itemize}
    \item In seguito all'incontro il Team prevede di perfezionare l'analisi dei requisiti secondo le indicazioni ricevute.
    \item Parallelamente il Team ha deciso di iniziare la produzione del PoC.
\end{itemize}



\vfill


\begin{flushleft}
\begin{tabular}{@{}p{0.5in}p{2in}@{}}
Data: & \hrulefill \\
\end{tabular}

\vspace{1em}

\begin{tabular}{@{}p{0.5in}p{2in}@{}}
Firma: & \hrulefill \\
& [Sanmarco Informatica SPA] \\
\end{tabular}
\end{flushleft}

\begin{flushright}
    \textit{7-ZPUs}
\end{flushright}

\end{document}
