\documentclass[a4paper,12pt]{article}

\usepackage[utf8]{inputenc}
\usepackage[T1]{fontenc}
\usepackage[italian]{babel}
\usepackage[sfdefault]{atkinson}
\usepackage{float}
\usepackage{microtype}
\usepackage{geometry}
\usepackage{setspace}
\usepackage{enumitem}
\usepackage{titlesec}
\usepackage{chngpage}
\usepackage{tocloft}
\usepackage{longtable}
\usepackage{array}
\usepackage{graphicx}
\usepackage{fancyhdr}
\usepackage{xcolor}
\usepackage{color,soul}
\usepackage[most]{tcolorbox}
\usepackage[colorlinks=true]{hyperref}



\hypersetup{
    linkcolor=black,
    urlcolor=blue
}

\definecolor{lightblack}{gray}{0.35}
\newcommand{\glossario}[1]{\textit{#1}\textsubscript{\textbf{\textit{\textcolor{lightblack}{G}}}}}
\newcommand{\ped}[1]{\textsubscript{#1}}

\pagestyle{fancy}
\setlength{\headwidth}{\textwidth}
\fancyhfoffset[L,R]{0pt}
\lhead{\rightmark}
\rhead{7-ZPUs}
\lfoot{Piano di Qualifica}
\rfoot{\thepage}
\cfoot{}
\renewcommand{\headrulewidth}{0.8pt}
\renewcommand{\footrulewidth}{0.8pt}

\renewcommand{\contentsname}{Indice}

\geometry{margin=2.5cm}
\setstretch{1.1}

\titleclass{\subsubsubsection}{straight}[\subsection]

\newcounter{subsubsubsection}[subsubsection]
\renewcommand\thesubsubsubsection{\thesubsubsection.\arabic{subsubsubsection}}
\renewcommand\theparagraph{\thesubsubsubsection.\arabic{paragraph}} % optional; useful if paragraphs are to be numbered

\titleformat{\subsubsubsection}
  {\normalfont\normalsize\bfseries}{\thesubsubsubsection}{1em}{}
\titlespacing*{\subsubsubsection}
{0pt}{3.25ex plus 1ex minus .2ex}{1.5ex plus .2ex}

\makeatletter
\renewcommand\paragraph{\@startsection{paragraph}{5}{\z@}%
  {3.25ex \@plus1ex \@minus.2ex}%
  {-1em}%
  {\normalfont\normalsize\bfseries}}
\renewcommand\subparagraph{\@startsection{subparagraph}{6}{\parindent}%
  {3.25ex \@plus1ex \@minus .2ex}%
  {-1em}%
  {\normalfont\normalsize\bfseries}}
\def\toclevel@subsubsubsection{4}
\def\toclevel@paragraph{5}
%\def\toclevel@paragraph{6}
\def\toclevel@subparagraph{6}
\def\l@subsubsubsection{\@dottedtocline{4}{7em}{4em}}
\def\l@paragraph{\@dottedtocline{5}{10em}{5em}}
\def\l@subparagraph{\@dottedtocline{6}{14em}{6em}}
\makeatother

\setcounter{secnumdepth}{4}
\setcounter{tocdepth}{4}

\titleformat{\section}{\Large\bfseries}{\thesection}{1em}{}
\titleformat{\subsection}{\large\bfseries}{\thesubsection}{1em}{}
\titleformat{\subsubsection}{\normalsize\bfseries}{\thesubsubsection}{1em}{}
\titleformat{\paragraph}{\large\bfseries}{\theparagraph}{1em}{}
\titleformat{\subparagraph}{\normalsize\bfseries}{\thesubparagraph}{1em}{}

\begin{document}

\begin{center}
    \includegraphics[width=9.5cm]{../assets/logo7ZPUs.jpg}\\
    \small\hspace{10cm} 7zpus.swe@gmail.com\\
    \vspace{0.5cm}
    \Large \textbf{Piano di Qualifica}\\
\end{center}

\vspace{0.3cm}
\hrule
\vspace{0.5cm}


\section*{Tabella di Versionamento}
\begin{table}[H]
    \begin{adjustwidth}{-4cm}{-4cm}
    \centering
    \begin{tabular}{|c|c|c|c|c|}
        \hline
        \textbf{Versione} & \textbf{Data} & \textbf{Autore}  & \textbf{Verificatore} & \textbf{Descrizione} \\
        \hline
        0.1.0 & 07/11/2025 & Scrittore & Verificatore & \begin{tabular}[c]{@{}c@{}} Creazione del template\\ e stesura iniziale \end{tabular} \\
        \hline
    \end{tabular}
    \end{adjustwidth}
\end{table}

\tableofcontents

\newpage

\section{Introduzione}

\subsection{Scopo}
Lo scopo di questo documento è definire le metriche di qualità a supporto della verifica e validazione del ciclo di vita del progetto necessarie per poter fornire un prodotto corrispondente ai requisiti della proponente e agli obiettivi del team fornitore mantenendo standard qualitativi elevati.
La struttura è suddivisa in:
\begin{itemize}
    \item Obiettivi di qualità
    \item Test di verifica
    \item Cruscotto di valutazione e miglioramento
\end{itemize}

\subsection{Glossario}
Ogni termine tecnico o con particolare significato nell'ambito dell'\glossario{Ingegneria del Software} utilizzato nella documentazione di progetto viene definito nell'apposito documento \href{https://cdn.jsdelivr.net/gh/7-zpus/Docs@norme_in_lavorazione/2_RTB/Glossario.pdf}{\ul{Glossario1.0}\setulcolor{blue}}\ped{(ultimo accesso: 17/11/2025)}.

\subsection{Riferimenti}

\subsubsection{Riferimenti Normativi}
\begin{itemize}
    \item \href{https://cdn.jsdelivr.net/gh/7-zpus/Docs@norme_in_lavorazione/2_RTB/NormeDiProgetto.pdf}{\ul{NormeDiProgetto1.0}\setulcolor{blue}}\ped{(ultimo accesso: 3/12/2025)}
    \item \href{https://www.math.unipd.it/~tullio/IS-1/2025/Progetto/C3.pdf}{\ul{Capitolato C3: DIPReader}\setulcolor{blue}} \ped{(ultimo accesso: 01/12/2025)}
    \item \href{https://www.math.unipd.it/~tullio/IS-1/2025/Dispense/PD1.pdf}{\ul{Regolamento di Progetto Didattico a.a. 2025/2026}\setulcolor{blue}} \ped{(ultimo accesso: 17/11/2025)}
\end{itemize}
\subsubsection{Riferimenti Informativi}
\begin{itemize}
    \item \href{https://cdn.jsdelivr.net/gh/7-zpus/Docs@norme_in_lavorazione/2_RTB/Glossario.pdf}{\ul{Glossario1.0}\setulcolor{blue}}\ped{(ultimo accesso: 17/11/2025)}
    \item \href{https://iso25000.com}{\ul{The ISO/IEC 25000 Series of Standards}\setulcolor{blue}}
    \item \href{https://www.iso.org/standard/63712.html}{\ul{Standard ISO/IEC 9126-1:2001}\setulcolor{blue}}
    \item \href{https://www.iso.org/standard/71952.html}{\ul{Standard ISO/IEC 145981-1:1999}\setulcolor{blue}}
\end{itemize}

\section{Obiettivi di qualità}

\subsection{Metriche comuni}

\begin{table}[H]
    \begin{adjustwidth}{-4cm}{-4cm}
    \centering
    \begin{spacing}{1.5}
    \begin{tabular}{|c|c|c|c|}
        \hline
        \textbf{Codice} & \textbf{Nome metrica} & \textbf{Valore accettabile} & \textbf{Valore desiderabile} \\
        \hline
        MC-1 & Metriche soddisfatte & & \\
        \hline
        MC-2 & Requisiti obbligatori soddisfatti & & \\
        \hline
        MC-3 & Requisiti opzionali soddisfatti & & \\
        \hline
        MC-4 & Requisiti desiderabili soddisfatti & & \\
        \hline
    \end{tabular}
    \end{spacing}
    \end{adjustwidth}
\end{table}

\subsection{Qualità di processo}

\subsubsection{Processi primari}

\subsubsubsection{Fornitura}

\begin{table}[H]
    \begin{adjustwidth}{-4cm}{-4cm}
    \centering
    \begin{spacing}{1.5}
    \begin{tabular}{|c|c|c|c|}
        \hline
        \textbf{Codice} & \textbf{Nome metrica} & \begin{tabular}[c]{@{}c@{}}  \textbf{Valore}\\\textbf{accettabile} \end{tabular} & \begin{tabular}[c]{@{}c@{}}  \textbf{Valore}\\\textbf{desiderabile} \end{tabular} \\
        \hline
        MPC-1 & EA (Earned Value) & & \\
        \hline
        MPC-2 & PV (Planned Value) & & \\
        \hline
        MPC-3 & AC (Actual Cost) & & \\
        \hline
        MPC-4 & CPI (Cost Performance Index) & & \\
        \hline
        MPC-5 & SPI (Schedule Performance Index) & & \\
        \hline
        MPC-6 & TEAC (Time Estimate at Completion) & & \\
        \hline
        MPC-7 & ETC (Estimate to Complete) & & \\
        \hline
        MPC-8 & EAC (Estimate at Completion) & & \\
        \hline
        MPC-9 & BV (Budget Variance) & & \\
        \hline
        MPC-10 & SV (Schedule Variance) & & \\
        \hline
    \end{tabular}
    \end{spacing}
    \end{adjustwidth}
\end{table}

\subsubsubsection{Sviluppo}

\subsubsubsection{Operativo}



\subsubsection{Processi di supporto}

\subsubsection{Processi organizzativi}



\begin{table}[H]
    \begin{adjustwidth}{-4cm}{-4cm}
    \centering
    \begin{spacing}{1.5}
    \begin{tabular}{|c|c|c|c|}
        \hline
        \textbf{Codice} & \textbf{Nome metrica} & \textbf{Valore accettabile} & \textbf{Valore desiderabile} \\
        \hline
        MPC-9 & Rischi non previsti & & \\
        \hline
        MPC-10 & Efficacia contromisure rischi & & \\
        \hline
        MPC-11 & Stabilità dei requisiti & & \\
        \hline
        MPC-12 & Variazione dei requisiti & & \\
        \hline
        MPC-13 & Work in progress & & \\
        \hline
        MPC-14 & Code review turnaround time & & \\
        \hline
        MPC-15 & Deployment frequency & & \\
        \hline
        MPC-16 & Change failure rate & & \\
        \hline
        
    \end{tabular}
    \end{spacing}
    \end{adjustwidth}
\end{table}


\subsection{Qualità di prodotto}
In seguito vengono elencate le metriche ritenute necessarie per fornire un prodotto di qualità. 

\subsubsection{Funzionalità}

\subsubsection{Affidabilità}

\subsubsection{Usabilità}

\subsubsection{Efficienza}

\subsubsection{Manutenibilità}

\subsubsection{Portabilità}

\begin{table}[H]
    \begin{adjustwidth}{-4cm}{-4cm}
    \centering
    \begin{spacing}{1.5}
    \begin{tabular}{|c|c|c|c|}
        \hline
        \textbf{Codice} & \textbf{Nome metrica} & \textbf{Valore accettabile} & \textbf{Valore desiderabile} \\
        \hline
        MPD-1 & Tempo di risposta (Time To First Byte) & 1.5 s & 0.5 s \\
        \hline
        MPD-2 & Largest Contentful Paint & & \\
        \hline
        MPD-3 & Throughput & & \\
        \hline
        MPD-4 & Densità degli errori & & \\
        \hline
        MPD-5 & Statement coverage & & \\
        \hline
        MPD-6 & Branch coverage & & \\
        \hline
        MPD-7 & Complessità ciclomatica & & \\
        \hline
        MPD-8 & Accoppiamento tra classi & & \\
        \hline
        MPD-9 & Document Retrieval Time (DRT) & & \\
        \hline
        MPD-10 & Search Success Rate & & \\
        \hline
        MPD-11 & Profondità di navigazione & & \\
        \hline
        MPD-12 & Indice di Gulpease & & \\
        \hline
        MPD-13 & Browser supportati & & \\
        \hline
        MPD-14 & Tempo rendering prima pagina & & \\
        \hline
        MPD-15 & Code churn (\# righe aggiunte, modif. o canc.) & & \\
        \hline
        MPD-16 & Comment density & & \\
        \hline
        MPD-17 & Livello conformità WCAG & & \\
        \hline
        MPD-18 & Tasso errori HTTP & & \\
        \hline
        MPD-19 & Mean Reciprocal Rank (pos. risultato corr.) & & \\
        \hline
        MPD-20 & Average CPU usage & & \\
        \hline
        MPD-21 & Peak memory usage & & \\
        \hline
        MPD-22 & Densità commenti & & \\
        \hline
        
    \end{tabular}
    \end{spacing}
    \end{adjustwidth}
\end{table}

\section{Test di verifica}




minimo e desiderabile

\vfill
\begin{flushright}
    \textit{7-ZPUs}
\end{flushright}

\end{document}