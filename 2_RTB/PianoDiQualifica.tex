\documentclass[a4paper,12pt]{article}

\usepackage[utf8]{inputenc}
\usepackage[T1]{fontenc}
\usepackage[italian]{babel}
\usepackage[sfdefault]{atkinson}
\usepackage{float}
\usepackage{microtype}
\usepackage{geometry}
\usepackage{setspace}
\usepackage{enumitem}
\usepackage{titlesec}
\usepackage{chngpage}
\usepackage{tocloft}
\usepackage{longtable}
\usepackage{array}
\usepackage{graphicx}
\usepackage{fancyhdr}
\usepackage{xcolor}
\usepackage{color,soul}
\usepackage[most]{tcolorbox}
\usepackage[colorlinks=true]{hyperref}



\hypersetup{
    linkcolor=black,
    urlcolor=blue
}

\definecolor{lightblack}{gray}{0.35}
\newcommand{\glossario}[1]{\textit{#1}\textsubscript{\textbf{\textit{\textcolor{lightblack}{G}}}}}
\newcommand{\ped}[1]{\textsubscript{#1}}

\pagestyle{fancy}
\setlength{\headwidth}{\textwidth}
\fancyhfoffset[L,R]{0pt}
\lhead{\rightmark}
\rhead{7-ZPUs}
\lfoot{Piano di Qualifica}
\rfoot{\thepage}
\cfoot{}
\renewcommand{\headrulewidth}{0.8pt}
\renewcommand{\footrulewidth}{0.8pt}

\renewcommand{\contentsname}{Indice}

\geometry{margin=2.5cm}
\setstretch{1.1}

\titleclass{\subsubsubsection}{straight}[\subsection]

\newcounter{subsubsubsection}[subsubsection]
\renewcommand\thesubsubsubsection{\thesubsubsection.\arabic{subsubsubsection}}
\renewcommand\theparagraph{\thesubsubsubsection.\arabic{paragraph}} % optional; useful if paragraphs are to be numbered

\titleformat{\subsubsubsection}
  {\normalfont\normalsize\bfseries}{\thesubsubsubsection}{1em}{}
\titlespacing*{\subsubsubsection}
{0pt}{3.25ex plus 1ex minus .2ex}{1.5ex plus .2ex}

\makeatletter
\renewcommand\paragraph{\@startsection{paragraph}{5}{\z@}%
  {3.25ex \@plus1ex \@minus.2ex}%
  {-1em}%
  {\normalfont\normalsize\bfseries}}
\renewcommand\subparagraph{\@startsection{subparagraph}{6}{\parindent}%
  {3.25ex \@plus1ex \@minus .2ex}%
  {-1em}%
  {\normalfont\normalsize\bfseries}}
\def\toclevel@subsubsubsection{4}
\def\toclevel@paragraph{5}
%\def\toclevel@paragraph{6}
\def\toclevel@subparagraph{6}
\def\l@subsubsubsection{\@dottedtocline{4}{7em}{4em}}
\def\l@paragraph{\@dottedtocline{5}{10em}{5em}}
\def\l@subparagraph{\@dottedtocline{6}{14em}{6em}}
\makeatother

\setcounter{secnumdepth}{4}
\setcounter{tocdepth}{4}

\titleformat{\section}{\LARGE\bfseries}{\thesection}{1em}{}
\titleformat{\subsection}{\Large\bfseries}{\thesubsection}{1em}{}
\titleformat{\subsubsection}{\large\bfseries}{\thesubsubsection}{1em}{}
\titleformat{\paragraph}{\large\bfseries}{\theparagraph}{1em}{}
\titleformat{\subparagraph}{\normalsize\bfseries}{\thesubparagraph}{1em}{}

\begin{document}

\begin{center}
    \includegraphics[width=9.5cm]{../assets/logo7ZPUs.jpg}\\
    \small\hspace{10cm} 7zpus.swe@gmail.com\\
    \vspace{0.5cm}
    \Large \textbf{Piano di Qualifica}\\
\end{center}

\vspace{0.3cm}
\hrule
\vspace{0.5cm}


\section*{Tabella di Versionamento}
\begin{table}[H]
    \begin{adjustwidth}{-4cm}{-4cm}
    \centering
    \begin{tabular}{|c|c|c|c|c|}
        \hline
        \textbf{Versione} & \textbf{Data} & \textbf{Autore}  & \textbf{Verificatore} & \textbf{Descrizione} \\
        \hline
        0.3.0 & 2025/02/08 & Vigolo Davide & Soligo Lorenzo & \begin{tabular}[c]{@{}c@{}} Aggiunta sezione cruscotto \\ di valutazione \end{tabular} \\
        \hline
        0.2.0 & 2025/02/02 & Vigolo Davide & Soligo Lorenzo & \begin{tabular}[c]{@{}c@{}} Revisione metriche\\ e aggiornamento soglie \end{tabular} \\
        \hline
        0.1.0 & 2025/12/14 & Rocco Matteo A. & Vigolo Davide & \begin{tabular}[c]{@{}c@{}} Creazione del documento\\ e stesura iniziale \end{tabular} \\
        \hline
    \end{tabular}
    \end{adjustwidth}
\end{table}

\tableofcontents

\newpage

\section{Introduzione}

\subsection{Scopo}
Lo scopo di questo documento è di fondamentale importanza. Permette di definire misure quantitative per misurare la qualità di processo e di prodotto. Assieme al cruscotto di valutazione permette di monitorare l'efficacia e l'efficienza dei processi di ciclo di vita istanziati nel progetto.
Garantire una sufficiente qualità di processo e di prodotto, è condizione necessaria alla qualità di prodotto in uso, che è di interesse primario per la committente. Il Piano di Qualifica si compone di tre elementi:
\begin{itemize}
    \item Piano della Qualità: Definizione di obiettivi quantitativi di qualità, metriche e strategie per raggiungerla
    \item Controllo di Qualità: insieme di attività e tecniche per valutare che il piano stabilito sia efficace.
    \item Miglioramento continuo: stabilire eventuali azioni correttive alla luce dei risultati del controllo, adattando processi, obiettivi e vincoli.
\end{itemize}

\subsection{Glossario}
Ogni termine tecnico o con particolare significato nell'ambito dell'\glossario{Ingegneria del Software} utilizzato nella documentazione di progetto viene definito nell'apposito documento \href{https://cdn.jsdelivr.net/gh/7-zpus/Docs@norme_in_lavorazione/2_RTB/Glossario.pdf}{\ul{Glossario1.0}\setulcolor{blue}}\ped{(ultimo accesso: 17/11/2025)}.

\subsection{Riferimenti}

\subsubsection{Riferimenti Normativi}
\begin{itemize}
    \item \href{https://cdn.jsdelivr.net/gh/7-zpus/Docs@norme_in_lavorazione/2_RTB/NormeDiProgetto.pdf}{\ul{NormeDiProgetto1.0}\setulcolor{blue}}\ped{(ultimo accesso: 3/12/2025)}
    \item \href{https://www.math.unipd.it/~tullio/IS-1/2025/Progetto/C3.pdf}{\ul{Capitolato C3: DIPReader}\setulcolor{blue}} \ped{(ultimo accesso: 01/12/2025)}
    \item \href{https://www.math.unipd.it/~tullio/IS-1/2025/Dispense/PD1.pdf}{\ul{Regolamento di Progetto Didattico a.a. 2025/2026}\setulcolor{blue}} \ped{(ultimo accesso: 17/11/2025)}
\end{itemize}
\subsubsection{Riferimenti Informativi}
\begin{itemize}
    \item \href{https://cdn.jsdelivr.net/gh/7-zpus/Docs@norme_in_lavorazione/2_RTB/Glossario.pdf}{\ul{Glossario1.0}\setulcolor{blue}}\ped{(ultimo accesso: 17/11/2025)}
    \item \href{https://iso25000.com}{\ul{The ISO/IEC 25000 Series of Standards}\setulcolor{blue}}
    \item \href{https://www.iso.org/standard/63712.html}{\ul{Standard ISO/IEC 9126-1:2001}\setulcolor{blue}}
    \item \href{https://www.iso.org/standard/71952.html}{\ul{Standard ISO/IEC 145981-1:1999}\setulcolor{blue}}
\end{itemize}

\section{Metriche di qualità}

\subsection{Qualità di processo}

\subsubsection{Processi primari}

\subsubsubsection{Fornitura}

\begin{table}[H]
    \begin{adjustwidth}{-4cm}{-4cm}
    \centering
    \begin{spacing}{1.5}
    \begin{tabular}{|c|c|c|c|}
        \hline
        \textbf{Codice} & \textbf{Nome metrica} & \begin{tabular}[c]{@{}c@{}}  \textbf{Valore}\\\textbf{accettabile} \end{tabular} & \begin{tabular}[c]{@{}c@{}}  \textbf{Valore}\\\textbf{desiderabile} \end{tabular} \\
        \hline
        MPC-1 & EV (Earned Value) & $\geq$0 & PV \\
        \hline
        MPC-2 & PV (Planned Value) & $\geq$0 & - \\
        \hline
        MPC-3 & AC (Actual Cost) & $\geq$0 & $\leq$EV \\
        \hline
        MPC-4 & CPI (Cost Performance Index) & $\geq$0.5 & $\geq$1 \\
        \hline
        MPC-5 & SPI (Schedule Performance Index) & $\geq$0.5 & $\geq$1 \\
        \hline
        MPC-6 & ETC (Estimate to Complete) & $\geq$0 & $\leq$BAC\textsubscript{\textit{G}} - AC\textsubscript{\textit{G}} \\
        \hline
        MPC-7 & EAC (Estimate at Completion) & $\geq$0 & $\leq$BAC \\
        \hline
    \end{tabular}
    \end{spacing}
    \end{adjustwidth}
\end{table}

\subsubsubsection{Sviluppo}

\begin{table}[H]
    \begin{adjustwidth}{-4cm}{-4cm}
    \centering
    \begin{spacing}{1.5}
    \begin{tabular}{|c|c|c|c|}
        \hline
        \textbf{Codice} & \textbf{Nome metrica} & \begin{tabular}[c]{@{}c@{}}  \textbf{Valore}\\\textbf{accettabile} \end{tabular} & \begin{tabular}[c]{@{}c@{}}  \textbf{Valore}\\\textbf{desiderabile} \end{tabular} \\
        \hline
        MPC-8 & Deployment frequency & - & 1/gg \\
        \hline
        MPC-9 & Requirements Stability Index & $\geq$70\% & 100\% \\
        \hline
    \end{tabular}
    \end{spacing}
    \end{adjustwidth}
\end{table}

\subsubsubsection{Integrazione}

\begin{table}[H]
    \begin{adjustwidth}{-4cm}{-4cm}
    \centering
    \begin{spacing}{1.5}
    \begin{tabular}{|c|c|c|c|}
        \hline
        \textbf{Codice} & \textbf{Nome metrica} & \begin{tabular}[c]{@{}c@{}}  \textbf{Valore}\\\textbf{accettabile} \end{tabular} & \begin{tabular}[c]{@{}c@{}}  \textbf{Valore}\\\textbf{desiderabile} \end{tabular} \\
        \hline
        MPC-12 & Average build time & $\leq$15 min & $\leq$10 min \\
        \hline
    \end{tabular}
    \end{spacing}
    \end{adjustwidth}
\end{table}

\subsubsection{Processi di supporto}

\subsubsubsection{Documentazione}

\begin{table}[H]
    \begin{adjustwidth}{-4cm}{-4cm}
    \centering
    \begin{spacing}{1.5}
    \begin{tabular}{|c|c|c|c|}
        \hline
        \textbf{Codice} & \textbf{Nome metrica} & \begin{tabular}[c]{@{}c@{}}  \textbf{Valore}\\\textbf{accettabile} \end{tabular} & \begin{tabular}[c]{@{}c@{}}  \textbf{Valore}\\\textbf{desiderabile} \end{tabular} \\
        \hline
        MPC-10 & Indice di Gulpease & $\geq$60 & $\geq$75\\
        \hline
        MPC-11 & Correttezza ortografica & 0 & 0 \\
        \hline
    \end{tabular}
    \end{spacing}
    \end{adjustwidth}
\end{table}

\subsubsubsection{Verifica}

\begin{table}[H]
    \begin{adjustwidth}{-4cm}{-4cm}
    \centering
    \begin{spacing}{1.5}
    \begin{tabular}{|c|c|c|c|}
        \hline
        \textbf{Codice} & \textbf{Nome metrica} & \begin{tabular}[c]{@{}c@{}}  \textbf{Valore}\\\textbf{accettabile} \end{tabular} & \begin{tabular}[c]{@{}c@{}}  \textbf{Valore}\\\textbf{desiderabile} \end{tabular} \\
        \hline
        MPC-13 & Code review turnaround time & $\leq$72h & $\leq$24h\\
        \hline
        MPC-14 & Test success rate & 100\% & 100\%\\
        \hline
    \end{tabular}
    \end{spacing}
    \end{adjustwidth}
\end{table}

\subsubsection{Processi organizzativi}

\subsubsubsection{Gestione dei rischi}

\begin{table}[H]
    \begin{adjustwidth}{-4cm}{-4cm}
    \centering
    \begin{spacing}{1.5}
    \begin{tabular}{|c|c|c|c|}
        \hline
        \textbf{Codice} & \textbf{Nome metrica} & \begin{tabular}[c]{@{}c@{}}  \textbf{Valore}\\\textbf{accettabile} \end{tabular} & \begin{tabular}[c]{@{}c@{}}  \textbf{Valore}\\\textbf{desiderabile} \end{tabular} \\
        \hline
        MPC-15 & Rischi non previsti & $\geq$0 & 0\\
        \hline
    \end{tabular}
    \end{spacing}
    \end{adjustwidth}
\end{table}

\subsubsubsection{Gestione della Qualità}
\begin{table}[H]
    \begin{adjustwidth}{-4cm}{-4cm}
    \centering
    \begin{spacing}{1.5}
    \begin{tabular}{|c|c|c|c|}
        \hline
        \textbf{Codice} & \textbf{Nome metrica} & \begin{tabular}[c]{@{}c@{}}  \textbf{Valore}\\\textbf{accettabile} \end{tabular} & \begin{tabular}[c]{@{}c@{}}  \textbf{Valore}\\\textbf{desiderabile} \end{tabular} \\
        \hline
        MPC-16 & Metriche soddisfatte & $\geq$70\% & 100\% \\
        \hline
    \end{tabular}
    \end{spacing}
    \end{adjustwidth}
\end{table}


\subsection{Qualità di prodotto}
Le metriche definite in questa sezione riguardano principalmente caratteristiche di qualità "interne" del prodotto software. Raggiungere la qualità su queste caratteristiche abilita all'ottenimento di qualità in uso, o "esterna".
Suddividiamo le metriche secondo raggruppamenti logici qui di seguito elencati ed esplicitati:
\begin{itemize}
    \item \textbf{Funzionalità}: completezza, correttezza ed appropriatezza del prodotto
    \item \textbf{Affidabilità}: maturità, disponibilità, tolleranza ai guasti e riparabilità del prodotto
    \item \textbf{Usabilità}: apprendibilità, operabilità, UX e accessibilità del prodotto
    \item \textbf{Efficienza}: nel tempo, nelle altre risorse, nella capacità
    \item \textbf{Manutenibilità}: modularità, riusabilità, analizzabilità, modificabilità e verificabilità del prodotto
    \item \textbf{Portabilità}: adattabilità del prodotto a diversi ambienti
\end{itemize}


\subsubsection{Funzionalità}

\begin{table}[H]
    \begin{adjustwidth}{-4cm}{-4cm}
    \centering
    \begin{spacing}{1.5}
    \begin{tabular}{|c|c|c|c|}
        \hline
        \textbf{Codice} & \textbf{Nome metrica} & \begin{tabular}[c]{@{}c@{}}  \textbf{Valore}\\\textbf{accettabile} \end{tabular} & \begin{tabular}[c]{@{}c@{}}  \textbf{Valore}\\\textbf{desiderabile} \end{tabular} \\
        \hline
        MPD-2 & Requisiti obbligatori soddisfatti & $\geq$0\% & 100\% \\
        \hline
        MPD-3 & Requisiti opzionali soddisfatti & $\geq$0\% & 100\%\\
        \hline
        MPD-4 & Requisiti desiderabili soddisfatti & $\geq$0\% & 100\% \\
        \hline
    \end{tabular}
    \end{spacing}
    \end{adjustwidth}
\end{table}


\subsubsection{Affidabilità}

\begin{table}[H]
    \begin{adjustwidth}{-4cm}{-4cm}
    \centering
    \begin{spacing}{1.5}
    \begin{tabular}{|c|c|c|c|}
        \hline
        \textbf{Codice} & \textbf{Nome metrica} & \begin{tabular}[c]{@{}c@{}}  \textbf{Valore}\\\textbf{accettabile} \end{tabular} & \begin{tabular}[c]{@{}c@{}}  \textbf{Valore}\\\textbf{desiderabile} \end{tabular} \\
        \hline
        MPD-5 & Broken Links & 2 & 0\\
        \hline
        MPD-6 & Branch coverage & $\geq$80\% & $\geq$90\% \\
        \hline
        MPD-7 & Statement Coverage & $\geq$65\% & $\geq$80\% \\
        \hline
    \end{tabular}
    \end{spacing}
    \end{adjustwidth}
\end{table}

\subsubsection{Usabilità}

\begin{table}[H]
    \begin{adjustwidth}{-4cm}{-4cm}
    \centering
    \begin{spacing}{1.5}
    \begin{tabular}{|c|c|c|c|}
        \hline
        \textbf{Codice} & \textbf{Nome metrica} & \begin{tabular}[c]{@{}c@{}}  \textbf{Valore}\\\textbf{accettabile} \end{tabular} & \begin{tabular}[c]{@{}c@{}}  \textbf{Valore}\\\textbf{desiderabile} \end{tabular} \\
        \hline
        MPD-8 & Profondità di navigazione & $\geq$0 & $\leq$5 \\
        \hline
    \end{tabular}
    \end{spacing}
    \end{adjustwidth}
\end{table}


\subsubsection{Efficienza}

\begin{table}[H]
    \begin{adjustwidth}{-4cm}{-4cm}
    \centering
    \begin{spacing}{1.5}
    \begin{tabular}{|c|c|c|c|}
        \hline
        \textbf{Codice} & \textbf{Nome metrica} & \begin{tabular}[c]{@{}c@{}}  \textbf{Valore}\\\textbf{accettabile} \end{tabular} & \begin{tabular}[c]{@{}c@{}}  \textbf{Valore}\\\textbf{desiderabile} \end{tabular} \\
        \hline
        MPD-9 & Indexing Time & 2min & $\leq$30 s \\
        \hline
        MPD-10 & Search Time & $\leq$2 s & $\leq$1 s \\
        \hline
        MPD-11 & Average CPU usage & $\leq$30\% & $\leq$15\% \\
        \hline
        MPD-12 & Peak memory usage & $\leq$1 GB & $\leq$500 MB \\
        \hline
    \end{tabular}
    \end{spacing}
    \end{adjustwidth}
\end{table}



\subsubsection{Manutenibilità}

\begin{table}[H]
    \begin{adjustwidth}{-4cm}{-4cm}
    \centering
    \begin{spacing}{1.5}
    \begin{tabular}{|c|c|c|c|}
        \hline
        \textbf{Codice} & \textbf{Nome metrica} & \begin{tabular}[c]{@{}c@{}}  \textbf{Valore}\\\textbf{accettabile} \end{tabular} & \begin{tabular}[c]{@{}c@{}}  \textbf{Valore}\\\textbf{desiderabile} \end{tabular} \\
       \hline
        MPD-13 & Complessità ciclomatica & $\leq$15 & $\leq$10 \\
        \hline
        MPD-14 & Accoppiamento tra classi & $\leq$0.4 & $\leq$0.2 \\
        \hline
        MPD-15 & Code Smells & $\leq$15 & 0 \\
        \hline
    \end{tabular}
    \end{spacing}
    \end{adjustwidth}
\end{table}

\subsubsection{Portabilità}

\begin{table}[H]
    \begin{adjustwidth}{-4cm}{-4cm}
    \centering
    \begin{spacing}{1.5}
    \begin{tabular}{|c|c|c|c|}
        \hline
        \textbf{Codice} & \textbf{Nome metrica} & \begin{tabular}[c]{@{}c@{}}  \textbf{Valore}\\\textbf{accettabile} \end{tabular} & \begin{tabular}[c]{@{}c@{}}  \textbf{Valore}\\\textbf{desiderabile} \end{tabular} \\
        \hline
        MPD-16 & Sistemi operativi supportati & Windows, macOS, Linux & Windows, macOS, Linux \\
        \hline
    \end{tabular}
    \end{spacing}
    \end{adjustwidth}
\end{table}

\section{Test di verifica}
todo.

\section{Cruscotto di valutazione e miglioramento}

Il cruscotto di valutazione è strumento fondamentale per la corretta gestione di progetto. Permette di avere un visione oggettiva dell'andamento, e di identificare eventuali problemi o criticità. Nei grafici seguenti sono riportati i dati per ogni settimana. Questo perché un monitoraggio solamente a termine di periodo è una scelta miope e di marginale utilità. Si preferisce invece avere un monitoraggio costante. Si fa notare come nel primo periodo
il gruppo, non disponendo ancora del cruscotto, non abbia attivamente monitorato le metriche. Nonostante ciò, essendo esse ricavabili vengono riportate per completezza.

\subsection{MPC-1, MPC-2, MPC-3: Earned Value, Planned Value, Actual Cost}
\includegraphics[width=\textwidth]{../assets/pdq/ev_pv_ac.png}
Dal grafico si può apprezzare come il progetto sia partito lentamente durante le prime due settimane, dovuto ovviamente alle fasi iniziali di organizzazione del gruppo. Successivamente durante il terzo e il quarto sprint si può notare una differenza importante tr PV ed EV. Questo, come riportato nelle retrospettive, è dovuto ad una stima delle attività troppo ottimistica. Il gruppo ha intrapreso dunque un approccio più conservativo che ha portato al ricongiungersi dei due valori alla settimana 10-11. \\
\subsection{MPC-4, MPC-5: CPI, SPI}
\includegraphics[width=\textwidth]{../assets/pdq/cpi_spi.png}
Possiamo notare anche da CPI e SPI la pianificazione troppo ottimistica con un calo dello SPI.
Nonostante il rallentamento, il gruppo ha completato le attività con costi inferiori al previsto.
Successivamente invece si è riusciti a recuperare il ritardo. Lo SPI è tornato sopra 1, dovuto al fatto che si sono dovute recuperare le task in ritardo, tuttavia spendendo più del previsto. Il CPI si è stabilizzato nelle ultime settimane, dovuto al fatto che il gruppo ha familiarizzato e accumulato esperienza nello stimare task simili. \\
\subsection{MPC-7: EAC confrontato con BAC}
\includegraphics[width=\textwidth]{../assets/pdq/eac_bac.png}
Strettamente collegato, e conseguenza del CPI, è l'andamento di EAC e BAC. Tra le settimane 3 e 5 si osserva una diminuzione. Tale diminuzione non tiene conto della sottostima delle attività: nelle attività completate il costo reale è risultato inferiore. Successivamente l'EAC aumenta, segnalando un incremento dei costi. Questo aumento è dovuto alle numerose iterazioni, a volte molto distruttive, sul documento di Analisi dei Requisiti. Dopo gli incontri con il prof. Cardin il gruppo ha dovuto rivedere quasi completamente la struttura degli Use Case\textsubscript{\textit{G}}. \\
\subsection{MPC-1, MPC-2, MPC-3, MPC-6, MPC-7: EV, PV, AC, ETC, EAC}
\includegraphics[width=\textwidth]{../assets/pdq/ev_pv_ac_eac_etc.png}
Si può apprezzare dal grafico quanto detto fino ad ora. Si osserva un aumento dell EAC nella fase iniziale, con un suo successivo rientro. \\
\subsection{MPC-10: Indice di Gulpease}
\includegraphics[width=\textwidth]{../assets/pdq/gfi.png}
Il gruppo misura l'Indice di Gulpease per garantire la leggibilità, soprattutto dei documenti rivolti ad esterni. Nelle fasi iniziali non è stata posta particolare attenzione, ma i risultati sono stati comunque buoni. Un'eccezione sono le Norme di Progetto: più descrittive e con pochi elenchi, mostrano valori inferiori. Per garantire un limite minimo, nelle settimane 7--10 (SPRINT 4) è stato introdotto un controllo automatico che accetta in repository solo documenti con Gulpease $\geq$ 60. \\
\subsection{MPC-11: Correttezza ortografica}
\includegraphics[width=\textwidth]{../assets/pdq/eo.png}
Gli errori ortografici sono stati gestiti in maniera analoga e parallela al Gulpease Index. Il gruppo ha deciso di non accettare documenti con errori ortografici, implementando un controllo automatico. \\
\subsection{MPC-15: Rischi non previsti}
\includegraphics[width=\textwidth]{../assets/pdq/rnp.png}
Il gruppo ha registrato due rischi imprevisti in due sprint diversi. Il primo, coerente con i dati mostrati, è stato causato dalla sottostima delle attività. Nel secondo sprint si è verificato un rischio imprevedibile: un membro del gruppo è stato indisposto. \\
\subsection{MPC-9: Requirements Stability Index}
\includegraphics[width=\textwidth]{../assets/pdq/rsi.png}
Dal grafico si osserva che, all'inizio, la stabilità dei requisiti è rimasta costante mentre il gruppo definiva le Norme di Progetto e i requisiti non erano ancora esplorati. Successivamente si è verificato un calo significativo: l'attenzione si è spostata sullo studio del dominio applicativo e sulle aspettative della proponente. La scarsa conoscenza del dominio e la continua selezione e revisione dei requisiti, anche per motivi di fattibilità, hanno generato instabilità. La proponente non ha imposto vincoli rigidi sulle funzionalità; di conseguenza i requisiti sono stati proposti e discussi in modo progressivo. Questo ha fatto sì che non si sia reso spesso necessario modificare, aggiungere o eliminare requisiti, e l'indice si è stabilizzato nella fase finale.
\subsection{MPC-16: Metriche soddisfatte}
\includegraphics[width=\textwidth]{../assets/pdq/ms.png}
Si nota un calo iniziale, corrispondentemente al periodo di instabilità generale del progetto. Successivamente, grazie a misure correttive e ai controlli automatici di alcune metriche si è raggiunta la stabilità. Questo anche grazie alla crescente attenzione rivolta al soddisfacimento delle metriche di qualità.

\vfill

\begin{flushright}
    \textit{7-ZPUs}
\end{flushright}

\end{document}