\documentclass[a4paper,12pt]{article}

\usepackage[utf8]{inputenc}
\usepackage[T1]{fontenc}
\usepackage[italian]{babel}
\usepackage[sfdefault]{atkinson}
\usepackage{float}
\usepackage{microtype}
\usepackage{geometry}
\usepackage{setspace}
\usepackage{enumitem}
\usepackage{titlesec}
\usepackage{chngpage}
\usepackage{tocloft}
\usepackage{longtable}
\usepackage{array}
\usepackage{graphicx}
\usepackage{fancyhdr}
\usepackage{xcolor}
\usepackage{color,soul}
\usepackage[most]{tcolorbox}
\usepackage[colorlinks=true]{hyperref}



\hypersetup{
    linkcolor=black,
    urlcolor=blue
}

\definecolor{lightblack}{gray}{0.35}
\newcommand{\glossario}[1]{\textit{#1}\textsubscript{\textbf{\textit{\textcolor{lightblack}{G}}}}}
\newcommand{\ped}[1]{\textsubscript{#1}}

\pagestyle{fancy}
\setlength{\headwidth}{\textwidth}
\fancyhfoffset[L,R]{0pt}
\lhead{\rightmark}
\rhead{7-ZPUs}
\lfoot{Piano di Qualifica}
\rfoot{\thepage}
\cfoot{}
\renewcommand{\headrulewidth}{0.8pt}
\renewcommand{\footrulewidth}{0.8pt}

\renewcommand{\contentsname}{Indice}

\geometry{margin=2.5cm}
\setstretch{1.1}

\titleclass{\subsubsubsection}{straight}[\subsection]

\newcounter{subsubsubsection}[subsubsection]
\renewcommand\thesubsubsubsection{\thesubsubsection.\arabic{subsubsubsection}}
\renewcommand\theparagraph{\thesubsubsubsection.\arabic{paragraph}} % optional; useful if paragraphs are to be numbered

\titleformat{\subsubsubsection}
  {\normalfont\normalsize\bfseries}{\thesubsubsubsection}{1em}{}
\titlespacing*{\subsubsubsection}
{0pt}{3.25ex plus 1ex minus .2ex}{1.5ex plus .2ex}

\makeatletter
\renewcommand\paragraph{\@startsection{paragraph}{5}{\z@}%
  {3.25ex \@plus1ex \@minus.2ex}%
  {-1em}%
  {\normalfont\normalsize\bfseries}}
\renewcommand\subparagraph{\@startsection{subparagraph}{6}{\parindent}%
  {3.25ex \@plus1ex \@minus .2ex}%
  {-1em}%
  {\normalfont\normalsize\bfseries}}
\def\toclevel@subsubsubsection{4}
\def\toclevel@paragraph{5}
%\def\toclevel@paragraph{6}
\def\toclevel@subparagraph{6}
\def\l@subsubsubsection{\@dottedtocline{4}{7em}{4em}}
\def\l@paragraph{\@dottedtocline{5}{10em}{5em}}
\def\l@subparagraph{\@dottedtocline{6}{14em}{6em}}
\makeatother

\setcounter{secnumdepth}{4}
\setcounter{tocdepth}{4}

\titleformat{\section}{\LARGE\bfseries}{\thesection}{1em}{}
\titleformat{\subsection}{\Large\bfseries}{\thesubsection}{1em}{}
\titleformat{\subsubsection}{\large\bfseries}{\thesubsubsection}{1em}{}
\titleformat{\paragraph}{\large\bfseries}{\theparagraph}{1em}{}
\titleformat{\subparagraph}{\normalsize\bfseries}{\thesubparagraph}{1em}{}

\begin{document}

\begin{center}
    \includegraphics[width=9.5cm]{../assets/logo7ZPUs.jpg}\\
    \small\hspace{10cm} 7zpus.swe@gmail.com\\
    \vspace{0.5cm}
    \Large \textbf{Piano di Qualifica}\\
\end{center}

\vspace{0.3cm}
\hrule
\vspace{0.5cm}


\section*{Tabella di Versionamento}
\begin{table}[H]
    \begin{adjustwidth}{-4cm}{-4cm}
    \centering
    \begin{tabular}{|c|c|c|c|c|}
        \hline
        \textbf{Versione} & \textbf{Data} & \textbf{Autore}  & \textbf{Verificatore} & \textbf{Descrizione} \\
        \hline
        0.1.0 & 14/12/2025 & Rocco Matteo A. & Vigolo Davide & \begin{tabular}[c]{@{}c@{}} Creazione del documento\\ e stesura iniziale \end{tabular} \\
        \hline
    \end{tabular}
    \end{adjustwidth}
\end{table}

\tableofcontents

\newpage

\section{Introduzione}

\subsection{Scopo}
Lo scopo di questo documento è di fondamentale importanza. Permette di definire misure quantitative per misurare la qualità di processo e di prodotto. Assieme al cruscotto di valutazione permette di monitorare l'efficacia e l'efficienza dei processi di ciclo di vita istanziati nel progetto.
Garantire una sufficiente qualità di processo e di prodotto, è condizione necessaria alla qualità di prodotto in uso, che è di interesse primario per la committente. Il Piano di Qualifica si compone di tre elementi:
\begin{itemize}
    \item Piano della Qualità: Definizione di obiettivi quantitativi di qualità, metriche e strategie per raggiungerla
    \item Controllo di Qualità: insieme di attività e tecniche per valutare che il piano stabilito sia efficace.
    \item Miglioramento continuo: stabilire eventuali azioni correttive alla luce dei risultati del controllo, adattando processi, obiettivi e vincoli.
\end{itemize}

\subsection{Glossario}
Ogni termine tecnico o con particolare significato nell'ambito dell'\glossario{Ingegneria del Software} utilizzato nella documentazione di progetto viene definito nell'apposito documento \href{https://cdn.jsdelivr.net/gh/7-zpus/Docs@norme_in_lavorazione/2_RTB/Glossario.pdf}{\ul{Glossario1.0}\setulcolor{blue}}\ped{(ultimo accesso: 17/11/2025)}.

\subsection{Riferimenti}

\subsubsection{Riferimenti Normativi}
\begin{itemize}
    \item \href{https://cdn.jsdelivr.net/gh/7-zpus/Docs@norme_in_lavorazione/2_RTB/NormeDiProgetto.pdf}{\ul{NormeDiProgetto1.0}\setulcolor{blue}}\ped{(ultimo accesso: 3/12/2025)}
    \item \href{https://www.math.unipd.it/~tullio/IS-1/2025/Progetto/C3.pdf}{\ul{Capitolato C3: DIPReader}\setulcolor{blue}} \ped{(ultimo accesso: 01/12/2025)}
    \item \href{https://www.math.unipd.it/~tullio/IS-1/2025/Dispense/PD1.pdf}{\ul{Regolamento di Progetto Didattico a.a. 2025/2026}\setulcolor{blue}} \ped{(ultimo accesso: 17/11/2025)}
\end{itemize}
\subsubsection{Riferimenti Informativi}
\begin{itemize}
    \item \href{https://cdn.jsdelivr.net/gh/7-zpus/Docs@norme_in_lavorazione/2_RTB/Glossario.pdf}{\ul{Glossario1.0}\setulcolor{blue}}\ped{(ultimo accesso: 17/11/2025)}
    \item \href{https://iso25000.com}{\ul{The ISO/IEC 25000 Series of Standards}\setulcolor{blue}}
    \item \href{https://www.iso.org/standard/63712.html}{\ul{Standard ISO/IEC 9126-1:2001}\setulcolor{blue}}
    \item \href{https://www.iso.org/standard/71952.html}{\ul{Standard ISO/IEC 145981-1:1999}\setulcolor{blue}}
\end{itemize}

\section{Metriche di qualità}

\subsection{Qualità di processo}

\subsubsection{Processi primari}

\subsubsubsection{Fornitura}

\begin{table}[H]
    \begin{adjustwidth}{-4cm}{-4cm}
    \centering
    \begin{spacing}{1.5}
    \begin{tabular}{|c|c|c|c|}
        \hline
        \textbf{Codice} & \textbf{Nome metrica} & \begin{tabular}[c]{@{}c@{}}  \textbf{Valore}\\\textbf{accettabile} \end{tabular} & \begin{tabular}[c]{@{}c@{}}  \textbf{Valore}\\\textbf{desiderabile} \end{tabular} \\
        \hline
        MPC-1 & EV (Earned Value) & $\geq$0 & PV \\
        \hline
        MPC-2 & PV (Planned Value) & $\geq$0 & - \\
        \hline
        MPC-3 & AC (Actual Cost) & $\geq$0 & $\leq$EV \\
        \hline
        MPC-4 & CPI (Cost Performance Index) & $\geq$0.5 & $\geq$1 \\
        \hline
        MPC-5 & SPI (Schedule Performance Index) & $\geq$0.5 & $\geq$1 \\
        \hline
        MPC-6 & TEAC (Time Estimate at Completion) & $\geq$0 & $\leq$scadenza \\
        \hline
        MPC-7 & ETC (Estimate to Complete) & $\geq$0 & $\leq$BAC\textsubscript{\textit{G}} - AC\textsubscript{\textit{G}} \\
        \hline
        MPC-8 & EAC (Estimate at Completion) & $\geq$0 & $\leq$BAC \\
        \hline
        MPC-9 & BV (Budget Variance) & - & $>$0\\
        \hline
        MPC-10 & SV (Schedule Variance) & - & $>$0 \\
        \hline
    \end{tabular}
    \end{spacing}
    \end{adjustwidth}
\end{table}

\subsubsubsection{Sviluppo}

\begin{table}[H]
    \begin{adjustwidth}{-4cm}{-4cm}
    \centering
    \begin{spacing}{1.5}
    \begin{tabular}{|c|c|c|c|}
        \hline
        \textbf{Codice} & \textbf{Nome metrica} & \begin{tabular}[c]{@{}c@{}}  \textbf{Valore}\\\textbf{accettabile} \end{tabular} & \begin{tabular}[c]{@{}c@{}}  \textbf{Valore}\\\textbf{desiderabile} \end{tabular} \\
        \hline
        MPC-11 & Deployment frequency & - & 1/gg \\
        \hline
        MPC-12 & Stabilità dei requisiti & $\geq$70\% & 100\% \\
        \hline
    \end{tabular}
    \end{spacing}
    \end{adjustwidth}
\end{table}

\subsubsection{Processi di supporto}

\subsubsubsection{Documentazione}

\begin{table}[H]
    \begin{adjustwidth}{-4cm}{-4cm}
    \centering
    \begin{spacing}{1.5}
    \begin{tabular}{|c|c|c|c|}
        \hline
        \textbf{Codice} & \textbf{Nome metrica} & \begin{tabular}[c]{@{}c@{}}  \textbf{Valore}\\\textbf{accettabile} \end{tabular} & \begin{tabular}[c]{@{}c@{}}  \textbf{Valore}\\\textbf{desiderabile} \end{tabular} \\
        \hline
        MPC-13 & Indice di Gulpease & $\geq$60 & $\geq$75\\
        \hline
        MPC-14 & Correttezza ortografica & 0 & 0 \\
        \hline
    \end{tabular}
    \end{spacing}
    \end{adjustwidth}
\end{table}

\subsubsubsection{Configurazione}

\begin{table}[H]
    \begin{adjustwidth}{-4cm}{-4cm}
    \centering
    \begin{spacing}{1.5}
    \begin{tabular}{|c|c|c|c|}
        \hline
        \textbf{Codice} & \textbf{Nome metrica} & \begin{tabular}[c]{@{}c@{}}  \textbf{Valore}\\\textbf{accettabile} \end{tabular} & \begin{tabular}[c]{@{}c@{}}  \textbf{Valore}\\\textbf{desiderabile} \end{tabular} \\
        \hline
        MPC-15 & Average build time & $\leq$15 min & $\leq$10 min \\
        \hline
    \end{tabular}
    \end{spacing}
    \end{adjustwidth}
\end{table}

\subsubsubsection{Verifica}

\begin{table}[H]
    \begin{adjustwidth}{-4cm}{-4cm}
    \centering
    \begin{spacing}{1.5}
    \begin{tabular}{|c|c|c|c|}
        \hline
        \textbf{Codice} & \textbf{Nome metrica} & \begin{tabular}[c]{@{}c@{}}  \textbf{Valore}\\\textbf{accettabile} \end{tabular} & \begin{tabular}[c]{@{}c@{}}  \textbf{Valore}\\\textbf{desiderabile} \end{tabular} \\
        \hline
        MPC-16 & Code review turnaround time & $\leq$72h & $\leq$24h\\
        \hline
        MPC-17 & Test success rate & 1 & 1\\
        \hline
    \end{tabular}
    \end{spacing}
    \end{adjustwidth}
\end{table}

\subsubsubsection{Risoluzione dei problemi}

\begin{table}[H]
    \begin{adjustwidth}{-4cm}{-4cm}
    \centering
    \begin{spacing}{1.5}
    \begin{tabular}{|c|c|c|c|}
        \hline
        \textbf{Codice} & \textbf{Nome metrica} & \begin{tabular}[c]{@{}c@{}}  \textbf{Valore}\\\textbf{accettabile} \end{tabular} & \begin{tabular}[c]{@{}c@{}}  \textbf{Valore}\\\textbf{desiderabile} \end{tabular} \\
        \hline
        MPC-18 & Rischi non previsti & $\geq$0 & 0\\
        \hline
    \end{tabular}
    \end{spacing}
    \end{adjustwidth}
\end{table}

\subsubsubsection{Gestione della Qualità}
\begin{table}[H]
    \begin{adjustwidth}{-4cm}{-4cm}
    \centering
    \begin{spacing}{1.5}
    \begin{tabular}{|c|c|c|c|}
        \hline
        \textbf{Codice} & \textbf{Nome metrica} & \begin{tabular}[c]{@{}c@{}}  \textbf{Valore}\\\textbf{accettabile} \end{tabular} & \begin{tabular}[c]{@{}c@{}}  \textbf{Valore}\\\textbf{desiderabile} \end{tabular} \\
        \hline
        MPD-1 & Metriche soddisfatte & $\geq$70\% & 100\% \\
        \hline
    \end{tabular}
    \end{spacing}
    \end{adjustwidth}
\end{table}


\subsection{Qualità di prodotto}
Le metriche definite in questa sezione riguardano principalmente caratteristiche di qualità "interne" del prodotto software. Raggiungere la qualità su queste caratteristiche abilita all'ottenimento di qualità in uso, o "esterna".
Suddividiamo le metriche secondo raggruppamenti logici qui di seguito elencati ed esplicitati:
\begin{itemize}
    \item \textbf{Funzionalità}: completezza, correttezza ed appropriatezza del prodotto
    \item \textbf{Affidabilità}: maturità, disponibilità, tolleranza ai guasti e riparabilità del prodotto
    \item \textbf{Usabilità}: apprendibilità, operabilità, UX e accessibilità del prodotto
    \item \textbf{Efficienza}: nel tempo, nelle altre risorse, nella capacità
    \item \textbf{Manutenibilità}: modularità, riusabilità, analizzabilità, modificabilità e verificabilità del prodotto
    \item \textbf{Portabilità}: adattabilità del prodotto a diversi ambienti
\end{itemize}


\subsubsection{Funzionalità}

\begin{table}[H]
    \begin{adjustwidth}{-4cm}{-4cm}
    \centering
    \begin{spacing}{1.5}
    \begin{tabular}{|c|c|c|c|}
        \hline
        \textbf{Codice} & \textbf{Nome metrica} & \begin{tabular}[c]{@{}c@{}}  \textbf{Valore}\\\textbf{accettabile} \end{tabular} & \begin{tabular}[c]{@{}c@{}}  \textbf{Valore}\\\textbf{desiderabile} \end{tabular} \\
        \hline
        MPD-2 & Requisiti obbligatori soddisfatti & $\geq$0\% & 100\% \\
        \hline
        MPD-3 & Requisiti opzionali soddisfatti & $\geq$0\% & 100\%\\
        \hline
        MPD-4 & Requisiti desiderabili soddisfatti & $\geq$0\% & 100\% \\
        \hline
    \end{tabular}
    \end{spacing}
    \end{adjustwidth}
\end{table}


\subsubsection{Affidabilità}

\begin{table}[H]
    \begin{adjustwidth}{-4cm}{-4cm}
    \centering
    \begin{spacing}{1.5}
    \begin{tabular}{|c|c|c|c|}
        \hline
        \textbf{Codice} & \textbf{Nome metrica} & \begin{tabular}[c]{@{}c@{}}  \textbf{Valore}\\\textbf{accettabile} \end{tabular} & \begin{tabular}[c]{@{}c@{}}  \textbf{Valore}\\\textbf{desiderabile} \end{tabular} \\
        \hline
        MPD-5 & Broken Links & 2 & 0\\
        \hline
        MPD-6 & Branch coverage & $\geq$80\% & $\geq$90\% \\
        \hline
        MPD-7 & Statement Coverage & $\geq$65\% & $\geq$80\% \\
        \hline
    \end{tabular}
    \end{spacing}
    \end{adjustwidth}
\end{table}

\subsubsection{Usabilità}

\begin{table}[H]
    \begin{adjustwidth}{-4cm}{-4cm}
    \centering
    \begin{spacing}{1.5}
    \begin{tabular}{|c|c|c|c|}
        \hline
        \textbf{Codice} & \textbf{Nome metrica} & \begin{tabular}[c]{@{}c@{}}  \textbf{Valore}\\\textbf{accettabile} \end{tabular} & \begin{tabular}[c]{@{}c@{}}  \textbf{Valore}\\\textbf{desiderabile} \end{tabular} \\
        \hline
        MPD-8 & Profondità di navigazione & $\geq$0 & $\leq$5 \\
        \hline
    \end{tabular}
    \end{spacing}
    \end{adjustwidth}
\end{table}


\subsubsection{Efficienza}

\begin{table}[H]
    \begin{adjustwidth}{-4cm}{-4cm}
    \centering
    \begin{spacing}{1.5}
    \begin{tabular}{|c|c|c|c|}
        \hline
        \textbf{Codice} & \textbf{Nome metrica} & \begin{tabular}[c]{@{}c@{}}  \textbf{Valore}\\\textbf{accettabile} \end{tabular} & \begin{tabular}[c]{@{}c@{}}  \textbf{Valore}\\\textbf{desiderabile} \end{tabular} \\
        \hline
        MPD-9 & Indexing Time & 2min & $\leq$30 s \\
        \hline
        MPD-10 & Search Time & $\leq$2 s & $\leq$1 s \\
        \hline
        MPD-11 & Average CPU usage & $\leq$30\% & $\leq$15\% \\
        \hline
        MPD-12 & Peak memory usage & $\leq$1 GB & $\leq$500 MB \\
        \hline
    \end{tabular}
    \end{spacing}
    \end{adjustwidth}
\end{table}



\subsubsection{Manutenibilità}

\begin{table}[H]
    \begin{adjustwidth}{-4cm}{-4cm}
    \centering
    \begin{spacing}{1.5}
    \begin{tabular}{|c|c|c|c|}
        \hline
        \textbf{Codice} & \textbf{Nome metrica} & \begin{tabular}[c]{@{}c@{}}  \textbf{Valore}\\\textbf{accettabile} \end{tabular} & \begin{tabular}[c]{@{}c@{}}  \textbf{Valore}\\\textbf{desiderabile} \end{tabular} \\
       \hline
        MPD-13 & Complessità ciclomatica & $\leq$15 & $\leq$10 \\
        \hline
        MPD-14 & Accoppiamento tra classi & $\leq$0.4 & $\leq$0.2 \\
        \hline
        MPD-15 & Code Smells & $\leq$15 & 0 \\
        \hline
    \end{tabular}
    \end{spacing}
    \end{adjustwidth}
\end{table}

\subsubsection{Portabilità}

\begin{table}[H]
    \begin{adjustwidth}{-4cm}{-4cm}
    \centering
    \begin{spacing}{1.5}
    \begin{tabular}{|c|c|c|c|}
        \hline
        \textbf{Codice} & \textbf{Nome metrica} & \begin{tabular}[c]{@{}c@{}}  \textbf{Valore}\\\textbf{accettabile} \end{tabular} & \begin{tabular}[c]{@{}c@{}}  \textbf{Valore}\\\textbf{desiderabile} \end{tabular} \\
        \hline
        MPD-16 & Browser supportati & Chrome, Firefox, Edge  & Safari, Arc, Brave \\
        \hline
        MPD-17 & Sistemi operativi supportati & Windows, macOS, Linux & Windows, macOS, Linux \\
        \hline
    \end{tabular}
    \end{spacing}
    \end{adjustwidth}
\end{table}

\section{Test di verifica}
minimo e desiderabile

\section{Cruscotto di valutazione e miglioramento}


\vfill
\begin{flushright}
    \textit{7-ZPUs}
\end{flushright}

\end{document}