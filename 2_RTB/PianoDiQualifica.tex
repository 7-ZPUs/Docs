\documentclass[a4paper,12pt]{article}

\usepackage[T1]{fontenc}
\usepackage[italian]{babel}
\usepackage[sfdefault]{atkinson}
\renewcommand*\familydefault{\sfdefault}
\usepackage{float}
\usepackage{microtype}
\usepackage{geometry}
\usepackage{setspace}
\usepackage{enumitem}
\usepackage{titlesec}
\usepackage{chngpage}
\usepackage{tocloft}
\usepackage{longtable}
\usepackage{ltablex}
\usepackage{array}
\usepackage{graphicx}
\usepackage{fancyhdr}
\usepackage[table]{xcolor}
\usepackage{color,soul}
\usepackage[most]{tcolorbox}
\usepackage{nameref}
\usepackage[colorlinks=true]{hyperref}

% Colori ZPUS - Verde, Nero, Bianco
\definecolor{zpusgreen}{RGB}{4, 138, 55}
\definecolor{zpusdarkgreen}{RGB}{0, 100, 0}
\definecolor{zpusblack}{RGB}{0, 0, 0}
\definecolor{zpuswhite}{RGB}{255, 255, 255}
\definecolor{zpuslightgray}{RGB}{245, 245, 245}

\hypersetup{
    linkcolor=black,
    urlcolor=blue
}

\definecolor{lightblack}{gray}{0.35}
\newcommand{\glossario}[1]{\textit{#1}\textsubscript{\textbf{\textit{\textcolor{lightblack}{G}}}}}
\newcommand{\ped}[1]{\textsubscript{#1}}

\pagestyle{fancy}
\setlength{\headwidth}{\textwidth}
\fancyhfoffset[L,R]{0pt}
\lhead{\rightmark}
\rhead{7-ZPUs}
\lfoot{Piano di Qualifica}
\rfoot{\thepage}
\cfoot{}
\renewcommand{\headrulewidth}{0.8pt}
\renewcommand{\footrulewidth}{0.8pt}

\renewcommand{\contentsname}{Indice}

\geometry{margin=2.5cm}
\setstretch{1.1}

\titleclass{\subsubsubsection}{straight}[\subsection]

\newcounter{subsubsubsection}[subsubsection]
\renewcommand\thesubsubsubsection{\thesubsubsection.\arabic{subsubsubsection}}
\renewcommand\theparagraph{\thesubsubsubsection.\arabic{paragraph}} % optional; useful if paragraphs are to be numbered

\titleformat{\subsubsubsection}
  {\normalfont\normalsize\bfseries}{\thesubsubsubsection}{1em}{}
\titlespacing*{\subsubsubsection}
{0pt}{3.25ex plus 1ex minus .2ex}{1.5ex plus .2ex}

\makeatletter
\renewcommand\paragraph{\@startsection{paragraph}{5}{\z@}%
  {3.25ex \@plus1ex \@minus.2ex}%
  {-1em}%
  {\normalfont\normalsize\bfseries}}
\renewcommand\subparagraph{\@startsection{subparagraph}{6}{\parindent}%
  {3.25ex \@plus1ex \@minus .2ex}%
  {-1em}%
  {\normalfont\normalsize\bfseries}}
\def\toclevel@subsubsubsection{4}
\def\toclevel@paragraph{5}
%\def\toclevel@paragraph{6}
\def\toclevel@subparagraph{6}
\def\l@subsubsubsection{\@dottedtocline{4}{7em}{4em}}
\def\l@paragraph{\@dottedtocline{5}{10em}{5em}}
\def\l@subparagraph{\@dottedtocline{6}{14em}{6em}}
\makeatother

\setcounter{secnumdepth}{4}
\setcounter{tocdepth}{4}

\titleformat{\section}{\LARGE\bfseries}{\thesection}{1em}{}
\titleformat{\subsection}{\Large\bfseries}{\thesubsection}{1em}{}
\titleformat{\subsubsection}{\large\bfseries}{\thesubsubsection}{1em}{}
\titleformat{\paragraph}{\large\bfseries}{\theparagraph}{1em}{}
\titleformat{\subparagraph}{\normalsize\bfseries}{\thesubparagraph}{1em}{}

\begin{document}

\begin{center}
    \includegraphics[width=9.5cm]{../assets/logo7ZPUs.jpg}\\
    \small\hspace{10cm} 7zpus.swe@gmail.com\\
    \vspace{0.5cm}
    \Large \textbf{Piano di Qualifica}\\
\end{center}

\vspace{0.3cm}
\hrule
\vspace{0.5cm}


\section*{Tabella di Versionamento}
\begin{table}[H]
    \begin{adjustwidth}{-4cm}{-4cm}
    \centering
    \begin{tabular}{|c|c|c|c|c|}
        \hline
        \textbf{Versione} & \textbf{Data} & \textbf{Autore}  & \textbf{Verificatore} & \textbf{Descrizione} \\
        \hline
         0.4.0 & 2025/02/09 & Rocco Matteo A. & Gingillino Aaron & \begin{tabular}[c]{@{}c@{}} Aggiunta sezione\\ test di sistema \end{tabular} \\
        \hline
        0.3.0 & 2025/02/08 & Vigolo Davide & Soligo Lorenzo & \begin{tabular}[c]{@{}c@{}} Aggiunta sezione cruscotto \\ di valutazione \end{tabular} \\
        \hline
        0.2.0 & 2025/02/02 & Vigolo Davide & Soligo Lorenzo & \begin{tabular}[c]{@{}c@{}} Revisione metriche\\ e aggiornamento soglie \end{tabular} \\
        \hline
        0.1.0 & 2025/12/14 & Rocco Matteo A. & Vigolo Davide & \begin{tabular}[c]{@{}c@{}} Creazione del documento\\ e stesura iniziale \end{tabular} \\
        \hline
    \end{tabular}
    \end{adjustwidth}
\end{table}

\tableofcontents

\newpage

\section{Introduzione}

\subsection{Scopo}
Lo scopo di questo documento è di fondamentale importanza. Permette di definire misure quantitative per misurare la qualità di processo e di prodotto. Assieme al cruscotto di valutazione permette di monitorare l'efficacia e l'efficienza dei processi di ciclo di vita istanziati nel progetto.
Garantire una sufficiente qualità di processo e di prodotto, è condizione necessaria alla qualità di prodotto in uso, che è di interesse primario per la committente. Il Piano di Qualifica si compone di tre elementi:
\begin{itemize}
    \item Piano della Qualità: Definizione di obiettivi quantitativi di qualità, metriche e strategie per raggiungerla
    \item Controllo di Qualità: insieme di attività e tecniche per valutare che il piano stabilito sia efficace.
    \item Miglioramento continuo: stabilire eventuali azioni correttive alla luce dei risultati del controllo, adattando processi, obiettivi e vincoli.
\end{itemize}

\subsection{Glossario}
Ogni termine tecnico o con particolare significato nell'ambito dell'\glossario{Ingegneria del Software} utilizzato nella documentazione di progetto viene definito nell'apposito documento \href{https://cdn.jsdelivr.net/gh/7-zpus/Docs@norme_in_lavorazione/2_RTB/Glossario.pdf}{\ul{Glossario1.0}\setulcolor{blue}}\ped{(ultimo accesso: 17/11/2025)}.

\subsection{Riferimenti}

\subsubsection{Riferimenti Normativi}
\begin{itemize}
    \item \href{https://cdn.jsdelivr.net/gh/7-zpus/Docs@norme_in_lavorazione/2_RTB/NormeDiProgetto.pdf}{\ul{NormeDiProgetto1.0}\setulcolor{blue}}\ped{(ultimo accesso: 3/12/2025)}
    \item \href{https://www.math.unipd.it/~tullio/IS-1/2025/Progetto/C3.pdf}{\ul{Capitolato C3: DIPReader}\setulcolor{blue}} \ped{(ultimo accesso: 01/12/2025)}
    \item \href{https://www.math.unipd.it/~tullio/IS-1/2025/Dispense/PD1.pdf}{\ul{Regolamento di Progetto Didattico a.a. 2025/2026}\setulcolor{blue}} \ped{(ultimo accesso: 17/11/2025)}
\end{itemize}
\subsubsection{Riferimenti Informativi}
\begin{itemize}
    \item \href{https://cdn.jsdelivr.net/gh/7-zpus/Docs@norme_in_lavorazione/2_RTB/Glossario.pdf}{\ul{Glossario1.0}\setulcolor{blue}}\ped{(ultimo accesso: 17/11/2025)}
    \item \href{https://iso25000.com}{\ul{The ISO/IEC 25000 Series of Standards}\setulcolor{blue}}
    \item \href{https://www.iso.org/standard/63712.html}{\ul{Standard ISO/IEC 9126-1:2001}\setulcolor{blue}}
    \item \href{https://www.iso.org/standard/71952.html}{\ul{Standard ISO/IEC 145981-1:1999}\setulcolor{blue}}
\end{itemize}

\section{Metriche di qualità}

\subsection{Qualità di processo}

\subsubsection{Processi primari}

\subsubsubsection{Fornitura}

\begin{table}[H]
    \begin{adjustwidth}{-4cm}{-4cm}
    \centering
    \begin{spacing}{1.5}
    \begin{tabular}{|c|c|c|c|}
        \hline
        \textbf{Codice} & \textbf{Nome metrica} & \begin{tabular}[c]{@{}c@{}}  \textbf{Valore}\\\textbf{accettabile} \end{tabular} & \begin{tabular}[c]{@{}c@{}}  \textbf{Valore}\\\textbf{desiderabile} \end{tabular} \\
        \hline
        MPC-1 & EV (Earned Value) & $\geq$0 & PV \\
        \hline
        MPC-2 & PV (Planned Value) & $\geq$0 & - \\
        \hline
        MPC-3 & AC (Actual Cost) & $\geq$0 & $\leq$EV \\
        \hline
        MPC-4 & CPI (Cost Performance Index) & $\geq$0.5 & $\geq$1 \\
        \hline
        MPC-5 & SPI (Schedule Performance Index) & $\geq$0.5 & $\geq$1 \\
        \hline
        MPC-6 & ETC (Estimate to Complete) & $\geq$0 & $\leq$BAC\textsubscript{\textit{G}} - AC\textsubscript{\textit{G}} \\
        \hline
        MPC-7 & EAC (Estimate at Completion) & $\geq$0 & $\leq$BAC \\
        \hline
    \end{tabular}
    \end{spacing}
    \end{adjustwidth}
\end{table}

\subsubsubsection{Sviluppo}

\begin{table}[H]
    \begin{adjustwidth}{-4cm}{-4cm}
    \centering
    \begin{spacing}{1.5}
    \begin{tabular}{|c|c|c|c|}
        \hline
        \textbf{Codice} & \textbf{Nome metrica} & \begin{tabular}[c]{@{}c@{}}  \textbf{Valore}\\\textbf{accettabile} \end{tabular} & \begin{tabular}[c]{@{}c@{}}  \textbf{Valore}\\\textbf{desiderabile} \end{tabular} \\
        \hline
        MPC-8 & Deployment frequency & - & 1/gg \\
        \hline
        MPC-9 & Requirements Stability Index & $\geq$70\% & 100\% \\
        \hline
    \end{tabular}
    \end{spacing}
    \end{adjustwidth}
\end{table}

\subsubsubsection{Integrazione}

\begin{table}[H]
    \begin{adjustwidth}{-4cm}{-4cm}
    \centering
    \begin{spacing}{1.5}
    \begin{tabular}{|c|c|c|c|}
        \hline
        \textbf{Codice} & \textbf{Nome metrica} & \begin{tabular}[c]{@{}c@{}}  \textbf{Valore}\\\textbf{accettabile} \end{tabular} & \begin{tabular}[c]{@{}c@{}}  \textbf{Valore}\\\textbf{desiderabile} \end{tabular} \\
        \hline
        MPC-12 & Average build time & $\leq$15 min & $\leq$10 min \\
        \hline
    \end{tabular}
    \end{spacing}
    \end{adjustwidth}
\end{table}

\subsubsection{Processi di supporto}

\subsubsubsection{Documentazione}

\begin{table}[H]
    \begin{adjustwidth}{-4cm}{-4cm}
    \centering
    \begin{spacing}{1.5}
    \begin{tabular}{|c|c|c|c|}
        \hline
        \textbf{Codice} & \textbf{Nome metrica} & \begin{tabular}[c]{@{}c@{}}  \textbf{Valore}\\\textbf{accettabile} \end{tabular} & \begin{tabular}[c]{@{}c@{}}  \textbf{Valore}\\\textbf{desiderabile} \end{tabular} \\
        \hline
        MPC-10 & Indice di Gulpease & $\geq$60 & $\geq$75\\
        \hline
        MPC-11 & Correttezza ortografica & 0 & 0 \\
        \hline
    \end{tabular}
    \end{spacing}
    \end{adjustwidth}
\end{table}

\subsubsubsection{Verifica}

\begin{table}[H]
    \begin{adjustwidth}{-4cm}{-4cm}
    \centering
    \begin{spacing}{1.5}
    \begin{tabular}{|c|c|c|c|}
        \hline
        \textbf{Codice} & \textbf{Nome metrica} & \begin{tabular}[c]{@{}c@{}}  \textbf{Valore}\\\textbf{accettabile} \end{tabular} & \begin{tabular}[c]{@{}c@{}}  \textbf{Valore}\\\textbf{desiderabile} \end{tabular} \\
        \hline
        MPC-13 & Code review turnaround time & $\leq$72h & $\leq$24h\\
        \hline
        MPC-14 & Test success rate & 100\% & 100\%\\
        \hline
    \end{tabular}
    \end{spacing}
    \end{adjustwidth}
\end{table}

\subsubsection{Processi organizzativi}

\subsubsubsection{Gestione dei rischi}

\begin{table}[H]
    \begin{adjustwidth}{-4cm}{-4cm}
    \centering
    \begin{spacing}{1.5}
    \begin{tabular}{|c|c|c|c|}
        \hline
        \textbf{Codice} & \textbf{Nome metrica} & \begin{tabular}[c]{@{}c@{}}  \textbf{Valore}\\\textbf{accettabile} \end{tabular} & \begin{tabular}[c]{@{}c@{}}  \textbf{Valore}\\\textbf{desiderabile} \end{tabular} \\
        \hline
        MPC-15 & Rischi non previsti & $\geq$0 & 0\\
        \hline
    \end{tabular}
    \end{spacing}
    \end{adjustwidth}
\end{table}

\subsubsubsection{Gestione della Qualità}
\begin{table}[H]
    \begin{adjustwidth}{-4cm}{-4cm}
    \centering
    \begin{spacing}{1.5}
    \begin{tabular}{|c|c|c|c|}
        \hline
        \textbf{Codice} & \textbf{Nome metrica} & \begin{tabular}[c]{@{}c@{}}  \textbf{Valore}\\\textbf{accettabile} \end{tabular} & \begin{tabular}[c]{@{}c@{}}  \textbf{Valore}\\\textbf{desiderabile} \end{tabular} \\
        \hline
        MPC-16 & Metriche soddisfatte & $\geq$70\% & 100\% \\
        \hline
    \end{tabular}
    \end{spacing}
    \end{adjustwidth}
\end{table}


\subsection{Qualità di prodotto}
Le metriche definite in questa sezione riguardano principalmente caratteristiche di qualità "interne" del prodotto software. Raggiungere la qualità su queste caratteristiche abilita all'ottenimento di qualità in uso, o "esterna".
Suddividiamo le metriche secondo raggruppamenti logici qui di seguito elencati ed esplicitati:
\begin{itemize}
    \item \textbf{Funzionalità}: completezza, correttezza ed appropriatezza del prodotto
    \item \textbf{Affidabilità}: maturità, disponibilità, tolleranza ai guasti e riparabilità del prodotto
    \item \textbf{Usabilità}: apprendibilità, operabilità, UX e accessibilità del prodotto
    \item \textbf{Efficienza}: nel tempo, nelle altre risorse, nella capacità
    \item \textbf{Manutenibilità}: modularità, riusabilità, analizzabilità, modificabilità e verificabilità del prodotto
    \item \textbf{Portabilità}: adattabilità del prodotto a diversi ambienti
\end{itemize}


\subsubsection{Funzionalità}

\begin{table}[H]
    \begin{adjustwidth}{-4cm}{-4cm}
    \centering
    \begin{spacing}{1.5}
    \begin{tabular}{|c|c|c|c|}
        \hline
        \textbf{Codice} & \textbf{Nome metrica} & \begin{tabular}[c]{@{}c@{}}  \textbf{Valore}\\\textbf{accettabile} \end{tabular} & \begin{tabular}[c]{@{}c@{}}  \textbf{Valore}\\\textbf{desiderabile} \end{tabular} \\
        \hline
        MPD-2 & Requisiti obbligatori soddisfatti & $\geq$0\% & 100\% \\
        \hline
        MPD-3 & Requisiti opzionali soddisfatti & $\geq$0\% & 100\%\\
        \hline
        MPD-4 & Requisiti desiderabili soddisfatti & $\geq$0\% & 100\% \\
        \hline
    \end{tabular}
    \end{spacing}
    \end{adjustwidth}
\end{table}


\subsubsection{Affidabilità}

\begin{table}[H]
    \begin{adjustwidth}{-4cm}{-4cm}
    \centering
    \begin{spacing}{1.5}
    \begin{tabular}{|c|c|c|c|}
        \hline
        \textbf{Codice} & \textbf{Nome metrica} & \begin{tabular}[c]{@{}c@{}}  \textbf{Valore}\\\textbf{accettabile} \end{tabular} & \begin{tabular}[c]{@{}c@{}}  \textbf{Valore}\\\textbf{desiderabile} \end{tabular} \\
        \hline
        MPD-5 & Broken Links & 2 & 0\\
        \hline
        MPD-6 & Branch coverage & $\geq$80\% & $\geq$90\% \\
        \hline
        MPD-7 & Statement Coverage & $\geq$65\% & $\geq$80\% \\
        \hline
    \end{tabular}
    \end{spacing}
    \end{adjustwidth}
\end{table}

\subsubsection{Usabilità}

\begin{table}[H]
    \begin{adjustwidth}{-4cm}{-4cm}
    \centering
    \begin{spacing}{1.5}
    \begin{tabular}{|c|c|c|c|}
        \hline
        \textbf{Codice} & \textbf{Nome metrica} & \begin{tabular}[c]{@{}c@{}}  \textbf{Valore}\\\textbf{accettabile} \end{tabular} & \begin{tabular}[c]{@{}c@{}}  \textbf{Valore}\\\textbf{desiderabile} \end{tabular} \\
        \hline
        MPD-8 & Profondità di navigazione & $\geq$0 & $\leq$5 \\
        \hline
    \end{tabular}
    \end{spacing}
    \end{adjustwidth}
\end{table}


\subsubsection{Efficienza}

\begin{table}[H]
    \begin{adjustwidth}{-4cm}{-4cm}
    \centering
    \begin{spacing}{1.5}
    \begin{tabular}{|c|c|c|c|}
        \hline
        \textbf{Codice} & \textbf{Nome metrica} & \begin{tabular}[c]{@{}c@{}}  \textbf{Valore}\\\textbf{accettabile} \end{tabular} & \begin{tabular}[c]{@{}c@{}}  \textbf{Valore}\\\textbf{desiderabile} \end{tabular} \\
        \hline
        MPD-9 & Indexing Time & 2min & $\leq$30 s \\
        \hline
        MPD-10 & Search Time & $\leq$2 s & $\leq$1 s \\
        \hline
        MPD-11 & Average CPU usage & $\leq$30\% & $\leq$15\% \\
        \hline
        MPD-12 & Peak memory usage & $\leq$1 GB & $\leq$500 MB \\
        \hline
    \end{tabular}
    \end{spacing}
    \end{adjustwidth}
\end{table}



\subsubsection{Manutenibilità}

\begin{table}[H]
    \begin{adjustwidth}{-4cm}{-4cm}
    \centering
    \begin{spacing}{1.5}
    \begin{tabular}{|c|c|c|c|}
        \hline
        \textbf{Codice} & \textbf{Nome metrica} & \begin{tabular}[c]{@{}c@{}}  \textbf{Valore}\\\textbf{accettabile} \end{tabular} & \begin{tabular}[c]{@{}c@{}}  \textbf{Valore}\\\textbf{desiderabile} \end{tabular} \\
       \hline
        MPD-13 & Complessità ciclomatica & $\leq$15 & $\leq$10 \\
        \hline
        MPD-14 & Accoppiamento tra classi & $\leq$0.4 & $\leq$0.2 \\
        \hline
        MPD-15 & Code Smells & $\leq$15 & 0 \\
        \hline
    \end{tabular}
    \end{spacing}
    \end{adjustwidth}
\end{table}

\subsubsection{Portabilità}

\begin{table}[H]
    \begin{adjustwidth}{-4cm}{-4cm}
    \centering
    \begin{spacing}{1.5}
    \begin{tabular}{|c|c|c|c|}
        \hline
        \textbf{Codice} & \textbf{Nome metrica} & \begin{tabular}[c]{@{}c@{}}  \textbf{Valore}\\\textbf{accettabile} \end{tabular} & \begin{tabular}[c]{@{}c@{}}  \textbf{Valore}\\\textbf{desiderabile} \end{tabular} \\
        \hline
        MPD-16 & Sistemi operativi supportati & 3 & $\geq$3 \\
        \hline
    \end{tabular}
    \end{spacing}
    \end{adjustwidth}
\end{table}

\section{Test di verifica}

\subsection{Test di sistema}

\setlength{\LTleft}{-0.4cm}
\renewcommand{\arraystretch}{1.2}
\rowcolors{2}{gray!15}{white}
\begin{longtable}{|p{1.5cm}|p{8.5cm}|p{2.2cm}|c|}
\hline
\rowcolor{zpusgreen!30}
\textbf{Codice} & \textbf{Descrizione} & \textbf{Fonti} & \textbf{Implementato} \\
\hline
\endfirsthead
\rowcolor{zpusgreen!30}
\textbf{Codice} & \textbf{Descrizione} & \textbf{Fonti} & \textbf{Implementato} \\
\hline
\endhead
TS-1 & Verificare che l'utente possa visualizzare l'elenco di classi documentali nel DIP & R-1-F-Ob & NO \\
\hline
TS-2 & Verificare che in caso non vi siano classi documentali nel DIP, l'utente possa visualizzare un messaggio di errore & R-2-F-Ob & NO \\
\hline
TS-3 & Verificare che quando viene selezionata una classe documentale, l'utente possa visualizzare ciascuna classe documentale all'interno dell'elenco & R-3-F-Ob & NO \\
\hline
TS-4 & Verificare che l'utente possa visualizzare il nome della classe documentale & R-4-F-Ob & NO \\
\hline
TS-5 & Verificare che l'utente possa visualizzare lo stato di verifica della classe documentale & R-5-F-Ob & NO \\
\hline
TS-6 & Verificare che l'utente possa visualizzare la marcatura temporale della classe documentale & R-6-F-Ob & NO \\
\hline
TS-7 & Verificare che l'utente possa visualizzare l'elenco dei processi associati alla classe documentale & R-7-F-Ob & NO \\
\hline
TS-8 & Verificare che in caso non vi siano processi associati alla classe documentale, l'utente possa visualizzare un messaggio di errore & R-8-F-Ob & NO \\
\hline
TS-9 & Verificare che quando viene selezionato un processo, l'utente possa visualizzare ciascun processo all'interno dell'elenco & R-9-F-Ob & NO \\
\hline
TS-10 & Verificare che l'utente possa visualizzare il nome del processo & R-10-F-Ob & NO \\
\hline
TS-11 & Verificare che l'utente possa visualizzare l'elenco di documenti associati a un processo & R-11-F-Ob & NO \\
\hline
TS-12 & Verificare che in caso non vi siano documenti associati al processo, l'utente possa visualizzare un messaggio di errore & R-12-F-Ob & NO \\
\hline
TS-13 & Verificare che quando viene selezionato un documento, l'utente possa visualizzare ciascun documento all'interno dell'elenco & R-13-F-Ob & NO \\
\hline
TS-14 & Verificare che l'utente possa visualizzare il nome del documento & R-14-F-Ob & NO \\
\hline
TS-15 & Verificare che l'utente possa visualizzare lo stato di verifica del documento & R-15-F-Ob & NO \\
\hline
TS-16 & Verificare che l'utente possa visualizzare la marcatura temporale del documento & R-16-F-Ob & NO \\
\hline
TS-17 & Verificare che l'utente possa selezionare una classe documentale & R-17-F-Ob & NO \\
\hline
TS-18 & Verificare che l'utente possa selezionare un processo & R-18-F-Ob & NO \\
\hline
TS-19 & Verificare che l'utente possa visualizzare l'anteprima di un documento selezionato & R-19-F-Ob & NO \\
\hline
TS-20 & Verificare che se il documento selezionato non è visualizzabile in anteprima, l'utente possa visualizzare un messaggio di errore & R-20-F-Ob & NO \\
\hline
TS-21 & Verificare che l'utente possa ricercare un documento, un processo, o una classe documentale & R-21-F-Ob & NO \\
\hline
TS-22 & Verificare che l'utente possa cercare una classe documentale esclusivamente per nome & R-22-F-Ob & NO \\
\hline
TS-23 & Verificare che l'utente possa cercare un processo esclusivamente per uuid & R-23-F-Ob & NO \\
\hline
TS-24 & Verificare che l'utente possa effettuare una ricerca semantica basata sui metadati dei documenti presenti nel DIP & R-24-F-Op & NO \\
\hline
TS-25 & Verificare che il sistema renda disponibile un comando o opzione per avviare l'indicizzazione semantica dei documenti nel DIP per la ricerca & R-25-F-Op & NO \\
\hline
TS-26 & Verificare che quando viene selezionata l'indicizzazione semantica il sistema indicizzi i documenti presenti nel DIP & R-26-F-Op & NO \\
\hline
TS-27 & Verificare che se il sistema non riesce ad indicizzare i documenti presenti nel DIP, l'utente possa visualizzare un messaggio di errore & R-27-F-Op & NO \\
\hline
TS-28 & Verificare che l'utente possa visualizzare lo stato dell'indicizzazione semantica & R-28-F-Op & NO \\
\hline
TS-29 & Verificare che il sistema renda disponibile un campo di ricerca & R-29-F-Ob & NO \\
\hline
TS-30 & Verificare che il sistema comunichi all'utente quando inserisce un valore non valido nel campo di ricerca & R-30-F-Ob & NO \\
\hline
TS-31 & Verificare che il sistema renda disponibile l'opzione di ricerca per documenti & R-31-F-Ob & NO \\
\hline
TS-32 & Verificare che il sistema renda disponibile l'opzione di ricerca per processi & R-32-F-Ob & NO \\
\hline
TS-33 & Verificare che il sistema renda disponibile l'opzione di ricerca per classi documentali & R-33-F-Ob & NO \\
\hline
TS-34 & Verificare che il sistema renda disponibile la ricerca avanzata con filtri & R-34-F-Ob & NO \\
\hline
TS-35 & Verificare che il sistema renda disponibile la sezione di filtri di ricerca comuni & R-35-F-Ob & NO \\
\hline
TS-36 & Verificare che il sistema renda disponibile la sezione di filtri di ricerca specifici per tipo documentale & R-36-F-Ob & NO \\
\hline
TS-37 & Verificare che il sistema renda disponibile la sezione di filtri di ricerca specifici per metadati custom & R-37-F-Ob & NO \\
\hline
TS-38 & Verificare che il sistema permetta di applicare più filtri contemporaneamente & R-38-F-Ob & NO \\
\hline
TS-39 & Verificare che il sistema permetta di rimuovere i filtri applicati & R-39-F-Ob & NO \\
\hline
TS-40 & Verificare che il sistema permetta di rimuovere tutti i filtri applicati & R-40-F-Ob & NO \\
\hline
TS-41 & Verificare che il sistema permetta di selezionare filtri per tipo documentale per un singolo tipo di documento & R-41-F-Ob & NO \\
\hline
TS-42 & Verificare che il sistema racchiuda i filtri di ricerca comuni in una sezione espandibile dedicata & R-42-F-Ob & NO \\
\hline
TS-43 & Verificare che il sistema racchiuda i filtri di ricerca specifici per tipo documentale in una sezione espandibile dedicata & R-43-F-Ob & NO \\
\hline
TS-44 & Verificare che il sistema racchiuda i filtri di ricerca specifici per metadati custom in una sezione espandibile dedicata & R-44-F-Ob & NO \\
\hline
TS-45 & Verificare che il sistema permetta di visualizzare i filtri comuni applicabili & R-45-F-Ob & NO \\
\hline
TS-46 & Verificare che il sistema renda disponibile una sezione per il filtro comune "Chiave Descrittiva" & R-46-F-Ob & NO \\
\hline
TS-47 & Verificare che l'utente possa inserire il valore per i campi "Chiave Descrittiva", "Oggetto" e "Parole chiave" & R-47-F-Ob & NO \\
\hline
TS-48 & Verificare che il sistema renda disponibile una sezione per il filtro comune "Classificazione" & R-48-F-Ob & NO \\
\hline
TS-49 & Verificare che l'utente possa inserire il valore per i campi "Indice di classificazione", "Descrizione" e "Piano di fascicolo" & R-49-F-Ob & NO \\
\hline
TS-50 & Verificare che il sistema renda disponibile una sezione per il filtro comune "Tempo di conservazione" & R-50-F-Ob & NO \\
\hline
TS-51 & Verificare che l'utente possa inserire il valore per il filtro "Tempo di conservazione" & R-51-F-Ob & NO \\
\hline
TS-52 & Verificare che l'utente possa selezionare l'opzione "Perenne" per il filtro "Tempo di conservazione" & R-52-F-Ob & NO \\
\hline
TS-53 & Verificare che il sistema renda disponibile una sezione per il filtro comune "Note" & R-53-F-Ob & NO \\
\hline
TS-54 & Verificare che l'utente possa inserire il valore per il filtro "Note" & R-54-F-Ob & NO \\
\hline
TS-55 & Verificare che il sistema renda disponibile una sezione per il filtro comune "Tipo di documento" & R-55-F-Ob & NO \\
\hline
TS-56 & Verificare che l'utente possa selezionare il valore per il filtro "Tipo di documento" tra "Documento informatico", "Documento amministrativo informatico", "Aggregazione documentale" & R-56-F-Ob & NO \\
\hline
TS-57 & Verificare che il sistema renda disponibile una sezione per il filtro comune "Soggetto" & R-57-F-Ob & NO \\
\hline
TS-58 & Verificare che l'utente possa inserire il valore per il filtro "Soggetto" & R-58-F-Ob & NO \\
\hline
TS-59 & Verificare che il sistema renda disponibile una sezione per il filtro del Ruolo del Soggetto & R-59-F-Ob & NO \\
\hline
TS-60 & Verificare che all'interno della sezione del filtro del Ruolo del Soggetto, l'utente possa selezionare il ruolo del soggetto tra quelli disponibili & R-60-F-Ob & NO \\
\hline
TS-61 & Verificare che il sistema renda disponibile una sezione per il filtro del Tipo di Soggetto & R-61-F-Ob & NO \\
\hline
TS-62 & Verificare che all'interno della sezione del filtro del Tipo di Soggetto, l'utente possa selezionare il tipo di soggetto tra quelli disponibili & R-62-F-Ob & NO \\
\hline
TS-63 & Verificare che il sistema renda disponibile una sezione per il filtro "Dettagli" del Soggetto & R-63-F-Ob & NO \\
\hline
TS-64 & Verificare che se il soggetto selezionato è di tipo "PAI", all'interno della sezione del filtro "Dettagli" del Soggetto, l'utente possa inserire il valore per i campi "Denominazione Amministrazione/Codice IPA", "Denominazione Amministrazione AOO/Codice IPA AOO", "Denominazione Amministrazione UOR/Codice IPA UOR" e "Indirizzi digitali di riferimento" & R-64-F-Ob & NO \\
\hline
TS-65 & Verificare che se il soggetto selezionato è di tipo "PAE", all'interno della sezione del filtro "Dettagli" del Soggetto, l'utente possa inserire il valore per i campi "Denominazione Amministrazione", "Denominazione Ufficio" e "Indirizzi digitali di riferimento" & R-65-F-Ob & NO \\
\hline
TS-66 & Verificare che se il soggetto selezionato è di tipo "AS", all'interno della sezione del filtro "Dettagli" del Soggetto, l'utente possa inserire il valore per i campi "Cognome", "Nome", "Codice Fiscale", "Denominazione Amministrazione AOO/Codice IPA AOO", "Denominazione Amministrazione UOR/Codice IPA UOR" e "Indirizzi digitali di riferimento" & R-66-F-Ob & NO \\
\hline
TS-67 & Verificare che se il soggetto selezionato è di tipo "PG", all'interno della sezione del filtro "Dettagli" del Soggetto, l'utente possa inserire il valore per i campi "Denominazione Organizzazione", "Codice Fiscale/Partita IVA", "Denominazione Ufficio" e "Indirizzi digitali di riferimento" & R-67-F-Ob & NO \\
\hline
TS-68 & Verificare che se il soggetto selezionato è di tipo "PF", all'interno della sezione del filtro "Dettagli" del Soggetto, l'utente possa inserire il valore per i campi "Cognome", "Nome" e "Indirizzi digitali di riferimento" & R-68-F-Ob & NO \\
\hline
TS-69 & Verificare che se il soggetto selezionato è di tipo "RUP", all'interno della sezione del filtro "Dettagli" del Soggetto, l'utente possa inserire il valore per i campi "Cognome", "Nome", "Codice Fiscale" "Denominazione Amministrazione/Codice IPA", "Denominazione Amministrazione AOO/Codice IPA AOO", "Denominazione Amministrazione UOR/Codice IPA UOR" e "Indirizzi digitali di riferimento" & R-69-F-Ob & NO \\
\hline
TS-70 & Verificare che se il soggetto selezionato è di tipo "SW", all'interno della sezione del filtro "Dettagli" del Soggetto, l'utente possa inserire il valore per il campo "Denominazione Sistema" & R-70-F-Ob & NO \\
\hline
TS-71 & Verificare che all'interno della sezione di filtri per tipo documentale, il sistema permetta di selezionare i filtri specifici per il tipo "Documento Informatico e Amministrativo Informatico" & R-71-F-Ob & NO \\
\hline
TS-72 & Verificare che per il Documento Informatico e Amministrativo Informatico, il sistema mostri la lista di filtri specifici & R-72-F-Ob & NO \\
\hline
TS-73 & Verificare che per il Documento Informatico e Amministrativo Informatico, il sistema renda disponibile una sezione per il filtro "Dati di Registrazione" & R-73-F-Ob & NO \\
\hline
TS-74 & Verificare che per il Documento Informatico e Amministrativo Informatico, all'interno della sezione del filtro "Dati di Registrazione", l'utente possa inserire il valore per i campi "Tipologia di Flusso", "Tipo di Registro", "Data/Ora di Registrazione", "Numero Documento" e "Codice Registro" & R-74-F-Ob & NO \\
\hline
TS-75 & Verificare che per il Documento Informatico e Amministrativo Informatico, l'utente possa inserire il valore per il filtro "Tipologia Documentale" & R-75-F-Ob & NO \\
\hline
TS-76 & Verificare che per il Documento Informatico e Amministrativo Informatico, l'utente possa inserire il valore per il filtro "Modalità di Formazione" tra:
\begin{itemize}
    \item Creazione tramite l'utilizzo di strumenti software che assicurino la produzione di documenti nei formati previsti nell'Allegato 2 delle Linee Guida;
    \item Acquisizione di un documento informatico per via telematica o su supporto informatico, acquisizione della copia per immagine su supporto informatico di un documento analogico, acquisizione della copia informatica di un documento analogico;
    \item Memorizzazione su supporto informatico in formato digitale delle informazioni risultanti da transazioni o processi informatici o dalla presentazione telematica di dati attraverso moduli o formulari resi disponibili all'utente;
    \item Generazione o raggruppamento anche in via automatica di un insieme di dati o registrazioni, provenienti da una o più banche dati, anche appartenenti a più soggetti interoperanti, secondo una struttura logica predeterminata e memorizzata in forma statica;
\end{itemize}  & R-76-F-Ob & NO \\
\hline
TS-77 & Verificare che per il Documento Informatico e Amministrativo Informatico, l'utente possa inserire il valore per il filtro "Campo Riservato" tra "Vero" o "Falso" & R-77-F-Ob & NO \\
\hline
TS-78 & Verificare che per il Documento Informatico e Amministrativo Informatico, il sistema renda disponibile una sezione per il filtro "Identificativo del Formato" & R-78-F-Ob & NO \\
\hline
TS-79 & Verificare che per il Documento Informatico e Amministrativo Informatico, all'interno della sezione del filtro "Identificativo del Formato", l'utente possa inserire il valore per i campi "Formato" e "Prodotto Software" & R-79-F-Ob & NO \\
\hline
TS-80 & Verificare che per il Documento Informatico e Amministrativo Informatico, il sistema renda disponibile una sezione per il filtro "Dati di Verifica" & R-80-F-Ob & NO \\
\hline
TS-81 & Verificare che per il Documento Informatico e Amministrativo Informatico, all'interno della sezione del filtro "Dati di Verifica", l'utente possa inserire il valore booleano per i campi "Firmato Digitalmente", "Sigillato Elettronicamente", "Marcatura Temporale" e "conformità copie immagine su supporto informatico" & R-81-F-Ob & NO \\
\hline
TS-82 & Verificare che per il Documento Informatico e Amministrativo Informatico, l'utente possa inserire il valore per il filtro "Nome del Documento" & R-82-F-Ob & NO \\
\hline
TS-83 & Verificare che per il Documento Informatico e Amministrativo Informatico, l'utente possa inserire il valore per il filtro "Versione del Documento" & R-83-F-Ob & NO \\
\hline
TS-84 & Verificare che per il Documento Informatico e Amministrativo Informatico, l'utente possa inserire il valore per il filtro "Identificativo del Documento Primario" & R-84-F-Ob & NO \\
\hline
TS-85 & Verificare che per il Documento Informatico e Amministrativo Informatico, il sistema renda disponibile una sezione per il filtro "Tracciatura Modifiche di Documento" & R-85-F-Ob & NO \\
\hline
TS-86 & Verificare che per il Documento Informatico e Amministrativo Informatico, all'interno della sezione del filtro "Tracciatura Modifiche di Documento", l'utente possa inserire il valore per i campi "Tipo di Modifica", "Soggetto Autore della Modifica", "Data/Ora della Modifica" e "IdDoc versione precedente" & R-86-F-Ob & NO \\
\hline
TS-87 & Verificare che all'interno della sezione di filtri per tipo documentale, il sistema permetta di selezionare i filtri specifici per il tipo "Aggregazione Documentale Informatica" & R-87-F-Ob & NO \\
\hline
TS-88 & Verificare che per l'Aggregazione Documentale Informatica, il sistema mostri la lista di filtri specifici & R-88-F-Ob & NO \\
\hline
TS-89 & Verificare che per l'Aggregazione Documentale Informatica, il sistema renda disponibile una sezione per il filtro "Tipo di Aggregazione" & R-89-F-Ob & NO \\
\hline
TS-90 & Verificare che per l'Aggregazione Documentale Informatica, all'interno della sezione del filtro "Tipo di Aggregazione", l'utente possa inserire il valore per il campo "Tipo di Aggregazione" tra "Fascicolo", "Serie Documentale" o "Serie di Fascicoli" & R-90-F-Ob & NO \\
\hline
TS-91 & Verificare che per l'Aggregazione Documentale Informatica, l'utente possa inserire il valore per il filtro "Identificativo dell'Aggregazione Documentale" & R-91-F-Ob & NO \\
\hline
TS-92 & Verificare che per l'Aggregazione Documentale Informatica, l'utente possa inserire il valore per il filtro "Tipologia di Fascicolo" tra "Affare", "Attività", "Persona Fisica", "Persona Giuridica" e "Procedimento Amministrativo" & R-92-F-Ob & NO \\
\hline
TS-93 & Verificare che per l'Aggregazione Documentale Informatica, l'utente possa inserire il valore per il filtro "Id Aggregazione Primario" & R-93-F-Ob & NO \\
\hline
TS-94 & Verificare che per l'Aggregazione Documentale Informatica, l'utente possa inserire il valore per il filtro "Data Apertura" & R-94-F-Ob & NO \\
\hline
TS-95 & Verificare che per l'Aggregazione Documentale Informatica, l'utente possa inserire il valore per il filtro "Data Chiusura" & R-95-F-Ob & NO \\
\hline
TS-96 & Verificare che per l'Aggregazione Documentale Informatica, il sistema renda disponibile una sezione per il filtro "Procedimento Amministrativo" & R-96-F-Ob & NO \\
\hline
TS-97 & Verificare che per l'Aggregazione Documentale Informatica, all'interno della sezione del filtro "Procedimento Amministrativo", l'utente possa inserire il valore per i campi "Materia/Argomento/Struttura", "Procedimento", "Catalogo Procedimenti" e "Fasi" & R-97-F-Ob & NO \\
\hline
TS-98 & Verificare che per l'Aggregazione Documentale Informatica, all'interno della sezione del filtro "Procedimento Amministrativo" per il campo "Fasi", l'utente possa inserire il valore per il campo "Tipo Fase" tra "Preparatoria", "Istruttoria", "Consultiva", "Decisoria o Deliberativa" o "Integrazione dell'efficacia", il campo "Data Inizio Fase" e "Data Fine Fase" se presente & R-98-F-Ob & NO \\
\hline
TS-99 & Verificare che per l'Aggregazione Documentale Informatica, il sistema renda disponibile una sezione per il filtro "Assegnazione" & R-99-F-Ob & NO \\
\hline
TS-100 & Verificare che per l'Aggregazione Documentale Informatica, all'interno della sezione del filtro "Assegnazione", l'utente possa inserire una o più volte il valore per i campi "Tipo Assegnazione", "Soggetto Assegnatario", "Data Inizio Assegnazione" e "Data Fine Assegnazione" & R-100-F-Ob & NO \\
\hline
TS-101 & Verificare che per l'Aggregazione Documentale Informatica, l'utente possa inserire il valore per il filtro "Progressivo Aggregazione" & R-101-F-Ob & NO \\
\hline
TS-102 & Verificare che all'interno della sezione di filtri per custom metadata, il sistema permetta di selezionare i filtri specifici per i metadata presenti & R-102-F-Ob & NO \\
\hline
TS-103 & Verificare che per ciascun custom metadata, l'utente possa inserire il nome del metadato e il relativo valore & R-103-F-Ob & NO \\
\hline
TS-104 & Verificare che quando viene eseguita una ricerca, il sistema mostri i risultati della ricerca & R-104-F-Ob & NO \\
\hline
TS-105 & Verificare che il sistema mostri per ogni risultato le informazioni rilevanti: Nome del documento o aggregazione, Data di registrazione/creazione del documento/aggregazione, Tipo di elemento tra Documento, Aggregazione, Processo o Classe Documentale & R-105-F-Ob & NO \\
\hline
TS-106 & Verificare che se la ricerca non produce risultati, l'utente possa visualizzare un messaggio di errore & R-106-F-Ob & NO \\
\hline
TS-107 & Verificare che il sistema comunichi all'utente quando il formato del valore inserito non è valido & R-107-F-Ob & NO \\
\hline

TS-108 & Verificare che l'utente possa compilare il filtro selezionato con un valore & R-108-F-Ob & NO  \\
\hline
TS-109 & Verificare che se il valore inserito in un filtro non è corretto l'utente possa visualizzare un messaggio di errore & R-109-F-Ob & NO \\
\hline
TS-110 & Verificare che l'utente possa salvare il documento in locale in una cartella selezionata & R-110-F-Ob & NO \\
\hline
TS-111 & Verificare che l'utente possa salvare più documenti in una cartella selezionata & R-111-F-Ob & NO \\
\hline
TS-112 & Verificare che se il salvataggio di uno o più documenti fallisce, l'utente possa visualizzare un messaggio di errore & R-112-F-Ob & NO \\
\hline
TS-113 & Verificare che l'utente possa stampare un documento & R-113-F-Ob & NO \\
\hline
TS-114 & Verificare che l'utente possa stampare un insieme di documenti & R-114-F-Ob & NO \\
\hline
TS-115 & Verificare che se la stampa di uno o più documenti fallisce, l'utente possa visualizzare un messaggio di errore & R-115-F-Ob & NO \\
\hline
TS-116 & Verificare che se la stampa non è disponibile, l'utente possa visualizzare un messaggio di errore & R-116-F-Ob & NO \\
\hline
TS-117 & Verificare che l'utente possa avviare la verifica del DIP & R-117-F-Ob & NO \\
\hline
TS-118 & Verificare che l'utente possa visualizzare lo stato di verifica del DIP & R-118-F-Ob & NO \\
\hline
TS-119 & Verificare che l'utente possa avviare la verifica della classe documentale & R-119-F-Ob & NI\\
\hline
TS-120 & Verificare che l'utente possa visualizzare lo stato di verifica della classe documentale & R-120-F-Ob & NO \\
\hline
TS-121 & Verificare che l'utente possa avviare la verifica dell'integrità processo & R-121-F-Ob & NO \\
\hline
TS-122 & Verificare che l'utente possa visualizzare lo stato di verifica dell'integrità processo & R-122-F-Ob & NO \\
\hline
TS-123 & Verificare che l'utente possa avviare la verifica del documento & R-123-F-Ob & NO \\
\hline
TS-124 & Verificare che l'utente possa visualizzare lo stato di verifica del documento & R-124-F-Ob & NO \\
\hline





TS-125 & Verificare che l'utente possa visualizzare il report di integrità del DIP completo con le informazioni aggregate & R-125-F-Ob & NO \\
\hline
TS-126 & Verificare che il sistema mostri il conteggio totale delle classi documentali verificate nel report del DIP & R-126-F-Ob & NO \\
\hline
TS-127 & Verificare che il sistema mostri il numero di classi integre (stato "Valido") in colore verde nel report del DIP & R-127-F-Ob & NO \\
\hline
TS-128 & Verificare che il sistema mostri il numero di classi corrotte (stato "Non Valido") in colore rosso nel report del DIP & R-128-F-Ob & NO \\
\hline
TS-129 & Verificare che il sistema mostri l'elenco delle classi corrotte indicando nome e numero di processi corrotti per ciascuna & R-129-F-Ob & NO \\
\hline
TS-130 & Verificare che il sistema mostri la data e l'ora di inizio della verifica del DIP nel formato "GG/MM/AAAA HH:MM:SS" & R-130-F-Ob & NO \\
\hline
TS-131 & Verificare che l'utente possa visualizzare il report di integrità dettagliato per una singola classe documentale & R-131-F-Ob & NO \\
\hline
TS-132 & Verificare che il sistema mostri il conteggio dei processi verificati all'interno della classe documentale & R-132-F-Ob & NO \\
\hline
TS-133 & Verificare che il sistema mostri il numero di processi integri (stato "Valido") in colore verde nella classe documentale & R-133-F-Ob & NO \\
\hline
TS-134 & Verificare che il sistema mostri il numero di processi corrotti (stato "Non Valido") in colore rosso nella classe documentale & R-134-F-Ob & NO \\
\hline
TS-135 & Verificare che il sistema mostri la lista dei processi corrotti con il relativo numero di documenti compromessi & R-135-F-Ob & NO \\
\hline
TS-136 & Verificare che il sistema mostri la data e l'ora di inizio della verifica della classe nel formato "GG/MM/AAAA HH:MM:SS" & R-136-F-Ob & NO \\
\hline
TS-137 & Verificare che l'utente possa visualizzare il report di integrità dettagliato di un singolo processo & R-137-F-Ob & NO \\
\hline
TS-138 & Verificare che il sistema mostri il conteggio dei documenti verificati all'interno del processo selezionato & R-138-F-Ob & NO \\
\hline
TS-139 & Verificare che il sistema mostri il numero di documenti integri (stato "Valido") in colore verde nel processo selezionato & R-139-F-Ob & NO \\
\hline
TS-140 & Verificare che il sistema mostri il numero di documenti corrotti (stato "Non Valido") in colore rosso nel processo selezionato & R-140-F-Ob & NO \\
\hline
TS-141 & Verificare che il sistema mostri la lista dei documenti corrotti con indicazione del nome e dell'errore specifico riscontrato & R-141-F-Ob & NO \\
\hline
TS-142 & Verificare che il sistema mostri la data e l'ora di inizio della verifica del processo nel formato "GG/MM/AAAA HH:MM:SS" & R-142-F-Ob & NO \\
\hline
TS-143 & Verificare che l'utente possa visualizzare il report di integrità di un singolo documento & R-143-F-Ob & NO \\
\hline
TS-144 & Verificare che il sistema mostri il nome del documento all'interno del report di integrità & R-144-F-Ob & NO \\
\hline
TS-145 & Verificare che il sistema mostri lo stato della verifica (Valido / Non Valido) per il documento selezionato & R-145-F-Ob & NO \\
\hline
TS-146 & Verificare che il sistema mostri la data e l'ora di inizio della verifica del documento nel formato "GG/MM/AAAA HH:MM:SS" & R-146-F-Ob & NO \\
\hline
TS-147 & Verificare che il sistema mostri la descrizione tecnica del dettaglio dell'errore per i documenti con stato "Non Valido" & R-147-F-Ob & NO \\
\hline
TS-148 & Verificare che l'utente possa avviare la conversione del report di verifica visualizzato in formato PDF & R-148-F-Ob & NO \\
\hline
TS-149 & Verificare che il sistema mostri un messaggio di errore qualora la generazione del file PDF non vada a buon fine & R-149-F-Ob & NO \\
\hline
TS-150 & Verificare che l'utente possa scaricare un file in una cartella locale previa selezione della cartella di destinazione & R-150-F-Ob & NO \\
\hline
TS-151 & Verificare che il sistema mostri un messaggio di conferma indicando il percorso di destinazione al termine del salvataggio & R-151-F-Ob & NO \\
\hline
TS-152 & Verificare che il sistema impedisca il salvataggio di file all'interno della cartella sorgente del DIP per preservarne l'integrità & R-152-F-Ob & NO \\
\hline
TS-153 & Verificare che il sistema mostri un errore specifico se l'utente tenta di scaricare un file nel percorso protetto del DIP & R-153-F-Ob & NO \\
\hline
TS-154 & Verificare che l'utente possa visualizzare le informazioni dell'AiP di provenienza di un documento selezionato & R-154-F-Ob & NO \\
\hline
TS-155 & Verificare che il sistema mostri la classe documentale di appartenenza dell'AiP relativo al documento selezionato & R-155-F-Ob & NO \\
\hline
TS-156 & Verificare che il sistema mostri lo UUID dell'AiP relativo al documento selezionato & R-156-F-Ob & NO \\
\hline




TS-157 & Verificare che l'utente possa visualizzare le informazioni del processo di conservazione dell'AiP & R-157-F-Ob & NO \\
\hline
TS-158 & Verificare che l'utente possa visualizzare la data di inizio di un processo o sessione & R-158-F-Ob & NO \\
\hline
TS-159 & Verificare che l'utente possa visualizzare la data di fine di un processo o sessione & R-159-F-Ob & NO \\
\hline
TS-160 & Verificare che se il processo o la sessione non è ancora terminato/a, al posto della data di fine il sistema mostri un messaggio che indica l'assenza della data di fine & R-160-F-Ob & NO \\
\hline
TS-161 & Verificare che l'utente possa visualizzare lo UUID dell'utente attivatore di un processo o sessione & R-161-F-Ob & NO \\
\hline
TS-162 & Verificare che l'utente possa visualizzare lo UUID dell'utente terminatore di un processo o sessione & R-162-F-Ob & NO \\
\hline
TS-163 & Verificare che se il processo o la sessione non è ancora terminato/a, al posto dello UUID dell'utente terminatore il sistema mostri un messaggio che indica l'assenza dello UUID & R-163-F-Ob & NO \\
\hline
TS-164 & Verificare che l'utente possa visualizzare il nome del canale di attivazione di un processo o sessione & R-164-F-Ob & NO \\
\hline
TS-165 & Verificare che l'utente possa visualizzare il nome del canale di terminazione di un processo o sessione & R-165-F-Ob & NO \\
\hline
TS-166 & Verificare che se il processo o la sessione non è ancora terminato/a, al posto del nome del canale di terminazione il sistema mostri un messaggio che indica l'assenza del nome del canale di terminazione & R-166-F-Ob & NO \\
\hline
TS-167 & Verificare che l'utente possa visualizzare lo stato di un processo o sessione & R-167-F-Ob & NO \\
\hline
TS-168 & Verificare che l'utente possa visualizzare le informazioni della sessione di versamento del processo di conservazione selezionato & R-168-F-Ob & NO \\
\hline
TS-169 & Verificare che l'utente possa visualizzare la data di inizio della sessione di versamento & R-169-F-Ob & NO \\
\hline
TS-170 & Verificare che l'utente possa visualizzare la data di fine della sessione di versamento & R-170-F-Ob & NO \\
\hline
TS-171 & Verificare che se la sessione di versamento non è ancora terminata, al posto della data di fine il sistema mostri un messaggio che indica l'assenza della data di fine & R-171-F-Ob & NO \\
\hline
TS-172 & Verificare che l'utente possa visualizzare lo UUID dell'utente attivatore della sessione di versamento & R-172-F-Ob & NO \\
\hline
TS-173 & Verificare che l'utente possa visualizzare lo UUID dell'utente terminatore della sessione di versamento & R-173-F-Ob & NO \\
\hline
TS-174 & Verificare che se la sessione di versamento non è ancora terminata, al posto dello UUID dell'utente terminatore il sistema mostri un messaggio che indica l'assenza dello UUID & R-174-F-Ob & NO \\
\hline
TS-175 & Verificare che l'utente possa visualizzare il nome del canale di attivazione della sessione di versamento & R-175-F-Ob & NO \\
\hline
TS-176 & Verificare che l'utente possa visualizzare il nome del canale di terminazione della sessione di versamento & R-176-F-Ob & NO \\
\hline
TS-177 & Verificare che se la sessione di versamento non è ancora terminata, al posto del nome del canale di terminazione il sistema mostri un messaggio che indica l'assenza del nome del canale di terminazione & R-177-F-Ob & NO \\
\hline
TS-178 & Verificare che l'utente possa visualizzare lo stato della sessione di versamento & R-178-F-Ob & NO \\
\hline
TS-179 & Verificare che l'utente possa visualizzare le informazioni della sessione di conservazione del processo di conservazione selezionato & R-179-F-Ob & NO \\
\hline
TS-180 & Verificare che l'utente possa visualizzare la data di inizio della sessione di conservazione & R-180-F-Ob & NO \\
\hline
TS-181 & Verificare che l'utente possa visualizzare la data di fine della sessione di conservazione & R-181-F-Ob & NO \\
\hline
TS-182 & Verificare che se la sessione di conservazione non è ancora terminata, al posto della data di fine il sistema mostri un messaggio che indica l'assenza della data di fine & R-182-F-Ob & NO \\
\hline
TS-183 & Verificare che l'utente possa visualizzare lo UUID dell'utente attivatore della sessione di conservazione & R-183-F-Ob & NO \\
\hline
TS-184 & Verificare che l'utente possa visualizzare lo UUID dell'utente terminatore della sessione di conservazione & R-184-F-Ob & NO \\
\hline
TS-185 & Verificare che se la sessione di conservazione non è ancora terminata, al posto dello UUID dell'utente terminatore il sistema mostri un messaggio che indica l'assenza dello UUID & R-185-F-Ob & NO \\
\hline
TS-186 & Verificare che l'utente possa visualizzare il nome del canale di attivazione della sessione di conservazione & R-186-F-Ob & NO \\
\hline
TS-187 & Verificare che l'utente possa visualizzare il nome del canale di terminazione della sessione di conservazione & R-187-F-Ob & NO \\
\hline
TS-188 & Verificare che se la sessione di conservazione non è ancora terminata, al posto del nome del canale di terminazione il sistema mostri un messaggio che indica l'assenza del nome del canale di terminazione & R-188-F-Ob & NO \\
\hline
TS-189 & Verificare che l'utente possa visualizzare lo stato della sessione di conservazione & R-189-F-Ob & NO \\
\hline
TS-190 & Verificare che l'utente possa visualizzare la descrizione del documento selezionato & R-190-F-Ob & NO \\
\hline
TS-191 & Verificare che l'utente possa visualizzare la lista dei soggetti coinvolti nel documento selezionato & R-191-F-Ob & NO \\
\hline
TS-192 & Verificare che per ogni soggetto coinvolto nel documento, il sistema visualizzi il ruolo del soggetto nel documento & R-192-F-Ob & NO \\
\hline
TS-193 & Verificare che per ogni soggetto coinvolto nel documento, il sistema visualizzi il tipo di soggetto & R-193-F-Ob & NO \\
\hline
TS-194 & Verificare che se il soggetto coinvolto nel documento è di tipo Persona Fisica, il sistema visualizzi il nome del soggetto & R-194-F-Ob & NO \\
\hline
TS-195 & Verificare che se il soggetto coinvolto nel documento è di tipo Persona Fisica, il sistema visualizzi il cognome del soggetto & R-195-F-Ob & NO \\
\hline
TS-196 & Verificare che se il soggetto coinvolto nel documento è di tipo Persona Fisica, il sistema visualizzi il codice fiscale del soggetto & R-196-F-Ob & NO \\
\hline
TS-197 & Verificare che se il soggetto coinvolto nel documento è di tipo Persona Fisica, il sistema visualizzi gli indirizzi digitali di riferimento del soggetto & R-197-F-Ob & NO \\
\hline
TS-198 & Verificare che se il soggetto coinvolto nel documento è di tipo Persona Giuridica, il sistema visualizzi la denominazione dell'organizzazione del soggetto & R-198-F-Ob & NO \\
\hline
TS-199 & Verificare che se il soggetto coinvolto nel documento è di tipo Persona Giuridica, il sistema visualizzi la partita IVA del soggetto & R-199-F-Ob & NO \\
\hline
TS-200 & Verificare che se il soggetto coinvolto nel documento è di tipo Persona Giuridica, il sistema visualizzi il codice fiscale del soggetto & R-200-F-Ob & NO \\
\hline
TS-201 & Verificare che se il soggetto coinvolto nel documento è di tipo Persona Giuridica, il sistema visualizzi la denominazione dell'ufficio del soggetto & R-201-F-Ob & NO \\
\hline
TS-202 & Verificare che se il soggetto coinvolto nel documento è di tipo Persona Giuridica, il sistema visualizzi gli indirizzi digitali di riferimento del soggetto & R-202-F-Ob & NO \\
\hline
TS-203 & Verificare che se il soggetto coinvolto nel documento è di tipo AS, il sistema visualizzi il cognome del soggetto & R-203-F-Ob & NO \\
\hline
TS-204 & Verificare che se il soggetto coinvolto nel documento è di tipo AS, il sistema visualizzi il nome del soggetto & R-204-F-Ob & NO \\
\hline
TS-205 & Verificare che se il soggetto coinvolto nel documento è di tipo AS, il sistema visualizzi il codice fiscale del soggetto & R-205-F-Ob & NO \\
\hline
TS-206 & Verificare che se il soggetto coinvolto nel documento è di tipo AS, il sistema visualizzi la denominazione dell'organizzazione del soggetto & R-206-F-Ob & NO \\
\hline
TS-207 & Verificare che se il soggetto coinvolto nel documento è di tipo AS, il sistema visualizzi la denominazione dell'ufficio del soggetto & R-207-F-Ob & NO \\
\hline
TS-208 & Verificare che se il soggetto coinvolto nel documento è di tipo AS, il sistema visualizzi gli indirizzi digitali di riferimento del soggetto & R-208-F-Ob & NO \\
\hline
TS-209 & Verificare che se il soggetto coinvolto nel documento è di tipo PAI, il sistema visualizzi la denominazione dell'amministrazione e il codice IPA del soggetto & R-209-F-Ob & NO \\
\hline
TS-210 & Verificare che se il soggetto coinvolto nel documento è di tipo PAI, il sistema visualizzi la denominazione dell'amministrazione AOO e il codice IPA AOO del soggetto & R-210-F-Ob & NO \\
\hline
TS-211 & Verificare che se il soggetto coinvolto nel documento è di tipo PAI, il sistema visualizzi la denominazione dell'amministrazione UOR e il codice IPA UOR del soggetto & R-211-F-Ob & NO \\
\hline
TS-212 & Verificare che se il soggetto coinvolto nel documento è di tipo PAI, il sistema visualizzi gli indirizzi digitali di riferimento del soggetto & R-212-F-Ob & NO \\
\hline
TS-213 & Verificare che se il soggetto coinvolto nel documento è di tipo PAE, il sistema visualizzi la denominazione dell'amministrazione del soggetto & R-213-F-Ob & NO \\
\hline
TS-214 & Verificare che se il soggetto coinvolto nel documento è di tipo PAE, il sistema visualizzi la denominazione dell'ufficio del soggetto & R-214-F-Ob & NO \\
\hline
TS-215 & Verificare che se il soggetto coinvolto nel documento è di tipo PAE, il sistema visualizzi gli indirizzi digitali di riferimento del soggetto & R-215-F-Ob & NO \\
\hline
TS-216 & Verificare che se il soggetto coinvolto nel documento è di tipo SW, il sistema visualizzi la denominazione del sistema del soggetto & R-216-F-Ob & NO \\
\hline
TS-217 & Verificare che l'utente possa visualizzare l'indice di classificazione del documento selezionato & R-217-F-Ob & NO \\
\hline
TS-218 & Verificare che l'utente possa visualizzare la descrizione dell'indice di classificazione del documento selezionato & R-218-F-Ob & NO \\
\hline
TS-219 & Verificare che l'utente possa visualizzare l'URI del piano di classificazione del documento selezionato & R-219-F-Ob & NO \\
\hline
TS-220 & Verificare che se il documento selezionato ha un tempo di conservazione diverso da quello assegnato all'aggregazione documentale informatica a cui appartiene, l'utente possa visualizzare il tempo di conservazione effettivo del documento & R-220-F-Ob & NO \\
\hline
TS-221 & Verificare che se il tempo di conservazione del documento coincide con quello assegnato all'aggregazione documentale a cui appartiene, il sistema mostri un messaggio che indica questa coincidenza & R-221-F-Ob & NO \\
\hline
TS-222 & Verificare che l'utente possa visualizzare le note relative al documento selezionato & R-222-F-Ob & NO \\
\hline
TS-223 & Verificare che se le note del documento sono assenti o vuote, il sistema mostri un messaggio che indica l'assenza delle note & R-223-F-Ob & NO \\
\hline
TS-224 & Verificare che l'utente possa visualizzare la tipologia di flusso del documento selezionato & R-224-F-Ob & NO \\
\hline
TS-225 & Verificare che l'utente possa visualizzare il tipo di registro del documento selezionato & R-225-F-Ob & NO \\
\hline
TS-226 & Verificare che l'utente possa visualizzare la data di registrazione del documento selezionato & R-226-F-Ob & NO \\
\hline
TS-227 & Verificare che l'utente possa visualizzare il numero del documento selezionato & R-227-F-Ob & NO \\
\hline
TS-228 & Verificare che l'utente possa visualizzare il codice identificativo del registro di appartenenza del documento selezionato & R-228-F-Ob & NO \\
\hline
TS-229 & Verificare che l'utente possa visualizzare la tipologia documentale del documento selezionato & R-229-F-Ob & NO \\
\hline
TS-230 & Verificare che l'utente possa visualizzare la modalità di formazione del documento selezionato & R-230-F-Ob & NO \\
\hline
TS-231 & Verificare che l'utente possa visualizzare lo stato di riservatezza del documento selezionato & R-231-F-Ob & NO \\
\hline
TS-232 & Verificare che l'utente possa visualizzare il tipo di formato del documento selezionato & R-232-F-Ob & NO \\
\hline
TS-233 & Verificare che l'utente possa visualizzare il nome del prodotto software che ha generato il documento selezionato & R-233-F-Ob & NO \\
\hline
TS-234 & Verificare che l'utente possa visualizzare la versione del prodotto software che ha generato il documento selezionato & R-234-F-Ob & NO \\
\hline
TS-235 & Verificare che l'utente possa visualizzare il produttore del software che ha generato il documento selezionato & R-235-F-Ob & NO \\
\hline
TS-236 & Verificare che l'utente possa visualizzare se il documento selezionato è firmato digitalmente & R-236-F-Ob & NO \\
\hline
TS-237 & Verificare che l'utente possa visualizzare se il documento selezionato è sigillato elettronicamente & R-237-F-Ob & NO \\
\hline
TS-238 & Verificare che l'utente possa visualizzare se il documento selezionato è dotato di marcatura temporale & R-238-F-Ob & NO \\
\hline
TS-239 & Verificare che l'utente possa visualizzare se vi è conformità alle copie immagine su supporto informatico del documento selezionato & R-239-F-Ob & NO \\
\hline
TS-240 & Verificare che l'utente possa visualizzare la versione del documento selezionato & R-240-F-Ob & NO \\
\hline
TS-241 & Verificare che l'utente possa visualizzare il nome del documento selezionato & R-241-F-Ob & NO \\
\hline
TS-242 & Verificare che l'utente possa visualizzare il numero di allegati del documento selezionato & R-242-F-Ob & NO \\
\hline
TS-243 & Verificare che se il documento ha almeno un allegato, l'utente possa visualizzare l'identificativo di ciascun allegato & R-243-F-Ob & NO \\
\hline
TS-244 & Verificare che se l'informazione sull'identificativo dell'allegato non è disponibile, il sistema mostri un messaggio di errore & R-244-F-Ob & NO \\
\hline
TS-245 & Verificare che se il documento ha almeno un allegato, l'utente possa visualizzare la descrizione di ciascun allegato & R-245-F-Ob & NO \\
\hline
TS-246 & Verificare che se l'informazione sulla descrizione dell'allegato non è disponibile, il sistema mostri un messaggio di errore & R-246-F-Ob & NO \\
\hline
TS-247 & Verificare che se il documento non ha allegati, il sistema mostri un messaggio che indica l'assenza degli allegati & R-247-F-Ob & NO \\
\hline
TS-248 & Verificare che l'utente possa visualizzare il tipo di modifica per ogni modifica del documento selezionato tra: Annullamento, Rettifica, Integrazione e Annotazione & R-248-F-Ob & NO \\
\hline
TS-249 & Verificare che l'utente possa visualizzare le informazioni del soggetto autore di ogni modifica del documento selezionato & R-249-F-Ob & NO \\
\hline
TS-250 & Verificare che l'utente possa visualizzare la data e l'ora di ogni modifica del documento selezionato & R-250-F-Ob & NO \\
\hline
TS-251 & Verificare che l'utente possa visualizzare l'identificativo del documento alla versione precedente alla modifica & R-251-F-Ob & NO \\
\hline
TS-252 & Verificare che l'utente possa visualizzare il tipo di aggregazione dell'aggregazione documentale selezionata tra: Fascicolo, Serie Documentale e Serie di Fascicoli & R-252-F-Ob & NO \\
\hline
TS-253 & Verificare che l'utente possa visualizzare l'identificativo dell'aggregazione documentale selezionata & R-253-F-Ob & NO \\
\hline
TS-254 & Verificare che l'utente possa visualizzare la tipologia di fascicolo dell'aggregazione documentale selezionata tra: Affare, Attività, Persona Fisica, Persona Giuridica e Procedimento Amministrativo & R-254-F-Ob & NO \\
\hline
TS-255 & Verificare che l'utente possa visualizzare il tipo di assegnazione dell'aggregazione documentale selezionata tra: Per competenza e Per conoscenza & R-255-F-Ob & NO \\
\hline
TS-256 & Verificare che l'utente possa visualizzare le informazioni del soggetto assegnatario dell'aggregazione documentale selezionata & R-256-F-Ob & NO \\
\hline
TS-257 & Verificare che l'utente possa visualizzare la data e l'ora di inizio dell'assegnazione dell'aggregazione documentale selezionata & R-257-F-Ob & NO \\
\hline
TS-258 & Verificare che l'utente possa visualizzare la data e l'ora di fine dell'assegnazione dell'aggregazione documentale selezionata & R-258-F-Ob & NO \\
\hline
TS-259 & Verificare che l'utente possa visualizzare la data di apertura dell'aggregazione documentale selezionata & R-259-F-Ob & NO \\
\hline
TS-260 & Verificare che l'utente possa visualizzare la data di chiusura dell'aggregazione documentale selezionata & R-260-F-Ob & NO \\
\hline
TS-261 & Verificare che l'utente possa visualizzare il progressivo dell'aggregazione documentale selezionata & R-261-F-Ob & NO \\
\hline
TS-262 & Verificare che l'utente possa visualizzare l'indice della materia/argomento/struttura per la quale sono catalogati i procedimenti dell'aggregazione documentale selezionata & R-262-F-Ob & NO \\
\hline
TS-263 & Verificare che l'utente possa visualizzare la denominazione del procedimento amministrativo dell'aggregazione documentale selezionata & R-263-F-Ob & NO \\
\hline
TS-264 & Verificare che l'utente possa visualizzare il catalogo dei procedimenti come URI di pubblicazione dell'aggregazione documentale selezionata & R-264-F-Ob & NO \\
\hline
TS-265 & Verificare che l'utente possa visualizzare la lista delle fasi del procedimento amministrativo dell'aggregazione documentale selezionata & R-265-F-Ob & NO \\
\hline
TS-266 & Verificare che per ogni fase del procedimento amministrativo, il sistema visualizzi il tipo di fase tra: Preparatoria, Istruttoria, Consultiva, Decisoria o deliberativa e Integrazione dell'efficacia & R-266-F-Ob & NO \\
\hline
TS-267 & Verificare che per ogni fase del procedimento amministrativo, il sistema visualizzi la data e l'ora di inizio della fase & R-267-F-Ob & NO \\
\hline
TS-268 & Verificare che per ogni fase del procedimento amministrativo, il sistema visualizzi la data e l'ora di fine della fase & R-268-F-Ob & NO \\
\hline
TS-269 & Verificare che l'utente possa visualizzare l'indice dei documenti contenuti nell'aggregazione documentale selezionata & R-269-F-Ob & NO \\
\hline
TS-270 & Verificare che per ogni voce dell'indice dei documenti dell'aggregazione, il sistema visualizzi il tipo di documento contenuto & R-270-F-Ob & NO \\
\hline
TS-271 & Verificare che per ogni voce dell'indice dei documenti dell'aggregazione, il sistema visualizzi l'identificativo del documento contenuto & R-271-F-Ob & NO \\
\hline
TS-272 & Verificare che l'utente possa visualizzare la posizione fisica dell'aggregazione documentale selezionata & R-272-F-Ob & NO \\
\hline
TS-273 & Verificare che l'utente possa visualizzare l'identificativo dell'aggregazione primaria dell'aggregazione documentale selezionata & R-273-F-Ob & NO \\
\hline
TS-274 & Verificare che l'utente possa visualizzare il tempo di conservazione dell'aggregazione documentale selezionata & R-274-F-Ob & NO \\
\hline
TS-275 & Verificare che l'utente possa visualizzare il nome di ciascun metadato custom del documento selezionato & R-275-F-Ob & NO \\
\hline
TS-276 & Verificare che l'utente possa visualizzare il valore di ciascun metadato custom del documento selezionato & R-276-F-Ob & NO \\
\hline
TS-277 & Verificare che se il documento selezionato non dispone di metadati custom, il sistema mostri un messaggio che indica l'assenza dei metadati custom & R-277-F-Ob & NO \\
\hline

\caption{Test di sistema}
\label{tab:test-sistema}
\end{longtable}


\section{Cruscotto di valutazione e miglioramento}

Il cruscotto di valutazione è strumento fondamentale per la corretta gestione di progetto. Permette di avere un visione oggettiva dell'andamento, e di identificare eventuali problemi o criticità. Nei grafici seguenti sono riportati i dati per ogni settimana. Questo perché un monitoraggio solamente a termine di periodo è una scelta miope e di marginale utilità. Si preferisce invece avere un monitoraggio costante. Si fa notare come nel primo periodo
il gruppo, non disponendo ancora del cruscotto, non abbia attivamente monitorato le metriche. Nonostante ciò, essendo esse ricavabili vengono riportate per completezza.

\subsection{MPC-1, MPC-2, MPC-3: Earned Value, Planned Value, Actual Cost}
\includegraphics[width=\textwidth]{../assets/pdq/ev_pv_ac.png}
Dal grafico si può apprezzare come il progetto sia partito lentamente durante le prime due settimane, dovuto ovviamente alle fasi iniziali di organizzazione del gruppo. Successivamente durante il terzo e il quarto sprint si può notare una differenza importante tr PV ed EV. Questo, come riportato nelle retrospettive, è dovuto ad una stima delle attività troppo ottimistica. Il gruppo ha intrapreso dunque un approccio più conservativo che ha portato al ricongiungersi dei due valori alla settimana 10-11. \\
\subsection{MPC-4, MPC-5: CPI, SPI}
\includegraphics[width=\textwidth]{../assets/pdq/cpi_spi.png}
Possiamo notare anche da CPI e SPI la pianificazione troppo ottimistica con un calo dello SPI.
Nonostante il rallentamento, il gruppo ha completato le attività con costi inferiori al previsto.
Successivamente invece si è riusciti a recuperare il ritardo. Lo SPI è tornato sopra 1, dovuto al fatto che si sono dovute recuperare le task in ritardo, tuttavia spendendo più del previsto. Il CPI si è stabilizzato nelle ultime settimane, dovuto al fatto che il gruppo ha familiarizzato e accumulato esperienza nello stimare task simili. \\
\subsection{MPC-7: EAC confrontato con BAC}
\includegraphics[width=\textwidth]{../assets/pdq/eac_bac.png}
Strettamente collegato, e conseguenza del CPI, è l'andamento di EAC e BAC. Tra le settimane 3 e 5 si osserva una diminuzione. Tale diminuzione non tiene conto della sottostima delle attività: nelle attività completate il costo reale è risultato inferiore. Successivamente l'EAC aumenta, segnalando un incremento dei costi. Questo aumento è dovuto alle numerose iterazioni, a volte molto distruttive, sul documento di Analisi dei Requisiti. Dopo gli incontri con il prof. Cardin il gruppo ha dovuto rivedere quasi completamente la struttura degli Use Case\textsubscript{\textit{G}}. \\
\subsection{MPC-1, MPC-2, MPC-3, MPC-6, MPC-7: EV, PV, AC, ETC, EAC}
\includegraphics[width=\textwidth]{../assets/pdq/ev_pv_ac_eac_etc.png}
Si può apprezzare dal grafico quanto detto fino ad ora. Si osserva un aumento dell EAC nella fase iniziale, con un suo successivo rientro. \\
\subsection{MPC-10: Indice di Gulpease}
\includegraphics[width=\textwidth]{../assets/pdq/gfi.png}
Il gruppo misura l'Indice di Gulpease per garantire la leggibilità, soprattutto dei documenti rivolti ad esterni. Nelle fasi iniziali non è stata posta particolare attenzione, ma i risultati sono stati comunque buoni. Un'eccezione sono le Norme di Progetto: più descrittive e con pochi elenchi, mostrano valori inferiori. Per garantire un limite minimo, nelle settimane 7--10 (SPRINT 4) è stato introdotto un controllo automatico che accetta in repository solo documenti con Gulpease $\geq$ 60. \\
\subsection{MPC-11: Correttezza ortografica}
\includegraphics[width=\textwidth]{../assets/pdq/eo.png}
Gli errori ortografici sono stati gestiti in maniera analoga e parallela al Gulpease Index. Il gruppo ha deciso di non accettare documenti con errori ortografici, implementando un controllo automatico. \\
\subsection{MPC-15: Rischi non previsti}
\includegraphics[width=\textwidth]{../assets/pdq/rnp.png}
Il gruppo ha registrato due rischi imprevisti in due sprint diversi. Il primo, coerente con i dati mostrati, è stato causato dalla sottostima delle attività. Nel secondo sprint si è verificato un rischio imprevedibile: un membro del gruppo è stato indisposto. \\
\subsection{MPC-9: Requirements Stability Index}
\includegraphics[width=\textwidth]{../assets/pdq/rsi.png}
Dal grafico si osserva che, all'inizio, la stabilità dei requisiti è rimasta costante mentre il gruppo definiva le Norme di Progetto e i requisiti non erano ancora esplorati. Successivamente si è verificato un calo significativo: l'attenzione si è spostata sullo studio del dominio applicativo e sulle aspettative della proponente. La scarsa conoscenza del dominio e la continua selezione e revisione dei requisiti, anche per motivi di fattibilità, hanno generato instabilità. La proponente non ha imposto vincoli rigidi sulle funzionalità; di conseguenza i requisiti sono stati proposti e discussi in modo progressivo. Questo ha fatto sì che non si sia reso spesso necessario modificare, aggiungere o eliminare requisiti, e l'indice si è stabilizzato nella fase finale.
\subsection{MPC-16: Metriche soddisfatte}
\includegraphics[width=\textwidth]{../assets/pdq/ms.png}
Si nota un calo iniziale, corrispondentemente al periodo di instabilità generale del progetto. Successivamente, grazie a misure correttive e ai controlli automatici di alcune metriche si è raggiunta la stabilità. Questo anche grazie alla crescente attenzione rivolta al soddisfacimento delle metriche di qualità.

\vfill

\begin{flushright}
    \textit{7-ZPUs}
\end{flushright}

\end{document}