\documentclass[a4paper,12pt]{article}

\usepackage[utf8]{inputenc}
\usepackage[T1]{fontenc}
\usepackage[italian, provide=*]{babel}
\usepackage[sfdefault]{atkinson}
\renewcommand*\familydefault{\sfdefault}
\usepackage{float}
\usepackage{microtype}
\usepackage{geometry}
\usepackage{setspace}
\usepackage{enumitem}
\usepackage{titlesec}
\usepackage{chngpage}
\usepackage{tocloft}
\usepackage{longtable}
\usepackage{array}
\usepackage{graphicx}
\usepackage{fancyhdr}
\usepackage{xcolor}
\usepackage{color,soul}
\usepackage[most]{tcolorbox}
\usepackage{nameref}
\usepackage[colorlinks=true]{hyperref}

\hypersetup{
    linkcolor=black,
    urlcolor=blue
}

\definecolor{lightblack}{gray}{0.35}
\newcommand{\glossario}[1]{\textit{#1}\textsubscript{\textbf{\textit{\textcolor{lightblack}{G}}}}}
\newcommand{\ped}[1]{\textsubscript{#1}}

\pagestyle{fancy}
\setlength{\headwidth}{\textwidth}
\fancyhfoffset[L,R]{0pt}
\lhead{\rightmark}
\rhead{7-ZPUs}
\lfoot{Analisi dei requisiti}
\rfoot{\thepage}
\cfoot{}
\renewcommand{\headrulewidth}{0.8pt}
\renewcommand{\footrulewidth}{0.8pt}

\renewcommand{\contentsname}{Indice}
\renewcommand{\listoftables}{Elenco delle tabelle}
\renewcommand{\listoffigures}{Elenco delle immagini}
\geometry{margin=2.5cm}
\usepackage{titlesec}
\setcounter{tocdepth}{5}
\setcounter{secnumdepth}{5}


\titleformat{\section}{\LARGE\bfseries}{\thesection}{1em}{}
\titleformat{\subsection}{\Large\bfseries}{\thesubsection}{1em}{}
\titleformat{\subsubsection}{\large\bfseries}{}{1em}{}
\titleformat{\subsubsubsection}{\normalsize\bfseries}{}{1em}{}
\titleformat{\paragraph}{\large\bfseries}{}{1em}{}
\titleformat{\subparagraph}{\normalsize\bfseries}{}{1em}{}

% --- INIZIO DEFINIZIONE CONTATORI UC (METODO EREDITARIETÀ) ---

% 1. Contatore Principale (es. UC-1)
\newcounter{ucMain}
% Qui definiamo che la rappresentazione del contatore include già il prefisso "UC-"
\renewcommand{\theucMain}{UC-\arabic{ucMain}}

\newcommand{\usecase}[2]{% #1=Label (opzionale), #2=Titolo
    \refstepcounter{ucMain}%
    \setcounter{ucSub}{0}% Reset sotto-contatori
    \setcounter{ucVar}{0}%
    % Nota: Ora usiamo solo \theucMain perché include già "UC-"
    \subsubsection*{\theucMain\ - #2}%
    \addcontentsline{toc}{subsubsection}{\theucMain\ - #2}%
    \ifx&#1&\else\label{#1}\fi%
}

% 2. Contatore Varianti (es. UC-2a)
\newcounter{ucVar}[ucMain]
% Definiamo che la variante è: NomePadre (UC-2) + Lettera (a) -> UC-2a
\renewcommand{\theucVar}{\theucMain\alph{ucVar}}

\newcommand{\varusecase}[2]{%
    \stepcounter{ucMain}% Incrementa il padre (senza ref)
    \refstepcounter{ucVar}% Incrementa la variante (ref punta qui)
    \subsubsection*{\theucVar\ - #2}%
    \addcontentsline{toc}{subsubsection}{\theucVar\ - #2}%
    \ifx&#1&\else\label{#1}\fi%
}

% --- RAMO VARIANTI (Figli di UC-2a) ---

% 3. Contatore Figlio di Variante (es. UC-3c.1)
\newcounter{varSub}[ucVar]
% Ereditarietà: NomePadre (UC-3c) + Punto + Numero
\renewcommand{\thevarSub}{\theucVar.\arabic{varSub}}

\newcommand{\varsubusecase}[2]{%
    \refstepcounter{varSub}%
    \subsubsection*{\thevarSub\ - #2}%
    \addcontentsline{toc}{subsubsection}{\thevarSub\ - #2}%
    \ifx&#1&\else\label{#1}\fi%
}

% 4. Contatore Nipote di Variante (es. UC-3c.1.3)
\newcounter{varSubSub}[varSub]
\renewcommand{\thevarSubSub}{\thevarSub.\arabic{varSubSub}}

\newcommand{\varsubsubusecase}[2]{%
    \refstepcounter{varSubSub}%
    \paragraph*{\thevarSubSub\ - #2}%
    \addcontentsline{toc}{paragraph}{\thevarSubSub\ - #2}%
    \ifx&#1&\else\label{#1}\fi%
}

% 5. Livello profondo variante (es. UC-3c.1.1.1)
\newcounter{varDeep}[varSubSub]
\renewcommand{\thevarDeep}{\thevarSubSub.\arabic{varDeep}}

\newcommand{\vardeepusecase}[2]{%
    \refstepcounter{varDeep}%
    \paragraph*{\thevarDeep\ - #2}%
    \addcontentsline{toc}{paragraph}{\thevarDeep\ - #2}%
    \ifx&#1&\else\label{#1}\fi%
}

% 6. Livello massimo variante (es. UC-3c.1.1.1.1)
\newcounter{varSubDeep}[varDeep]
\renewcommand{\thevarSubDeep}{\thevarDeep.\arabic{varSubDeep}}

\newcommand{\varsubdeepusecase}[2]{%
    \refstepcounter{varSubDeep}%
    \paragraph*{\thevarSubDeep\ - #2}%
    \addcontentsline{toc}{paragraph}{\thevarSubDeep\ - #2}%
    \ifx&#1&\else\label{#1}\fi%
}

% --- RAMO STANDARD (Figli di UC-1) ---

% 3. Contatore Secondario (es. UC-1.1)
\newcounter{ucSub}[ucMain]
% Ereditarietà: NomePadre (UC-1) + Punto + Numero
\renewcommand{\theucSub}{\theucMain.\arabic{ucSub}}

\newcommand{\subusecase}[2]{%
    \refstepcounter{ucSub}%
    \subsubsection*{\theucSub\ - #2}%
    \addcontentsline{toc}{paragraph}{\theucSub\ - #2}%
    \ifx&#1&\else\label{#1}\fi%
}

% 4. Contatore Terziario (es. UC-3.1.1)
\newcounter{ucSubSub}[ucSub]
\renewcommand{\theucSubSub}{\theucSub.\arabic{ucSubSub}}

\newcommand{\subsubusecase}[2]{%
    \refstepcounter{ucSubSub}%
    \paragraph*{\theucSubSub\ - #2}%
    \addcontentsline{toc}{subparagraph}{\theucSubSub\ - #2}%
    \ifx&#1&\else\label{#1}\fi%
}

% 5. Contatore Quaternario (es. UC-3.2.1.1)
\newcounter{ucDeep}[ucSubSub]
\renewcommand{\theucDeep}{\theucSubSub.\arabic{ucDeep}}

\newcommand{\deepusecase}[2]{%
    \refstepcounter{ucDeep}%
    \paragraph*{\theucDeep\ - #2}%
    \addcontentsline{toc}{subparagraph}{\protect\hspace{3em}\theucDeep\ - #2}%
    \ifx&#1&\else\label{#1}\fi%
}

% 6. Contatore 5 (es. UC-3.2.1.1.5)
\newcounter{ucSubDeep}[ucSubSub] % NOTA: Nel tuo originale era su ucSubSub, forse intendevi su ucDeep?
% Ho corretto ipotizzando erediti da ucDeep per coerenza
\renewcommand{\theucSubDeep}{\theucDeep.\arabic{ucSubDeep}}

\newcommand{\subdeepusecase}[2]{%
    \refstepcounter{ucSubDeep}%
    \paragraph*{\theucSubDeep\ - #2}%
    \addcontentsline{toc}{subparagraph}{\protect\hspace{6em}\theucSubDeep\ - #2}%
    \ifx&#1&\else\label{#1}\fi%
}

% --- FINE DEFINIZIONE CONTATORI UC ---

% --- Configurazione Indentazione Indice (TOC) ---

% 1. Livello Padre (Use Case principale) -> Mappato su subsubsection
% Lo teniamo allineato a sinistra (o con un rientro minimo)
\setlength{\cftsubsubsecindent}{0pt} 
\setlength{\cftsubsubsecnumwidth}{3em} % Spazio per il numero (es. UC-1)

% 2. Livello Figlio (Sub Use Case) -> Mappato su paragraph
% Rientro ridotto (es. 1.5em invece del default che è molto largo)
\setlength{\cftparaindent}{1.5em} 
\setlength{\cftparanumwidth}{3.5em} 

% 3. Livello Nipote (Sub Sub Use Case) -> Mappato su subparagraph
% Rientro progressivo ridotto (es. 3em)
\setlength{\cftsubparaindent}{3em} 
\setlength{\cftsubparanumwidth}{4em}

\begin{document}
\pagenumbering{roman}

\pagenumbering{arabic}
\setcounter{page}{1}

\begin{center}
    \includegraphics[width=9.5cm]{../assets/logo7ZPUs.jpg}\\
    \small\hspace{10cm} 7zpus.swe@gmail.com\\
    \vspace{0.5cm}
    \LARGE \textbf{Analisi dei Requisiti}\\
\end{center}

\vspace{0.3cm}
\hrule
\vspace{0.5cm}

\setlength{\LTleft}{-1cm}
\section*{Tabella di Versionamento}
    \begin{center}
    \begin{longtable}{|c|c|c|c|c|}
        \hline        
        \textbf{Versione} & \textbf{Data} & \textbf{Autore}  & \textbf{Verificatore} & \textbf{Descrizione} \\
        \hline
        0.14.0 & 2026/02/08 & Gingillino Aaron & Rocco Matteo A. & \begin{tabular}[c]{@{}c@{}} Stesura requisiti, \\tracciamento e \\diagrammi UC24-UC36 \end{tabular} \\
        \hline
        0.13.0 & 2026/02/06 & Georgescu Diana & Laoud Zakaria & \begin{tabular}[c]{@{}c@{}} Stesura requisiti, \\tracciamento e \\diagrammi UC37-UC46 \end{tabular} \\
        \hline
        0.12.0 & 2026/02/04 & Laoud Zakaria & Rocco Matteo A. & \begin{tabular}[c]{@{}c@{}} Stesura requisiti, \\tracciamento e \\diagrammi UC47-UC72 \end{tabular} \\
        \hline
        0.11.0 & 2026/02/04 & Rocco Matteo A. & Laoud Zakaria & \begin{tabular}[c]{@{}c@{}} Stesura requisiti, \\tracciamento e \\diagrammi UC1-UC23 \end{tabular} \\
        \hline
        0.10.1 & 2026/01/14 & Rocco Matteo A. & Georgescu Diana & \begin{tabular}[c]{@{}c@{}} Revisione e correzione \\generale del documento \end{tabular} \\
        \hline
        0.10.0 & 2025/12/5 & Soligo Lorenzo & Rocco Matteo A. & \begin{tabular}[c]{@{}c@{}} Aggiunta dettaglio \\ UC-Visualizza risultati\end{tabular}\\
        \hline
        0.9.3 & 2025/12/5 & Vigolo Davide & Rocco Matteo A. & \begin{tabular}[c]{@{}c@{}}Allineamento degli\\ Use Case\end{tabular} \\
        \hline
        0.9.2 & 2025/12/30 & Soligo Lorenzo & Vigolo Davide & \begin{tabular}[c]{@{}c@{}}Automatizzazione \\ codici Use Case\end{tabular} \\
        \hline
        0.9.1 & 2025/12/28 & \begin{tabular}[c]{@{}c@{}}Fattoni Antonio \\ Soligo Lorenzo \\ Vigolo Davide\end{tabular}& \begin{tabular}[c]{@{}c@{}}Fattoni Antonio \\ Soligo Lorenzo \\ Vigolo Davide\end{tabular} & \begin{tabular}[c]{@{}c@{}}Revisione Congiunta \\ degli Use Case\end{tabular} \\
        \hline
        0.9.0 & 2025/12/28 & Soligo Lorenzo & Vigolo Davide & \begin{tabular}[c]{@{}c@{}} Aggiunta UC27 $\rightarrow$ UC42 su \\ branch comune\end{tabular} \\
        \hline
        0.8.0 & 2025/12/28 & Vigolo Davide & Soligo Lorenzo & \begin{tabular}[c]{@{}c@{}} Aggiunta UC14 $\rightarrow$ UC26 su \\ branch comune\end{tabular} \\
        \hline
        0.7.0 & 2025/12/28 & Vigolo Davide & Soligo Lorenzo & \begin{tabular}[c]{@{}c@{}} Aggiunta UC10 $\rightarrow$ UC13 su \\ branch comune\end{tabular} \\
        \hline
        0.6.0 & 2025/12/28 & Fattoni Antonio & Vigolo Davide & \begin{tabular}[c]{@{}c@{}} Aggiunta UC5 $\rightarrow$ UC8 su \\ branch comune\end{tabular} \\
        \hline
        0.5.0 & 2025/12/28 & Soligo Lorenzo & Fattoni Antonio & \begin{tabular}[c]{@{}c@{}} Aggiunta UC1 $\rightarrow$ UC4 su \\ branch comune\end{tabular} \\
        \hline
        0.4.0 & 2025/12/24 & Soligo Lorenzo & Fattoni Antonio & \begin{tabular}[c]{@{}c@{}} Revisione Bozza UC1 $\rightarrow$ UC4\\ dopo incontro con \\ Prof. Cardin \end{tabular} \\
        \hline
        0.3.0 & 2025/12/21 & Soligo Lorenzo & Fattoni Antonio & \begin{tabular}[c]{@{}c@{}} Revisione Bozza UC1 $\rightarrow$ UC4 \\ dopo incontro Riunione \\ Interna\end{tabular} \\
        \hline
        0.2.0 & 12/12/2025 & Soligo Lorenzo & Rocco Matteo A. & \begin{tabular}[c]{@{}c@{}} Modularizzazione\\ file Tex \end{tabular} \\
        \hline
        0.1.0 & 07/11/2025 & Georgescu Diana & Laoud Zakaria & \begin{tabular}[c]{@{}c@{}} Creazione del template\\ e stesura iniziale \end{tabular} \\
        \hline
    \end{longtable}
    \end{center}

\tableofcontents
\newpage
\listoftables
\newpage
\listoffigures
\newpage

\section{Introduzione}

\subsection{Scopo}
Questo documento si pone l'obiettivo di delineare in modo chiaro le caratteristiche del software da realizzare, partendo dall'analisi dei bisogni e delle aspettative della \glossario{proponente}.
L'elaborazione dei requisiti trae origine dallo studio preliminare del \glossario{capitolato}, al fine di individuare gli \glossario{attori} coinvolti e le funzionalità attese.\\
 
\noindent Il presente documento sarà utilizzato come punto di riferimento per tutto l'arco dello sviluppo del prodotto, dalla \glossario{progettazione} alla \glossario{validazione}, e permetterà il tracciamento di ogni decisione progettuale, consentendoci di soddisfare le aspettative della proponente.\\

\noindent Il documento di \glossario{Analisi dei Requisiti} è redatto dagli \glossario{analisti} del team di progetto ed è destinato principalmente a tre categorie di soggetti.
In primo luogo, al \glossario{\textbf{committente}}, che attraverso la sua consultazione può verificare che i \glossario{requisiti} siano stati correttamente compresi e formalizzati in linea con le proprie aspettative.
In secondo luogo, al team di \textbf{\glossario{progettisti} e \glossario{programmatori}}, per i quali il documento rappresenta una guida di riferimento essenziale durante la fase di sviluppo del prodotto.
Infine, al \textbf{team di \glossario{verificatori}}, che si baserà sulle informazioni contenute nel suddetto documento per definire i \glossario{test} e verificare la conformità del prodotto ai requisiti.
Il documento sarà inoltre disponibile per \glossario{amministratori} e \glossario{responsabili di progetto}, offrendo una panoramica completa delle caratteristiche e delle funzionalità previste per il prodotto.
Data la natura incrementale del processo di sviluppo, questo documento verrà aggiornato periodicamente per riflettere eventuali modifiche o integrazioni ai requisiti.

\subsection{Glossario}
Per una corretta comprensione del documento, si rimanda al documento di \href{https://cdn.jsdelivr.net/gh/7-zpus/Docs@main/2_RTB/Glossario.pdf}{\ul{Glossario}\setulcolor{black}} contenente la definizione dei termini contrassegnati dalla \textit{G} a pedice (\glossario{Glossario}).

\subsection{Riferimenti}
Il documento cerca di aderire il più possibile, seppur non in modo vincolante, alla struttura e ai contenuti previsti dallo standard \glossario{IEEE 830:1998} per la specifica dei requisiti software e di rispettare le indicazioni fornite dalle dispense didattiche del corso di Ingegneria del Software dell'Università di Padova.

\subsubsection{Riferimenti Normativi}
\begin{itemize}
    \item \href{https://www.math.unipd.it/~tullio/IS-1/2025/Progetto/C3.pdf}{\ul{Capitolato C3: DIPReader}\setulcolor{black}} \ped{(ultimo accesso: 13/11/2025)}
    \item \href{https://www.math.unipd.it/~tullio/IS-1/2025/Dispense/PD1.pdf}{\ul{Regolamento di Progetto Didattico a.a. 2025/2026}\setulcolor{black}} \ped{(ultimo accesso: 17/11/2025)}
    \item \href{https://cdn.jsdelivr.net/gh/7-zpus/Docs@main/2_RTB/NormeDiProgetto.pdf}{\ul{Norme di Progetto}\setulcolor{black}} \ped{(ultimo accesso: 11/02/2026)}
\end{itemize}

\subsubsection{Riferimenti Informativi}
\begin{itemize}
    \item \href{https://cdn.jsdelivr.net/gh/7-zpus/Docs@main/2_RTB/Glossario.pdf}{\ul{Glossario di progetto}\setulcolor{black}} \ped{(ultimo accesso: 11/02/2026)}
    \item \href{https://ieeexplore.ieee.org/document/720574}{\ul{Standard IEEE 830:1998}\setulcolor{black}} \ped{(ultimo accesso: 24/11/2025)}
    \item Dispense del corso di Ingegneria del Software 2025/2026:
    \begin{itemize}
        \item \href{https://www.math.unipd.it/~tullio/IS-1/2025/Dispense/T05.pdf}{\ul{https://www.math.unipd.it/~tullio/IS-1/2025/Dispense/T05.pdf}\setulcolor{black}} \ped{(ultimo accesso: 24/11/2025)}
        \item \href{https://www.math.unipd.it/~rcardin/swea/2022/Diagrammi%20Use%20Case.pdf}{\ul{https://www.math.unipd.it/~rcardin/swea/2022/Diagrammi\%20Use\%20Case.pdf}\setulcolor{black}} \ped{(ultimo accesso: 24/11/2025)}
        \item \href{https://www.math.unipd.it/~rcardin/swea/2023/Diagrammi%20delle%20Classi.pdf}{\ul{https://www.math.unipd.it/~rcardin/swea/2023/Diagrammi\%20delle\%20Classi.pdf}\setulcolor{black}} \ped{(ultimo accesso: 24/11/2025)}
        \item \href{https://kurzy.kpi.fei.tuke.sk/zsi/resources/CockburnBookDraft.pdf}{\ul{A. Cockburn, Writing Effective Use Cases}\setulcolor{black}} \ped{(ultimo accesso: 30/12/2025)}
        \item \href{https://www.omg.org/spec/UML/2.5.1/PDF}{\ul{OMG UML 2.5.1 Specification}\setulcolor{black}} \ped{(ultimo accesso: 30/12/2025)}
    \end{itemize}
\end{itemize}

\section{Descrizione del Prodotto}

\subsection{Panoramica}
Il software oggetto di sviluppo è \textbf{DIPReader}, un'applicazione per l'accesso e la visualizzazione di documenti informatici all'interno di un Distribution Information Package (\glossario{DIP}), ovvero un archivio compresso distribuito da un sistema di conservazione centralizzato. \\

\noindent Un DIP contiene un insieme di cartelle e documenti tecnici con valore legale, strutturati secondo specifiche normative che ne garantiscono autenticità, integrità, affidabilità e ottimizzazione dello spazio di archiviazione.
I pacchetti DIP sono generalmente distribuiti come archivi compressi in formato .zip e possono includere una grande quantità di documenti eterogenei tra loro.
La loro struttura interna può differire in modo significativo da quella comunemente utilizzata nei sistemi di memorizzazione personali; risulta pertanto complesso, per l'utente, consultarli tramite strumenti generici di decompressione e visualizzazione.
DIPReader nasce per affrontare questa difficoltà, fornendo una soluzione dedicata che consenta di importare, esplorare, ricercare e validare in modo efficiente e intuitivo i contenuti dei pacchetti DIP, supportando al contempo le esigenze operative e normative del contesto in cui vengono utilizzati.

\subsection{Funzionalità}
Il prodotto da realizzare deve offrire le seguenti funzionalità fondamentali:
\begin{itemize}
    \item Possibilità di navigare in modo intuitivo all'interno di un DIP.
    \item Possibilità di visualizzare i contenuti di ciascuna cartella.
    \item Possibilità di visualizzare alcuni formati di documenti tramite anteprima.
    \item Possibilità di ricercare i documenti sia per nome che per attributi significativi (\glossario{metadata}).
    \item Possibilità di filtrare i documenti visualizzati per agevolare la ricerca attraverso i metadata
    \item Possibilità di salvare uno o più documenti nel computer dell'utente.
    \item Possibilità di verificare l'autenticità di un file attraverso il suo \glossario{processo di conservazione}.
    \item Mantenere la portabilità del DIP senza necessitare installazioni su disco.
\end{itemize}
Il prodotto deve inoltre operare in modo fluido durante le attività di navigazione, ricerca e visualizzazione anche in presenza di grandi quantità di dati, scenario molto probabile per natura stessa del contesto in cui opera.
È inoltre apprezzato lo sviluppo delle seguenti, e ulteriori se ritenute utili, funzionalità opzionali:
\begin{itemize}
    \item Possibilità di stampare i documenti selezionati.
    \item \glossario{Ricerca semantica} dei documenti mediante strumenti di intelligenza artificiale.
    \item Possibilità di accedere anche a pacchetti di distribuzione direttamente dal cloud.
    \item Possibilità di verificare l'eventuale firma digitale associata al file.
\end{itemize}

\subsection{Utenti di destinazione}
Il prodotto costituisce una soluzione essenziale per utenti operanti nei settori giudiziario e tecnico, quali l'Agenzia delle Entrate, la Guardia di Finanza e i magistrati che necessitano di consultare documenti digitali complessi presenti nei DIP.


\section{Casi d'Uso}

\subsection{Introduzione}
Per facilitare la comprensione dei casi d'uso, questi saranno descritti da un diagramma ciascuno, che rispetta la sintassi UML (Unified Modeling Language).
La descrizione testuale deve contenere le seguenti informazioni:
\begin{itemize}
    \item \textbf{Attori (principali o secondari)}: rappresentano un ruolo che un'entità esterna al sistema assume quando interagisce con esso per raggiungere un obiettivo.
    \item \textbf{Precondizioni}: condizioni che devono essere vere nello stato del sistema prima che il caso d'uso inizi la sua esecuzione.
    \item \textbf{Postcondizioni}: condizioni che devono essere vere nello stato del sistema dopo che il caso d'uso è terminato.
    \item \textbf{Scenario principale}: interazioni tra attore e sistema che porta al raggiungimento dell'obiettivo del caso d'uso con successo
    \item \textbf{Scenari alternativi}: variazione rispetto al flusso principale.
\end{itemize}

\subsection{Elenco Casi d'Uso}

\usecase{classiDocumentali}{Visualizza elenco classi documentali}
\begin{figure}[H]
    \centering
    \includegraphics[width=0.9\textwidth]{../assets/uml/UC1.png}
    \caption{UC1 - Visualizza elenco classi documentali}
    \label{fig:uc_classiDocumentali}
\end{figure}
\begin{itemize}
    \item \textbf{Attore Primario}: Utente
    \item \textbf{Precondizioni}: L'Utente ha avviato l'applicazione.
    \item \textbf{Postcondizioni}: L'Utente visualizza l'elenco delle classi documentali.
    \item \textbf{Flusso Principale}:
    \begin{enumerate}
        \item Il sistema mostra a video l'elenco delle classi documentali presenti nel DIP.
    \end{enumerate}
    \item \textbf{Flusso Alternativo}:
    \begin{itemize}
        \item Se non sono presenti classi documentali, il sistema mostra un elenco vuoto. (\ref{elencoVuoto})
    \end{itemize}
    \item \textbf{Estensioni}: \ref{elencoVuoto} Visualizza elenco vuoto
\end{itemize}

% inclusioni: nomeClasseDocumentale
\usecase{classeDocumentale}{Visualizza classe documentale in elenco}
\begin{figure}[H]
    \centering
    \includegraphics[width=0.6\textwidth]{../assets/uml/UC2.png}
    \caption{UC2 - Visualizza classe documentale in elenco}
    \label{fig:uc_classeDocumentale}
\end{figure}
\begin{itemize}
    \item \textbf{Attore Primario}: Utente
    \item \textbf{Precondizioni}: L'Utente ha avviato l'applicazione.
    \item \textbf{Postcondizioni}: L'Utente visualizza l'elenco delle classi documentali.
    \item \textbf{Flusso Principale}:
    \begin{enumerate}
        \item Per ogni singola classe documentale presente nell'elenco viene mostrato:
        \begin{itemize}
            \item Nome della classe documentale (\ref{nomeClasseDocumentale})
            \item Stato di verifica della classe documentale (\ref{statoVerificaElemento})
            \item Marcatura temporale della classe documentale (\ref{marcaturaTemporaleElemento})
        \end{itemize}
    \end{enumerate}
    \item \textbf{Inclusioni}:
    \begin{itemize}
        \item \ref{nomeClasseDocumentale} Visualizza Nome Classe Documentale
        \item \ref{statoVerificaElemento} Visualizza Stato di Verifica Elemento
        \item \ref{marcaturaTemporaleElemento} Visualizza Marcatura Temporale Elemento
    \end{itemize}
\end{itemize}
\begin{figure}[H]
    \centering
    \includegraphics[width=0.5\textwidth]{../assets/uml/UC2INC.png}
    \caption{Inclusioni UC2 - Visualizza elenco processi di una classe documentale}
    \label{fig:inclusioniClasseDocumentale}
\end{figure}

\subusecase{nomeClasseDocumentale}{Visualizza Nome Classe Documentale}
\begin{itemize}
    \item \textbf{Attore Primario}: Utente
    \item \textbf{Precondizioni}: L'Utente ha avviato l'applicazione.
    \item \textbf{Postcondizioni}: L'Utente visualizza il nome della classe documentale.
    \item \textbf{Flusso Principale}:
          \begin{enumerate}
              \item Il sistema mostra a video il nome della classe documentale.
          \end{enumerate}
\end{itemize}

\subusecase{statoVerificaElemento}{Visualizza Stato di Verifica Elemento}
\begin{itemize}
    \item \textbf{Attore Primario}: Utente
    \item \textbf{Precondizioni}: L'Utente ha selezionato un elemento.
    \item \textbf{Postcondizioni}: L'utente visualizza lo stato di verifica dell' elemento selezionato.
    \item \textbf{Flusso Principale}:
    \begin{enumerate}
              \item Il sistema mostra a video lo stato di verifica dell'elemento selezionato
    \end{enumerate}
\end{itemize}

\subusecase{marcaturaTemporaleElemento}{Visualizza Marcatura Temporale Elemento}
\begin{itemize}
    \item \textbf{Attore Primario}: Utente
    \item \textbf{Precondizioni}: L'Utente ha selezionato un elemento.
    \item \textbf{Postcondizioni}: L'utente visualizza la marcatura temporale dell' elemento selezionato.
    \item \textbf{Flusso Principale}:
    \begin{enumerate}
              \item Il sistema mostra a video la marcatura temporale dell' elemento selezionato
    \end{enumerate}
\end{itemize}

\usecase{processiClasseDocumentale}{Visualizza elenco dei processi di una classe documentale}
\begin{figure}[H]
    \centering
    \includegraphics[width=0.9\textwidth]{../assets/uml/UC3.png}
    \caption{UC3 - Visualizza elenco dei processi di una classe documentale}
    \label{fig:uc_processiClasseDocumentale}
\end{figure}
\begin{itemize}
    \item \textbf{Attore Primario}: Utente
    \item \textbf{Precondizioni}: \begin{itemize}
        \item L'utente ha avviato l'applicazione
        \item L'Utente ha selezionato una delle Classi Documentali del DIP
    \end{itemize}
    \item \textbf{Postcondizioni}: L'Utente visualizza i processi associati alla classe selezionata.
    \item \textbf{Flusso Principale}: 
          \begin{enumerate}
              \item Il sistema mostra a video l'elenco dei processi della classe documentale selezionata.
          \end{enumerate}
    \item \textbf{Flusso Alternativo}:
    \begin{itemize}
        \item Se la classe documentale non contiene processi, il sistema mostra un elenco vuoto.
    \end{itemize}
    \item \textbf{Estensioni}: \ref{elencoVuoto} Visualizza elenco vuoto
\end{itemize}

\usecase{processoClasseDocumentale}{Visualizza processi in elenco}
\begin{figure}[H]
    \centering
    \includegraphics[width=0.6\textwidth]{../assets/uml/UC4.png}
    \caption{UC4 - Visualizza processi in elenco}
    \label{fig:uc_processoClasseDocumentale}
\end{figure}
\begin{itemize}
    \item \textbf{Attore Primario}: Utente
    \item \textbf{Precondizioni}:
    \begin{itemize}
        \item L'utente ha avviato l'applicazione
        \item L'Utente ha selezionato una delle Classi Documentali del DIP
    \end{itemize}
    \item \textbf{Postcondizioni}: L'Utente visualizza l'elenco dei processi .
    \item \textbf{Flusso Principale}:
    \begin{enumerate}
        \item Per ogni singolo processo presente nell'elenco viene mostrato:
        \begin{itemize}
            \item Id del processo (\ref{idProcessoClasse})
            \item Stato di verifica del processo (\ref{statoVerificaElemento})
            \item Marcatura temporale del processo (\ref{marcaturaTemporaleElemento})
        \end{itemize}
    \end{enumerate}
    \item \textbf{Inclusioni}:
    \begin{itemize}
        \item \ref{idProcessoClasse} Visualizza Id Processo
        \item \ref{statoVerificaElemento} Visualizza Stato di Verifica Elemento
        \item \ref{marcaturaTemporaleElemento} Visualizza Marcatura Temporale Elemento
    \end{itemize}
\end{itemize}
\begin{figure}[H]
    \centering
    \includegraphics[width=0.5\textwidth]{../assets/uml/UC4INC.png}
    \caption{Inclusioni UC4 - Visualizza elenco dei documenti di un processo}
    \label{fig:inclusioniProcessoClasseDocumentale}
\end{figure}

\subusecase{idProcessoClasse}{Visualizza Id Processo}
\begin{itemize}
    \item \textbf{Attore Primario}: Utente
    \item \textbf{Precondizioni}: \begin{itemize}
        \item L'utente ha avviato l'applicazione
        \item L'Utente ha selezionato una delle Classi Documentali del DIP
    \end{itemize}
    \item \textbf{Postcondizioni}: L'Utente visualizza il nome del processo.
    \item \textbf{Flusso Principale}:
          \begin{enumerate}
              \item Il sistema mostra a video il nome del processo.
          \end{enumerate}
\end{itemize}


\usecase{documentiProcesso}{Visualizza elenco dei documenti di un processo}
\begin{figure}[H]
    \centering
    \includegraphics[width=0.9\textwidth]{../assets/uml/UC5.png}
    \caption{UC5 - Visualizza elenco dei documenti di un processo}
    \label{fig:uc_documentiProcesso}
\end{figure}
\begin{itemize}
    \item \textbf{Attore Primario}: Utente
    \item \textbf{Precondizioni}: 
    \begin{itemize}
        \item L'utente ha avviato l'applicazione
        \item L'Utente ha selezionato una delle Classi Documentali del DIP
        \item L'Utente ha selezionato un processo associato alla classe documentale selezionata
    \end{itemize}
    \item \textbf{Postcondizioni}: L'Utente visualizza l'elenco dei documenti associati al processo selezionato.
    \item \textbf{Flusso Principale}:
    \begin{enumerate}
        \item Il sistema mostra a video l'elenco dei documenti della classe documentale selezionata.
    \end{enumerate}
    \item \textbf{Flusso Alternativo}:
    \begin{itemize}
        \item Se il processo non contiene documenti, il sistema mostra un elenco vuoto.
    \end{itemize}
    \item \textbf{Estensioni}: \ref{elencoVuoto} Visualizza elenco vuoto
\end{itemize}

\usecase{documentoProcesso}{Visualizza documento in elenco}
\begin{figure}[H]
    \centering
    \includegraphics[width=0.6\textwidth]{../assets/uml/UC6.png}
    \caption{UC6 - Visualizza documento in elenco}
    \label{fig:uc_documentoProcesso}
\end{figure}
\begin{itemize}
    \item \textbf{Attore Primario}: Utente
    \item \textbf{Precondizioni}: 
    \begin{itemize}
        \item L'utente ha avviato l'applicazione
        \item L'Utente ha selezionato una delle Classi Documentali del DIP
        \item L'Utente ha selezionato un processo associato alla classe documentale selezionata
    \end{itemize}
    \item \textbf{Postcondizioni}: L'Utente visualizza l'elenco delle classi documentali.
    \item \textbf{Flusso Principale}:
    \item Per ogni documento presente nell'elenco viene mostrato:
        \begin{itemize}
            \item Nome del documento (\ref{nomeDocumentoProcesso})
            \item Stato di verifica del documento (\ref{statoVerificaElemento})
            \item Marcatura temporale del documento (\ref{marcaturaTemporaleElemento})
        \end{itemize}
    \item \textbf{Inclusioni}:
    \begin{itemize}
        \item \ref{nomeDocumentoProcesso} Visualizza Nome Documento
        \item \ref{statoVerificaElemento} Visualizza Stato di Verifica Elemento
        \item \ref{marcaturaTemporaleElemento} Visualizza Marcatura Temporale Elemento
    \end{itemize}
\end{itemize}
\begin{figure}[H]
    \centering
    \includegraphics[width=0.5\textwidth]{../assets/uml/UC6INC.png}
    \caption{Inclusioni UC6 - Visualizza elenco dei documenti di un processo}
    \label{fig:inclusioniDocumentoProcesso}
\end{figure}

\subusecase{nomeDocumentoProcesso}{Visualizza Nome Documento}
\begin{itemize}
    \item \textbf{Attore Primario}: Utente
    \item \textbf{Precondizioni}: L'Utente ha avviato l'applicazione.
    \item \textbf{Postcondizioni}: L'Utente visualizza il nome del documento.
    \item \textbf{Flusso Principale}:
    \begin{enumerate}
        \item Il sistema mostra a video il nome del documento.
    \end{enumerate}
\end{itemize}

\usecase{elencoVuoto}{Elenco Vuoto}
\begin{itemize}
    \item \textbf{Attore Primario}: Utente
    \item \textbf{Precondizioni}: \begin{itemize}
        \item L'utente ha avviato l'applicazione
        \item L'Utente sta visualizzando un elenco di elementi del DIP, che risulta vuoto.
    \end{itemize}
    \item \textbf{Postcondizioni}: L'utente viene notificato che l'elenco è vuoto.
    \item \textbf{Flusso Principale}: 
          \begin{enumerate}
              \item Il sistema mostra a video un messaggio che indica che l'elenco è vuoto.
          \end{enumerate}
\end{itemize}

\usecase{selezioneElemento}{Selezione Elemento del DIP}
\begin{figure}[H]
    \centering
    \includegraphics[width=0.6\textwidth]{../assets/uml/UC8.png}
    \caption{UC8 - Selezione Elemento del DIP}
    \label{fig:uc_selezioneElemento}
\end{figure}
\begin{itemize}
    \item \textbf{Attore Primario}: Utente
    \item \textbf{Precondizioni}: 
        \begin{itemize}
            \item L'utente ha avviato l'applicazione
            \item L'utente sta visualizzando una lista di elementi del DIP (Classi Documentali, Processi o Documenti)
        \end{itemize}
    \item \textbf{Postcondizioni}: Viene selezionato l'elemento scelto dall'utente.
    \item \textbf{Flusso Principale}:
    \begin{enumerate}
              \item Il sistema seleziona l'elemento scelto dall'utente.
    \end{enumerate}
\end{itemize}

\usecase{selezionaClasseDocumentale}{Seleziona classe documentale}
\begin{figure}[H]
    \centering
    \includegraphics[width=0.6\textwidth]{../assets/uml/UC9.png}
    \caption{UC9 - Seleziona classe documentale}
    \label{fig:uc_selezionaClasseDocumentale}
\end{figure}
\begin{itemize}
    \item \textbf{Attore Primario}: Utente
    \item \textbf{Precondizioni}:
    \begin{itemize}
        \item L'utente ha avviato l'applicazione
        \item L'utente sta visualizzando una lista di classe documentale
    \end{itemize}
    \item \textbf{Postcondizioni}:
    \begin{itemize}
        \item Viene selezionata una classe documentale sulla quale effettuare un operazione
    \end{itemize}
    \item \textbf{Flusso principale}:
    \begin{enumerate}
        \item L'utente seleziona una classe documentale dalla lista
    \end{enumerate}
\end{itemize}

\usecase{selezionaProcesso}{Seleziona processo}
\begin{figure}[H]
    \centering
    \includegraphics[width=0.6\textwidth]{../assets/uml/UC10.png}
    \caption{UC10 - Seleziona processo}
    \label{fig:uc_selezionaProcesso}
\end{figure}
\begin{itemize}
    \item \textbf{Attore Primario}: Utente
    \item \textbf{Precondizioni}:
    \begin{itemize}
        \item L'utente ha avviato l'applicazione
        \item L'utente sta visualizzando una lista di processi
    \end{itemize}
    \item \textbf{Postcondizioni}:
    \begin{itemize}
        \item Viene selezionato un processo sul quale effettuare un operazione
    \end{itemize}
    \item \textbf{Flusso principale}:
    \begin{enumerate}
        \item L'utente seleziona un processo dalla lista
    \end{enumerate}
\end{itemize}

\usecase{selezionaDocumento}{Seleziona documento}
\begin{figure}[H]
    \centering
    \includegraphics[width=0.6\textwidth]{../assets/uml/UC11.png}
    \caption{UC11 - Seleziona documento}
    \label{fig:uc_selezionaDocumento}
\end{figure}
\begin{itemize}
    \item \textbf{Attore Primario}: Utente
    \item \textbf{Precondizioni}:
    \begin{itemize}
        \item L'utente ha avviato l'applicazione
        \item L'utente sta visualizzando una lista di documenti
    \end{itemize}
    \item \textbf{Postcondizioni}:
    \begin{itemize}
        \item Viene selezionato un documento sul quale effettuare un operazione
    \end{itemize}
    \item \textbf{Flusso principale}:
    \begin{enumerate}
        \item L'utente seleziona un documento dalla lista
    \end{enumerate}
\end{itemize}

% questo non serve più per quello che ha detto cardin il 9/01,che non c'è use case selezione multipla ma direttamente "download file multipli" o "stampa file multipli"
%\usecase{selezionaPiuDocumenti}{Seleziona più documenti} 
%\begin{itemize}
%    \item \textbf{Attore Primario}: Utente
%    \item \textbf{Precondizioni}:
%    \begin{itemize}
%        \item L'utente ha avviato l'applicazione
%        \item L'utente sta visualizzando una lista di documenti
%    \end{itemize}
%    \item \textbf{Postcondizioni}:
%    \begin{itemize}
%        \item Viene selezionati più documenti sul quale effettuare un operazione
%    \end{itemize}
%    \item \textbf{Flusso principale}:
%    \begin{itemize}
%        \item L'utente seleziona un insieme di documenti dalla lista
%    \end{itemize}
%\end{itemize}



% \subusecase{informazioniElemento}{Visualizza Informazioni di un elemento}
% \begin{itemize}
%     \item \textbf{Attore Primario}: Utente
%     \item \textbf{Precondizioni}: L'Utente ha selezionato un elemento del DIP.
%     \item \textbf{Postcondizioni}: L'Utente visualizza i metadati e le azioni disponibili per l'elemento selezionato tra Documento, Processo, Classe Documentale e DIP.
%     \item \textbf{Flusso Principale}:\begin{enumerate}
%               \item L'Utente seleziona una cartella (Classe Documentale o Processo) o un documento dalla struttura del DIP (\ref{strutturaDIP}).
%               \item Il sistema mostra i metadati associati, tra cui:
%                     \begin{itemize}
%                       \item Verifica l'integrità (\ref{verificaIntegritaDIPCompleto}, \ref{verificaIntegritaProcesso}, \ref{verificaIntegritaClasseDocumentale})
%                       \item Stampa (\ref{stampaSingoloDoc} e \ref{stampaInsiemeDoc})
%                       \item Salva in locale (\ref{scaricaFile} e \ref{salvaPiuDOcs})
%                       \item Visualizza l'anteprima (\ref{anteprimaDocumento})
%                       \item Visualizza Informazioni sull'AIP (\ref{visualizzaInfoAiP})
%                       \item Visualizza tutti i metadati associati (\ref{ })
%                     \end{itemize}
%           \end{enumerate}
%     \item \textbf{Flussi Alternativi}: \begin{itemize}
%               \item Se l'Utente seleziona un elemento non valido, il sistema non mostra alcuna informazione.
%               \item L'Utente esegue un'azione sull'elemento selezionato:
%                 \begin{itemize}
%           
%                 \end{itemize}
%           \end{itemize}
%     \item \textbf{Inclusioni}: \ref{nomeElemento}, \ref{statoVerificaElemento}, 
%     \item \textbf{Estensioni}: \ref{verificaIntegritaDIPCompleto}, \ref{verificaIntegritaProcesso}, \ref{verificaIntegritaClasseDocumentale}, \ref{stampaSingoloDoc}, \ref{stampaInsiemeDoc}, \ref{scaricaFile}, \ref{salvaPiuDOcs}, \ref{anteprimaDocumento}, \ref{visualizzaInfoAiP}
% \end{itemize}

\usecase{anteprimaDocumento}{Visualizza Anteprima Documento}
\begin{figure}[H]
    \centering
    \includegraphics[width=0.9\textwidth]{../assets/uml/UC12.png}
    \caption{UC12 - Visualizza Anteprima Documento}
    \label{fig:uc_anteprimaDocumento}
\end{figure}
\begin{itemize}
    \item \textbf{Attore Primario}: Utente
    \item \textbf{Precondizioni}: 
    \begin{itemize}
        \item L'utente ha avviato l'applicazione
        \item L'Utente ha selezionato un documento dalla struttura del DIP.
    \end{itemize}
    \item \textbf{Postcondizioni}: L'Utente visualizza un'anteprima del documento selezionato.
    \item \textbf{Flusso Principale}:
    \begin{enumerate}
        \item L'Utente seleziona l'opzione per visualizzare l'anteprima del documento.
        \item Il sistema apre una finestra di anteprima che mostra il contenuto del documento.
    \end{enumerate}
    \item \textbf{Flusso Alternativo}:
    \begin{itemize}
        \item Se il formato del documento non è supportato per l'anteprima, il sistema mostra un messaggio che specifica perché il tipo non è supportato.
    \end{itemize}
    \item \textbf{Estensioni}: \ref{formatoDocumentoNonSupportato} Formato Documento non Supportato dal Sistema per l'anteprima
\end{itemize}

\usecase{formatoDocumentoNonSupportato}{Formato Documento non Supportato dal Sistema per l'anteprima}
\begin{itemize}
    \item \textbf{Attore Primario}: Utente
    \item \textbf{Precondizioni}: 
    \begin{itemize}
        \item L'utente ha avviato l'applicazione
        \item L'Utente ha selezionato un documento
        \item L'Utente ha selezionato l'opzione di Visualizza anteprima per il documento selezionato.
        \item Il formato del documento selezionato non è supportato per l'anteprima.
    \end{itemize}
    \item \textbf{Postcondizioni}: 
    \begin{itemize}
        \item Non viene mostrata l'anteprima del documento.
        \item Viene mostrato un messaggio di errore che indica che il formato del documento non è supportato per l'anteprima.
    \end{itemize}
    \item \textbf{Flusso Principale}:
    \begin{enumerate}
        \item Il sistema chiude annulla l'operazione di anteprima.
        \item Il sistema mostra a video un messaggio di errore che indica che il formato del documento non è supportato per l'anteprima.
    \end{enumerate}
\end{itemize}


\subsubsection{UC-2 - Ricerca Nel DIP}

\subsubsection{UC-2.1 - Ricercare Classe Documentale con filtri}
\begin{itemize}
      \item \textbf{Attore Primario}: utente
      \item \textbf{Precondizioni}: L'utente sta visualizzando la lista delle Classi Documentali.
      \item \textbf{Postcondizioni}: Viene visualizzato l'elenco delle Classi Documentali corrispondenti ai filtri applicati.
      \item \textbf{Flusso Principale}:
            \begin{enumerate}
                  \item L'utente inserisce i filtri per la ricerca secondo i metadati delle Classi
                        Documentali.
                  \item Il sistema mostra i risultati della ricerca (UC-2.4).
            \end{enumerate}
      \item \textbf{Flusso Alternativo (Nessun risultato)}: La ricerca non produce risultati e l'utente visualizza un elenco vuoto.
\end{itemize}

\subsubsection{UC-2.2 - Ricercare Processo con filtri}
\begin{itemize}
      \item \textbf{Attore Primario}: utente
      \item \textbf{Precondizioni}: L'utente sta visualizzando la lista dei Processi.
      \item \textbf{Postcondizioni}: Viene visualizzato l'elenco dei Processi corrispondenti ai filtri applicati.
      \item \textbf{Flusso Principale}:
            \begin{enumerate}
                  \item L'utente inserisce i filtri per la ricerca secondo i metadati dei Processi.
                  \item Il sistema mostra i risultati della ricerca (UC-2.4).
            \end{enumerate}
      \item \textbf{Flusso Alternativo (Nessun risultato)}: La ricerca non produce risultati e l'utente visualizza un elenco vuoto.
\end{itemize}

\subsubsection{UC-2.3 - Ricercare Documento con filtri}
\begin{itemize}
      \item \textbf{Attore Primario}: utente
      \item \textbf{Precondizioni}: L'utente ha accesso alla funzionalità di ricerca documentale.
      \item \textbf{Postcondizioni}: Visualizzazione di un elenco di documenti che corrispondono ai criteri inseriti.
      \item \textbf{Flusso Principale}:
            \begin{enumerate}
                  \item L'utente imposta uno o più filtri per la ricerca, scegliendo tra campi comuni
                        (UC-2.3.1) o campi specifici per tipo documentale (UC-2.3.2).
                  \item L'utente avvia la ricerca.
                  \item Il sistema mostra i risultati della ricerca (UC-2.4).
            \end{enumerate}
      \item \textbf{Flusso Alternativo (Nessun risultato)}: La ricerca non produce risultati e il sistema informa l'utente.
\end{itemize}

\subsubsection{UC-2.3.1 - Ricerca per Campi Comuni}
\begin{itemize}
      \item \textbf{Attore Primario}: utente
      \item \textbf{Precondizioni}: L'utente si trova nella schermata di ricerca documentale.
      \item \textbf{Postcondizioni}: I risultati della ricerca vengono filtrati in base ai campi comuni specificati.
      \item \textbf{Flusso Principale}:
            \begin{enumerate}
                  \item L'utente seleziona uno o più campi comuni su cui basare la ricerca (da
                        UC-2.3.1.1 a UC-2.3.1.5).
                  \item L'utente compila i valori per i campi selezionati.
                  \item Il sistema utilizza questi parametri per la ricerca.
            \end{enumerate}
      \item \textbf{Flusso Alternativo (Nessun risultato)}: La ricerca non produce risultati.
\end{itemize}

\subsubsection{UC-2.3.1.1 - Ricerca per Chiave Descrittiva}
\begin{itemize}
      \item \textbf{Attore Primario}: utente
      \item \textbf{Precondizioni}: L'utente sta effettuando una ricerca per campi comuni.
      \item \textbf{Postcondizioni}: I risultati della ricerca vengono filtrati in base alla Chiave Descrittiva, Oggetto o Parole Chiave.
      \item \textbf{Flusso Principale}:
            \begin{enumerate}
                  \item L'utente digita nella casella di ricerca i valori per:
                        \begin{itemize}
                              \item Chiave Descrittiva
                              \item Oggetto
                              \item Parole Chiave.
                        \end{itemize}
                  \item Il sistema restituisce la lista dei documenti che corrispondono ai parametri
                        inseriti (\nameref{risultatiRicerca}).
            \end{enumerate}
\end{itemize}

\subsubsection{UC-2.3.1.2 - Ricerca per Soggetti}
\begin{itemize}
      \item \textbf{Attore Primario}: utente
      \item \textbf{Precondizioni}: L'utente sta effettuando una ricerca per campi comuni.
      \item \textbf{Postcondizioni}: I risultati della ricerca vengono filtrati in base ai soggetti specificati.
      \item \textbf{Flusso Principale}:
            \begin{enumerate}
                  \item L'utente decide di aggiungere uno o più soggetti come filtro di ricerca.
                  \item Per ogni soggetto, l'utente ne specifica i dettagli (Ruolo e altri campi
                        pertinenti, come descritto da UC-2.3.1.2.1 a UC-2.3.1.2.7).
                  \item Il sistema restituisce la lista dei documenti che corrispondono ai parametri
                        inseriti (\nameref{risultatiRicerca}).
            \end{enumerate}
\end{itemize}

\subsubsection{UC-2.3.1.2.1 - Ricerca per Dettagli di un Soggetto: Ruolo = PAI}
\begin{itemize}
      \item \textbf{Attore Primario}: utente
      \item \textbf{Precondizioni}: L'utente ha scelto di filtrare la ricerca per Soggetto e ha selezionato il ruolo PAI.
      \item \textbf{Postcondizioni}: I risultati sono filtrati per i dettagli del soggetto PAI.
      \item \textbf{Flusso Principale}:
            \begin{enumerate}
                  \item L'utente inserisce i valori per i campi:
                        \begin{itemize}
                              \item Denominazione Amministrazione/ Codice IPA
                              \item Denominazione Amministrazione AOO/ Codice IPA AOO
                              \item Denominazione Amministrazione UOR/ Codice IPA UOR
                              \item Indirizzi Digitali di Riferimento
                        \end{itemize}
                  \item Il sistema usa questi valori per filtrare la ricerca.
            \end{enumerate}
\end{itemize}

\subsubsection{UC-2.3.1.2.2 - Ricerca per Dettagli di un Soggetto: Ruolo = PAE}
\begin{itemize}
      \item \textbf{Attore Primario}: utente
      \item \textbf{Precondizioni}: L'utente ha scelto di filtrare la ricerca per Soggetto e ha selezionato il ruolo PAE.
      \item \textbf{Postcondizioni}: I risultati sono filtrati per i dettagli del soggetto PAE.
      \item \textbf{Flusso Principale}:
            \begin{enumerate}
                  \item L'utente inserisce i valori per i campi:
                        \begin{itemize}
                              \item Denominazione Amministrazione
                              \item Denominazione Ufficio
                              \item Indirizzi Digitali di Riferimento
                        \end{itemize}
                  \item Il sistema usa questi valori per filtrare la ricerca.
            \end{enumerate}
\end{itemize}

\subsubsection{UC-2.3.1.2.3 - Ricerca per Dettagli di un Soggetto: Ruolo = AS}
\begin{itemize}
      \item \textbf{Attore Primario}: utente
      \item \textbf{Precondizioni}: L'utente ha scelto di filtrare la ricerca per Soggetto e ha selezionato il ruolo AS.
      \item \textbf{Postcondizioni}: I risultati sono filtrati per i dettagli del soggetto AS.
      \item \textbf{Flusso Principale}:
            \begin{enumerate}
                  \item L'utente inserisce i valori per i campi:
                        \begin{itemize}
                              \item Cognome
                              \item Nome
                              \item Codice Fiscale
                              \item Denominazione Amministrazione/ Codice IPA
                              \item Denominazione Amministrazione AOO/ Codice IPA AOO
                              \item Denominazione Amministrazione UOR/ Codice IPA UOR
                              \item Indirizzi Digitali di Riferimento
                        \end{itemize}
                  \item Il sistema usa questi valori per filtrare la ricerca.
            \end{enumerate}
\end{itemize}

\subsubsection{UC-2.3.1.2.4 - Ricerca per Dettagli di un Soggetto: Ruolo = PG}
\begin{itemize}
      \item \textbf{Attore Primario}: utente
      \item \textbf{Precondizioni}: L'utente ha scelto di filtrare la ricerca per Soggetto e ha selezionato il ruolo PG.
      \item \textbf{Postcondizioni}: I risultati sono filtrati per i dettagli del soggetto PG.
      \item \textbf{Flusso Principale}:
            \begin{enumerate}
                  \item L'utente inserisce i valori per i campi:
                        \begin{itemize}
                              \item Denominazione Organizzazione
                              \item Codice fiscale / Partita Iva
                              \item Denominazione Ufficio
                              \item Indirizzi Digitali di Riferimento
                        \end{itemize}
                  \item Il sistema usa questi valori per filtrare la ricerca.
            \end{enumerate}
\end{itemize}

\subsubsection{UC-2.3.1.2.5 - Ricerca per Dettagli di un Soggetto: Ruolo = PF}
\begin{itemize}
      \item \textbf{Attore Primario}: utente
      \item \textbf{Precondizioni}: L'utente ha scelto di filtrare la ricerca per Soggetto e ha selezionato il ruolo PF.
      \item \textbf{Postcondizioni}: I risultati sono filtrati per i dettagli del soggetto PF.
      \item \textbf{Flusso Principale}:
            \begin{enumerate}
                  \item  L'utente inserisce i valori per i campi:
                        \begin{itemize}
                              \item Cognome
                              \item Nome
                              \item Indirizzi Digitali di Riferimento
                        \end{itemize}
                  \item Il sistema usa questi valori per filtrare la ricerca.
            \end{enumerate}
\end{itemize}

\subsubsection{UC-2.3.1.2.6 - Ricerca per Dettagli di un Soggetto: Ruolo = RUP}
\begin{itemize}
      \item \textbf{Attore Primario}: utente
      \item \textbf{Precondizioni}: L'utente ha scelto di filtrare la ricerca per Soggetto e ha selezionato il ruolo RUP.
      \item \textbf{Postcondizioni}: I risultati sono filtrati per i dettagli del soggetto RUP.
      \item \textbf{Flusso Principale}:
            \begin{enumerate}
                  \item L'utente inserisce i valori per i campi:
                        \begin{itemize}
                              \item Cognome
                              \item Nome
                              \item Denominazione Amministrazione/ Codice IPA
                              \item Denominazione Amministrazione AOO/ Codice IPA AOO
                              \item Denominazione Amministrazione UOR/ Codice IPA UOR
                              \item Indirizzi Digitali di Riferimento
                        \end{itemize}
                  \item Il sistema usa questi valori per filtrare la ricerca.
            \end{enumerate}
\end{itemize}

\subsubsection{UC-2.3.1.2.7 - Ricerca per Dettagli di un Soggetto: Ruolo = SW}
\begin{itemize}
      \item \textbf{Attore Primario}: utente
      \item \textbf{Precondizioni}: L'utente ha scelto di filtrare la ricerca per Soggetto e ha selezionato il ruolo SW.
      \item \textbf{Postcondizioni}: I risultati sono filtrati per i dettagli del soggetto SW.
      \item \textbf{Flusso Principale}:
            \begin{enumerate}
                  \item L'utente inserisce la Denominazione Sistema
                  \item Il sistema usa questi valori per filtrare la ricerca.
            \end{enumerate}
\end{itemize}

\subsubsection{UC-2.3.1.3 - Ricerca per Classificazione}
\begin{itemize}
      \item \textbf{Attore Primario}: utente
      \item \textbf{Precondizioni}: L'utente sta effettuando una ricerca per campi comuni.
      \item \textbf{Postcondizioni}: I risultati della ricerca vengono filtrati in base alla Classificazione.
      \item \textbf{Flusso Principale}:
            \begin{enumerate}
                  \item L'utente compila i parametri di Classificazione tra:
                        \begin{itemize}
                              \item Indice di Classificazione
                              \item Descrizione
                              \item Piano di Fascicolo
                        \end{itemize}.
                  \item Viene restituita la lista dei documenti nei quali la Classificazione, per
                        qualcuno dei parametri cercati, corrisponde al parametro ricercato
                        (\nameref{risultatiRicerca}).
            \end{enumerate}
\end{itemize}

\subsubsection{UC-2.3.1.4 - Ricerca per Tempo di Conservazione}
\begin{itemize}
      \item \textbf{Attore Primario}: utente
      \item \textbf{Precondizioni}: L'utente sta effettuando una ricerca per campi comuni.
      \item \textbf{Postcondizioni}: I risultati della ricerca vengono filtrati in base al Tempo di Conservazione.
      \item \textbf{Flusso Principale}:
            \begin{enumerate}
                  \item L'utente digita nella casella di ricerca il tempo di conservazione o seleziona
                        "Perenne".
                  \item Il sistema restituisce la lista dei documenti che corrispondono al parametro
                        inserito (\nameref{risultatiRicerca}).
            \end{enumerate}
\end{itemize}

\subsubsection{UC-2.3.1.5 - Ricerca per Note}
\begin{itemize}
      \item \textbf{Attore Primario}: utente
      \item \textbf{Precondizioni}: L'utente sta effettuando una ricerca per campi comuni.
      \item \textbf{Postcondizioni}: I risultati della ricerca vengono filtrati in base al contenuto delle Note.
      \item \textbf{Flusso Principale}:
            \begin{enumerate}
                  \item L'utente digita nella casella di ricerca il testo della Nota.
                  \item Il sistema restituisce la lista dei documenti che corrispondono al parametro
                        inserito (\nameref{risultatiRicerca}).
            \end{enumerate}
\end{itemize}

\subsubsection{UC-2.3.2 - Ricerca per Tipo Documentale}
\begin{itemize}
      \item \textbf{Attore Primario}: utente
      \item \textbf{Precondizioni}: L'utente si trova nella schermata di ricerca documentale.
      \item \textbf{Postcondizioni}: Vengono mostrati campi di ricerca specifici per il tipo documentale scelto.
      \item \textbf{Flusso Principale}:
            \begin{enumerate}
                  \item L'utente imposta la ricerca per tipo e seleziona tra: Aggregazione Documentale,
                        Documento Informatico, Documento Amministrativo Informatico.
                  \item Il sistema presenta i filtri specifici per il tipo selezionato (descritto negli
                        UC da 2.3.2.1 a 2.3.2.3).
            \end{enumerate}
      \item \textbf{Flusso Alternativo}: Il tipo documentale cercato non è valido o non è presente.
\end{itemize}

\subsubsection{UC-2.3.2.1 - Ricerca per Campi di Documento Informatico e Amministrativo Informatico}
\begin{itemize}
      \item \textbf{Descrizione}: Questo caso d'uso raggruppa i filtri applicabili sia al Documento Informatico sia al Documento Amministrativo Informatico.
\end{itemize}

\subsubsection{UC-2.3.2.1.1 - Ricerca per Dati di Registrazione}
\begin{itemize}
      \item \textbf{Attore Primario}: utente
      \item \textbf{Precondizioni}: L'utente ha selezionato "Documento Informatico" o "Documento Amministrativo Informatico" come tipo documentale.
      \item \textbf{Postcondizioni}: I risultati sono filtrati per Dati di Registrazione.
      \item \textbf{Flusso Principale}:
            \begin{enumerate}
                  \item L'utente sceglie la Tipologia di Flusso tra: Uscita, Entrata, Interno.
                  \item L'utente sceglie il Tipo di Registro tra: Nessuno, Protocollo
                        Ordinario/Protocollo di Emergenza, Repertorio/Registro.
                  \item L'utente inserisce la Data/Ora di Registrazione (nel caso di un documento
                        protocollato tali parametri fanno riferimento alla protocollazione).
                  \item L'utente inserisce il codice identificativo del Registro.
                  \item Il sistema restituisce la lista dei documenti corrispondenti
                        (\nameref{risultatiRicerca}).
            \end{enumerate}
\end{itemize}

\subsubsection{UC-2.3.2.1.2 - Ricerca per Tipologia Documentale}
\begin{itemize}
      \item \textbf{Attore Primario}: utente
      \item \textbf{Precondizioni}: L'utente ha selezionato "Documento Informatico" o "Documento Amministrativo Informatico".
      \item \textbf{Postcondizioni}: I risultati sono filtrati per Tipologia Documentale.
      \item \textbf{Flusso Principale}:
            \begin{enumerate}
                  \item L'utente inserisce la Tipologia Documentale (es. fatture, delibere, determine).
                  \item Il sistema restituisce la lista dei documenti corrispondenti
                        (\nameref{risultatiRicerca}).
            \end{enumerate}
\end{itemize}

\subsubsection{UC-2.3.2.1.3 - Ricerca per Modalità di Formazione}
\begin{itemize}
      \item \textbf{Attore Primario}: utente
      \item \textbf{Precondizioni}: L'utente ha selezionato "Documento Informatico" o "Documento Amministrativo Informatico".
      \item \textbf{Postcondizioni}: I risultati sono filtrati per Modalità di Formazione.
      \item \textbf{Flusso Principale}:
            \begin{enumerate}
                  \item L'utente sceglie la Modalità di Formazione tra:
                        \begin{itemize}
                              \item creazione tramite l'utilizzo di strumenti software che assicurino la produzione
                                    di documenti nei formati previsti nell’Allegato 2 delle Linee Guida
                              \item acquisizione di un documento informatico per via telematica o su supporto
                                    informatico, acquisizione della copia per immagine su supporto informatico di
                                    un documento analogico, acquisizione della copia informatica di un documento
                                    analogico
                              \item memorizzazione su supporto informatico in formato digitale delle informazioni
                                    risultanti da transazioni o processi informatici o dalla presentazione
                                    telematica di dati attraverso moduli o formulari resi disponibili all’utente;
                              \item generazione o raggruppamento anche in via automatica di un insieme di dati o
                                    registrazioni, provenienti da una o più banche dati, anche appartenenti a più
                                    soggetti interoperanti, secondo una struttura logica predeterminata e
                                    memorizzata in forma statica.
                        \end{itemize}
                  \item Il sistema restituisce la lista dei documenti corrispondenti
                        (\nameref{risultatiRicerca}).
            \end{enumerate}
\end{itemize}

\subsubsection{UC-2.3.2.1.4 - Ricerca per campo Riservato}
\begin{itemize}
      \item \textbf{Attore Primario}: utente
      \item \textbf{Precondizioni}: L'utente ha selezionato "Documento Informatico" o "Documento Amministrativo Informatico".
      \item \textbf{Postcondizioni}: I risultati sono filtrati in base al metadato "Riservato".
      \item \textbf{Flusso Principale}:
            \begin{enumerate}
                  \item L'utente sceglie se il file è Riservato o meno.
                  \item Il sistema restituisce la lista dei documenti corrispondenti
                        (\nameref{risultatiRicerca}).
            \end{enumerate}
\end{itemize}

\subsubsection{UC-2.3.2.1.5 - Ricerca per Identificativo di Formato}
\begin{itemize}
      \item \textbf{Attore Primario}: utente
      \item \textbf{Precondizioni}: L'utente ha selezionato "Documento Informatico" o "Documento Amministrativo Informatico".
      \item \textbf{Postcondizioni}: I risultati sono filtrati per Identificativo di Formato.
      \item \textbf{Flusso Principale}:
            \begin{enumerate}
                  \item L'utente sceglie la Tipologia di Formato all'interno di quelli Previsti dalle
                        linee Guida.
                  \item L'utente inserisce il Nome del Prodotto utilizzato per la creazione del
                        Documento.
                  \item L'utente inserisce la Versione del Prodotto utilizzato per la creazione del
                        Documento
                  \item L'utente inserisce il Produttore del Prodotto utilizzato per la creazione del
                        Documento.
                  \item Il sistema restituisce la lista dei documenti corrispondenti
                        (\nameref{risultatiRicerca}).
            \end{enumerate}
\end{itemize}

\subsubsection{UC-2.3.2.1.6 - Ricerca per Dati di Verifica}
\begin{itemize}
      \item \textbf{Attore Primario}: utente
      \item \textbf{Precondizioni}: L'utente ha selezionato "Documento Informatico" o "Documento Amministrativo Informatico".
      \item \textbf{Postcondizioni}: I risultati sono filtrati in base ai dati di verifica.
      \item \textbf{Flusso Principale}:
            \begin{enumerate}
                  \item L'utente sceglie se il file è Firmato Digitalmente o meno.
                  \item L'utente sceglie se il file è Sigillato Elettronicamente o meno.
                  \item L'utente sceglie se il file ha una Marcatura Temporale o meno.
                  \item L'utente sceglie se vi è conformità copie immagine su supporto informatico o
                        meno.
                  \item Il sistema restituisce la lista dei documenti corrispondenti
                        (\nameref{risultatiRicerca}).
            \end{enumerate}
\end{itemize}

\subsubsection{UC-2.3.2.1.7 - Ricerca per Nome del Documento}
\begin{itemize}
      \item \textbf{Attore Primario}: utente
      \item \textbf{Precondizioni}: L'utente ha selezionato "Documento Informatico" o "Documento Amministrativo Informatico".
      \item \textbf{Postcondizioni}: I risultati sono filtrati per Nome del Documento.
      \item \textbf{Flusso Principale}:
            \begin{enumerate}
                  \item L'utente inserisce il Nome del Documento.
                  \item Il sistema restituisce la lista dei documenti corrispondenti
                        (\nameref{risultatiRicerca}).
            \end{enumerate}
\end{itemize}

\subsubsection{UC-2.3.2.1.8 - Ricerca per Versione del Documento}
\begin{itemize}
      \item \textbf{Attore Primario}: utente
      \item \textbf{Precondizioni}: L'utente ha selezionato "Documento Informatico" o "Documento Amministrativo Informatico".
      \item \textbf{Postcondizioni}: I risultati sono filtrati per Versione del Documento.
      \item \textbf{Flusso Principale}:
            \begin{enumerate}
                  \item L'utente inserisce la Versione del Documento.
                  \item Il sistema restituisce la lista dei documenti corrispondenti
                        (\nameref{risultatiRicerca}).
            \end{enumerate}
\end{itemize}

\subsubsection{UC-2.3.2.1.9 - Ricerca per Identificativo del Documento Primario}
\begin{itemize}
      \item \textbf{Attore Primario}: utente
      \item \textbf{Precondizioni}: L'utente ha selezionato "Documento Informatico" o "Documento Amministrativo Informatico".
      \item \textbf{Postcondizioni}: I risultati sono filtrati per Identificativo del Documento Primario.
      \item \textbf{Flusso Principale}:
            \begin{enumerate}
                  \item L'utente inserisce l'Identificativo del Documento Primario.
                  \item Il sistema restituisce la lista dei documenti corrispondenti
                        (\nameref{risultatiRicerca}).
            \end{enumerate}
\end{itemize}

\subsubsection{UC-2.3.2.1.10 - Ricerca per Tracciature Modifiche di Documento}
\begin{itemize}
      \item \textbf{Attore Primario}: utente
      \item \textbf{Precondizioni}: L'utente ha selezionato "Documento Informatico" o "Documento Amministrativo Informatico".
      \item \textbf{Postcondizioni}: I risultati sono filtrati per le tracciature di modifica.
      \item \textbf{Flusso Principale}:
            \begin{enumerate}
                  \item L'utente sceglie il Tipo di modifica tra:
                        \begin{itemize}
                              \item Annullamento
                              \item Registrazione
                              \item Integrazione
                              \item Annotazione
                        \end{itemize}.
                  \item L'Utente sceglie il Soggetto che ha effettuato la modifica.
                  \item L'Utente inserisce la Data/Ora della Modifica.
                  \item L'Utente inserisce l'identificativo documento versione precedente.
                  \item Il sistema restituisce la lista dei documenti corrispondenti
                        (\nameref{risultatiRicerca}).
            \end{enumerate}
\end{itemize}

\subsubsection{UC-2.3.2.2 - Ricerca per Campi di Documento Informatico}
\begin{itemize}
      \item \textbf{Descrizione}: Questo caso d'uso raggruppa i filtri applicabili esclusivamente al Documento Informatico.
\end{itemize}

\subsubsection{UC-2.3.2.2.1 - Ricerca per Soggetti del Documento Informatico}
\begin{itemize}
      \item \textbf{Attore Primario}: utente
      \item \textbf{Precondizioni}: L'utente ha scelto come tipo di documento il Documento Informatico.
      \item \textbf{Postcondizioni}: I risultati sono filtrati in base ai soggetti specifici.
      \item \textbf{Flusso Principale}:
            \begin{enumerate}
                  \item L'Utente sceglie il Ruolo e i sottocampi specifici tra:
                        \begin{itemize}
                              \item Assegnatario (AS)
                              \item Autore (PF, PG, PAI valido solo nei flussi in entrata)
                              \item Destinatario (PF, PG, PAI valido solo come mittente nei flussi in entrata, come
                                    destinatario nei flussi in uscita, PAE valido solo come mittente nei flussi in
                                    entrata, come destinatario nei flussi in uscita)
                              \item Mittente (PF, PG, PAI valido solo come mittente nei flussi in entrata, come
                                    destinatario nei flussi in uscita, PAE valido solo come mittente nei flussi in
                                    entrata, come destinatario nei flussi in uscita)
                              \item Operatore (PF)
                              \item Produttore (SW)
                              \item RGD (PF)
                              \item RSP (PF)
                              \item Soggetto che effettua la registrazione (PF, PG)
                              \item Altro(PF, PG, PAI valido solo come mittente nei flussi in entrata, come
                                    destinatario nei flussi in uscita, PAE valido solo come mittente nei flussi in
                                    entrata, come destinatario nei flussi in uscita)
                        \end{itemize}
                  \item Per ogni ruolo, l'utente può specificare ulteriori dettagli (UC-2.3.1.2).
                  \item Il sistema restituisce la lista dei documenti corrispondenti
                        (\nameref{risultatiRicerca}).
            \end{enumerate}
\end{itemize}

\subsubsection{UC-2.3.2.3 - Ricerca per Campi di Documento Amministrativo Informatico}
\begin{itemize}
      \item \textbf{Descrizione}: Questo caso d'uso raggruppa i filtri applicabili esclusivamente al Documento Amministrativo Informatico.
\end{itemize}

\subsubsection{UC-2.3.2.3.1 - Ricerca per Soggetti del Documento Amministrativo Informatico}
\begin{itemize}
      \item \textbf{Attore Primario}: utente
      \item \textbf{Precondizioni}: L'utente ha scelto come tipo di documento il Documento Amministrativo Informatico.
      \item \textbf{Postcondizioni}: I risultati sono filtrati in base ai soggetti specifici.
      \item \textbf{Flusso Principale}:
            \begin{enumerate}
                  \item L'utente sceglie il Ruolo tra quelli previsti per il Documento Amministrativo
                        Informatico (Amministratore, Assegnatario, etc.).
                  \item Per ogni ruolo, l'utente può specificare ulteriori dettagli (UC-2.3.1.2).
                  \item Il sistema restituisce la lista dei documenti corrispondenti
                        (\nameref{risultatiRicerca}).
            \end{enumerate}
\end{itemize}

\subsubsection{UC-2.3.2.4 - Ricerca per Campi di Aggregazione Documentale}
\begin{itemize}
      \item \textbf{Descrizione}: Questo caso d'uso raggruppa i filtri applicabili esclusivamente all'Aggregazione Documentale.
\end{itemize}

\subsubsection{UC-2.3.2.4.1 - Ricerca per Tipo di Aggregazione}
\begin{itemize}
      \item \textbf{Attore Primario}: utente
      \item \textbf{Precondizioni}: L'utente ha scelto come tipo di documento l'Aggregazione Documentale.
      \item \textbf{Postcondizioni}: I risultati sono filtrati per Tipo di Aggregazione.
      \item \textbf{Flusso Principale}:
            \begin{enumerate}
                  \item L'utente sceglie tra Fascicolo, Serie Documentale, Serie di Fascicoli.
                  \item Il sistema restituisce la lista delle aggregazioni corrispondenti
                        (\nameref{risultatiRicerca}).
            \end{enumerate}
\end{itemize}

\subsubsection{UC-2.3.2.4.2 - Ricerca per Id dell'Aggregazione}
\begin{itemize}
      \item \textbf{Attore Primario}: utente
      \item \textbf{Precondizioni}: L'utente ha scelto come tipo di documento l'Aggregazione Documentale.
      \item \textbf{Postcondizioni}: I risultati sono filtrati per Id dell'Aggregazione.
      \item \textbf{Flusso Principale}:
            \begin{enumerate}
                  \item L'utente inserisce l'Id dell'Aggregazione.
                  \item Il sistema restituisce la lista delle aggregazioni corrispondenti
                        (\nameref{risultatiRicerca}).
            \end{enumerate}
\end{itemize}

\subsubsection{UC-2.3.2.4.3 - Ricerca per Tipologia di Fascicolo}
\begin{itemize}
      \item \textbf{Attore Primario}: utente
      \item \textbf{Precondizioni}: L'utente sta cercando un'Aggregazione Documentale di tipo "Fascicolo".
      \item \textbf{Postcondizioni}: I risultati sono filtrati per Tipologia di Fascicolo.
      \item \textbf{Flusso Principale}:
            \begin{enumerate}
                  \item L'utente sceglie tra: Affare, Attività, Persona Fisica, Persona Giuridica,
                        Procedimento Amministrativo.
                  \item Il sistema restituisce la lista dei fascicoli corrispondenti
                        (\nameref{risultatiRicerca}).
            \end{enumerate}
\end{itemize}

\subsubsection{UC-2.3.2.4.4 - Ricerca per Id Aggregazione Primario}
\begin{itemize}
      \item \textbf{Attore Primario}: utente
      \item \textbf{Precondizioni}: L'utente ha scelto come tipo di documento l'Aggregazione Documentale.
      \item \textbf{Postcondizioni}: I risultati sono filtrati per Id dell'Aggregazione Primaria.
      \item \textbf{Flusso Principale}:
            \begin{enumerate}
                  \item L'utente inserisce l'Id dell'Aggregazione "padre".
                  \item Il sistema restituisce la lista delle aggregazioni corrispondenti
                        (\nameref{risultatiRicerca}).
            \end{enumerate}
\end{itemize}

\subsubsection{UC-2.3.2.4.5 - Ricerca per Soggetti dell'Aggregazione Documentale}
\begin{itemize}
      \item \textbf{Attore Primario}: utente
      \item \textbf{Precondizioni}: L'utente ha scelto come tipo di documento l'Aggregazione Documentale.
      \item \textbf{Postcondizioni}: I risultati sono filtrati per Soggetti.
      \item \textbf{Flusso Principale}:
            \begin{enumerate}
                  \item L'Utente sceglie il Ruolo e i sottocampi specifici tra:
                        \begin{itemize}
                              \item Amministrazione titolare (PAI)
                              \item Amministrazioni partecipanti (PAI o PAE)
                              \item Assegnatario (AS)
                              \item Soggetto intestatario persona fisica (PF)
                              \item Soggetto intestatario persona giuridica (PG, PAI o PAE)
                              \item RUP
                        \end{itemize}
                  \item Per ogni ruolo, l'utente può specificare ulteriori dettagli (UC-2.3.1.2).
                  \item Il sistema restituisce la lista delle aggregazioni corrispondenti
                        (\nameref{risultatiRicerca}).
            \end{enumerate}
\end{itemize}

\subsubsection{UC-2.3.2.4.6 - Ricerca per Data Apertura}
\begin{itemize}
      \item \textbf{Attore Primario}: utente
      \item \textbf{Precondizioni}: L'utente ha scelto come tipo di documento l'Aggregazione Documentale.
      \item \textbf{Postcondizioni}: I risultati sono filtrati per Data di Apertura.
      \item \textbf{Flusso Principale}:
            \begin{enumerate}
                  \item L'utente inserisce la Data di Apertura.
                  \item Il sistema restituisce la lista delle aggregazioni corrispondenti
                        (\nameref{risultatiRicerca}).
            \end{enumerate}
\end{itemize}

\subsubsection{UC-2.3.2.4.7 - Ricerca per Data Chiusura}
\begin{itemize}
      \item \textbf{Attore Primario}: utente
      \item \textbf{Precondizioni}: L'utente ha scelto come tipo di documento l'Aggregazione Documentale.
      \item \textbf{Postcondizioni}: I risultati sono filtrati per Data di Chiusura.
      \item \textbf{Flusso Principale}:
            \begin{enumerate}
                  \item L'utente inserisce la Data di Chiusura.
                  \item Il sistema restituisce la lista delle aggregazioni corrispondenti
                        (\nameref{risultatiRicerca}).
            \end{enumerate}
\end{itemize}

\subsubsection{UC-2.3.2.4.8 - Ricerca per Procedimento Amministrativo}
\begin{itemize}
      \item \textbf{Attore Primario}: utente
      \item \textbf{Precondizioni}: L'utente ha scelto come tipo di documento l'Aggregazione Documentale.
      \item \textbf{Postcondizioni}: I risultati sono filtrati per i dati del Procedimento Amministrativo.
      \item \textbf{Flusso Principale}:
            \begin{enumerate}
                  \item L'utente inserisce la Materia, Argomento e Struttura per i procedimenti.
                  \item L'utente inserisce il Procedimento come denominazione.
                  \item L'utente inserisce il Catalogo dei Procedimenti come URI di pubblicazione del
                        catalogo.
                  \item  L'utente può aggiungere una o più Fasi come filtro (UC-2.3.2.4.8.1).
                  \item Viene restituita la lista dei documenti nei quali i parametri di un
                        Procedimento amministrativo corrispondono ai parametri ricercati
                        (\nameref{risultatiRicerca}).
            \end{enumerate}
\end{itemize}

\subsubsection{UC-2.3.2.4.8.1 - Ricerca per Fasi del Procedimento}
\begin{itemize}
      \item \textbf{Attore Primario}: utente
      \item \textbf{Precondizioni}: L'utente sta cercando per Procedimento Amministrativo.
      \item \textbf{Postcondizioni}: La ricerca viene filtrata per i dati di una o più fasi.
      \item \textbf{Flusso Principale}:
            \begin{enumerate}
                  \item L'utente aggiunge un filtro per "Fase".
                  \item L'utente inserisce i dati della fase:
                        \begin{itemize}
                              \item Tipo Fase (Preparatoria, Istruttoria, Consultiva, Decisoria o deliberativa,
                                    Integrazione dell'efficacia)
                              \item Data di inizio
                              \item Data di fine
                        \end{itemize}
                  \item L'utente può aggiungere altre fasi.
                  \item Il sistema usa questi parametri per la ricerca.
            \end{enumerate}
\end{itemize}

\subsubsection{UC-2.3.2.4.8.1 - Ricerca per Fasi del Procedimento}
\begin{itemize}
      \item \textbf{Attore Primario}: utente
      \item \textbf{Precondizioni}: L'utente sta cercando per Procedimento Amministrativo.
      \item \textbf{Postcondizioni}: La ricerca viene filtrata per i dati di una o più fasi.
      \item \textbf{Flusso Principale}:
            \begin{enumerate}
                  \item L'utente aggiunge un filtro per "Fase".
                  \item L'utente sceglie il Tipo Fase tra: Preparatoria, Istruttoria, Consultiva,
                        Decisoria o deliberativa, Integrazione dell'efficacia.
                  \item L'utente inserisce la Data di inizio della fase.
                  \item L'utente inserisce la Data di fine della fase, se terminata.
                  \item L'utente può aggiungere altre fasi.
                  \item Il sistema usa questi parametri per la ricerca.

                  \item Viene restituita la lista dei documenti nei quali i parametri di Fase
                        corrispondono ai parametri ricercati (\nameref{risultatiRicerca}).
            \end{enumerate}
\end{itemize}

\subsubsection{UC-2.3.2.4.9 - Ricerca per Assegnazione}
\begin{itemize}
      \item \textbf{Attore Primario}: utente
      \item \textbf{Precondizioni}: L'utente ha scelto come tipo di documento l'Aggregazione Documentale.
      \item \textbf{Postcondizioni}: La ricerca è filtrata per i dati di Assegnazione.
      \item \textbf{Flusso Principale}:
            \begin{enumerate}
                  \item L'utente aggiunge un filtro per "Assegnazione".
                  \item L'utente inserisce i dati:
                        \begin{itemize}
                              \item Tipo Assegnazione
                              \item Soggetto Assegnatario (UC-2.3.)
                              \item Data di Inizio
                              \item Data di Fine
                        \end{itemize}
                  \item L'utente può aggiungere altre assegnazioni.
                  \item Il sistema restituisce la lista delle aggregazioni corrispondenti
                        (\nameref{risultatiRicerca}).
            \end{enumerate}
\end{itemize}

\subsubsection{UC-2.3.2.4.10 - Ricerca per Progressivo}
\begin{itemize}
      \item \textbf{Attore Primario}: utente
      \item \textbf{Precondizioni}: L'utente ha scelto come tipo di documento l'Aggregazione Documentale.
      \item \textbf{Postcondizioni}: I risultati sono filtrati per numero Progressivo.
      \item \textbf{Flusso Principale}:
            \begin{enumerate}
                  \item L'utente inserisce il numero Progressivo dell'Aggregazione.
                  \item Il sistema restituisce la lista delle aggregazioni corrispondenti
                        (\nameref{risultatiRicerca}).
            \end{enumerate}
\end{itemize}

\subsubsection{UC-2.4 - Visualizzazione Risultati di Ricerca} \label{risultatiRicerca}
\begin{itemize}
      \item \textbf{Attore Primario}: utente
      \item \textbf{Precondizioni}: L'utente ha eseguito una ricerca (tramite UC-2.1, UC-2.2, UC-2.3 o UC-2.5).
      \item \textbf{Postcondizioni}: I risultati della ricerca sono presentati all'utente.
      \item \textbf{Flusso Principale}:
            \begin{enumerate}
                  \item Il sistema visualizza un elenco di documenti o aggregazioni che corrispondono
                        ai criteri di ricerca.
                  \item Per ogni elemento, vengono mostrate le informazioni principali (Nome, Data,
                        Tipo).
            \end{enumerate}
      \item \textbf{Flusso Alternativo (Nessun risultato)}: Il sistema comunica che la ricerca non ha prodotto alcun risultato.
\end{itemize}

\subsubsection{UC-2.5 - Ricerca Semantica}
\begin{itemize}
      \item \textbf{Attore Primario}: utente
      \item \textbf{Attore Secondario}: LLM in remoto
      \item \textbf{Precondizioni}: L'utente ha accesso alla funzionalità di ricerca semantica e dispone di una connessione internet.
      \item \textbf{Postcondizioni}: Visualizzazione di un elenco di documenti o aggregazioni pertinenti alla richiesta in linguaggio naturale.
      \item \textbf{Flusso Principale}:
            \begin{enumerate}
                  \item L'utente descrive in linguaggio naturale cosa sta cercando nell'apposita barra
                        di ricerca.
                  \item Il sistema invia la richiesta al servizio LLM.
                  \item Il sistema riceve i risultati e li visualizza (UC-2.4).
            \end{enumerate}
      \item \textbf{Flusso Alternativo}:
            \begin{itemize}
                  \item La ricerca non riporta alcun risultato pertinente.
                  \item La connessione internet viene a mancare, impedendo di completare la richiesta.
            \end{itemize}
\end{itemize}

\subsubsection{UC -? - Salvataggio di più Documenti}
\begin{itemize}
    \item \textbf{Attore Primario}: Utente
    \item \textbf{Precondizioni}: L'Utente sta visualizzando un folder con più documenti o ha selezionato più documenti
    \item \textbf{Postcondizioni}: Il sistema salva i documenti in locale in una cartella selezionata dall'utente
    \item \textbf{Flusso Principale}: L'Utente cliccando può selezionare più documenti di cui effettuare il salvataggio in una cartella del proprio file system
    \item \textbf{Flusso Alternativo}: L'Utente cerca di salvare i documenti nella cartella del DIP e questa operazione viene bloccata per garantire lo stato del DIP. Il DIP è in sola lettura.
\end{itemize}


\subsubsection{UC-4 - Stampa documento (LS)}

\subsubsection{UC-4.1 - Stampa di un singolo documento}
\begin{itemize}
    \item \textbf{Attore Primario}: Utente
    \item \textbf{Precondizioni}: L'Utente sta visualizzando il documento
    \item \textbf{Postcondizioni}: Il documento viene stampato
    \item \textbf{Flusso Principale}: L'Utente cliccando può selezionare la stampa del documento
    \item \textbf{Flusso Alternativo}:
    \begin{itemize}
        \item L'Utente cerca di stampare un documento ma un errore esterno al programma e legato alla stampante interrompe l'azione
        \item L'Utente cerca di stampare un documento non stampabile (video, audio per esempio)
    \end{itemize}
\end{itemize}

\subsubsection{UC-4.2 - Stampa di un insieme di documenti}
\begin{itemize}
    \item \textbf{Attore Primario}: Utente
    \item \textbf{Precondizioni}: L'Utente sta visualizzando più Documenti o un'aggregazione di essi, un folder contenente più file
    \item \textbf{Postcondizioni}: I documenti vengono stampati uno dopo l'altro
    \item \textbf{Flusso Principale}: L'Utente cliccando può selezionare tutti i documenti e quindi stamparli uno dopo l'altro
    \item \textbf{Flusso Alternativo}:
    \begin{itemize}
        \item L'Utente cerca di stampare i documenti ma un errore esterno al programma e legato alla stampante interrompe l'azione
        \item L'Utente cerca di stampare uno o più documenti non stampabili (video, audio per esempio)
    \end{itemize}
\end{itemize}
% DA UC-37

\usecase{verificaIntegritaDIPCompleto}{Verifica integrità DIP completo}
\begin{figure}[H]
      \centering
      \includegraphics[width=0.6\textwidth]{../assets/uml/UC33.png}
      \caption{UC33 - Verifica integrità DIP completo}
      \label{fig:uc_verificaIntegritaDIPCompleto}
\end{figure}
\begin{itemize}
    \item \textbf{Attore Primario}: Utente
    \item \textbf{Precondizioni}:
        \begin{itemize}
            \item L'Utente ha avviato l'applicazione
        \end{itemize}
    \item \textbf{Postcondizioni}: 
        \begin{enumerate}
            \item Il sistema compone il report di integrità dell'intero DIP
            \item Stato di verifica DIP aggiornato (Non Verificato / Valido / Non Valido).
        \end{enumerate}      
    \item \textbf{Flusso Principale}:
        \begin{enumerate}
            \item L'utente seleziona "Verifica Integrità dell'intero DIP"
            \item Il sistema verifica il DIP completo
            \item Il sistema aggiorna lo stato della verifica del DIP.
        \end{enumerate}
\end{itemize}

\usecase{verificaIntegritaClasseDocumentale}{Verifica integrità classe documentale}
\begin{figure}[H]
      \centering
      \includegraphics[width=0.6\textwidth]{../assets/uml/UC34.png}
      \caption{UC34 - Verifica integrità classe documentale}
      \label{fig:uc_verificaIntegritaClasseDocumentale}
\end{figure}
\begin{itemize}
    \item \textbf{Attore Primario}: Utente
    \item \textbf{Precondizioni}: 
        \begin{itemize}
            \item L'Utente ha avviato l'applicazione
            \item L'Utente si trova nella vista principale del DIP
            \item L'Utente ha selezionato una classe documentale.
        \end{itemize} 
    \item \textbf{Postcondizioni}: 
        \begin{enumerate}
            \item Il sistema compone il Report di integrità classe documentale
            \item Stato di verifica classe aggiornato (Non Verificato / Valido / Non Valido).
        \end{enumerate}      
    \item \textbf{Flusso Principale}:
        \begin{enumerate}
            \item L'utente seleziona "Verifica Integrità" sulla classe documentale
            \item Il sistema verifica la classe documentale
            \item Il sistema aggiorna lo stato della verifica della classe documentale.
        \end{enumerate}
\end{itemize}

\usecase{verificaIntegritaProcesso}{Verifica integrità processo}
\begin{figure}[H]
      \centering
      \includegraphics[width=0.6\textwidth]{../assets/uml/UC35.png}
      \caption{UC35 - Verifica integrità processo}
      \label{fig:uc_verificaIntegritaProcesso}
\end{figure}
\begin{itemize}
    \item \textbf{Attore Primario}: Utente
    \item \textbf{Precondizioni}: 
        \begin{itemize}
            \item L'Utente ha avviato l'applicazione
            \item L'Utente si trova nella vista principale del DIP
            \item L'Utente ha selezionato un processo.
        \end{itemize} 
    \item \textbf{Postcondizioni}: 
        \begin{enumerate}
            \item Il sistema compone il report di integrità processo
            \item Stato di verifica processo aggiornato (Non Verificato / Valido / Non Valido).
        \end{enumerate}      
    \item \textbf{Flusso Principale}:
        \begin{enumerate}
            \item L'utente selezione "Verifica Integrità" sul processo
            \item Il sistema verifica il processo
            \item Il sistema aggiorna lo stato della verifica del processo.
        \end{enumerate}
\end{itemize}

\usecase{verificaIntegritaDocumento}{Verifica integrità documento}
\begin{figure}[H]
      \centering
      \includegraphics[width=0.6\textwidth]{../assets/uml/UC36.png}
      \caption{UC36 - Verifica integrità documento}
      \label{fig:uc_verificaIntegritaDocumento}
\end{figure}
\begin{itemize}
    \item \textbf{Attore Primario}: Utente
    \item \textbf{Precondizioni}: 
        \begin{itemize}
            \item L'Utente ha avviato l'applicazione
            \item L'Utente si trova nella vista principale del DIP
            \item L'Utente ha selezionato un documento.
        \end{itemize} 
    \item \textbf{Postcondizioni}: 
        \begin{enumerate}
            \item Il sistema compone il report di integrità documento
            \item Stato di verifica documento aggiornato (Non Verificato / Valido / Non Valido).
        \end{enumerate}      
    \item \textbf{Flusso Principale}:
        \begin{enumerate}
            \item L'utente selezione "Verifica Integrità" sul documento
            \item Il sistema esegue la verifica del documento
            \item Il sistema aggiorna lo stato della verifica del documento.
        \end{enumerate}
\end{itemize}

\usecase{visualizzazioneReportIntegritaDIPCompleto}{Visualizzazione report integrità DIP completo}
\begin{figure}[H]
    \centering
    \includegraphics[width=0.6\textwidth]{../assets/uml/UC37.png}
      \caption{UC37 - Visualizzazione report integrità DIP completo}
    \label{fig:uc_visualizzazioneReportIntegritaDIPCompleto}
\end{figure}
\begin{itemize}
    \item \textbf{Attore Primario}: Utente
    \item \textbf{Precondizioni}: 
        \begin{itemize}
            \item L'Utente ha avviato l'applicazione
            \item È stata completata la verifica di integrità del DIP completo.
        \end{itemize} 
    \item \textbf{Postcondizioni}: Il report dettagliato del DIP viene visualizzato.     
    \item \textbf{Flusso Principale}:
        \begin{enumerate}
            \item L'utente visualizza il report della verifica del DIP completo
            \item L'utente visualizza un pannello con le informazioni aggregate del DIP.
        \end{enumerate}
    \item \textbf{Inclusioni}: 
        \begin{itemize}
            \item \ref{visualizzazioneNumeroClassiVerificate} Visualizzazione numero classi verificate
            \item \ref{visualizzazioneNumeroClassiIntegre} Visualizzazione numero classi integre
            \item \ref{visualizzazioneNumeroClassiCorrotte} Visualizzazione numero classi corrotte
            \item \ref{visualizzazioneListaClassiCorrotte} Visualizzazione lista classi corrotte
            \item \ref{visualizzazioneDataEOraVerificaDIP} Visualizzazione data e ora verifica DIP.
        \end{itemize}
\end{itemize}
\begin{figure}[H]
    \centering
    \includegraphics[width=1\textwidth]{../assets/uml/IncUC37.png}
      \caption{Inclusioni UC37 - Visualizzazione report integrità DIP completo}
    \label{fig:inclusioniVisualizzazioneReportIntegritaDIPCompleto}
\end{figure}

\subusecase{visualizzazioneNumeroClassiVerificate}{Visualizzazione numero classi verificate}
\begin{itemize}
    \item \textbf{Attore Primario}: Utente
    \item \textbf{Precondizioni}: 
        \begin{itemize}
            \item L'Utente ha avviato l'applicazione
            \item È stata completata la verifica di integrità del DIP completo
            \item È stato generato un report di verifica del DIP completo.
        \end{itemize} 
    \item \textbf{Postcondizioni}: Numero totale di classi verificate visualizzato.     
    \item \textbf{Flusso Principale}:
        \begin{enumerate}
            \item Il sistema conta tutte le classi documentali sottoposte a verifica
            \item Il sistema mostra il conteggio con l'etichetta "Classi verificate: [N]".
        \end{enumerate}
\end{itemize}

\subusecase{visualizzazioneNumeroClassiIntegre}{Visualizzazione numero classi integre}
\begin{itemize}
    \item \textbf{Attore Primario}: Utente
    \item \textbf{Precondizioni}: 
        \begin{itemize}
            \item L'Utente ha avviato l'applicazione
            \item È stata completata la verifica di integrità del DIP completo
            \item È stato generato un report di verifica del DIP completo.
        \end{itemize} 
    \item \textbf{Postcondizioni}: Numero di classi integre visualizzato.     
    \item \textbf{Flusso Principale}:
        \begin{enumerate}
            \item Il sistema conta le classi con stato "Valido"
            \item Il sistema mostra il conteggio con l'etichetta "Classi integre: [N]" in colore verde.
        \end{enumerate}
\end{itemize}

\subusecase{visualizzazioneNumeroClassiCorrotte}{Visualizzazione numero classi corrotte}
\begin{itemize}
    \item \textbf{Attore Primario}: Utente
    \item \textbf{Precondizioni}: 
        \begin{itemize}
            \item L'Utente ha avviato l'applicazione
            \item È stata completata la verifica di integrità del DIP completo
            \item È stato generato un report di verifica del DIP completo.
        \end{itemize} 
    \item \textbf{Postcondizioni}: Numero di classi corrotte visualizzato.     
    \item \textbf{Flusso Principale}:
        \begin{enumerate}
            \item Il sistema conta le classi con stato "Non Valido"
            \item Il sistema mostra il conteggio con l'etichetta "Classi corrotte: [N]" in colore rosso.
        \end{enumerate}
\end{itemize}

\subusecase{visualizzazioneListaClassiCorrotte}{Visualizzazione lista classi corrotte}
\begin{itemize}
    \item \textbf{Attore Primario}: Utente
    \item \textbf{Precondizioni}: 
        \begin{itemize}
            \item L'Utente ha avviato l'applicazione
            \item È stata completata la verifica di integrità del DIP completo
            \item È stato generato un report di verifica del DIP completo.
            \item Esistono classi corrotte (numero classi corrotte > 0).
        \end{itemize} 
    \item \textbf{Postcondizioni}: Lista completa delle classi corrotte visualizzata.     
    \item \textbf{Flusso Principale}:
        \begin{enumerate}
            \item Il sistema recupera l'elenco di tutte le classi con stato "Non Valido"
            \item Per ogni classe corrotta il sistema mostra il nome della classe e il numero di processi corrotti nella classe.
        \end{enumerate}
\end{itemize}

\subusecase{visualizzazioneDataEOraVerificaDIP}{Visualizzazione data e ora verifica DIP}
\begin{itemize}
    \item \textbf{Attore Primario}: Utente
    \item \textbf{Precondizioni}: 
        \begin{itemize}
            \item L'Utente ha avviato l'applicazione
            \item È stata completata la verifica di integrità del DIP completo
            \item È stato generato un report di verifica del DIP completo.
        \end{itemize} 
    \item \textbf{Postcondizioni}: Timestamp della verifica visualizzato.     
    \item \textbf{Flusso Principale}:
        \begin{enumerate}
            \item Il sistema recupera la data e ora di inizio della verifica del DIP
            \item Il sistema mostra il timestamp nel formato "GG/MM/AAAA HH:MM:SS".
        \end{enumerate}
\end{itemize}

\usecase{visualizzazioneReportIntegritaClasseDocumentale}{Visualizzazione report integrità classe documentale}
\begin{figure}[H]
    \centering
    \includegraphics[width=0.6\textwidth]{../assets/uml/UC38.png}
      \caption{UC38 - Visualizzazione report integrità classe documentale}
    \label{fig:uc_visualizzazioneReportIntegritaClasseDocumentale}
\end{figure}
\begin{itemize}
    \item \textbf{Attore Primario}: Utente
    \item \textbf{Precondizioni}: 
        \begin{itemize}
            \item L'Utente ha avviato l'applicazione
            \item È stata completata la verifica di integrità della classe documentale selezionata.
        \end{itemize} 
    \item \textbf{Postcondizioni}: Report dettagliato della classe visualizzato.     
    \item \textbf{Flusso Principale}:
        \begin{enumerate}
            \item Il sistema mostra il report aggregando i dati di tutti i processi della classe
            \item Il sistema mostra un pannello con le informazioni aggregate della classe.
        \end{enumerate}
    \item \textbf{Inclusioni}: 
        \begin{itemize}
            \item \ref{visualizzazioneNumeroProcessiVerificati} Visualizzazione Numero Processi Verificati
            \item \ref{visualizzazioneNumeroProcessiIntegri} Visualizzazione Numero Processi Integri
            \item \ref{visualizzazioneNumeroProcessiCorrotti} Visualizzazione Numero Processi Corrotti
            \item \ref{visualizzazioneListaProcessiCorrotti} Visualizzazione Lista Processi Corrotti
            \item \ref{visualizzazioneDataEOraVerificaClasse} Visualizzazione Data e Ora Verifica Classe.
        \end{itemize}
\end{itemize}
\begin{figure}[H]
    \centering
    \includegraphics[width=1\textwidth]{../assets/uml/IncUC38.png}
    \caption{Inclusioni UC38 - Visualizzazione Report Integrità Classe Documentale}
    \label{fig:inclusioniVisualizzazioneReportIntegritaClasseDocumentale}
\end{figure}

\subusecase{visualizzazioneNumeroProcessiVerificati}{Visualizzazione Numero Processi Verificati}
\begin{itemize}
    \item \textbf{Attore Primario}: Utente
    \item \textbf{Precondizioni}: 
        \begin{itemize}
            \item L'Utente ha avviato l'applicazione
            \item È stata completata una verifica di integrità della classe documentale
            \item È stato generato un report di verifica della classe.
        \end{itemize} 
    \item \textbf{Postcondizioni}: Numero totale di processi verificati visualizzato.     
    \item \textbf{Flusso Principale}:
        \begin{enumerate}
            \item Il sistema conta tutti i processi della classe sottoposti a verifica
            \item Il sistema mostra il conteggio con l'etichetta "Processi verificati: [N]".
        \end{enumerate}
\end{itemize}

\subusecase{visualizzazioneNumeroProcessiIntegri}{Visualizzazione numero processi integri}
\begin{itemize}
    \item \textbf{Attore Primario}: Utente
    \item \textbf{Precondizioni}: 
        \begin{itemize}
            \item L'Utente ha avviato l'applicazione
            \item È stata completata una verifica di integrità della classe documentale
            \item È stato generato un report di verifica della classe.
        \end{itemize} 
    \item \textbf{Postcondizioni}: Numero di processi integri visualizzato.     
    \item \textbf{Flusso Principale}:
        \begin{enumerate}
            \item Il sistema conta i processi con stato "Valido"
            \item Il sistema mostra il conteggio con l'etichetta "Processi integri: [N]" in colore verde.
        \end{enumerate}
\end{itemize}

\subusecase{visualizzazioneNumeroProcessiCorrotti}{Visualizzazione numero processi corrotti}
\begin{itemize}
    \item \textbf{Attore Primario}: Utente
    \item \textbf{Precondizioni}: 
        \begin{itemize}
            \item L'Utente ha avviato l'applicazione
            \item È stata completata una verifica di integrità della classe documentale
            \item È stato generato un report di verifica della classe.
        \end{itemize} 
    \item \textbf{Postcondizioni}: Numero di processi corrotti visualizzato.     
    \item \textbf{Flusso Principale}:
        \begin{enumerate}
            \item Il sistema conta i processi con stato "Non Valido"
            \item Il sistema mostra il conteggio con l'etichetta "Processi corrotti: [N]" in colore rosso.
        \end{enumerate}
\end{itemize}

\subusecase{visualizzazioneListaProcessiCorrotti}{Visualizzazione lista processi corrotti}
\begin{itemize}
    \item \textbf{Attore Primario}: Utente
    \item \textbf{Precondizioni}: 
        \begin{itemize}
            \item L'Utente ha avviato l'applicazione
            \item È stata completata una verifica di integrità della classe documentale
            \item È stato generato un report di verifica della classe
            \item Esistono processi corrotti (numero processi corrotti > 0).
        \end{itemize}      
    \item \textbf{Postcondizioni}: Lista completa dei processi corrotti visualizzata.     
    \item \textbf{Flusso Principale}:
        \begin{enumerate}
            \item Il sistema recupera l'elenco di tutti i processi con stato "Non Valido"
            \item Per ogni processo corrotto, il sistema mostra il nome del processo e il numero dei documenti corrotti.
        \end{enumerate}
\end{itemize}

\subusecase{visualizzazioneDataEOraVerificaClasse}{Visualizzazione data e ora verifica classe}
\begin{itemize}
    \item \textbf{Attore Primario}: Utente
    \item \textbf{Precondizioni}: 
        \begin{itemize}
            \item L'Utente ha avviato l'applicazione
            \item È stata completata una verifica di integrità della classe documentale
            \item È stato generato un report di verifica della classe.
        \end{itemize}  
    \item \textbf{Postcondizioni}: Timestamp della verifica visualizzato.
    \item \textbf{Flusso Principale}:
        \begin{enumerate}
            \item Il sistema recupera la data e ora di inizio della verifica della classe
            \item Il sistema mostra il timestamp nel formato "GG/MM/AAAA HH:MM:SS".
        \end{enumerate}
\end{itemize}

\usecase{visualizzazioneReportIntegritaProcesso}{Visualizzazione report integrità processo}
\begin{figure}[H]
    \centering
    \includegraphics[width=0.6\textwidth]{../assets/uml/UC39.png}
      \caption{UC39 - Visualizzazione report integrità processo}
    \label{fig:uc_visualizzazioneReportIntegritaProcesso}
\end{figure}
\begin{itemize}
    \item \textbf{Attore Primario}: Utente
    \item \textbf{Precondizioni}: 
        \begin{itemize}
            \item L'Utente ha avviato l'applicazione
            \item È stata completata una verifica di integrità di un processo.
        \end{itemize} 
    \item \textbf{Postcondizioni}: Report dettagliato del processo visualizzato.     
    \item \textbf{Flusso Principale}:
        \begin{enumerate}
            \item Il sistema mostra il report aggregando i dati di tutti i documenti del processo
            \item Il sistema mostra un pannello con le informazioni aggregate del processo.
        \end{enumerate}
    \item \textbf{Inclusioni}: 
        \begin{itemize}
            \item \ref{visualizzazioneNumeroDocumentiVerificati} Visualizzazione numero documenti verificati
            \item \ref{visualizzazioneNumeroDocumentiIntegri} Visualizzazione numero documenti integri
            \item \ref{visualizzazioneNumeroDocumentiCorrotti} Visualizzazione numero documenti corrotti
            \item \ref{visualizzazioneListaDocumentiCorrotti} Visualizzazione lista documenti corrotti
            \item \ref{visualizzazioneDataEOraVerificaProcesso} Visualizzazione data e ora verifica processo
        \end{itemize}
\end{itemize}
\begin{figure}[H]
    \centering
    \includegraphics[width=1\textwidth]{../assets/uml/IncUC39.png}
      \caption{Inclusioni UC39 - Visualizzazione report integrità processo}
    \label{fig:inclusioniVisualizzazioneReportIntegritaProcesso}
\end{figure}

\subusecase{visualizzazioneNumeroDocumentiVerificati}{Visualizzazione numero documenti verificati}
\begin{itemize}
    \item \textbf{Attore Primario}: Utente
    \item \textbf{Precondizioni}: 
        \begin{itemize}
            \item L'Utente ha avviato l'applicazione
            \item È stata completata una verifica di integrità di un processo
            \item È stato generato un report di verifica del processo.
        \end{itemize} 
    \item \textbf{Postcondizioni}: Numero totale di documenti verificati visualizzato.     
    \item \textbf{Flusso Principale}:
        \begin{enumerate}
            \item Il sistema conta tutti i documenti del processo sottoposti a verifica
            \item Il sistema mostra il conteggio con l'etichetta "Documenti verificati: [N]".
        \end{enumerate}
\end{itemize}

\subusecase{visualizzazioneNumeroDocumentiIntegri}{Visualizzazione numero documenti integri}
\begin{itemize}
    \item \textbf{Attore Primario}: Utente
    \item \textbf{Precondizioni}: 
        \begin{itemize}
            \item L'Utente ha avviato l'applicazione
            \item È stata completata una verifica di integrità di un processo
            \item È stato generato un report di verifica del processo.
        \end{itemize} 
    \item \textbf{Postcondizioni}: Numero di documenti integri visualizzato.     
    \item \textbf{Flusso Principale}:
        \begin{enumerate}
            \item Il sistema conta i documenti con stato "Valido"
            \item Il sistema mostra il conteggio con l'etichetta "Documenti integri: [N]" in colore verde.
        \end{enumerate}
\end{itemize}

\subusecase{visualizzazioneNumeroDocumentiCorrotti}{Visualizzazione numero documenti corrotti}
\begin{itemize}
    \item \textbf{Attore Primario}: Utente
    \item \textbf{Precondizioni}:
        \begin{itemize}
            \item L'Utente ha avviato l'applicazione
            \item È stata completata una verifica di integrità di un processo
            \item È stato generato un report di verifica del processo.
        \end{itemize} 
    \item \textbf{Postcondizioni}: Numero di documenti corrotti visualizzato.     
    \item \textbf{Flusso Principale}:
        \begin{enumerate}
            \item Il sistema conta i documenti con stato "Non Valido"
            \item Il sistema mostra il conteggio con l'etichetta "Documenti corrotti: [N]" in colore rosso.
        \end{enumerate}
\end{itemize}

\subusecase{visualizzazioneListaDocumentiCorrotti}{Visualizzazione lista documenti corrotti}
\begin{itemize}
    \item \textbf{Attore Primario}: Utente
    \item \textbf{Precondizioni}: 
        \begin{itemize}
            \item L'Utente ha avviato l'applicazione
            \item È stata completata una verifica di integrità di un processo
            \item È stato generato un report di verifica del processo
            \item Esistono documenti corrotti (numero documenti corrotti > 0).
        \end{itemize}    
    \item \textbf{Postcondizioni}: Lista completa dei documenti corrotti visualizzata.     
    \item \textbf{Flusso Principale}:
        \begin{enumerate}
            \item Il sistema recupera l'elenco di tutti i documenti con stato "Non Valido"
            \item Per ogni documento corrotto, il sistema mostra il nome del documento e l'errore specifico.
        \end{enumerate}
\end{itemize}

\subusecase{visualizzazioneDataEOraVerificaProcesso}{Visualizzazione data e ora verifica processo}
\begin{itemize}
    \item \textbf{Attore Primario}: Utente
    \item \textbf{Precondizioni}: 
        \begin{itemize}
            \item L'Utente ha avviato l'applicazione
            \item È stata completata una verifica di integrità di un processo
            \item È stato generato un report di verifica del processo.
        \end{itemize} 
    \item \textbf{Postcondizioni}: Timestamp della verifica visualizzato.
    \item \textbf{Flusso Principale}:
        \begin{enumerate}
            \item Il sistema recupera la data e ora di inizio della verifica del processo
            \item Il sistema mostra il timestamp nel formato "GG/MM/AAAA HH:MM:SS".
        \end{enumerate}
\end{itemize}

\usecase{visualizzazioneReportIntegritaDocumento}{Visualizzazione report integrità documento}
\begin{figure}[H]
    \centering
    \includegraphics[width=0.6\textwidth]{../assets/uml/UC40.png}
      \caption{UC40 - Visualizzazione report integrità documento}
    \label{fig:uc_visualizzazioneReportIntegritaDocumento}
\end{figure}
\begin{itemize}
    \item \textbf{Attore Primario}: Utente
    \item \textbf{Precondizioni}: 
        \begin{itemize}
            \item L'Utente ha avviato l'applicazione
            \item È stata completata una verifica di integrità di un documento.
        \end{itemize} 
    \item \textbf{Postcondizioni}: Report dettagliato del documento visualizzato.     
    \item \textbf{Flusso Principale}:
        \begin{enumerate}
            \item Il sistema mostra:
                \begin{itemize}
                    \item Nome del documento (\ref{visualizzazioneNomeDocumento})
                    \item Stato della verifica (Valido / Non Valido) (\ref{visualizzazioneStatoVerificaDocumento})
                    \item Data e ora della verifica (\ref{visualizzazioneDataEOraVerificaDocumento})
                    \item Dettagli dell'errore (se presenti) (\ref{visualizzazioneDettagliErroreDocumento})
                \end{itemize}
        \end{enumerate}
    \item \textbf{Inclusioni}: 
        \begin{itemize}
            \item \ref{visualizzazioneNomeDocumento} Visualizzazione nome documento
            \item \ref{visualizzazioneStatoVerificaDocumento} Visualizzazione stato verifica documento
            \item \ref{visualizzazioneDataEOraVerificaDocumento} Visualizzazione data e ora verifica documento
            \item \ref{visualizzazioneDettagliErroreDocumento} Visualizzazione dettaglio errore documento (se presente)
        \end{itemize}
\end{itemize}
\begin{figure}[H]
    \centering
    \includegraphics[width=1\textwidth]{../assets/uml/IncUC40.png}
      \caption{Inclusioni UC40 - Visualizzazione report integrità documento}
    \label{fig:inclusioniVisualizzazioneReportIntegritaDocumento}
\end{figure}



\subusecase{visualizzazioneStatoVerificaDocumento}{Visualizzazione stato verifica documento}
\begin{itemize}
    \item \textbf{Attore Primario}: Utente
    \item \textbf{Precondizioni}: 
        \begin{itemize}
            \item L'Utente ha avviato l'applicazione
            \item È stata completata una verifica di integrità di un documento
            \item È stato generato un report di verifica del documento.
        \end{itemize} 
    \item \textbf{Postcondizioni}: Stato della verifica visualizzato.     
    \item \textbf{Flusso Principale}:
        \begin{enumerate}
            \item Il sistema mostra lo stato della verifica.
        \end{enumerate}
\end{itemize}

\subusecase{visualizzazioneDataEOraVerificaDocumento}{Visualizzazione data e ora verifica documento}
\begin{itemize}
    \item \textbf{Attore Primario}: Utente
    \item \textbf{Precondizioni}: 
        \begin{itemize}
            \item L'Utente ha avviato l'applicazione
            \item È stata completata una verifica di integrità di un documento
            \item È stato generato un report di verifica del documento.
        \end{itemize} 
    \item \textbf{Postcondizioni}: Timestamp della verifica visualizzato.     
    \item \textbf{Flusso Principale}:
        \begin{enumerate}
            \item Il sistema recupera la data e ora di inizio della verifica del documento
            \item Il sistema mostra il timestamp nel formato "GG/MM/AAAA HH:MM:SS".
        \end{enumerate}
\end{itemize}

\subusecase{visualizzazioneDettagliErroreDocumento}{Visualizzazione dettagli errore documento}
\begin{itemize}
    \item \textbf{Attore Primario}: Utente
    \item \textbf{Precondizioni}: 
        \begin{itemize}
            \item L'Utente ha avviato l'applicazione
            \item È stata completata una verifica di integrità di un documento
            \item È stato generato un report di verifica del documento.
            \item Il documento ha stato "Non Valido".
        \end{itemize} 
    \item \textbf{Postcondizioni}: Dettagli dell'errore visualizzati.     
    \item \textbf{Flusso Principale}:
        \begin{enumerate}
            \item Il sistema mostra la descrizione dell'errore (ad esempio: hash calcolato non coincide con hash associato al documento, firma digitale non valida o scaduta).
        \end{enumerate}
\end{itemize}

\usecase{visualizzazioneNomeDocumento}{Visualizzazione nome documento}
\begin{itemize}
    \item \textbf{Attore Primario}: Utente
    \item \textbf{Precondizioni}: 
        \begin{itemize}
            \item L'Utente ha avviato l'applicazione
        \end{itemize}
    \item \textbf{Postcondizioni}: Nome documento visualizzato.     
    \item \textbf{Flusso Principale}:
        \begin{enumerate}
            \item Il sistema mostra il nome del documento.
        \end{enumerate}
\end{itemize}

\usecase{convertiReportVerificaPDF}{Converti report verifica in PDF}
\begin{figure}[H]
    \centering
    \includegraphics[width=0.8\textwidth]{../assets/uml/UC42.png}
      \caption{UC42 - Converti report verifica in PDF}
    \label{fig:uc_convertiReportVerificaPDF}
\end{figure}
\begin{itemize}
    \item \textbf{Attore Primario}: Utente
    \item \textbf{Precondizioni}: 
        \begin{itemize}
            \item L'Utente ha avviato l'applicazione
            \item È stata completata una verifica di integrità (DIP completo / Classe Documentale / Processo / Documento)
            \item È stato generato un report di verifica (DIP completo / Classe Documentale / Processo / Documento).
        \end{itemize} 
    \item \textbf{Postcondizioni}: Il report visualizzato viene convertito in PDF.     
    \item \textbf{Flusso Principale}:
        \begin{enumerate}
            \item L'utente seleziona "Salva Report"
            \item Il sistema converte il report corrente in formato PDF.
        \end{enumerate}
    \item \textbf{Flusso Alternativo}:
    \begin{itemize}
        \item La conversione non va a buon fine: viene visualizzato un messaggio di errore "Impossibile generare il PDF. Riprovare.".
    \end{itemize}
    \item \textbf{Estensioni}: \ref{erroreGenerazionePDF} Errore generazione PDF
\end{itemize}

\usecase{scaricaFile}{Scarica file}
\begin{figure}[H]
    \centering
    \includegraphics[width=0.8\textwidth]{../assets/uml/UC43.png}
    \caption{UC43 - Scarica File}
    \label{fig:uc_scaricaFile}
\end{figure}
\begin{itemize}
    \item \textbf{Attore Primario}: Utente
    \item \textbf{Precondizioni}: 
        \begin{itemize}
            \item L'Utente ha avviato l'applicazione
            \item L'Utente ha selezionato un file da scaricare (Documento o Report).
        \end{itemize} 
    \item \textbf{Postcondizioni}: L'utente ha scaricato il File nella cartella selezionata.     
    \item \textbf{Flusso Principale}:
        \begin{enumerate}
            \item Il sistema apre un dialogo di selezione cartella
            \item L'utente seleziona la cartella di destinazione
            \item Il sistema salva il File nella cartella selezionata
            \item Il sistema conferma il salvataggio con un messaggio "File salvato con successo in [percorso]".
        \end{enumerate}
    \item \textbf{Flusso Alternativo}:
        \begin{itemize}
            \item L'utente tenta di salvare il documento nella cartella del DIP: il sistema blocca l'operazione e mostra il messaggio "Impossibile salvare nel DIP. Selezionare un'altra cartella per preservare l'integrità del DIP."
            \item L'utente annulla l'operazione: il sistema chiude il dialogo senza salvare.
        \end{itemize}
      \item \textbf{Estensioni}: \ref{erroreScaricamentoFile} Errore scaricamento file
\end{itemize}

\usecase{erroreScaricamentoFile}{Errore scaricamento file}
\begin{itemize}
    \item \textbf{Attore Primario}: Utente
    \item \textbf{Precondizioni}: 
        \begin{itemize}
            \item L'Utente ha avviato l'applicazione
            \item L'Utente ha selezionato un file da scaricare (Documento o Report).
        \end{itemize} 
    \item \textbf{Postcondizioni}: L'utente visualizza un messaggio d'errore.
    \item \textbf{Flusso Principale}:
        \begin{enumerate}
            \item Il sistema apre un dialogo di selezione cartella
            \item L'utente seleziona la cartella di destinazione
            \item Il sistema blocca l'operazione e mostra il messaggio "Impossibile salvare nel DIP. Selezionare un'altra cartella per preservare l'integrità del DIP."
            \item L'utente annulla l'operazione: il sistema chiude il dialogo senza salvare.
        \end{enumerate}
\end{itemize}

\usecase{erroreGenerazionePDF}{Errore generazione PDF}
\begin{itemize}
    \item \textbf{Attore Primario}: Utente
    \item \textbf{Precondizioni}: 
        \begin{itemize}
            \item L'Utente ha avviato l'applicazione
            \item È stata completata una verifica di integrità (DIP completo / Classe Documentale / Processo / Documento)
            \item È stato generato un report di verifica (DIP completo / Classe Documentale / Processo / Documento)
            \item È stata effettuata una conversione del report in PDF
            \item Una conversione PDF non è andata a buon fine.
        \end{itemize} 
    \item \textbf{Postcondizioni}: L'utente visualizza un messaggio d'errore.     
    \item \textbf{Flusso Principale}:
        \begin{enumerate}
            \item L'utente ha provato a convertire un Report come PDF
            \item La conversione non va a buon fine
            \item L'utente visualizza un messaggio d'errore.
        \end{enumerate}
\end{itemize}

\usecase{visualizzaInfoAiP}{Visualizzazione informazioni AiP di provenienza documento  }
\begin{itemize}
    \item \textbf{Attore Primario}: Utente
    \item \textbf{Precondizioni}:
    \begin{enumerate}
        \item L'utente ha avviato l'applicazione
        \item L'utente ha selezionato un documento
    \end{enumerate}
    \item \textbf{Postcondizioni}: L'utente ha ottenuto le informazioni sull'AiP di provenienza relativo al documento: UUID e classe documentale.
    \item \textbf{Flusso Principale}: Il sistema mostra a video le informazioni sull'AiP:
    \begin{itemize}[label=$\rhd$]
        \item Classe documentale AiP $\rightarrow$ Vedi \hyperref[validazioneTestoLibero]{[UC-10.\ref{validazioneTestoLibero}]}
        \item UUID AiP $\rightarrow$ Vedi \hyperref[validazioneIdentificativiIPA]{[UC-10.\ref{validazioneIdentificativiIPA}]}
    \end{itemize}

\item \textbf{Inclusioni}: 
    \begin{itemize}[label=$\rhd$]
        \item \hyperref[validazioneTestoLibero]{[UC-10.\ref{validazioneTestoLibero}]}
        \item \hyperref[validazioneIdentificativiIPA]{[UC-10.\ref{validazioneIdentificativiIPA}]}
    \end{itemize}
\end{itemize}

\subusecase{visualizzaClasseDocumentaleAiP}{Visualizza classe documentale di appartenenza AiP  }
\begin{itemize}
    \item \textbf{Attore Primario}: Utente
    \item \textbf{Precondizioni}:
    \begin{enumerate}
        \item L'utente ha avviato l'applicazione
        \item L'utente sta visualizzando le informazioni di un AiP
    \end{enumerate}
    \item \textbf{Postcondizioni}: L'utente visualizza la classe documentale dell'AiP relativo al documento selezionato
    \item \textbf{Flusso Principale}: Il sistema mostra a video la classe documentale dell'AiP relativo al documento selezionato
\end{itemize}

\subusecase{visualizzaUUIDAiP}{Visualizza UUID AiP  }
\begin{itemize}
    \item \textbf{Attore Primario}: Utente
    \item \textbf{Precondizioni}:
    \begin{enumerate}
        \item L'utente ha avviato l'applicazione
        \item L'utente sta visualizzando le informazioni di un AiP
    \end{enumerate}
    \item \textbf{Postcondizioni}: L'utente visualizza lo UUID dell'AiP relativo al documento selezionato
    \item \textbf{Flusso Principale}: Il sistema mostra a video lo UUID dell'AiP relativo al documento selezionato
\end{itemize}

\usecase{visualizzaInfoProcessoConservazioneAiP}{Visualizza informazioni processo di conservazione AiP  }
\begin{itemize}
    \item \textbf{Attore Primario}: Utente
    \item \textbf{Precondizioni}:
    \begin{enumerate}
        \item L'utente ha avviato il programma
        \item L'utente sta visualizzando le informazioni di un AiP
    \end{enumerate}
    \item \textbf{Postcondizioni}: L'utente visualizza le informazioni del processo di conservazione, quali: data inizio, data fine, uuid utente responsabile, canale di attivazione, stato del processo, sessione di versamento, sessione di conservazione.
    \item \textbf{Flusso Principale}: Vengono mostrati a video, in riferimento all'AiP selezionato:
    \item \textbf{Flusso Principale}: Il sistema mostra a video le informazioni sul processo:
    \begin{itemize}[label=$\rhd$]
        \item Data inizio $\rightarrow$ Vedi \hyperref[-]{[UC-11.\ref{-}]}
        \item Data fine $\rightarrow$ Vedi \hyperref[-]{[UC-11.\ref{-}]}
        \item UUID utente attivatore $\rightarrow$ Vedi \hyperref[-]{[UC-11.\ref{-}]}
        \item UUID utente terminatore $\rightarrow$ Vedi \hyperref[-]{[UC-11.\ref{-}]}
        \item Nome canale di attivazione $\rightarrow$ Vedi \hyperref[-]{[UC-11.\ref{-}]}
        \item Nome canale di terminazione $\rightarrow$ Vedi \hyperref[-]{[UC-11.\ref{-}]}
        \item Stato del processo $\rightarrow$ Vedi \hyperref[-]{[UC-11.\ref{-}]}
        \item Informazioni sulla sessione di versamento $\rightarrow$ Vedi \hyperref[salvaPiuDOcs]{[UC-\ref{salvaPiuDOcs}]}
        \item Informazioni sulla sessione di conservazione $\rightarrow$ Vedi \hyperref[stampaSingoloDoc]{[UC-\ref{stampaSingoloDoc}]}
    \end{itemize}

\item \textbf{Inclusioni}: 
    \begin{itemize}[label=$\rhd$]
        \item 
        \hyperref[-]{[UC-11.\ref{-}]}, 
        \hyperref[-]{[UC-11.\ref{-}]},
        \hyperref[-]{[UC-11.\ref{-}]},
        \hyperref[-]{[UC-11.\ref{-}]}, 
        \hyperref[-]{[UC-11.\ref{-}]}, 
        \hyperref[-]{[UC-11.\ref{-}]}, 
        \hyperref[-]{[UC-11.\ref{-}]}, 
        \hyperref[salvaPiuDOcs]{[UC-\ref{salvaPiuDOcs}]}, \hyperref[stampaSingoloDoc]{[UC-\ref{stampaSingoloDoc}]}
    \end{itemize}
\end{itemize}

\subusecase{visualizzaDataInizioProcessoSessione}{Visualizza data inizio processo/sessione  }
\begin{itemize}
    \item \textbf{Attore Primario}: Utente
    \item \textbf{Precondizioni}:
    \begin{enumerate}
        \item L'utente ha avviato l'applicazione
        \item L'utente sta visualizzando le informazioni di un processo/sessione
    \end{enumerate}
    \item \textbf{Postcondizioni}: L'utente visualizza la data di inizio del/la processo/sessione
    \item \textbf{Flusso Principale}: Il sistema mostra a video la data di inizio del/la processo/sessione
\end{itemize}

\subusecase{visualizzaDataFineProcessoSessione}{Visualizza data fine processo/sessione  }
\begin{itemize}
    \item \textbf{Attore Primario}: Utente
    \item \textbf{Precondizioni}:
    \begin{enumerate}
        \item L'utente ha avviato l'applicazione
        \item L'utente sta visualizzando le informazioni di un processo/sessione
    \end{enumerate}
    \item \textbf{Postcondizioni}: L'utente visualizza la data di fine del/la processo/sessione
    \item \textbf{Flusso Principale}: Il sistema mostra a video la data di fine del/la processo/sessione
    \item \textbf{Flusso Alternativo}: La data di fine è assente in quanto il/la processo/sessione non è terminato/a
\end{itemize}

\subusecase{visualizzaUUIDUtenteAttivatore}{Visualizza UUID utente attivatore di un processo/sessione  }
\begin{itemize}
    \item \textbf{Attore Primario}: Utente
    \item \textbf{Precondizioni}:
    \begin{enumerate}
        \item L'utente ha avviato l'applicazione
        \item L'utente sta visualizzando le informazioni di un processo/sessione
    \end{enumerate}
    \item \textbf{Postcondizioni}: L'utente visualizza lo UUID dell'utente che è responsabile dell'attivazione del/la processo/sessione
    \item \textbf{Flusso Principale}: Il sistema mostra a video lo UUID dell'utente attivatore del/la processo/sessione
\end{itemize}

\subusecase{visualizzaUUIDUtenteTerminatore}{Visualizza UUID utente terminatore di un processo/sessione  }
\begin{itemize}
    \item \textbf{Attore Primario}: Utente
    \item \textbf{Precondizioni}:
    \begin{enumerate}
        \item L'utente ha avviato l'applicazione
        \item L'utente sta visualizzando le informazioni di un processo/sessione
    \end{enumerate}
    \item \textbf{Postcondizioni}: L'utente visualizza lo UUID dell'utente che è responsabile della terminazione del/la processo/sessione
    \item \textbf{Flusso Principale}: Il sistema mostra a video lo UUID dell'utente terminatore del/la processo/sessione
    \item \textbf{Flusso Alternativo}: Non viene mostrato a video lo UUID dell'utente terminatore del/la processo/sessione è assente in quanto il/la processo/sessione non è ancora terminato/a.
\end{itemize}

\subusecase{visualizzaNomeCanaleAttivazione}{Visualizza nome canale di attivazione di un processo/sessione  }
\begin{itemize}
    \item \textbf{Attore Primario}: Utente
    \item \textbf{Precondizioni}:
    \begin{enumerate}
        \item L'utente ha avviato l'applicazione
        \item L'utente sta visualizzando le informazioni di un processo/sessione
    \end{enumerate}
    \item \textbf{Postcondizioni}: L'utente visualizza il nome del canale di attivazione del/la processo/sessione
    \item \textbf{Flusso Principale}: Il sistema mostra a video il nome del canale di attivazione del/la processo/sessione
\end{itemize}

\subusecase{visualizzaNomeCanaleTerminazione}{Visualizza nome canale di terminazione di un processo/sessione  }
\begin{itemize}
    \item \textbf{Attore Primario}: Utente
    \item \textbf{Precondizioni}:
    \begin{enumerate}
        \item L'utente ha avviato l'applicazione
        \item L'utente sta visualizzando le informazioni di un processo/sessione
    \end{enumerate}
    \item \textbf{Postcondizioni}: L'utente visualizza il nome del canale di terminazione del/la processo/sessione
    \item \textbf{Flusso Principale}: Il sistema mostra a video il nome del canale di terminazione del/la processo/sessione
    \item \textbf{Flusso Alternativo}: Non viene mostrato a video il nome del canale di terminazione del/la processo/sessione in quanto il/la processo/sessione non è ancora terminato/a.
\end{itemize}

\subusecase{visualizzaStatoProcessoSessione}{Visualizza stato di un processo/sessione  }
\begin{itemize}
    \item \textbf{Attore Primario}: Utente
    \item \textbf{Precondizioni}:
    \begin{enumerate}
        \item L'utente ha avviato l'applicazione
        \item L'utente sta visualizzando le informazioni di un processo/sessione
    \end{enumerate}
    \item \textbf{Postcondizioni}: L'utente visualizza lo stato del/la processo/sessione
    \item \textbf{Flusso Principale}: Il sistema mostra a video lo stato del/la processo/sessione
\end{itemize}

\usecase{visualizzaInfoSessioneVersamento}{Visualizza Informazioni sulla Sessione di Versamento}
\begin{itemize}
    \item \textbf{Attore Primario}: Utente
    \item \textbf{Precondizioni}:
    \begin{enumerate}
        \item L'utente ha avviato il programma
        \item L'utente sta visualizzando i dettagli di un processo di conservazione
    \end{enumerate}
    \item \textbf{Postcondizioni}: L'utente visualizza i dettagli della sessione di versamento, quali: data inizio, data fine, UUID utente attivatore, UUID utente terminatore, nome canale di attivazione, nome canale di terminazione, stato del processo
    \item \textbf{Flusso Principale}: Vengono mostrati a video, relativamente alla sessione di versamento del processo selezionato:
    \begin{itemize}
       \item Data inizio $\rightarrow$ Vedi \hyperref[-]{[UC-11.\ref{-}]}
        \item Data fine $\rightarrow$ Vedi \hyperref[-]{[UC-11.\ref{-}]}
        \item UUID utente attivatore $\rightarrow$ Vedi \hyperref[-]{[UC-11.\ref{-}]}
        \item UUID utente terminatore $\rightarrow$ Vedi \hyperref[-]{[UC-11.\ref{-}]}
        \item Nome canale di attivazione $\rightarrow$ Vedi \hyperref[-]{[UC-11.\ref{-}]}
        \item Nome canale di terminazione $\rightarrow$ Vedi \hyperref[-]{[UC-11.\ref{-}]}
        \item Stato del processo $\rightarrow$ Vedi \hyperref[-]{[UC-11.\ref{-}]}
    \end{itemize}
    \item \textbf{Inclusioni}: 
    \begin{itemize}[label=$\rhd$]
        \item 
        \hyperref[-]{[UC-11.\ref{-}]}, 
        \hyperref[-]{[UC-11.\ref{-}]},
        \hyperref[-]{[UC-11.\ref{-}]},
        \hyperref[-]{[UC-11.\ref{-}]}, 
        \hyperref[-]{[UC-11.\ref{-}]}, 
        \hyperref[-]{[UC-11.\ref{-}]}, 
        \hyperref[-]{[UC-11.\ref{-}]}, 
    \end{itemize}
\end{itemize}


\usecase{visualizzaInfoSessioneConservazione}{Visualizza Informazioni sulla Sessione di Conservazione}
\begin{itemize}
    \item \textbf{Attore Primario}: Utente
    \item \textbf{Precondizioni}:
    \begin{enumerate}
        \item L'utente ha avviato il programma
        \item L'utente sta visualizzando i dettagli di un processo di conservazione
    \end{enumerate}
    \item \textbf{Postcondizioni}: L'utente visualizza i dettagli della sessione di conservazione, quali: data inizio, data fine, UUID utente attivatore, UUID utente terminatore, nome canale di attivazione, nome canale di terminazione, stato del processo
       \item \textbf{Flusso Principale}: Vengono mostrati a video, relativamente alla sessione di versamento del processo selezionato:
    \begin{itemize}
       \item Data inizio $\rightarrow$ Vedi \hyperref[-]{[UC-11.\ref{-}]}
        \item Data fine $\rightarrow$ Vedi \hyperref[-]{[UC-11.\ref{-}]}
        \item UUID utente attivatore $\rightarrow$ Vedi \hyperref[-]{[UC-11.\ref{-}]}
        \item UUID utente terminatore $\rightarrow$ Vedi \hyperref[-]{[UC-11.\ref{-}]}
        \item Nome canale di attivazione $\rightarrow$ Vedi \hyperref[-]{[UC-11.\ref{-}]}
        \item Nome canale di terminazione $\rightarrow$ Vedi \hyperref[-]{[UC-11.\ref{-}]}
        \item Stato del processo $\rightarrow$ Vedi \hyperref[-]{[UC-11.\ref{-}]}
    \end{itemize}
    \item \textbf{Inclusioni}: 
    \begin{itemize}[label=$\rhd$]
        \item 
        \hyperref[-]{[UC-11.\ref{-}]}, 
        \hyperref[-]{[UC-11.\ref{-}]},
        \hyperref[-]{[UC-11.\ref{-}]},
        \hyperref[-]{[UC-11.\ref{-}]}, 
        \hyperref[-]{[UC-11.\ref{-}]}, 
        \hyperref[-]{[UC-11.\ref{-}]}, 
        \hyperref[-]{[UC-11.\ref{-}]}, 
    \end{itemize}
\end{itemize}
% DA UC-45

\usecase{visualizzaDescrizioneDocumento}{Visualizza descrizione documento}
\begin{figure}[H]
    \centering
    \includegraphics[width=0.6\textwidth]{../assets/uml/UC50.png}
    \caption{UC50 - Visualizzazione descrizione documento}
    \label{fig:uc_visualizzaDescrizioneDocumento}
\end{figure}
\begin{itemize}
    \item \textbf{Attore Primario}: Utente
    \item \textbf{Precondizioni}:
    \begin{enumerate}
        \item L'utente ha avviato l'applicazione
        \item L'utente ha selezionato un documento
    \end{enumerate}
    \item \textbf{Postcondizioni}: L'utente visualizza la descrizione del documento selezionato
    \item \textbf{Flusso Principale}: 
    \begin{enumerate}
        \item Il sistema mostra a video la descrizione del documento selezionato
    \end{enumerate}
\end{itemize}

\usecase{visualizzaListaSoggettiCoinvolti}{Visualizza lista soggetti coinvolti nel documento}
\begin{figure}[H]
    \centering
    \includegraphics[width=0.6\textwidth]{../assets/uml/UC51.png}
    \caption{UC51 - Visualizzazione lista soggetti coinvolti nel documento}
    \label{fig:uc_visualizzaListaSoggettiCoinvolti}
\end{figure}
\begin{itemize}
    \item \textbf{Attore Primario}: Utente
    \item \textbf{Precondizioni}:
    \begin{enumerate}
        \item L'utente ha avviato l'applicazione
        \item L'utente ha selezionato un documento
    \end{enumerate}
    \item \textbf{Postcondizioni}: L'utente visualizza la lista dei soggetti coinvolti nel documento selezionato
    \item \textbf{Flusso Principale}:
        \begin{enumerate}
            \item L'utente visualizza la lista dei soggetti coinvolti nel documento selezionato, per ognuno mostra:
            \begin{itemize}
                \item Informazioni del soggetto coinvolto (\ref{visualizzaInfoSoggettoCoinvolto})
            \end{itemize}
        \end{enumerate} 
    \item \textbf{Inclusioni}: \ref{visualizzaInfoSoggettoCoinvolto} Visualizza informazioni del soggetto coinvolto in un documento
\end{itemize}
\begin{figure}[H]
    \centering
    \includegraphics[width=0.6\textwidth]{../assets/uml/IncUC51.png}
    \caption{Inclusioni UC51 - Visualizzazione lista soggetti coinvolti nel documento}
    \label{fig:inclusioniVisualizzaListaSoggettiCoinvolti}
\end{figure}

\subusecase{visualizzaInfoSoggettoCoinvolto}{Visualizza informazioni soggetto coinvolto in un documento}
\begin{figure}[H]
    \centering
    \includegraphics[width=1\textwidth]{../assets/uml/GenUC51.1.png}
    \caption{Generalizzazione UC51.1 - Visualizzazione informazioni soggetto coinvolto in un documento}
    \label{fig:generalizzazioneVisualizzaInfoSoggettoCoinvolto}
\end{figure}
\begin{itemize}
    \item \textbf{Attore Primario}: Utente
    \item \textbf{Precondizioni}:
    \begin{enumerate}
        \item L'utente ha avviato l'applicazione
        \item L'utente sta visualizzando una lista di soggetti o un singolo soggetto coinvolto in un documento
    \end{enumerate}

    \item \textbf{Postcondizioni}:
    \begin{itemize}
        \item L'utente visualizza le informazioni del soggetto coinvolto nel documento selezionato
    \end{itemize}

    \item \textbf{Flusso Principale}:
    \begin{enumerate}
        \item Il sistema visualizza le informazioni comuni del soggetto, quali:
        \begin{itemize}
            \item Ruolo del soggetto nel documento (\ref{visualizzaRuoloSoggetto})
            \item Tipo di soggetto (\ref{visualizzaTipoSoggetto})
        \end{itemize}
    \end{enumerate}
    \item \textbf{Inclusioni}:
    \begin{itemize}
        \item \ref{visualizzaRuoloSoggetto} Visualizza ruolo soggetto nel documento
        \item \ref{visualizzaTipoSoggetto} Visualizza tipo di soggetto
    \end{itemize}
\end{itemize}
\begin{figure}[H]
    \centering
    \includegraphics[width=0.6\textwidth]{../assets/uml/IncUC51.1.png}
    \caption{Inclusioni UC51.1 - Visualizza informazioni soggetto coinvolto in un documento}
    \label{fig:inclusioniVisualizzaInfoSoggettoCoinvolto}
\end{figure}

\subsubusecase{visualizzaInfoPF}{Visualizza informazioni del soggetto persona fisica}
\begin{itemize}
    \item \textbf{Attore Primario}: Utente

    \item \textbf{Precondizioni}:
    \begin{enumerate}
        \item Sono soddisfatte le precondizioni di \ref{visualizzaInfoSoggettoCoinvolto}
        \item Il soggetto è di tipo Persona Fisica
    \end{enumerate}

    \item \textbf{Postcondizioni}:
    \begin{itemize}
        \item L'utente visualizza i dati anagrafici della Persona Fisica
    \end{itemize}

    \item \textbf{Flusso Principale}:
    \begin{enumerate}
        \item Il sistema visualizza le informazioni comuni del soggetto
        \item Il sistema visualizza le informazioni identificative del soggetto, quali:
        \begin{itemize}
            \item Nome \ref{visualizzaNomeSoggetto}
            \item Cognome \ref{visualizzaCognomeSoggetto}
            \item Codice Fiscale \ref{visualizzaCodiceFiscaleSoggetto}
            \item Indirizzi digitali di riferimento \ref{visualizzaIndirizziDigitaliSoggetto}
        \end{itemize}
    \end{enumerate}
    \item \textbf{Inclusioni}:
    \begin{itemize}
        \item \ref{visualizzaNomeSoggetto} Visualizza nome soggetto
        \item \ref{visualizzaCognomeSoggetto} Visualizza cognome soggetto
        \item \ref{visualizzaCodiceFiscaleSoggetto} Visualizza codice fiscale soggetto
        \item \ref{visualizzaIndirizziDigitaliSoggetto} Visualizza indirizzi digitali di riferimento soggetto
    \end{itemize}
    \item \textbf{Specializza}: \ref{visualizzaInfoSoggettoCoinvolto} Visualizza informazioni del soggetto coinvolto in un documento
\end{itemize}
\begin{figure}[H]
    \centering
    \includegraphics[width=1\textwidth]{../assets/uml/IncUC51.1.1.png}
      \caption{Inclusioni UC51.1.1 - Visualizza informazioni del soggetto persona fisica}
    \label{fig:inclusioniVisualizzaInfoPF}
\end{figure}
\deepusecase{visualizzaNomeSoggetto}{Visualizza nome soggetto}
\begin{itemize}
    \item \textbf{Attore Primario}: Utente
    \item \textbf{Precondizioni}:
    \begin{enumerate}
        \item L'utente ha avviato l'applicazione
        \item L'utente sta visualizzando le informazioni di una Persona Fisica coinvolta in un documento, oppure\\L'utente sta visualizzando le informazioni di un soggetto AS coinvolto in un documento
    \end{enumerate}
    \item \textbf{Postcondizioni}: L'utente visualizza il nome del soggetto coinvolto nel documento selezionato
    \item \textbf{Flusso Principale}:
    \begin{enumerate}
        \item Il sistema mostra a video il nome del soggetto coinvolto nel documento selezionato
    \end{enumerate}
\end{itemize}

\deepusecase{visualizzaCognomeSoggetto}{Visualizza cognome soggetto}
\begin{itemize}
    \item \textbf{Attore Primario}: Utente
    \item \textbf{Precondizioni}:
    \begin{enumerate}
        \item L'utente ha avviato l'applicazione
        \item L'utente sta visualizzando le informazioni di una Persona Fisica coinvolta in un documento, oppure\\L'utente sta visualizzando le informazioni di un soggetto AS coinvolto in un documento
    \end{enumerate}
    \item \textbf{Postcondizioni}: L'utente visualizza il cognome del soggetto coinvolto nel documento selezionato
    \item \textbf{Flusso Principale}:
    \begin{enumerate}
        \item Il sistema mostra a video il cognome del soggetto coinvolto nel documento selezionato
    \end{enumerate}
\end{itemize}

\deepusecase{visualizzaCodiceFiscaleSoggetto}{Visualizza codice fiscale soggetto}
\begin{itemize}
    \item \textbf{Attore Primario}: Utente
    \item \textbf{Precondizioni}:
    \begin{enumerate}
        \item L'utente ha avviato l'applicazione
        \item L'utente sta visualizzando le informazioni di una Persona Fisica coinvolta in un documento, oppure\\L'utente sta visualizzando le informazioni di una Persona Giuridica coinvolta in un documento, oppure\\L'utente sta visualizzando le informazioni di un soggetto AS coinvolto in un documento
    \end{enumerate}
    \item \textbf{Postcondizioni}: L'utente visualizza il codice fiscale del soggetto coinvolto nel documento selezionato
    \item \textbf{Flusso Principale}:
    \begin{enumerate}
        \item Il sistema mostra a video il codice fiscale del soggetto coinvolto nel documento selezionato
    \end{enumerate}
\end{itemize}

\deepusecase{visualizzaIndirizziDigitaliSoggetto}{Visualizza indirizzi digitali di riferimento soggetto}
\begin{itemize}
    \item \textbf{Attore Primario}: Utente
    \item \textbf{Precondizioni}:
    \begin{enumerate}
        \item L'utente ha avviato l'applicazione
        \item L'utente sta visualizzando le informazioni di una Persona Fisica coinvolta in un documento, oppure\\L'utente sta visualizzando le informazioni di una Persona Giuridica coinvolta in un documento, oppure\\L'utente sta visualizzando le informazioni di un soggetto AS coinvolto in un documento, oppure\\L'utente sta visualizzando le informazioni di un soggetto PAE coinvolto in un documento
    \end{enumerate}
    \item \textbf{Postcondizioni}: L'utente visualizza gli indirizzi digitali di riferimento del soggetto coinvolto nel documento selezionato
    \item \textbf{Flusso Principale}:
    \begin{enumerate}
        \item Il sistema mostra a video gli indirizzi digitali di riferimento del soggetto coinvolto nel documento selezionato
    \end{enumerate}
\end{itemize}

\subsubusecase{visualizzaInfoPG}{Visualizza informazioni del soggetto persona giuridica}
\begin{itemize}
    \item \textbf{Attore Primario}: Utente

    \item \textbf{Precondizioni}:
    \begin{enumerate}
        \item Sono soddisfatte le precondizioni di \ref{visualizzaInfoSoggettoCoinvolto}
        \item Il soggetto è di tipo Persona Giuridica
    \end{enumerate}

    \item \textbf{Postcondizioni}:
    \begin{itemize}
        \item L'utente visualizza i dati identificativi della Persona Giuridica
    \end{itemize}

    \item \textbf{Flusso Principale}:
    \begin{enumerate}
        \item Il sistema visualizza le informazioni comuni del soggetto
        \item Il sistema visualizza le informazioni identificative del soggetto, quali:
        \begin{itemize}
            \item Denominazione dell'organizzazione \ref{visualizzaDenominazioneOrganizzazioneSoggetto}
            \item Codice Fiscale / Partita IVA \ref{visualizzaCodiceFiscaleSoggetto} / \ref{visualizzaPartitaIVA}
            \item Denominazione dell'ufficio \ref{visualizzaDenominazioneUfficioSoggetto}
            \item Indirizzi digitali di riferimento \ref{visualizzaIndirizziDigitaliSoggetto}
        \end{itemize}
    \end{enumerate}
    \item \textbf{Inclusioni}: 
    \begin{itemize}
        \item \ref{visualizzaDenominazioneOrganizzazioneSoggetto} Visualizza denominazione organizzazione soggetto
        \item \ref{visualizzaCodiceFiscaleSoggetto} Visualizza codice fiscale soggetto
        \item \ref{visualizzaPartitaIVA} Visualizza partita IVA soggetto
        \item \ref{visualizzaDenominazioneUfficioSoggetto} Visualizza denominazione ufficio soggetto
        \item \ref{visualizzaIndirizziDigitaliSoggetto} Visualizza indirizzi digitali di riferimento soggetto
    \end{itemize}
    \item \textbf{Specializza}: \ref{visualizzaInfoSoggettoCoinvolto} Visualizza informazioni del soggetto coinvolto in un documento
\end{itemize}
\begin{figure}[H]
    \centering
    \includegraphics[width=1\textwidth]{../assets/uml/IncUC51.1.2.png}
      \caption{Inclusioni UC51.1.2 - Visualizza informazioni del soggetto persona giuridica}
    \label{fig:inclusioniVisualizzaInfoPG}
\end{figure}

\deepusecase{visualizzaDenominazioneOrganizzazioneSoggetto}{Visualizza denominazione organizzazione soggetto}
\begin{itemize}
    \item \textbf{Attore Primario}: Utente
    \item \textbf{Precondizioni}:
    \begin{enumerate}
        \item L'utente ha avviato l'applicazione
        \item L'utente sta visualizzando le informazioni di una Persona Giuridica coinvolta in un documento, oppure\\L'utente sta visualizzando le informazioni di un soggetto AS coinvolto in un documento
    \end{enumerate}
    \item \textbf{Postcondizioni}: L'utente visualizza la denominazione dell'organizzazione del soggetto coinvolto nel documento selezionato
    \item \textbf{Flusso Principale}:
    \begin{enumerate}
        \item Il sistema mostra a video la denominazione dell'organizzazione del soggetto coinvolto nel documento selezionato
    \end{enumerate}
\end{itemize}

\deepusecase{visualizzaPartitaIVA}{Visualizza partita IVA soggetto}
\begin{itemize}
    \item \textbf{Attore Primario}: Utente
    \item \textbf{Precondizioni}:
    \begin{enumerate}
        \item L'utente ha avviato l'applicazione
        \item L'utente sta visualizzando le informazioni di una Persona Giuridica coinvolta in un documento
    \end{enumerate}
    \item \textbf{Postcondizioni}: L'utente visualizza la partita IVA della Persona Giuridica coinvolta nel documento selezionato
    \item \textbf{Flusso Principale}:
    \begin{enumerate}
        \item Il sistema mostra a video la partita IVA della Persona Giuridica coinvolta nel documento selezionato
    \end{enumerate}
\end{itemize}

\deepusecase{visualizzaDenominazioneUfficioSoggetto}{Visualizza denominazione ufficio soggetto}
\begin{itemize}
    \item \textbf{Attore Primario}: Utente
    \item \textbf{Precondizioni}:
    \begin{enumerate}
        \item L'utente ha avviato l'applicazione
        \item L'utente sta visualizzando le informazioni di una Persona Giuridica coinvolta in un documento, oppure\\L'utente sta visualizzando le informazioni di un soggetto AS coinvolto in un documento, oppure\\L'utente sta visualizzando le informazioni di un soggetto PAE coinvolto in un documento
    \end{enumerate}
    \item \textbf{Postcondizioni}: L'utente visualizza la denominazione dell'ufficio del soggetto coinvolto nel documento selezionato
    \item \textbf{Flusso Principale}:
    \begin{enumerate}
        \item Il sistema mostra a video la denominazione dell'ufficio del soggetto coinvolto nel documento selezionato
    \end{enumerate}
\end{itemize}

\subsubusecase{visualizzaInfoAS}{Visualizza informazioni del soggetto AS}
\begin{itemize}
    \item \textbf{Attore Primario}: Utente

    \item \textbf{Precondizioni}:
    \begin{enumerate}
        \item Sono soddisfatte le precondizioni di \ref{visualizzaInfoSoggettoCoinvolto}
        \item Il soggetto è di tipo AS
    \end{enumerate}

    \item \textbf{Postcondizioni}:
    \begin{itemize}
        \item L'utente visualizza le informazioni identificative del soggetto AS
    \end{itemize}

    \item \textbf{Flusso Principale}:
    \begin{enumerate}
        \item Il sistema visualizza le informazioni comuni del soggetto
        \item Il sistema visualizza le informazioni identificative del soggetto, quali:
        \begin{itemize}
            \item Cognome (\ref{visualizzaCognomeSoggetto})
            \item Nome (\ref{visualizzaNomeSoggetto})
            \item Codice Fiscale (\ref{visualizzaCodiceFiscaleSoggetto})
            \item Denominazione dell'organizzazione (\ref{visualizzaDenominazioneOrganizzazioneSoggetto})
            \item Denominazione dell'ufficio (\ref{visualizzaDenominazioneUfficioSoggetto})
            \item Indirizzi digitali di riferimento (\ref{visualizzaIndirizziDigitaliSoggetto})
        \end{itemize}
    \end{enumerate}
    \item \textbf{Inclusioni}: 
    \begin{itemize}
        \item \ref{visualizzaNomeSoggetto} Visualizza nome soggetto
        \item \ref{visualizzaCognomeSoggetto} Visualizza cognome soggetto
        \item \ref{visualizzaCodiceFiscaleSoggetto} Visualizza codice fiscale soggetto
        \item \ref{visualizzaDenominazioneOrganizzazioneSoggetto} Visualizza denominazione organizzazione soggetto
        \item \ref{visualizzaDenominazioneUfficioSoggetto} Visualizza denominazione ufficio soggetto
        \item \ref{visualizzaIndirizziDigitaliSoggetto} Visualizza indirizzi digitali di riferimento soggetto
    \end{itemize}
    \item \textbf{Specializza}: \ref{visualizzaInfoSoggettoCoinvolto} Visualizza informazioni del soggetto coinvolto in un documento
\end{itemize}
\begin{figure}[H]
    \centering
    \includegraphics[width=1\textwidth]{../assets/uml/IncUC51.1.3.png}
    \caption{Inclusioni UC51.1.3 - Visualizza informazioni del soggetto AS}
    \label{fig:inclusioniVisualizzaInfoAS}
\end{figure}

\subsubusecase{visualizzaInfoPAI}{Visualizza informazioni del soggetto PAI}
\begin{itemize}
    \item \textbf{Attore Primario}: Utente

    \item \textbf{Precondizioni}:
    \begin{enumerate}
        \item Sono soddisfatte le precondizioni di \ref{visualizzaInfoSoggettoCoinvolto}
        \item Il soggetto è di tipo PAI
    \end{enumerate}

    \item \textbf{Postcondizioni}:
    \begin{itemize}
        \item L'utente visualizza le informazioni identificative del soggetto PAI
    \end{itemize}

    \item \textbf{Flusso Principale}:
    \begin{enumerate}
        \item Il sistema visualizza le informazioni comuni del soggetto
        \item Il sistema visualizza le informazioni identificative del soggetto, quali:
        \begin{itemize}
            \item Denominazione Amministrazione / Codice IPA (\ref{visualizzaDenominazioneAmministrazioneCodiceIPA})
            \item Denominazione Amministrazione AOO / Codice IPA AOO (\ref{visualizzaDenominazioneAmministrazioneAOOCodiceIPAOOO})
            \item Denominazione Amministrazione UOR / Codice IPA UOR (\ref{visualizzaDenominazioneAmministrazioneUORCodiceIPAUOR})
            \item Indirizzi digitali di riferimento (\ref{visualizzaIndirizziDigitaliSoggetto})
        \end{itemize}
    \end{enumerate}
    \item \textbf{Inclusioni}: 
    \begin{itemize}
        \item \ref{visualizzaDenominazioneAmministrazioneCodiceIPA} Visualizza denominazione amministrazione / codice IPA soggetto
        \item \ref{visualizzaDenominazioneAmministrazioneAOOCodiceIPAOOO} Visualizza denominazione amministrazione AOO / codice IPA AOO soggetto
        \item \ref{visualizzaDenominazioneAmministrazioneUORCodiceIPAUOR} Visualizza denominazione amministrazione UOR / codice IPA UOR soggetto
        \item \ref{visualizzaIndirizziDigitaliSoggetto} Visualizza indirizzi digitali di riferimento soggetto
    \end{itemize}
    \item \textbf{Specializza}: \ref{visualizzaInfoSoggettoCoinvolto} Visualizza informazioni del soggetto coinvolto in un documento
\end{itemize}
\begin{figure}[H]
    \centering
    \includegraphics[width=1\textwidth]{../assets/uml/IncUC51.1.4.png}
    \caption{Inclusioni UC51.1.4 - Visualizza informazioni del soggetto PAI}
    \label{fig:inclusioniVisualizzaInfoPAI}
\end{figure}

\deepusecase{visualizzaDenominazioneAmministrazioneCodiceIPA}{Visualizza denominazione amministrazione / codice IPA soggetto}
\begin{itemize}
    \item \textbf{Attore Primario}: Utente
    \item \textbf{Precondizioni}:
    \begin{enumerate}
        \item L'utente ha avviato l'applicazione
        \item L'utente sta visualizzando le informazioni di un soggetto PAI coinvolto in un documento
    \end{enumerate}
    \item \textbf{Postcondizioni}: L'utente visualizza la denominazione dell'amministrazione e il codice IPA del soggetto PAI coinvolto nel documento selezionato
    \item \textbf{Flusso Principale}:
        \begin{enumerate}
            \item Il sistema mostra a video la denominazione dell'amministrazione e il codice IPA del soggetto PAI coinvolto nel documento selezionato
        \end{enumerate}
\end{itemize}

\deepusecase{visualizzaDenominazioneAmministrazioneAOOCodiceIPAOOO}{Visualizza denominazione amministrazione AOO / codice IPA AOO soggetto}
\begin{itemize}
    \item \textbf{Attore Primario}: Utente
    \item \textbf{Precondizioni}:
    \begin{enumerate}
        \item L'utente ha avviato l'applicazione
        \item L'utente sta visualizzando le informazioni di un soggetto PAI coinvolto in un documento
    \end{enumerate}
    \item \textbf{Postcondizioni}: L'utente visualizza la denominazione dell'amministrazione AOO e il codice IPA AOO del soggetto PAI coinvolto nel documento selezionato
    \item \textbf{Flusso Principale}:
        \begin{enumerate}
            \item Il sistema mostra a video la denominazione dell'amministrazione AOO e il codice IPA AOO del soggetto PAI coinvolto nel documento selezionato
        \end{enumerate}
\end{itemize}

\deepusecase{visualizzaDenominazioneAmministrazioneUORCodiceIPAUOR}{Visualizza denominazione amministrazione UOR / codice IPA UOR soggetto}
\begin{itemize}
    \item \textbf{Attore Primario}: Utente
    \item \textbf{Precondizioni}:
    \begin{enumerate}
        \item L'utente ha avviato l'applicazione
        \item L'utente sta visualizzando le informazioni di un soggetto PAI coinvolto in un documento
    \end{enumerate}
    \item \textbf{Postcondizioni}: L'utente visualizza la denominazione dell'amministrazione UOR e il codice IPA UOR del soggetto PAI coinvolto nel documento selezionato
    \item \textbf{Flusso Principale}:
        \begin{enumerate}
            \item Il sistema mostra a video la denominazione dell'amministrazione UOR e il codice IPA UOR del soggetto PAI coinvolto nel documento selezionato
        \end{enumerate}
\end{itemize}

\subsubusecase{visualizzaInfoPAE}{Visualizza informazioni del soggetto PAE}
\begin{itemize}
    \item \textbf{Attore Primario}: Utente

    \item \textbf{Precondizioni}:
    \begin{enumerate}
        \item Sono soddisfatte le precondizioni di \ref{visualizzaInfoSoggettoCoinvolto}
        \item Il soggetto è di tipo PAE
    \end{enumerate}

    \item \textbf{Postcondizioni}:
    \begin{itemize}
        \item L'utente visualizza le informazioni identificative del soggetto PAE
    \end{itemize}

    \item \textbf{Flusso Principale}:
    \begin{enumerate}
        \item Il sistema visualizza le informazioni comuni del soggetto
        \item Il sistema visualizza le informazioni identificative del soggetto, quali:
        \begin{itemize}
            \item Denominazione Amministrazione (\ref{visualizzaDenominazioneAmministrazioneSoggetto})
            \item Denominazione Ufficio (\ref{visualizzaDenominazioneUfficioSoggetto})
            \item Indirizzi Digitali Di Riferimento (\ref{visualizzaIndirizziDigitaliSoggetto})
        \end{itemize}
    \end{enumerate}
    \item \textbf{Inclusioni}: 
    \begin{itemize}
        \item \ref{visualizzaIndirizziDigitaliSoggetto} Visualizza indirizzi digitali di riferimento soggetto
        \item \ref{visualizzaDenominazioneUfficioSoggetto} Visualizza denominazione ufficio soggetto
        \item \ref{visualizzaDenominazioneAmministrazioneSoggetto} Visualizza denominazione organizzazione soggetto
    \end{itemize}
    \item \textbf{Specializza}: \ref{visualizzaInfoSoggettoCoinvolto} Visualizza informazioni del soggetto coinvolto in un documento
\end{itemize}
\begin{figure}[H]
    \centering
    \includegraphics[width=1\textwidth]{../assets/uml/IncUC51.1.5.png}
    \caption{Inclusioni UC51.1.5 - Visualizza informazioni del soggetto PAE}
    \label{fig:inclusioniVisualizzaInfoPAE}
\end{figure}

\deepusecase{visualizzaDenominazioneAmministrazioneSoggetto}{Visualizza denominazione amministrazione soggetto}
\begin{itemize}
    \item \textbf{Attore Primario}: Utente
    \item \textbf{Precondizioni}:
    \begin{enumerate}
        \item L'utente ha avviato l'applicazione
        \item L'utente sta visualizzando le informazioni di un soggetto PAE coinvolto in un documento
    \end{enumerate}
    \item \textbf{Postcondizioni}: L'utente visualizza la denominazione dell'amministrazione del soggetto PAE coinvolto nel documento selezionato
    \item \textbf{Flusso Principale}:
        \begin{enumerate}
            \item Il sistema mostra a video la denominazione dell'amministrazione del soggetto PAE coinvolto nel documento selezionato
        \end{enumerate}
\end{itemize}

\subsubusecase{visualizzaInfoSW}{Visualizza informazioni del soggetto SW}
\begin{itemize}
    \item \textbf{Attore Primario}: Utente

    \item \textbf{Precondizioni}:
    \begin{enumerate}
        \item Sono soddisfatte le precondizioni di \ref{visualizzaInfoSoggettoCoinvolto}
        \item Il soggetto è di tipo SW
    \end{enumerate}

    \item \textbf{Postcondizioni}:
    \begin{itemize}
        \item L'utente visualizza le informazioni identificative del soggetto SW
    \end{itemize}

    \item \textbf{Flusso Principale}:
    \begin{enumerate}
        \item Il sistema visualizza le informazioni comuni del soggetto
        \item Il sistema visualizza le informazioni identificative del soggetto, quali:
        \begin{itemize}
            \item Denominazione Sistema \ref{visualizzaDenominazioneSistemaSoggetto}
        \end{itemize}
    \end{enumerate}

    \item \textbf{Specializza}: \ref{visualizzaInfoSoggettoCoinvolto} Visualizza informazioni del soggetto coinvolto in un documento
      \item \textbf{Inclusioni}: \ref{visualizzaDenominazioneSistemaSoggetto} Visualizza denominazione sistema soggetto
\end{itemize}
\begin{figure}[H]
    \centering
    \includegraphics[width=0.6\textwidth]{../assets/uml/IncUC51.1.6.png}
    \caption{Inclusioni UC51.1.6 - Visualizza informazioni del soggetto SW}
    \label{fig:inclusioniVisualizzaInfoSW}
\end{figure}

\deepusecase{visualizzaDenominazioneSistemaSoggetto}{Visualizza denominazione sistema soggetto}
\begin{itemize}
    \item \textbf{Attore Primario}: Utente
    \item \textbf{Precondizioni}:
    \begin{enumerate}
        \item L'utente ha avviato l'applicazione
        \item L'utente sta visualizzando le informazioni di un soggetto SW coinvolto in un documento
    \end{enumerate}
    \item \textbf{Postcondizioni}: L'utente visualizza la denominazione del sistema del soggetto SW coinvolto nel documento selezionato
    \item \textbf{Flusso Principale}:
        \begin{enumerate}
            \item Il sistema mostra a video la denominazione del sistema del soggetto SW coinvolto nel documento selezionato
        \end{enumerate}
\end{itemize}

\subusecase{visualizzaRuoloSoggetto}{Visualizza ruolo soggetto nel documento}
\begin{itemize}
    \item \textbf{Attore Primario}: Utente
    \item \textbf{Precondizioni}:
    \begin{enumerate}
        \item L'utente ha avviato l'applicazione
        \item L'utente sta visualizzando le informazioni di un soggetto coinvolto in un documento
    \end{enumerate}
    \item \textbf{Postcondizioni}: L'utente visualizza il ruolo del soggetto coinvolto nel documento selezionato
    \item \textbf{Flusso Principale}:
        \begin{enumerate}
            \item Il sistema mostra a video il ruolo del soggetto coinvolto nel documento selezionato
        \end{enumerate}
\end{itemize}

\subusecase{visualizzaTipoSoggetto}{Visualizza tipo di soggetto}
\begin{itemize}
    \item \textbf{Attore Primario}: Utente
    \item \textbf{Precondizioni}:
    \begin{enumerate}
        \item L'utente ha avviato l'applicazione
        \item L'utente sta visualizzando le informazioni di un soggetto coinvolto in un documento
    \end{enumerate}
    \item \textbf{Postcondizioni}: L'utente visualizza il tipo del soggetto coinvolto nel documento selezionato
    \item \textbf{Flusso Principale}:
        \begin{enumerate}
            \item Il sistema mostra a video il tipo del soggetto coinvolto nel documento selezionato
        \end{enumerate}
\end{itemize}


\usecase{visualizzaInfoClassificazioneDocumento}{Visualizza informazioni di classificazione del documento}
\begin{figure}[H]
    \centering
    \includegraphics[width=0.6\textwidth]{../assets/uml/UC52.png}
    \caption{UC52 - Visualizzazione informazioni di classificazione del documento}
    \label{fig:uc_visualizzaInfoClassificazioneDocumento}
\end{figure}
\begin{itemize}
    \item \textbf{Attore Primario}: Utente
    \item \textbf{Precondizioni}:
    \begin{enumerate}
        \item L'utente ha avviato l'applicazione
        \item L'utente ha selezionato un documento
    \end{enumerate}
    \item \textbf{Postcondizioni}: L'utente visualizza le informazioni di classificazione del documento selezionato
    \item \textbf{Flusso Principale}:
        \begin{enumerate}
            \item Il sistema mostra a video le informazioni di classificazione, quali:
            \begin{itemize}
                \item Indice di classificazione (\ref{visualizzaIndiceClassificazioneDocumento})
                \item Descrizione (\ref{visualizzaDescrizioneIndiceClassificazioneDocumento})
                \item Piano di classificazione (\ref{visualizzaURIPianoClassificazioneDocumento})
            \end{itemize}
        \end{enumerate}
    \item \textbf{Inclusioni}:
    \begin{itemize}
        \item \ref{visualizzaIndiceClassificazioneDocumento} Visualizza Indice di classificazione documento
        \item \ref{visualizzaDescrizioneIndiceClassificazioneDocumento} Visualizza Descrizione dell'Indice di classificazione documento
        \item \ref{visualizzaURIPianoClassificazioneDocumento} Visualizza URI Piano di Classificazione documento
    \end{itemize}
\end{itemize}
\begin{figure}[H]
    \centering
    \includegraphics[width=1\textwidth]{../assets/uml/IncUC52.png}
    \caption{Inclusioni UC52 - Visualizzazione informazioni di classificazione del documento}
    \label{fig:inclusioniVisualizzaInfoClassificazioneDocumento}
\end{figure}

\subusecase{visualizzaIndiceClassificazioneDocumento}{Visualizza indice di classificazione documento}
\begin{itemize}
    \item \textbf{Attore Primario}: Utente
    \item \textbf{Precondizioni}:
    \begin{enumerate}
        \item L'utente ha avviato l'applicazione
        \item L'utente sta visualizzando le informazioni di classificazione di un documento
    \end{enumerate}
    \item \textbf{Postcondizioni}: L'utente visualizza l'indice di classificazione del documento selezionato
    \item \textbf{Flusso Principale}:
        \begin{enumerate}
            \item Il sistema mostra a video l'indice di classificazione del documento selezionato
        \end{enumerate}
\end{itemize}

\subusecase{visualizzaDescrizioneIndiceClassificazioneDocumento}{Visualizza descrizione dell'indice di classificazione documento}
\begin{itemize}
    \item \textbf{Attore Primario}: Utente
    \item \textbf{Precondizioni}:
    \begin{enumerate}
        \item L'utente ha avviato l'applicazione
        \item L'utente sta visualizzando le informazioni di classificazione di un documento
    \end{enumerate}
    \item \textbf{Postcondizioni}: L'utente visualizza la descrizione dell'indice di classificazione del documento selezionato
    \item \textbf{Flusso Principale}:
        \begin{enumerate}
            \item Il sistema mostra a video la descrizione dell'indice di classificazione del documento selezionato
        \end{enumerate}
\end{itemize}

\subusecase{visualizzaURIPianoClassificazioneDocumento}{Visualizza URI piano di classificazione documento}
\begin{itemize}
    \item \textbf{Attore Primario}: Utente
    \item \textbf{Precondizioni}:
    \begin{enumerate}
        \item L'utente ha avviato l'applicazione
        \item L'utente sta visualizzando le informazioni di classificazione di un documento
    \end{enumerate}
    \item \textbf{Postcondizioni}: L'utente visualizza l'URI del piano di classificazione del documento selezionato
    \item \textbf{Flusso Principale}:
        \begin{enumerate}
            \item Il sistema mostra a video l'URI del piano di classificazione del documento selezionato
        \end{enumerate}
\end{itemize}

\usecase{visualizzaTempoConservazioneEffettivoDocumento}{Visualizza tempo di conservazione effettivo documento}
\begin{figure}[H]
    \centering
    \includegraphics[width=0.8\textwidth]{../assets/uml/UC53-ExtUC54.png}
    \caption{UC53 - Visualizza tempo di conservazione effettivo documento}
    \label{fig:uc_visualizzaTempoConservazioneEffettivoDocumento}
\end{figure}
\begin{itemize}
    \item \textbf{Attore Primario}: Utente
    \item \textbf{Precondizioni}:
    \begin{enumerate}
        \item L'utente ha avviato l'applicazione
        \item L'utente ha selezionato un documento
        \item Il documento ha un tempo di conservazione diverso da quello assegnato all'aggregazione documentale informatica a cui appartiene
    \end{enumerate}
    \item \textbf{Postcondizioni}: L'utente visualizza il numero di anni effettivi per cui è stato conservato il documento selezionato
    \item \textbf{Flusso Principale}:
        \begin{enumerate}
            \item Il sistema mostra a video il numero di anni per cui il documento è stato conservato
        \end{enumerate}
    \item \textbf{Flusso Alternativo}:
    \begin{itemize}
        \item il tempo di conservazione coincide con quello assegnato all'aggregazione documentale a cui il documento appartiene
    \end{itemize}
    \item \textbf{Estensioni}: \ref{erroreTempoConservazioneEffettivoDocumento} Errore Visualizzazione Tempo di Conservazione Effettivo documento
\end{itemize}

\usecase{erroreTempoConservazioneEffettivoDocumento}{Errore visualizzazione tempo di conservazione effettivo documento}
\begin{itemize}
    \item \textbf{Attore Primario}: Utente
    \item \textbf{Precondizioni}:
    \begin{enumerate}
        \item L'utente ha avviato l'applicazione
        \item L'utente ha selezionato un documento
        \item Il documento ha un tempo di conservazione uguale a quello assegnato all'aggregazione documentale informatica a cui appartiene
    \end{enumerate}
    \item \textbf{Postcondizioni}: L'utente visualizza un messaggio di errore relativo alla visualizzazione del tempo di conservazione effettivo del documento selezionato
    \item \textbf{Flusso Principale}:
        \begin{enumerate}
            \item Il sistema mostra a video il messaggio: Il tempo di conservazione del documento coincide con quello assegnato all'aggregazione documentale a cui appartiene.
        \end{enumerate}
\end{itemize}

\usecase{visualizzaNoteDocumento}{Visualizza note documento}
\begin{figure}[H]
    \centering
    \includegraphics[width=0.8\textwidth]{../assets/uml/UC55-ExtUC56.png}
    \caption{UC55 - Visualizza note documento}
    \label{fig:uc_visualizzaNoteDocumento}
\end{figure}
\begin{itemize}
    \item \textbf{Attore Primario}: Utente
    \item \textbf{Precondizioni}:
    \begin{enumerate}
        \item L'utente ha avviato l'applicazione
        \item L'utente ha selezionato un documento
    \end{enumerate}
    \item \textbf{Postcondizioni}: L'utente visualizza le note relative al documento selezionato
    \item \textbf{Flusso Principale}:
        \begin{enumerate}
            \item Il sistema mostra a video le note relative al documento selezionato
        \end{enumerate}
    \item \textbf{Flusso Alternativo}:
    \begin{itemize}
        \item Le note del documento sono assenti o vuote.
    \end{itemize}
      \item \textbf{Estensioni}: \ref{erroreNoteDocumento} Errore visualizzazione note documento
\end{itemize}

\usecase{erroreNoteDocumento}{Errore visualizzazione note documento}
\begin{itemize}
    \item \textbf{Attore Primario}: Utente
    \item \textbf{Precondizioni}:
    \begin{enumerate}
        \item L'utente ha avviato l'applicazione
        \item L'utente ha selezionato un documento
    \end{enumerate}
    \item \textbf{Postcondizioni}: L'utente visualizza un messaggio di errore relativo alla visualizzazione delle note del documento selezionato
    \item \textbf{Flusso Principale}:
        \begin{enumerate}
            \item Le note del documento sono assenti o vuote.
        \end{enumerate}
\end{itemize}

\usecase{visualizzaDatiRegistrazioneDocumento}{Visualizza Dati di Registrazione documento}
\begin{figure}[H]
    \centering
    \includegraphics[width=0.6\textwidth]{../assets/uml/UC57.png}
    \caption{UC57 - Visualizza Dati di Registrazione documento}
    \label{fig:uc_visualizzaDatiRegistrazioneDocumento}
\end{figure}
\begin{itemize}
    \item \textbf{Attore Primario}: Utente
    \item \textbf{Precondizioni}:
    \begin{enumerate}
        \item L'utente ha avviato l'applicazione
        \item L'utente ha selezionato un documento
    \end{enumerate}
    \item \textbf{Postcondizioni}: L'utente visualizza i dati di registrazione del documento selezionato
    \item \textbf{Flusso Principale}:
        \begin{enumerate}
            \item Il sistema mostra a video i dati di registrazione del documento, quali:
            \begin{itemize}
                \item Tipologia di flusso (\ref{visualizzaTipologiaFlussoDocumento})
                \item Tipo registro (\ref{visualizzaTipoRegistroDocumento})
                \item Data registrazione (\ref{visualizzaDataRegistrazioneDocumento})
                \item Numero documento (\ref{visualizzaNumeroDocumento})
                \item Codice registro (\ref{visualizzaCodiceIdentificativoRegistroAppartenenzaDocumento})
            \end{itemize}
        \end{enumerate}
    \item \textbf{Inclusioni}:
    \begin{itemize}
        \item \ref{visualizzaTipologiaFlussoDocumento} Visualizza tipologia di flusso documento
        \item \ref{visualizzaTipoRegistroDocumento} Visualizza tipo di registro documento
        \item \ref{visualizzaDataRegistrazioneDocumento} Visualizza data di registrazione documento
        \item \ref{visualizzaNumeroDocumento} Visualizza numero documento
        \item \ref{visualizzaCodiceIdentificativoRegistroAppartenenzaDocumento} Visualizza codice identificativo del registro di appartenenza documento
    \end{itemize}
\end{itemize}
\begin{figure}[H]
    \centering
    \includegraphics[width=1\textwidth]{../assets/uml/IncUC57.png}
    \caption{Inclusioni UC57 - Visualizza Dati di Registrazione documento}
    \label{fig:inclusioniVisualizzaDatiRegistrazioneDocumento}
\end{figure}
\subusecase{visualizzaTipologiaFlussoDocumento}{Visualizza Tipologia di Flusso documento}
\begin{itemize}
    \item \textbf{Descrizione}: L'utente vuole visualizzare la tipologia di flusso del documento, che indica se si tratta di un documento in uscita, in entrata o interno.
    \item \textbf{Attore Primario}: Utente
    \item \textbf{Precondizioni}:
    \begin{enumerate}
        \item L'utente ha avviato l'applicazione
        \item L'utente ha selezionato un documento
    \end{enumerate}
    \item \textbf{Postcondizioni}: L'utente visualizza la tipologia di flusso del documento selezionato
    \item \textbf{Flusso Principale}:
        \begin{enumerate}
            \item Il sistema mostra a video la tipologia del flusso del documento selezionato
        \end{enumerate}
\end{itemize}

\subusecase{visualizzaTipoRegistroDocumento}{Visualizza Tipo di Registro documento}
\begin{itemize}
    \item \textbf{Descrizione}: L'utente vuole visualizzare il tipo di registro del documento, che indica il sistema di registrazione adottato: protocollo ordinario/protocollo emergenza, o Repertorio/Registro.
    \item \textbf{Attore Primario}: Utente
    \item \textbf{Precondizioni}:
    \begin{enumerate}
        \item L'utente ha avviato l'applicazione
        \item L'utente ha selezionato un documento
    \end{enumerate}
    \item \textbf{Postcondizioni}: L'utente visualizza il Tipo di Registro del documento selezionato
    \item \textbf{Flusso Principale}:
        \begin{enumerate}
            \item Il sistema mostra a video il Tipo di Registro 
            del documento selezionato
        \end{enumerate}
\end{itemize}

\subusecase{visualizzaDataRegistrazioneDocumento}{Visualizza Data di Registrazione documento}
\begin{itemize}
    \item \textbf{Descrizione}: L'utente vuole visualizzare la data di registrazione del documento, che indica nel caso di documento non protocollato:
    \begin{itemize}
        \item Data di registrazione del Documento/Ora di registrazione del Documento
    \end{itemize}
    nel caso di documento protocollato:
    \begin{itemize}
        \item Data di registrazione di protocollo/Ora di protocollazione del Documento
    \end{itemize}
    \item \textbf{Attore Primario}: Utente
    \item \textbf{Precondizioni}:
    \begin{enumerate}
        \item L'utente ha avviato l'applicazione
        \item L'utente ha selezionato un documento
    \end{enumerate}
    \item \textbf{Postcondizioni}: L'utente visualizza la Data di Registrazione del documento selezionato
    \item \textbf{Flusso Principale}:
        \begin{enumerate}
            \item Il sistema mostra a video la Data di Registrazione
            del documento selezionato
        \end{enumerate}
\end{itemize}

\subusecase{visualizzaNumeroDocumento}{Visualizza Numero documento}
\begin{itemize}
    \item \textbf{Descrizione}: L'utente vuole visualizzare il numero del documento, che indica nel caso di documento non protocollato:
    \begin{itemize}
        \item Numero di registrazione del documento
    \end{itemize}
    mentre nel caso di documento protocollato:
    \begin{itemize}
        \item Numero di protocollo
    \end{itemize}
    \item \textbf{Attore Primario}: Utente
    \item \textbf{Precondizioni}:
    \begin{enumerate}
        \item L'utente ha avviato l'applicazione
        \item L'utente ha selezionato un documento
    \end{enumerate}
    \item \textbf{Postcondizioni}: L'utente visualizza il Numero del documento selezionato
    \item \textbf{Flusso Principale}:
        \begin{enumerate}
            \item Il sistema mostra a video il Numero del documento selezionato
        \end{enumerate}
\end{itemize}

\subusecase{visualizzaCodiceIdentificativoRegistroAppartenenzaDocumento}{Visualizza Codice identificativo del Registro di appartenenza documento}
\begin{itemize}
    \item \textbf{Descrizione}: L'utente vuole visualizzare il codice identificativo del registro in cui il documento viene registrato.
    \item \textbf{Attore Primario}: Utente
    \item \textbf{Precondizioni}:
    \begin{enumerate}
        \item L'utente ha avviato l'applicazione
        \item L'utente ha selezionato un documento
    \end{enumerate}
    \item \textbf{Postcondizioni}: L'utente visualizza il Codice identificativo del Registro in cui è stato registrato il documento selezionato
    \item \textbf{Flusso Principale}:
        \begin{enumerate}
            \item Il sistema mostra a video il Codice identificativo del Registro in cui è stato registrato il documento selezionato
        \end{enumerate}
\end{itemize}

\usecase{visualizzaTipologiaDocumentaleDocumento}{Visualizza Tipologia documentale documento}
\begin{figure}[H]
    \centering
    \includegraphics[width=0.6\textwidth]{../assets/uml/UC58.png}
    \caption{UC58 - Visualizza Tipologia documentale documento}
    \label{fig:uc_visualizzaTipologiaDocumentaleDocumento}
\end{figure}
\begin{itemize}
    \item \textbf{Attore Primario}: Utente
    \item \textbf{Precondizioni}:
    \begin{enumerate}
        \item L'utente ha avviato l'applicazione
        \item L'utente ha selezionato un documento
    \end{enumerate}
    \item \textbf{Postcondizioni}: L'utente visualizza la Tipologia documentale del documento selezionato
    \item \textbf{Flusso Principale}:
        \begin{enumerate}
            \item Il sistema mostra a video la tipologia documentale del documento selezionato
        \end{enumerate}
\end{itemize}

\usecase{visualizzaModalitaFormazioneDocumento}{Visualizza Modalità di formazione documento}
\begin{figure}[H]
    \centering
    \includegraphics[width=0.6\textwidth]{../assets/uml/UC59.png}
    \caption{UC59 - Visualizza Modalità di formazione documento}
    \label{fig:uc_visualizzaModalitaFormazioneDocumento}
\end{figure}
\begin{itemize}
    \item \textbf{Attore Primario}: Utente
    \item \textbf{Precondizioni}:
    \begin{enumerate}
        \item L'utente ha avviato l'applicazione
        \item L'utente ha selezionato un documento
    \end{enumerate}
    \item \textbf{Postcondizioni}: L'utente visualizza la Modalità di formazione del documento selezionato
    \item \textbf{Flusso Principale}:
        \begin{enumerate}
            \item Il sistema mostra a video la Modalità di formazione del documento selezionato
        \end{enumerate}
\end{itemize}

\usecase{visualizzaStatoRiservatezzaDocumento}{Visualizza Stato di riservatezza documento}
\begin{figure}[H]
    \centering
    \includegraphics[width=0.6\textwidth]{../assets/uml/UC60.png}
    \caption{UC60 - Visualizza Stato di riservatezza documento}
    \label{fig:uc_visualizzaStatoRiservatezzaDocumento}
\end{figure}
\begin{itemize}
    \item \textbf{Descrizione}: L'utente vuole visualizzare lo stato di riservatezza del documento, che indica se quest'ultimo è accessibile solo al personale autorizzato
    \item \textbf{Attore Primario}: Utente
    \item \textbf{Precondizioni}:
    \begin{enumerate}
        \item L'utente ha avviato l'applicazione
        \item L'utente ha selezionato un documento
    \end{enumerate}
    \item \textbf{Postcondizioni}: L'utente visualizza lo Stato di riservatezza del documento selezionato
    \item \textbf{Flusso Principale}:
        \begin{enumerate}
            \item Il sistema mostra a video lo Stato di riservatezza del documento selezionato
        \end{enumerate}
\end{itemize}


\subsubsection{UC-23 - Visualizza informazioni del identificativo di formato}
\begin{itemize}
    \item \textbf{Attore Primario}: Utente

    \item \textbf{Precondizioni}:
    \begin{enumerate}
        \item L'utente ha avviato l'applicazione
        \item L'utente ha selezionato un documento di tipo: "informatico" o "amministrativo informatico"
    \end{enumerate}

    \item \textbf{Postcondizioni}:
    \begin{itemize}
        \item Il sistema visualizza i dati sul formato e il software che ha generato il documento
    \end{itemize}

    \item \textbf{Flusso Principale}:
    \begin{enumerate}
        \item Il sistema visualizza il tipo di formato del documento tra quelli indicati nelle Linee guida AGID
        \item Il sistema visualizza le informazioni identificative del prodotto software, quali:
        \begin{itemize}
            \item Nome del prodotto
            \item Versione del prodottotem
            \item Produttore
        \end{itemize}
    \end{enumerate}
\end{itemize}

\subsubsection{UC-24 - Visualizza informazioni di verifica}
\begin{itemize}
    \item \textbf{Attore Primario}: Utente

    \item \textbf{Precondizioni}:
    \begin{enumerate}
        \item L'utente ha avviato l'applicazione
        \item L'utente ha selezionato un documento di tipo: "informatico" o "amministrativo informatico"
    \end{enumerate}

    \item \textbf{Postcondizioni}:
    \begin{itemize}
        \item Il sistema visualizza le caratteristiche della verifica del documento
    \end{itemize}

    \item \textbf{Flusso Principale}:
    \begin{enumerate}
        \item Il sistema visualizza se è Firmato Digitalmente o meno
        \item Il sistema visualizza se è Sigillato Elettronicamente o meno
        \item Il sistema visualizza se è dotato di Marcatura Temporale o meno
        \item Il sistema visualizza se vi è conformità alle copie immagine su supporto informatico
    \end{enumerate}
\end{itemize}

\subsubsection{UC-25 - Visualizza la versione del documento}
\begin{itemize}
    \item \textbf{Attore Primario}: Utente

    \item \textbf{Precondizioni}:
    \begin{enumerate}
        \item L'utente ha avviato l'applicazione
        \item L'utente ha selezionato un documento di tipo: "informatico" o "amministrativo informatico"
    \end{enumerate}

    \item \textbf{Postcondizioni}:
    \begin{itemize}
        \item Il sistema visualizza la versione del documento
    \end{itemize}

    \item \textbf{Flusso Principale}:
    \begin{enumerate}
        \item Il sistema visualizza la versione del documento
    \end{enumerate}
\end{itemize}

\subsubsection{UC-26 - Visualizza nome del documento}
\begin{itemize}
    \item \textbf{Attore Primario}: Utente

    \item \textbf{Precondizioni}:
    \begin{enumerate}
        \item L'utente ha avviato l'applicazione
        \item L'utente ha selezionato un documento di tipo: "informatico" o "amministrativo informatico"
    \end{enumerate}

    \item \textbf{Postcondizioni}:
    \begin{itemize}
        \item Il sistema visualizza il nome del documento
    \end{itemize}

    \item \textbf{Flusso Principale}:
    \begin{enumerate}
        \item Il sistema visualizza il nome completo del documento
    \end{enumerate}
\end{itemize}

\subsubsection{UC-27 - Visualizza informazioni sugli allegati}
\begin{itemize}
    \item \textbf{Attore Primario}: Utente

    \item \textbf{Precondizioni}:
    \begin{enumerate}
        \item L'utente ha avviato l'applicazione
        \item L'utente ha selezionato un documento di tipo: "informatico" o "amministrativo informatico"
    \end{enumerate}

    \item \textbf{Postcondizioni}:
    \begin{itemize}
        \item Il sistema visualizza le informazioni sugli allegati del documento
    \end{itemize}

    \item \textbf{Flusso Principale}:
    \begin{enumerate}
        \item Il sistema visualizza il numero di allegati
        \item Il sistema visualizza le informazioni identificative degli allegati, se presenti, quali:
        \begin{itemize}
            \item Identificativo dell'allegato
            \item Descrizione dell'allegato
        \end{itemize}    \end{enumerate}
\end{itemize}

\subsubsection{UC-28 - Visualizza informazioni sulle modifiche di un documento}
\begin{itemize}
    \item \textbf{Attore Primario}: Utente

    \item \textbf{Precondizioni}:
    \begin{enumerate}
        \item L'utente ha avviato l'applicazione
        \item L'utente ha selezionato un documento di tipo: "informatico" o "amministrativo informatico"
    \end{enumerate}

    \item \textbf{Postcondizioni}:
    \begin{itemize}
        \item Il sistema visualizza le informazioni sulle modifiche del documento
    \end{itemize}

    \item \textbf{Flusso Principale}:
    \begin{enumerate}
        \item Il sistema visualizza per ogni modifica del documento:
        \begin{itemize}
                \item Tipo di modifica tra Annullamento, Rettifica, Integrgazione, Annotazione
                \item Soggetto autore della modifica come definito UC-16
                \item Data Ora della modifica
                \item Identificativo del documento alla versione precedente alla modifica
        \end{itemize} 
           \end{enumerate}
\end{itemize}

\subsubsection{UC-29 - Visualizza tipo di aggregazione}
\begin{itemize}
    \item \textbf{Attore Primario}: Utente

    \item \textbf{Precondizioni}:
    \begin{enumerate}
        \item L'utente ha avviato l'applicazione
        \item L'utente ha selezionato un documento, di tipo: "aggregazione documentale"
    \end{enumerate}

    \item \textbf{Postcondizioni}:
    \begin{itemize}
        \item Il sistema visualizza il tipo di aggregazione dell'aggregazione selezionata
    \end{itemize}

    \item \textbf{Flusso Principale}:
    \begin{enumerate}
        \item Il sistema visualizza il tipo di aggregazione tra:
        \begin{itemize}
                \item Fascicolo
                \item Serie Documentale
                \item Serie di Fascicoli
        \end{itemize} 
           \end{enumerate}
\end{itemize}

\subsubsection{UC-30 - Visualizza identificativo di aggregazione}
\begin{itemize}
    \item \textbf{Attore Primario}: Utente

    \item \textbf{Precondizioni}:
    \begin{enumerate}
        \item L'utente ha avviato l'applicazione
        \item L'utente ha selezionato un documento, di tipo: "aggregazione documentale"
    \end{enumerate}

    \item \textbf{Postcondizioni}:
    \begin{itemize}
        \item Il sistema visualizza l'identificativo di aggregazione dell'aggregazione selezionata
    \end{itemize}

    \item \textbf{Flusso Principale}:
    \begin{enumerate}
        \item Il sistema visualizza l'identificativo di aggregazione dell'aggregazione selezionata
           \end{enumerate}
\end{itemize}

\subsubsection{UC-31 - Visualizza tipologia di fascicolo}
\begin{itemize}
    \item \textbf{Attore Primario}: Utente

    \item \textbf{Precondizioni}:
    \begin{enumerate}
        \item L'utente ha avviato l'applicazione
        \item L'utente ha selezionato un documento, di tipo: "aggregazione documentale di tipo fascicolo"
    \end{enumerate}

    \item \textbf{Postcondizioni}:
    \begin{itemize}
        \item Il sistema visualizza il tipo di di fascicolo dell'aggregazione selezionata
    \end{itemize}

    \item \textbf{Flusso Principale}:
    \begin{enumerate}
        \item Il sistema visualizza il tipo di di fascicolo tra:
        \begin{itemize}
                \item affare
                \item attività
                \item persona fisica
                \item presona giuridica
                \item procedimento amministrativo
        \end{itemize} 
           \end{enumerate}
\end{itemize}

\subsubsection{UC-32 - Visualizza assegnazione di aggregazione}
\begin{itemize}
    \item \textbf{Attore Primario}: Utente

    \item \textbf{Precondizioni}:
    \begin{enumerate}
        \item L'utente ha avviato l'applicazione
        \item L'utente ha selezionato un documento, di tipo: "aggregazione documentale informatica"
    \end{enumerate}

    \item \textbf{Postcondizioni}:
    \begin{itemize}
        \item Il sistema visualizza le informazioni di assegnazione dell'aggregazione selezionata
    \end{itemize}

    \item \textbf{Flusso Principale}:
    \begin{enumerate}
        \item Il sistema visualizza le informazioni di assegnazione, quali:
        \begin{itemize}
            \item Tipo di assegnazione tra: Per competenza e Per conoscenza
            \item Soggetto assegnatario come definito UC-16
            \item Data Ora di inizio assegnazione
            \item Data Ora di fine assegnazione
        \end{itemize}
           \end{enumerate}
\end{itemize}

\subsubsection{UC-33 - Visualizza data di apertura di aggregazione}
\begin{itemize}
    \item \textbf{Attore Primario}: Utente

    \item \textbf{Precondizioni}:
    \begin{enumerate}
        \item L'utente ha avviato l'applicazione
        \item L'utente ha selezionato un documento, di tipo: "aggregazione documentale informatica"
    \end{enumerate}

    \item \textbf{Postcondizioni}:
    \begin{itemize}
        \item Il sistema visualizza la data di apertura dell'aggregazione selezionata
    \end{itemize}

    \item \textbf{Flusso Principale}:
    \begin{enumerate}
        \item Il sistema visualizza la data di apertura dell'aggregazione selezionata
           \end{enumerate}
\end{itemize}

\subsubsection{UC-34 - Visualizza data di chiusura di aggregazione}
\begin{itemize}
    \item \textbf{Attore Primario}: Utente

    \item \textbf{Precondizioni}:
    \begin{enumerate}
        \item L'utente ha avviato l'applicazione
        \item L'utente ha selezionato un documento, di tipo: "aggregazione documentale informatica"
    \end{enumerate}

    \item \textbf{Postcondizioni}:
    \begin{itemize}
        \item Il sistema visualizza la data di chiusura dell'aggregazione selezionata
    \end{itemize}

    \item \textbf{Flusso Principale}:
    \begin{enumerate}
        \item Il sistema visualizza la data di chiusura dell'aggregazione selezionata
           \end{enumerate}
\end{itemize}

\subsubsection{UC-35 - Visualizza progressivo di aggregazione}
\begin{itemize}
    \item \textbf{Attore Primario}: Utente

    \item \textbf{Precondizioni}:
    \begin{enumerate}
        \item L'utente ha avviato l'applicazione
        \item L'utente ha selezionato un documento, di tipo: "aggregazione documentale informatica"
    \end{enumerate}

    \item \textbf{Postcondizioni}:
    \begin{itemize}
        \item Il sistema visualizza il progressivo dell'aggregazione selezionata
    \end{itemize}

    \item \textbf{Flusso Principale}:
    \begin{enumerate}
        \item Il sistema visualizza il progressivo dell'aggregazione selezionata
           \end{enumerate}
\end{itemize}

\subsubsection{UC-36 - Visualizza procedimento amministrativo di aggregazione}
\begin{itemize}
    \item \textbf{Attore Primario}: Utente

    \item \textbf{Precondizioni}:
    \begin{enumerate}
        \item L'utente ha avviato l'applicazione
        \item L'utente ha selezionato un documento, di tipo: "aggregazione documentale informatica"
    \end{enumerate}

    \item \textbf{Postcondizioni}:
    \begin{itemize}
        \item Il sistema visualizza le informazioni del procedimento amministrativo dell'aggregazione selezionata
    \end{itemize}

    \item \textbf{Flusso Principale}:
    \begin{enumerate}
        \item Il sistema visualizza le informazioni del procedimento amministrativo, quali:
        \begin{itemize}
            \item Materia/Argomento/Struttura per la quale sono catalogati i procedimenti
            \item Denominazione del procedimento
            \item Catalogo dei procedimenti come URI di pubblicazione
            \item Fasi del procedimento amministrativo
        \end{itemize}
           \end{enumerate}
\end{itemize}

\subsubsection{UC-36.1 - Visualizza fasi di un procedimento amministrativo}
\begin{itemize}
    \item \textbf{Attore Primario}: Utente

    \item \textbf{Precondizioni}:
    \begin{enumerate}
        \item L'utente ha avviato l'applicazione
        \item L'utente ha selezionato un documento, di tipo: "aggregazione documentale informatica"
    \end{enumerate}

    \item \textbf{Postcondizioni}:
    \begin{itemize}
        \item Il sistema visualizza le informazioni delle fasi di un procedimento amministrativo dell'aggregazione selezionata
    \end{itemize}

    \item \textbf{Flusso Principale}:
    \begin{enumerate}
        \item Il sistema visualizza le informazioni delle fasi di un procedimento, quali:
        \begin{itemize}
            \item Tipo di fase tra:
            \begin{itemize}
                \item Preparatoria
                \item Istruttoria
                \item Consultiva
                \item  Decisoria o deliberativa
                \item Integrazione dell'efficacia
            \end{itemize}
            \item Data Ora della fase
            \item Data Ora della fase
        \end{itemize}
           \end{enumerate}
\end{itemize}


\subsubsection{UC-37 - Visualizza progressivo di aggregazione}
\begin{itemize}
    \item \textbf{Attore Primario}: Utente

    \item \textbf{Precondizioni}:
    \begin{enumerate}
        \item L'utente ha avviato l'applicazione
        \item L'utente ha selezionato un documento, di tipo: "aggregazione documentale informatica"
    \end{enumerate}

    \item \textbf{Postcondizioni}:
    \begin{itemize}
        \item Il sistema visualizza il progressivo dell'aggregazione selezionata
    \end{itemize}

    \item \textbf{Flusso Principale}:
    \begin{enumerate}
        \item Il sistema visualizza il progressivo dell'aggregazione selezionata
           \end{enumerate}
\end{itemize}

\subsubsection{UC-38 - Visualizza indice dei documenti di aggregazione}
\begin{itemize}
    \item \textbf{Attore Primario}: Utente

    \item \textbf{Precondizioni}:
    \begin{enumerate}
        \item L'utente ha avviato l'applicazione
        \item L'utente ha selezionato un documento, di tipo: "aggregazione documentale informatica"
    \end{enumerate}

    \item \textbf{Postcondizioni}:
    \begin{itemize}
        \item Il sistema visualizza l'indice dei documenti dell'aggregazione selezionata
    \end{itemize}

    \item \textbf{Flusso Principale}:
    \begin{enumerate}
        \item Il sistema visualizza il tipo dei documenti contenuti dall'aggregazione selezionata
        \item Il sistema visualizza l'identificativo dei documenti contenuti dall'aggregazione selezionata
           \end{enumerate}
\end{itemize}

\subsubsection{UC-39 - Visualizza posizione fisica di aggregazione}
\begin{itemize}
    \item \textbf{Attore Primario}: Utente

    \item \textbf{Precondizioni}:
    \begin{enumerate}
        \item L'utente ha avviato l'applicazione
        \item L'utente ha selezionato un documento, di tipo: "aggregazione documentale informatica"
    \end{enumerate}

    \item \textbf{Postcondizioni}:
    \begin{itemize}
        \item Il sistema visualizza la posizione fisica dell'aggregazione selezionata
    \end{itemize}

    \item \textbf{Flusso Principale}:
    \begin{enumerate}
        \item Il sistema visualizza la posizione fisica dell'aggregazione selezionata
           \end{enumerate}
\end{itemize}

\subsubsection{UC-40 - Visualizza identificativo dell'aggregazione primaria di aggregazione}
\begin{itemize}
    \item \textbf{Attore Primario}: Utente

    \item \textbf{Precondizioni}:
    \begin{enumerate}
        \item L'utente ha avviato l'applicazione
        \item L'utente ha selezionato un documento, di tipo: "aggregazione documentale informatica"
    \end{enumerate}

    \item \textbf{Postcondizioni}:
    \begin{itemize}
        \item Il sistema visualizza l'identificativo dell'aggregazione primaria dell'aggregazione selezionata
    \end{itemize}

    \item \textbf{Flusso Principale}:
    \begin{enumerate}
        \item Il sistema visualizza l'identificativo dell'aggregazione primaria dell'aggregazione selezionata
           \end{enumerate}
\end{itemize}

\subsubsection{UC-41 - Visualizza tempo conservazione di aggregazione}
\begin{itemize}
    \item \textbf{Attore Primario}: Utente

    \item \textbf{Precondizioni}:
    \begin{enumerate}
        \item L'utente ha avviato l'applicazione
        \item L'utente ha selezionato un documento, di tipo: "aggregazione documentale informatica"
    \end{enumerate}

    \item \textbf{Postcondizioni}:
    \begin{itemize}
        \item Il sistema visualizza il tempo di conservazione dell'aggregazione selezionata
    \end{itemize}

    \item \textbf{Flusso Principale}:
    \begin{enumerate}
        \item Il sistema visualizza il tempo di conservazione dell'aggregazione selezionata
    \end{enumerate}
\end{itemize}

\subsubsection{UC-42 Visualizza metadati custom documento}
\begin{itemize}
    \item \textbf{Attore Primario}: Utente
    \item \textbf{Precondizioni}:
    \begin{enumerate}
        \item L'utente ha avviato l'applicazione
        \item L'utente ha selezionato un documento
    \end{enumerate}
    \item \textbf{Postcondizioni}:
        \begin{itemize}
            \item L'utente visualizza per ogni metadato custom relativo al documento selezionato, il nome e il valore del metadato custom
        \end{itemize}
    \item \textbf{Flusso principale}:
        \begin{itemize}
            \item Il sistema visualizza per ogni metadato custom, il suo valore come stringa (UC 42.1)
        \end{itemize}
    \item \textbf{Flusso alternativo}:
        \begin{itemize}
            \item Il documento selezionato non dispone di metadati custom
        \end{itemize}
\end{itemize}

\subsubsection{UC-42.1 Visualizza metadato custom documento}
\begin{itemize}
    \item \textbf{Attore Primario}: Utente
    \item \textbf{Precondizioni}:
    \begin{enumerate}
        \item L'utente ha avviato l'applicazione
        \item L'utente sta visualizzando i metadati custom di un documento
    \end{enumerate}
    \item \textbf{Postcondizioni}:
        \begin{itemize}
            \item L'utente visualizza il nome del metadato e il suo valore come stringa
        \end{itemize}
    \item \textbf{Flusso principale}:
        \begin{itemize}
            \item Il sistema mostra a video il nome ed il valore del metadato custom
        \end{itemize}
\end{itemize}
\section{Requisiti}

\subsection{Introduzione}
I requisiti vengono classificati secondo le seguenti categorie:
\begin{itemize}
    \item \textbf{Requisiti Funzionali (F)}: descrivono le funzionalità del sistema
    \item \textbf{Requisiti di Qualità (Q)}: descrivono le caratteristiche qualitative del sistema
    \item \textbf{Requisiti di Vincolo (V)}: descrivono i vincoli tecnologici e normativi
\end{itemize}

Ogni requisito è identificato da un codice univoco nella forma:
\begin{center}
R-[ID]-[Tipo]-[Priorità]
\end{center}

Dove:
\begin{itemize}
    \item \textbf{ID}: numero progressivo del requisito
    \item \textbf{Tipo}: F (Funzionale), Q (Qualità), V (Vincolo)
    \item \textbf{Priorità}: Ob (Obbligatorio), De (Desiderabile), Op (Opzionale)
\end{itemize}

\newpage

\subsection{Requisiti Funzionali}

\setlength{\LTleft}{0cm}
\renewcommand{\arraystretch}{1.3}

\rowcolors{2}{gray!15}{white}
\begin{longtable}{|p{2.5cm}|p{9.5cm}|p{2.5cm}|}
\hline
\rowcolor{zpusgreen!30}
\textbf{Codice} & \textbf{Descrizione} & \textbf{Fonti} \\
\hline
\endfirsthead
\rowcolor{zpusgreen!30}
\textbf{Codice} & \textbf{Descrizione} & \textbf{Fonti} \\
\hline
\endhead
R-1-F-Ob & L'utente deve poter visualizzare l'elenco di classi documentali nel DIP & \ref{classiDocumentali} \\
\hline
R-2-F-Ob &  In caso non vi siano classi documentali nel DIP, l'utente deve poter visualizzare un messaggio di errore & \ref{elencoVuoto} \\
\hline
R-3-F-Ob &  Quando viene selezionata una classe documentale, l'utente deve poter visualizzare ciascuna classe documentale all'interno dell'elenco & \ref{classeDocumentale} \\
\hline
R-4-F-Ob &  L'utente deve poter visualizzare il nome della classe documentale & \ref{nomeClasseDocumentale} \\
\hline
R-5-F-Ob &  L'utente deve poter visualizzare lo stato di verifica della classe documentale & \ref{statoVerificaElemento} \\
\hline
R-6-F-Ob &  L'utente deve poter visualizzare la marcatura temporale della classe documentale & \ref{marcaturaTemporaleElemento} \\
\hline
R-7-F-Ob &  L'utente deve poter visualizzare l'elenco dei processi associati alla classe documentale & \ref{processiClasseDocumentale} \\
\hline
R-8-F-Ob &  In caso non vi siano processi associati alla classe documentale, l'utente deve poter visualizzare un messaggio di errore & \ref{elencoVuoto} \\
\hline
R-9-F-Ob &  Quando viene selezionato un processo, l'utente deve poter visualizzare ciascun processo all'interno dell'elenco & \ref{processoClasseDocumentale} \\
\hline
R-10-F-Ob &  L'utente deve poter visualizzare il nome del processo & \ref{idProcessoClasse} \\
\hline
R-11-F-Ob &  L'utente deve poter visualizzare l'elenco di documenti associati a un processo & \ref{documentiProcesso} \\
\hline
R-12-F-Ob &  In caso non vi siano documenti associati al processo, l'utente deve poter visualizzare un messaggio di errore & \ref{elencoVuoto} \\
\hline
R-13-F-Ob &  Quando viene selezionato un documento, l'utente deve poter visualizzare ciascun documento all'interno dell'elenco & \ref{documentoProcesso} \\
\hline
R-14-F-Ob &  L'utente deve poter visualizzare il nome del documento & \ref{nomeDocumentoProcesso} \\
\hline
R-15-F-Ob &  L'utente deve poter visualizzare lo stato di verifica del documento & \ref{statoVerificaElemento} \\
\hline
R-16-F-Ob &  L'utente deve poter visualizzare la marcatura temporale del documento & \ref{marcaturaTemporaleElemento} \\
\hline
R-17-F-Ob &  L'utente deve poter selezionare una classe documentale  & \ref{selezionaClasseDocumentale} \\
\hline
R-18-F-Ob &  L'utente deve poter selezionare un processo  & \ref{selezionaProcesso} \\
\hline
R-19-F-Ob &  L'utente deve poter visualizzare l'anteprima di un documento selezionato  & \ref{anteprimaDocumento} \\
\hline
R-20-F-Ob &  Se il documento selezionato non è visualizzabile in anteprima, l'utente deve poter visualizzare un messaggio di errore  & \ref{formatoDocumentoNonSupportato} \\
\hline
R-21-F-Ob &  L'utente deve poter ricercare un documento, un processo, o una classe documentale & \ref{ricercaDIP} \\
\hline
R-22-F-Ob &  L'utente può cercare una classe documentale esclusivamente per nome & \begin{tabular}[t]{@{}l} \ref{ricercaClasse} \\ \ref{inserimentoNomeClasseDocumentale} \\ \ref{compilaValoreFiltro} \end{tabular} \\
\hline
R-23-F-Ob &  L'utente può cercare un processo esclusivamente per uuid & \begin{tabular}[t]{@{}l} \ref{ricercaProcesso} \\ \ref{inserimentoIdProcesso} \\ \ref{compilaValoreFiltro} \end{tabular} \\
\hline
R-24-F-Op &  L'utente deve poter effettuare una ricerca semantica basata sui metadati dei documenti presenti nel DIP & \ref{ricercaDIPSemantica} \\
\hline
R-25-F-Op &  Il sistema deve rendere disponibile un comando o opzione per avviare l'indicizzazione semantica dei documenti nel DIP per la ricerca & \ref{indicizzazioneSemantica} \\
\hline
R-26-F-Op &  Quando viene selezionata l'indicizzazione semantica il sistema deve indicizzare i documenti presenti nel DIP  & \ref{indicizzazioneSemantica} \\
\hline
R-27-F-Op &  Se il sistema non riesce ad indicizzare i documenti presenti nel DIP, l'utente deve poter visualizzare un messaggio di errore & \ref{erroreIndicizzazioneSemantica} \\
\hline
R-28-F-Op & L'utente deve poter visualizzare lo stato dell'indicizzazione semantica & \ref{StatoIndicizzazione} \\
\hline
R-29-F-Ob &  Il sistema deve rendere disponibile un campo di ricerca & \ref{ricercaDIP} \\
\hline
R-30-F-Ob &  Il sistema deve comunicare all'utente quando inserisce un valore non valido nel campo di ricerca & \ref{campoNonValido} \\
\hline
R-31-F-Ob &  Il sistema deve rendere disponibile l'opzione di ricerca per documenti & \ref{specificaFiltriRicercaDocumento} \\
\hline
R-32-F-Ob &  Il sistema deve rendere disponibile l'opzione di ricerca per processi & \ref{ricercaProcesso} \\
\hline
R-33-F-Ob &  Il sistema deve rendere disponibile l'opzione di ricerca per classi documentali & \ref{ricercaClasse} \\
\hline
R-34-F-Ob &  Il sistema deve rendere disponibile la ricerca avanzata con filtri & \ref{ricercaDIPConFiltri} \\
\hline
R-35-F-Ob &  Il sistema deve rendere disponibile la sezione di filtri di ricerca comuni & \ref{specificaFiltriComuni} \\
\hline
R-36-F-Ob &  Il sistema deve rendere disponibile la sezione di filtri di ricerca specifici per tipo documentale & \ref{addFiltriTipoDocumento} \\
\hline
R-37-F-Ob &  Il sistema deve rendere disponibile la sezione di filtri di ricerca specifici per metadati custom & \ref{addCustomMetadata} \\
\hline
R-38-F-Ob &  Il sistema deve permettere di applicare più filtri contemporaneamente & \ref{ricercaDIPConFiltri} \\
\hline
R-39-F-Ob &  Il sistema deve permettere di rimuovere i filtri applicati & \ref{ricercaDIPConFiltri} \\
\hline
R-40-F-Ob &  Il sistema deve permettere di rimuovere tutti i filtri applicati & \ref{ricercaDIPConFiltri} \\
\hline
R-41-F-Ob &  Il sistema deve permettere di selezionare filtri per tipo documentale per un singolo tipo di documento & \ref{addFiltriTipoDocumento} \\
\hline
R-42-F-Ob &  Il sistema deve racchiudere i filtri di ricerca comuni in una sezione espandibile dedicata & \ref{specificaFiltriComuni} \\
\hline
R-43-F-Ob &  Il sistema deve racchiudere i filtri di ricerca specifici per tipo documentale in una sezione espandibile dedicata & \ref{addFiltriTipoDocumento} \\
\hline
R-44-F-Ob &  Il sistema deve racchiudere i filtri di ricerca specifici per metadati custom in una sezione espandibile dedicata & \ref{addCustomMetadata} \\
\hline
R-45-F-Ob &  Il sistema deve permettere di visualizzare i filtri comuni applicabili & \ref{visualizzaListaCampiComuni} \\
\hline
R-46-F-Ob &  Il sistema deve rendere disponibile una sezione per il filtro comune "Chiave Descrittiva" & \ref{addChiaveDescrittiva} \\
\hline
R-47-F-Ob &  L'utente deve poter inserire il valore per i campi "Chiave Descrittiva", "Oggetto" e "Parole chiave" & \begin{tabular}[t]{@{}l} \ref{addChiaveDescrittiva} \\ \ref{compilaValoreFiltro} \end{tabular} \\
\hline
R-48-F-Ob &  Il sistema deve rendere disponibile una sezione per il filtro comune "Classificazione" & \ref{addClassificazione} \\
\hline
R-49-F-Ob &  L'utente deve poter inserire il valore per i campi "Indice di classificazione", "Descrizione" e "Piano di fascicolo" & \begin{tabular}[t]{@{}l} \ref{addClassificazione} \\ \ref{compilaValoreFiltro} \end{tabular} \\
\hline
R-50-F-Ob &  Il sistema deve rendere disponibile una sezione per il filtro comune "Tempo di conservazione" & \ref{addTempoConserva} \\
\hline
R-51-F-Ob &  L'utente deve poter inserire il valore per il filtro "Tempo di conservazione" & \begin{tabular}[t]{@{}l} \ref{addTempoConserva} \\ \ref{compilaValoreFiltro} \end{tabular} \\
\hline
R-52-F-Ob &  L'utente deve poter selezionare l'opzione "Perenne" per il filtro "Tempo di conservazione" & \ref{addTempoConserva} \\
\hline
R-53-F-Ob &  Il sistema deve rendere disponibile una sezione per il filtro comune "Note" & \ref{addNote} \\
\hline
R-54-F-Ob &  L'utente deve poter inserire il valore per il filtro "Note" & \begin{tabular}[t]{@{}l} \ref{addNote} \\ \ref{compilaValoreFiltro} \end{tabular} \\
\hline
R-55-F-Ob &  Il sistema deve rendere disponibile una sezione per il filtro comune "Tipo di documento" & \ref{addTipoDocumento} \\
\hline
R-56-F-Ob &  L'utente deve poter selezionare il valore per il filtro "Tipo di documento" tra "Documento informatico", "Documento amministrativo informatico", "Aggregazione documentale" & \begin{tabular}[t]{@{}l} \ref{addTipoDocumento} \\ \ref{compilaValoreFiltro} \end{tabular} \\
\hline
R-57-F-Ob &  Il sistema deve rendere disponibile una sezione per il filtro comune "Soggetto" & \ref{addSoggetti} \\
\hline
R-58-F-Ob &  L'utente deve poter inserire il valore per il filtro "Soggetto" & \ref{addSoggetti} \\
\hline
R-59-F-Ob &  Il sistema deve rendere disponibile una sezione per il filtro del Ruolo del Soggetto & \ref{specificaRuoloSoggetto} \\
\hline
R-60-F-Ob &  All'interno della sezione del filtro del Ruolo del Soggetto, l'utente deve poter selezionare il ruolo del soggetto tra quelli disponibili & \ref{specificaRuoloSoggetto} \\
\hline
R-61-F-Ob &  Il sistema deve rendere disponibile una sezione per il filtro del Tipo di Soggetto & \ref{specificaTipoSoggetto} \\
\hline
R-62-F-Ob &  All'interno della sezione del filtro del Tipo di Soggetto, l'utente deve poter selezionare il tipo di soggetto tra quelli disponibili & \ref{specificaTipoSoggetto} \\
\hline
R-63-F-Ob &  Il sistema deve rendere disponibile una sezione per il filtro "Dettagli" del Soggetto & \ref{addDettagliSoggetto} \\
\hline
R-64-F-Ob &  Se il soggetto selezionato è di tipo "PAI", all'interno della sezione del filtro "Dettagli" del Soggetto, l'utente deve poter inserire il valore per i campi "Denominazione Amministrazione/Codice IPA", "Denominazione Amministrazione AOO/Codice IPA AOO", "Denominazione Amministrazione UOR/Codice IPA UOR" e "Indirizzi digitali di riferimento" & \begin{tabular}[t]{@{}l} \ref{addDettagliPAI} \\ \ref{compilaValoreFiltro} \end{tabular} \\
\hline
R-65-F-Ob &  Se il soggetto selezionato è di tipo "PAE", all'interno della sezione del filtro "Dettagli" del Soggetto, l'utente deve poter inserire il valore per i campi "Denominazione Amministrazione", "Denominazione Ufficio" e "Indirizzi digitali di riferimento" & \begin{tabular}[t]{@{}l} \ref{addDettagliPAE} \\ \ref{compilaValoreFiltro} \end{tabular} \\
\hline
R-66-F-Ob &  Se il soggetto selezionato è di tipo "AS", all'interno della sezione del filtro "Dettagli" del Soggetto, l'utente deve poter inserire il valore per i campi "Cognome", "Nome", "Codice Fiscale", "Denominazione Amministrazione AOO/Codice IPA AOO", "Denominazione Amministrazione UOR/Codice IPA UOR" e "Indirizzi digitali di riferimento" & \begin{tabular}[t]{@{}l} \ref{addDettagliAS} \\ \ref{compilaValoreFiltro} \end{tabular} \\
\hline
R-67-F-Ob &  Se il soggetto selezionato è di tipo "PG", all'interno della sezione del filtro "Dettagli" del Soggetto, l'utente deve poter inserire il valore per i campi "Denominazione Organizzazione", "Codice Fiscale/Partita IVA", "Denominazione Ufficio" e "Indirizzi digitali di riferimento" & \begin{tabular}[t]{@{}l} \ref{addDettagliPG} \\ \ref{compilaValoreFiltro} \end{tabular} \\
\hline
R-68-F-Ob &  Se il soggetto selezionato è di tipo "PF", all'interno della sezione del filtro "Dettagli" del Soggetto, l'utente deve poter inserire il valore per i campi "Cognome", "Nome" e "Indirizzi digitali di riferimento" & \begin{tabular}[t]{@{}l} \ref{addDettagliPF} \\ \ref{compilaValoreFiltro} \end{tabular} \\
\hline
R-69-F-Ob &  Se il soggetto selezionato è di tipo "RUP", all'interno della sezione del filtro "Dettagli" del Soggetto, l'utente deve poter inserire il valore per i campi "Cognome", "Nome", "Codice Fiscale" "Denominazione Amministrazione/Codice IPA", "Denominazione Amministrazione AOO/Codice IPA AOO", "Denominazione Amministrazione UOR/Codice IPA UOR" e "Indirizzi digitali di riferimento" & \begin{tabular}[t]{@{}l} \ref{addDettagliRUP} \\ \ref{compilaValoreFiltro} \end{tabular} \\
\hline
R-70-F-Ob &  Se il soggetto selezionato è di tipo "SW", all'interno della sezione del filtro "Dettagli" del Soggetto, l'utente deve poter inserire il valore per il campo "Denominazione Sistema" & \begin{tabular}[t]{@{}l} \ref{addDettagliSW} \\ \ref{compilaValoreFiltro} \end{tabular} \\
\hline
R-71-F-Ob &  All'interno della sezione di filtri per tipo documentale, il sistema deve permettere di selezionare i filtri specifici per il tipo "Documento Informatico e Amministrativo Informatico" & \ref{selezionaCampiDI-DAI} \\
\hline
R-72-F-Ob &  Per il Documento Informatico e Amministrativo Informatico, il sistema deve mostrare la lista di filtri specifici & \ref{visualizzaListaCampiDIDAI} \\
\hline
R-73-F-Ob &  Per il Documento Informatico e Amministrativo Informatico, il sistema deve rendere disponibile una sezione per il filtro "Dati di Registrazione" & \ref{addDatiRegistrazione} \\
\hline
R-74-F-Ob &  Per il Documento Informatico e Amministrativo Informatico, all'interno della sezione del filtro "Dati di Registrazione", l'utente deve poter inserire il valore per i campi "Tipologia di Flusso", "Tipo di Registro", "Data/Ora di Registrazione", "Numero Documento" e "Codice Registro" & \begin{tabular}[t]{@{}l} \ref{addDatiRegistrazione} \\ \ref{compilaValoreFiltro} \end{tabular} \\
\hline
R-75-F-Ob &  Per il Documento Informatico e Amministrativo Informatico, l'utente deve poter inserire il valore per il filtro "Tipologia Documentale" & \begin{tabular}[t]{@{}l} \ref{addTipologiaDocumentale} \\ \ref{compilaValoreFiltro} \end{tabular} \\
\hline
R-76-F-Ob &  Per il Documento Informatico e Amministrativo Informatico, l'utente deve poter inserire il valore per il filtro "Modalità di Formazione" tra:
\begin{itemize}
    \item Creazione tramite l'utilizzo di strumenti software che assicurino la produzione di documenti nei formati previsti nell'Allegato 2 delle Linee Guida;
    \item Acquisizione di un documento informatico per via telematica o su supporto informatico, acquisizione della copia per immagine su supporto informatico di un documento analogico, acquisizione della copia informatica di un documento analogico;
    \item Memorizzazione su supporto informatico in formato digitale delle informazioni risultanti da transazioni o processi informatici o dalla presentazione telematica di dati attraverso moduli o formulari resi disponibili all'utente;
    \item Generazione o raggruppamento anche in via automatica di un insieme di dati o registrazioni, provenienti da una o più banche dati, anche appartenenti a più soggetti interoperanti, secondo una struttura logica predeterminata e memorizzata in forma statica;
\end{itemize}  & \begin{tabular}[t]{@{}l} \ref{addModalitaFormazione} \\ \ref{compilaValoreFiltro} \end{tabular} \\
\hline
R-77-F-Ob &  Per il Documento Informatico e Amministrativo Informatico, l'utente deve poter inserire il valore per il filtro "Campo Riservato" tra "Vero" o "Falso" & \begin{tabular}[t]{@{}l} \ref{addRiservato} \\ \ref{compilaValoreFiltro} \end{tabular} \\
\hline
R-78-F-Ob &  Per il Documento Informatico e Amministrativo Informatico, il sistema deve rendere disponibile una sezione per il filtro "Identificativo del Formato" & \ref{addIdentificativoFormato} \\
\hline
R-79-F-Ob &  Per il Documento Informatico e Amministrativo Informatico, all'interno della sezione del filtro "Identificativo del Formato", l'utente deve poter inserire il valore per i campi "Formato" e "Prodotto Software" & \begin{tabular}[t]{@{}l} \ref{addIdentificativoFormato} \\ \ref{compilaValoreFiltro} \end{tabular} \\
\hline
R-80-F-Ob &  Per il Documento Informatico e Amministrativo Informatico, il sistema deve rendere disponibile una sezione per il filtro "Dati di Verifica" & \ref{addDatiVerifica} \\
\hline
R-81-F-Ob &  Per il Documento Informatico e Amministrativo Informatico, all'interno della sezione del filtro "Dati di Verifica", l'utente deve poter inserire il valore booleano per i campi "Firmato Digitalmente", "Sigillato Elettronicamente", "Marcatura Temporale" e "conformità copie immagine su supporto informatico" & \begin{tabular}[t]{@{}l} \ref{addDatiVerifica} \\ \ref{compilaValoreFiltro} \end{tabular} \\
\hline
R-82-F-Ob &  Per il Documento Informatico e Amministrativo Informatico, l'utente deve poter inserire il valore per il filtro "Nome del Documento" & \begin{tabular}[t]{@{}l} \ref{addNomeDocumento} \\ \ref{compilaValoreFiltro} \end{tabular} \\
\hline
R-83-F-Ob &  Per il Documento Informatico e Amministrativo Informatico, l'utente deve poter inserire il valore per il filtro "Versione del Documento" & \begin{tabular}[t]{@{}l} \ref{addVersioneDocumento} \\ \ref{compilaValoreFiltro} \end{tabular} \\
\hline
R-84-F-Ob &  Per il Documento Informatico e Amministrativo Informatico, l'utente deve poter inserire il valore per il filtro "Identificativo del Documento Primario" & \begin{tabular}[t]{@{}l} \ref{addIdentificativoDocumentoPrimario} \\ \ref{compilaValoreFiltro} \end{tabular} \\
\hline
R-85-F-Ob &  Per il Documento Informatico e Amministrativo Informatico, il sistema deve rendere disponibile una sezione per il filtro "Tracciatura Modifiche di Documento" & \ref{addTracciatureModificheDocumento} \\
\hline
R-86-F-Ob &  Per il Documento Informatico e Amministrativo Informatico, all'interno della sezione del filtro "Tracciatura Modifiche di Documento", l'utente deve poter inserire il valore per i campi "Tipo di Modifica", "Soggetto Autore della Modifica", "Data/Ora della Modifica" e "IdDoc versione precedente" & \begin{tabular}[t]{@{}l} \ref{addTracciatureModificheDocumento} \\ \ref{compilaValoreFiltro} \end{tabular} \\
\hline
R-87-F-Ob &  All'interno della sezione di filtri per tipo documentale, il sistema deve permettere di selezionare i filtri specifici per il tipo "Aggregazione Documentale Informatica" & \ref{specificaFiltriAggregazione} \\
\hline
R-88-F-Ob &  Per l'Aggregazione Documentale Informatica, il sistema deve mostrare la lista di filtri specifici & \ref{visualizzaListaCampiAggregazioneDocumentale} \\
\hline
R-89-F-Ob &  Per l'Aggregazione Documentale Informatica, il sistema deve rendere disponibile una sezione per il filtro "Tipo di Aggregazione" & \ref{addTipoAggregazione} \\
\hline
R-90-F-Ob &  Per l'Aggregazione Documentale Informatica, all'interno della sezione del filtro "Tipo di Aggregazione", l'utente deve poter inserire il valore per il campo "Tipo di Aggregazione" tra "Fascicolo", "Serie Documentale" o "Serie di Fascicoli" & \begin{tabular}[t]{@{}l} \ref{addTipoAggregazione} \\ \ref{compilaValoreFiltro} \end{tabular} \\
\hline
R-91-F-Ob &  Per l'Aggregazione Documentale Informatica, l'utente deve poter inserire il valore per il filtro "Identificativo dell'Aggreagazione Documentale" & \begin{tabular}[t]{@{}l} \ref{addIdAggregazione} \\ \ref{compilaValoreFiltro} \end{tabular} \\
\hline
R-92-F-Ob &  Per l'Aggregazione Documentale Informatica, l'utente deve poter inserire il valore per il filtro "Tipologia di Fascicolo" tra "Affare", "Attività", "Persona Fisica", "Persona Giuridica" e "Procedimento Amministrativo" & \begin{tabular}[t]{@{}l} \ref{addTipologiaFascicolo} \\ \ref{compilaValoreFiltro} \end{tabular} \\
\hline
R-93-F-Ob &  Per l'Aggregazione Documentale Informatica, l'utente deve poter inserire il valore per il filtro "Id Aggregazione Primario" & \begin{tabular}[t]{@{}l} \ref{addIdAggregazionePrimario} \\ \ref{compilaValoreFiltro} \end{tabular} \\
\hline
R-94-F-Ob &  Per l'Aggregazione Documentale Informatica, l'utente deve poter inserire il valore per il filtro "Data Apertura" & \begin{tabular}[t]{@{}l} \ref{addDataApertura} \\ \ref{compilaValoreFiltro} \end{tabular} \\
\hline
R-95-F-Ob &  Per l'Aggregazione Documentale Informatica, l'utente deve poter inserire il valore per il filtro "Data Chiusura" & \begin{tabular}[t]{@{}l} \ref{addDataChiusura} \\ \ref{compilaValoreFiltro} \end{tabular} \\
\hline
R-96-F-Ob &  Per l'Aggregazione Documentale Informatica, il sistema deve rendere disponibile una sezione per il filtro "Procedimento Amministrativo" & \ref{addProcedimentoAmministrativo} \\
\hline
R-97-F-Ob &  Per l'Aggregazione Documentale Informatica, all'interno della sezione del filtro "Procedimento Amministrativo", l'utente deve poter inserire il valore per i campi "Materia/Argomento/Struttura", "Procedimento", "Catalogo Procedimenti" e "Fasi" & \begin{tabular}[t]{@{}l} \ref{addProcedimentoAmministrativo} \\ \ref{compilaValoreFiltro} \end{tabular} \\
\hline
R-98-F-Ob &  Per l'Aggregazione Documentale Informatica, all'interno della sezione del filtro "Procedimento Amministrativo" per il campo "Fasi", l'utente deve poter inserire il valore per il campo "Tipo Fase" tra "Preparatoria", "Istruttoria", "Consultiva", "Decisoria o Deliberativa" o "Integrazione dell'efficacia", il campo "Data Inizio Fase" e "Data Fine Fase" se presente & \begin{tabular}[t]{@{}l} \ref{addFasiProcedimentoAmministrativo} \\ \ref{compilaValoreFiltro} \end{tabular} \\
\hline
R-99-F-Ob &  Per l'Aggregazione Documentale Informatica, il sistema deve rendere disponibile una sezione per il filtro "Assegnazione" & \ref{addAssegnazione} \\
\hline
R-100-F-Ob &  Per l'Aggregazione Documentale Informatica, all'interno della sezione del filtro "Assegnazione", l'utente deve poter inserire una o più volte il valore per i campi "Tipo Assegnazione", "Soggetto Assegnatario", "Data Inizio Assegnazione" e "Data Fine Assegnazione" & \begin{tabular}[t]{@{}l} \ref{addAssegnazione} \\ \ref{compilaValoreFiltro} \end{tabular} \\
\hline
R-101-F-Ob &  Per l'Aggregazione Documentale Informatica, l'utente deve poter inserire il valore per il filtro "Progressivo Aggregazione" & \begin{tabular}[t]{@{}l} \ref{addProgressivoAggregazione} \\ \ref{compilaValoreFiltro} \end{tabular} \\
\hline
R-102-F-Ob &  All'interno della sezione di filtri per custom metadata, il sistema deve permettere di selezionare i filtri specifici per i metadata presenti & \ref{addCustomMetadata} \\
\hline
R-103-F-Ob &  Per ciascun custom metadata, l'utente deve poter inserire il nome del metadato e il relativo valore & \begin{tabular}[t]{@{}l} \ref{addFiltroMetadatoCustom} \\ \ref{compilaValoreFiltro} \end{tabular} \\
\hline
R-104-F-Ob &  Quando viene eseguita una ricerca, il sistema deve mostrare i risultati della ricerca & \ref{visualizzaRisultati} \\
\hline
R-105-F-Ob &  Il sistema deve mostrare per ogni risultato le informazioni rilevanti: Nome del documento o aggregazione, Data di registrazione/creazione del documento/aggregazione, Tipo di elemento tra Documento, Aggregazione, Processo o Classe Documentale & \ref{infoRisultatiRicerca} \\
\hline
R-106-F-Ob &  Se la ricerca non produce risultati, l'utente deve poter visualizzare un messaggio di errore & \ref{nessunRisultato} \\
\hline
R-107-F-Ob &  Il sistema deve comunicare all'utente quando il formato del valore inserito non è valido & \ref{formatoNonCorretto} \\
\hline

R-108-F-Ob & L'utente deve poter compilare il filtro selezionato con un valore & \ref{compilaValoreFiltro}  \\
\hline
R-109-F-Ob & Se il valore inserito  in un filtro non è corretto l'utente deve poter visualizzare un messaggio di errore & \ref{formatoNonCorretto} \\
\hline
R-110-F-Ob & L'utente deve poter salvare il documento in locale in una cartella selezionata & \ref{salvaDocumento} \\
\hline
R-111-F-Ob & L'utente deve poter salvare più documenti in una cartella selezionata & \ref{salvaPiuDocumenti} \\
\hline
R-112-F-Ob & Se il salvataggio di uno o più documenti fallisce, l'utente deve poter visualizzare un messaggio di errore & \ref{salvataggioFallito} \\
\hline
R-113-F-Ob & L'utente deve poter stampare un documento & \ref{stampaSingoloDoc} \\
\hline
R-114-F-Ob & L'utente deve poter stampare un insieme di documenti & \ref{stampaInsiemeDoc} \\
\hline
R-115-F-Ob & Se la stampa di uno o più documenti fallisce, l'utente deve poter visualizzare un messaggio di errore & \ref{stampaFallita} \\
\hline
R-116-F-Ob & Se la stampa non è disponibile, l'utente deve poter visualizzare un messaggio di errore & \ref{stampaNonDisponibile} \\
\hline
R-117-F-Ob & L'utente deve poter avviare la verifica del DIP & \ref{verificaIntegritaDIPCompleto} \\
\hline
R-118-F-Ob & L'utente deve poter visualizzare lo stato di verifica del DIP & \ref{verificaIntegritaDIPCompleto} \\
\hline
R-119-F-Ob & L'utente deve poter avviare la verifica della classe documentale  & \ref{verificaIntegritaClasseDocumentale}\\
\hline
R-120-F-Ob & L'utente deve poter visualizzare lo stato di verifica della classe documentale  & \ref{verificaIntegritaClasseDocumentale} \\
\hline
R-121-F-Ob & L'utente deve poter avviare la verifica dell'integrità processo & \ref{verificaIntegritaProcesso} \\
\hline
R-122-F-Ob & L'utente deve poter visualizzare lo stato di verifica dell'integrità processo & \ref{verificaIntegritaProcesso} \\
\hline
R-123-F-Ob & L'utente deve poter avviare la verifica del documento & \ref{verificaIntegritaDocumento} \\
\hline
R-124-F-Ob & L'utente deve poter visualizzare lo stato di verifica del documento & \ref{verificaIntegritaDocumento} \\
\hline





R-125-F-Ob & L'utente deve poter visualizzare il report di integrità del DIP completo con le informazioni aggregate & \ref{visualizzazioneReportIntegritaDIPCompleto} \\
\hline
R-126-F-Ob & Il sistema deve mostrare il conteggio totale delle classi documentali verificate nel report del DIP & \ref{visualizzazioneNumeroClassiVerificate} \\
\hline
R-127-F-Ob & Il sistema deve mostrare il numero di classi integre (stato "Valido") in colore verde nel report del DIP & \ref{visualizzazioneNumeroClassiIntegre} \\
\hline
R-128-F-Ob & Il sistema deve mostrare il numero di classi corrotte (stato "Non Valido") in colore rosso nel report del DIP & \ref{visualizzazioneNumeroClassiCorrotte} \\
\hline
R-129-F-Ob & Il sistema deve mostrare l'elenco delle classi corrotte indicando nome e numero di processi corrotti per ciascuna & \ref{visualizzazioneListaClassiCorrotte} \\
\hline
R-130-F-Ob & Il sistema deve mostrare la data e l'ora di inizio della verifica del DIP nel formato "GG/MM/AAAA HH:MM:SS" & \ref{visualizzazioneDataEOraVerificaDIP} \\
\hline
R-131-F-Ob & L'utente deve poter visualizzare il report di integrità dettagliato per una singola classe documentale & \ref{visualizzazioneReportIntegritaClasseDocumentale} \\
\hline
R-132-F-Ob & Il sistema deve mostrare il conteggio dei processi verificati all'interno della classe documentale & \ref{visualizzazioneNumeroProcessiVerificati} \\
\hline
R-133-F-Ob & Il sistema deve mostrare il numero di processi integri (stato "Valido") in colore verde nella classe documentale & \ref{visualizzazioneNumeroProcessiIntegri} \\
\hline
R-134-F-Ob & Il sistema deve mostrare il numero di processi corrotti (stato "Non Valido") in colore rosso nella classe documentale & \ref{visualizzazioneNumeroProcessiCorrotti} \\
\hline
R-135-F-Ob & Il sistema deve mostrare la lista dei processi corrotti con il relativo numero di documenti compromessi & \ref{visualizzazioneListaProcessiCorrotti} \\
\hline
R-136-F-Ob & Il sistema deve mostrare la data e l'ora di inizio della verifica della classe nel formato "GG/MM/AAAA HH:MM:SS" & \ref{visualizzazioneDataEOraVerificaClasse} \\
\hline
R-137-F-Ob & L'utente deve poter visualizzare il report di integrità dettagliato di un singolo processo & \ref{visualizzazioneReportIntegritaProcesso} \\
\hline
R-138-F-Ob & Il sistema deve mostrare il conteggio dei documenti verificati all'interno del processo selezionato & \ref{visualizzazioneNumeroDocumentiVerificati} \\
\hline
R-139-F-Ob & Il sistema deve mostrare il numero di documenti integri (stato "Valido") in colore verde nel processo selezionato & \ref{visualizzazioneNumeroDocumentiIntegri} \\
\hline
R-140-F-Ob & Il sistema deve mostrare il numero di documenti corrotti (stato "Non Valido") in colore rosso nel processo selezionato & \ref{visualizzazioneNumeroDocumentiCorrotti} \\
\hline
R-141-F-Ob & Il sistema deve mostrare la lista dei documenti corrotti con indicazione del nome e dell'errore specifico riscontrato & \ref{visualizzazioneListaDocumentiCorrotti} \\
\hline
R-142-F-Ob & Il sistema deve mostrare la data e l'ora di inizio della verifica del processo nel formato "GG/MM/AAAA HH:MM:SS" & \ref{visualizzazioneDataEOraVerificaProcesso} \\
\hline
R-143-F-Ob & L'utente deve poter visualizzare il report di integrità di un singolo documento & \ref{visualizzazioneReportIntegritaDocumento} \\
\hline
R-144-F-Ob & Il sistema deve mostrare il nome del documento all'interno del report di integrità & \ref{visualizzazioneNomeDocumento} \\
\hline
R-145-F-Ob & Il sistema deve mostrare lo stato della verifica (Valido / Non Valido) per il documento selezionato & \ref{visualizzazioneStatoVerificaDocumento} \\
\hline
R-146-F-Ob & Il sistema deve mostrare la data e l'ora di inizio della verifica del documento nel formato "GG/MM/AAAA HH:MM:SS" & \ref{visualizzazioneDataEOraVerificaDocumento} \\
\hline
R-147-F-Ob & Il sistema deve mostrare la descrizione tecnica del dettaglio dell'errore per i documenti con stato "Non Valido" & \ref{visualizzazioneDettagliErroreDocumento} \\
\hline
R-148-F-Ob & L'utente deve poter avviare la conversione del report di verifica visualizzato in formato PDF & \ref{convertiReportVerificaPDF} \\
\hline
R-149-F-Ob & Il sistema deve mostrare un messaggio di errore qualora la generazione del file PDF non vada a buon fine & \ref{erroreGenerazionePDF} \\
\hline
R-150-F-Ob & L'utente deve poter scaricare un file in una cartella locale previa selezione della cartella di destinazione & \ref{scaricaFile} \\
\hline
R-151-F-Ob & Il sistema deve mostrare un messaggio di conferma indicando il percorso di destinazione al termine del salvataggio & \ref{scaricaFile} \\
\hline
R-152-F-Ob & Il sistema deve impedire il salvataggio di file all'interno della cartella sorgente del DIP per preservarne l'integrità & \ref{scaricaFile}, \ref{erroreScaricamentoFile} \\
\hline
R-153-F-Ob & Il sistema deve mostrare un errore specifico se l'utente tenta di scaricare un file nel percorso protetto del DIP & \ref{erroreScaricamentoFile} \\
\hline
R-154-F-Ob & L'utente deve poter visualizzare le informazioni dell'AiP di provenienza di un documento selezionato & \ref{visualizzaInfoAiP} \\
\hline
R-155-F-Ob & Il sistema deve mostrare la classe documentale di appartenenza dell'AiP relativo al documento selezionato & \ref{visualizzaClasseDocumentaleAiP} \\
\hline
R-156-F-Ob & Il sistema deve mostrare lo UUID dell'AiP relativo al documento selezionato & \ref{visualizzaUUIDAiP} \\
\hline




R-157-F-Ob & L'utente deve poter visualizzare le informazioni del processo di conservazione dell'AiP & \ref{visualizzaInfoProcessoConservazioneAiP} \\
\hline
R-158-F-Ob & L'utente deve poter visualizzare la data di inizio di un processo o sessione & \ref{visualizzaDataInizioProcessoSessione} \\
\hline
R-159-F-Ob & L'utente deve poter visualizzare la data di fine di un processo o sessione & \ref{visualizzaDataFineProcessoSessione} \\
\hline
R-160-F-Ob & Se il processo o la sessione non è ancora terminato/a, al posto della data di fine il sistema deve mostrare un messaggio che indica l'assenza della data di fine & \ref{erroreDataFineProcessoSessione} \\
\hline
R-161-F-Ob & L'utente deve poter visualizzare lo UUID dell'utente attivatore di un processo o sessione & \ref{visualizzaUUIDUtenteAttivatore} \\
\hline
R-162-F-Ob & L'utente deve poter visualizzare lo UUID dell'utente terminatore di un processo o sessione & \ref{visualizzaUUIDUtenteTerminatore} \\
\hline
R-163-F-Ob & Se il processo o la sessione non è ancora terminato/a, al posto dello UUID dell'utente terminatore il sistema deve mostrare un messaggio che indica l'assenza dello UUID & \ref{erroreUUIDUtenteTerminatore} \\
\hline
R-164-F-Ob & L'utente deve poter visualizzare il nome del canale di attivazione di un processo o sessione & \ref{visualizzaNomeCanaleAttivazione} \\
\hline
R-165-F-Ob & L'utente deve poter visualizzare il nome del canale di terminazione di un processo o sessione & \ref{visualizzaNomeCanaleTerminazione} \\
\hline
R-166-F-Ob & Se il processo o la sessione non è ancora terminato/a, al posto del nome del canale di terminazione il sistema deve mostrare un messaggio che indica l'assenza del nome del canale di terminazione & \ref{erroreNomeCanaleTerminazione} \\
\hline
R-167-F-Ob & L'utente deve poter visualizzare lo stato di un processo o sessione & \ref{visualizzaStatoProcessoSessione} \\
\hline
R-168-F-Ob & L'utente deve poter visualizzare le informazioni della sessione di versamento del processo di conservazione selezionato & \ref{visualizzaInfoSessioneVersamento} \\
\hline
R-169-F-Ob & L'utente deve poter visualizzare la data di inizio della sessione di versamento & \ref{visualizzaDataInizioProcessoSessione} \\
\hline
R-170-F-Ob & L'utente deve poter visualizzare la data di fine della sessione di versamento & \ref{visualizzaDataFineProcessoSessione} \\
\hline
R-171-F-Ob & Se la sessione di versamento non è ancora terminata, al posto della data di fine il sistema deve mostrare un messaggio che indica l'assenza della data di fine & \ref{erroreDataFineProcessoSessione} \\
\hline
R-172-F-Ob & L'utente deve poter visualizzare lo UUID dell'utente attivatore della sessione di versamento & \ref{visualizzaUUIDUtenteAttivatore} \\
\hline
R-173-F-Ob & L'utente deve poter visualizzare lo UUID dell'utente terminatore della sessione di versamento & \ref{visualizzaUUIDUtenteTerminatore} \\
\hline
R-174-F-Ob & Se la sessione di versamento non è ancora terminata, al posto dello UUID dell'utente terminatore il sistema deve mostrare un messaggio che indica l'assenza dello UUID & \ref{erroreUUIDUtenteTerminatore} \\
\hline
R-175-F-Ob & L'utente deve poter visualizzare il nome del canale di attivazione della sessione di versamento & \ref{visualizzaNomeCanaleAttivazione} \\
\hline
R-176-F-Ob & L'utente deve poter visualizzare il nome del canale di terminazione della sessione di versamento & \ref{visualizzaNomeCanaleTerminazione} \\
\hline
R-177-F-Ob & Se la sessione di versamento non è ancora terminata, al posto del nome del canale di terminazione il sistema deve mostrare un messaggio che indica l'assenza del nome del canale di terminazione & \ref{erroreNomeCanaleTerminazione} \\
\hline
R-178-F-Ob & L'utente deve poter visualizzare lo stato della sessione di versamento & \ref{visualizzaStatoProcessoSessione} \\
\hline
R-179-F-Ob & L'utente deve poter visualizzare le informazioni della sessione di conservazione del processo di conservazione selezionato & \ref{visualizzaInfoSessioneConservazione} \\
\hline
R-180-F-Ob & L'utente deve poter visualizzare la data di inizio della sessione di conservazione & \ref{visualizzaDataInizioProcessoSessione} \\
\hline
R-181-F-Ob & L'utente deve poter visualizzare la data di fine della sessione di conservazione & \ref{visualizzaDataFineProcessoSessione} \\
\hline
R-182-F-Ob & Se la sessione di conservazione non è ancora terminata, al posto della data di fine il sistema deve mostrare un messaggio che indica l'assenza della data di fine & \ref{erroreDataFineProcessoSessione} \\
\hline
R-183-F-Ob & L'utente deve poter visualizzare lo UUID dell'utente attivatore della sessione di conservazione & \ref{visualizzaUUIDUtenteAttivatore} \\
\hline
R-184-F-Ob & L'utente deve poter visualizzare lo UUID dell'utente terminatore della sessione di conservazione & \ref{visualizzaUUIDUtenteTerminatore} \\
\hline
R-185-F-Ob & Se la sessione di conservazione non è ancora terminata, al posto dello UUID dell'utente terminatore il sistema deve mostrare un messaggio che indica l'assenza dello UUID & \ref{erroreUUIDUtenteTerminatore} \\
\hline
R-186-F-Ob & L'utente deve poter visualizzare il nome del canale di attivazione della sessione di conservazione & \ref{visualizzaNomeCanaleAttivazione} \\
\hline
R-187-F-Ob & L'utente deve poter visualizzare il nome del canale di terminazione della sessione di conservazione & \ref{visualizzaNomeCanaleTerminazione} \\
\hline
R-188-F-Ob & Se la sessione di conservazione non è ancora terminata, al posto del nome del canale di terminazione il sistema deve mostrare un messaggio che indica l'assenza del nome del canale di terminazione & \ref{erroreNomeCanaleTerminazione} \\
\hline
R-189-F-Ob & L'utente deve poter visualizzare lo stato della sessione di conservazione & \ref{visualizzaStatoProcessoSessione} \\
\hline
R-190-F-Ob & L'utente deve poter visualizzare la descrizione del documento selezionato & \ref{visualizzaDescrizioneDocumento} \\
\hline
R-191-F-Ob & L'utente deve poter visualizzare la lista dei soggetti coinvolti nel documento selezionato & \ref{visualizzaListaSoggettiCoinvolti} \\
\hline
R-192-F-Ob & Per ogni soggetto coinvolto nel documento, il sistema deve visualizzare il ruolo del soggetto nel documento & \ref{visualizzaRuoloSoggetto} \\
\hline
R-193-F-Ob & Per ogni soggetto coinvolto nel documento, il sistema deve visualizzare il tipo di soggetto & \ref{visualizzaTipoSoggetto} \\
\hline
R-194-F-Ob & Se il soggetto coinvolto nel documento è di tipo Persona Fisica, il sistema deve visualizzare il nome del soggetto & \ref{visualizzaNomeSoggetto} \\
\hline
R-195-F-Ob & Se il soggetto coinvolto nel documento è di tipo Persona Fisica, il sistema deve visualizzare il cognome del soggetto & \ref{visualizzaCognomeSoggetto} \\
\hline
R-196-F-Ob & Se il soggetto coinvolto nel documento è di tipo Persona Fisica, il sistema deve visualizzare il codice fiscale del soggetto & \ref{visualizzaCodiceFiscaleSoggetto} \\
\hline
R-197-F-Ob & Se il soggetto coinvolto nel documento è di tipo Persona Fisica, il sistema deve visualizzare gli indirizzi digitali di riferimento del soggetto & \ref{visualizzaIndirizziDigitaliSoggetto} \\
\hline
R-198-F-Ob & Se il soggetto coinvolto nel documento è di tipo Persona Giuridica, il sistema deve visualizzare la denominazione dell'organizzazione del soggetto & \ref{visualizzaDenominazioneOrganizzazioneSoggetto} \\
\hline
R-199-F-Ob & Se il soggetto coinvolto nel documento è di tipo Persona Giuridica, il sistema deve visualizzare la partita IVA del soggetto & \ref{visualizzaPartitaIVA} \\
\hline
R-200-F-Ob & Se il soggetto coinvolto nel documento è di tipo Persona Giuridica, il sistema deve visualizzare il codice fiscale del soggetto & \ref{visualizzaCodiceFiscaleSoggetto} \\
\hline
R-201-F-Ob & Se il soggetto coinvolto nel documento è di tipo Persona Giuridica, il sistema deve visualizzare la denominazione dell'ufficio del soggetto & \ref{visualizzaDenominazioneUfficioSoggetto} \\
\hline
R-202-F-Ob & Se il soggetto coinvolto nel documento è di tipo Persona Giuridica, il sistema deve visualizzare gli indirizzi digitali di riferimento del soggetto & \ref{visualizzaIndirizziDigitaliSoggetto} \\
\hline
R-203-F-Ob & Se il soggetto coinvolto nel documento è di tipo AS, il sistema deve visualizzare il cognome del soggetto & \ref{visualizzaCognomeSoggetto} \\
\hline
R-204-F-Ob & Se il soggetto coinvolto nel documento è di tipo AS, il sistema deve visualizzare il nome del soggetto & \ref{visualizzaNomeSoggetto} \\
\hline
R-205-F-Ob & Se il soggetto coinvolto nel documento è di tipo AS, il sistema deve visualizzare il codice fiscale del soggetto & \ref{visualizzaCodiceFiscaleSoggetto} \\
\hline
R-206-F-Ob & Se il soggetto coinvolto nel documento è di tipo AS, il sistema deve visualizzare la denominazione dell'organizzazione del soggetto & \ref{visualizzaDenominazioneOrganizzazioneSoggetto} \\
\hline
R-207-F-Ob & Se il soggetto coinvolto nel documento è di tipo AS, il sistema deve visualizzare la denominazione dell'ufficio del soggetto & \ref{visualizzaDenominazioneUfficioSoggetto} \\
\hline
R-208-F-Ob & Se il soggetto coinvolto nel documento è di tipo AS, il sistema deve visualizzare gli indirizzi digitali di riferimento del soggetto & \ref{visualizzaIndirizziDigitaliSoggetto} \\
\hline
R-209-F-Ob & Se il soggetto coinvolto nel documento è di tipo PAI, il sistema deve visualizzare la denominazione dell'amministrazione e il codice IPA del soggetto & \ref{visualizzaDenominazioneAmministrazioneCodiceIPA} \\
\hline
R-210-F-Ob & Se il soggetto coinvolto nel documento è di tipo PAI, il sistema deve visualizzare la denominazione dell'amministrazione AOO e il codice IPA AOO del soggetto & \ref{visualizzaDenominazioneAmministrazioneAOOCodiceIPAOOO} \\
\hline
R-211-F-Ob & Se il soggetto coinvolto nel documento è di tipo PAI, il sistema deve visualizzare la denominazione dell'amministrazione UOR e il codice IPA UOR del soggetto & \ref{visualizzaDenominazioneAmministrazioneUORCodiceIPAUOR} \\
\hline
R-212-F-Ob & Se il soggetto coinvolto nel documento è di tipo PAI, il sistema deve visualizzare gli indirizzi digitali di riferimento del soggetto & \ref{visualizzaIndirizziDigitaliSoggetto} \\
\hline
R-213-F-Ob & Se il soggetto coinvolto nel documento è di tipo PAE, il sistema deve visualizzare la denominazione dell'amministrazione del soggetto & \ref{visualizzaDenominazioneAmministrazioneSoggetto} \\
\hline
R-214-F-Ob & Se il soggetto coinvolto nel documento è di tipo PAE, il sistema deve visualizzare la denominazione dell'ufficio del soggetto & \ref{visualizzaDenominazioneUfficioSoggetto} \\
\hline
R-215-F-Ob & Se il soggetto coinvolto nel documento è di tipo PAE, il sistema deve visualizzare gli indirizzi digitali di riferimento del soggetto & \ref{visualizzaIndirizziDigitaliSoggetto} \\
\hline
R-216-F-Ob & Se il soggetto coinvolto nel documento è di tipo SW, il sistema deve visualizzare la denominazione del sistema del soggetto & \ref{visualizzaDenominazioneSistemaSoggetto} \\
\hline
R-217-F-Ob & L'utente deve poter visualizzare l'indice di classificazione del documento selezionato & \ref{visualizzaIndiceClassificazioneDocumento} \\
\hline
R-218-F-Ob & L'utente deve poter visualizzare la descrizione dell'indice di classificazione del documento selezionato & \ref{visualizzaDescrizioneIndiceClassificazioneDocumento} \\
\hline
R-219-F-Ob & L'utente deve poter visualizzare l'URI del piano di classificazione del documento selezionato & \ref{visualizzaURIPianoClassificazioneDocumento} \\
\hline
R-220-F-Ob & Se il documento selezionato ha un tempo di conservazione diverso da quello assegnato all'aggregazione documentale informatica a cui appartiene, l'utente deve poter visualizzare il tempo di conservazione effettivo del documento & \ref{visualizzaTempoConservazioneEffettivoDocumento} \\
\hline
R-221-F-Ob & Se il tempo di conservazione del documento coincide con quello assegnato all'aggregazione documentale a cui appartiene, il sistema deve mostrare un messaggio che indica questa coincidenza & \ref{erroreTempoConservazioneEffettivoDocumento} \\
\hline
R-222-F-Ob & L'utente deve poter visualizzare le note relative al documento selezionato & \ref{visualizzaNoteDocumento} \\
\hline
R-223-F-Ob & Se le note del documento sono assenti o vuote, il sistema deve mostrare un messaggio che indica l'assenza delle note & \ref{erroreNoteDocumento} \\
\hline
R-224-F-Ob & L'utente deve poter visualizzare la tipologia di flusso del documento selezionato & \ref{visualizzaTipologiaFlussoDocumento} \\
\hline
R-225-F-Ob & L'utente deve poter visualizzare il tipo di registro del documento selezionato & \ref{visualizzaTipoRegistroDocumento} \\
\hline
R-226-F-Ob & L'utente deve poter visualizzare la data di registrazione del documento selezionato & \ref{visualizzaDataRegistrazioneDocumento} \\
\hline
R-227-F-Ob & L'utente deve poter visualizzare il numero del documento selezionato & \ref{visualizzaNumeroDocumento} \\
\hline
R-228-F-Ob & L'utente deve poter visualizzare il codice identificativo del registro di appartenenza del documento selezionato & \ref{visualizzaCodiceIdentificativoRegistroAppartenenzaDocumento} \\
\hline
R-229-F-Ob & L'utente deve poter visualizzare la tipologia documentale del documento selezionato & \ref{visualizzaTipologiaDocumentaleDocumento} \\
\hline
R-230-F-Ob & L'utente deve poter visualizzare la modalità di formazione del documento selezionato & \ref{visualizzaModalitaFormazioneDocumento} \\
\hline
R-231-F-Ob & L'utente deve poter visualizzare lo stato di riservatezza del documento selezionato & \ref{visualizzaStatoRiservatezzaDocumento} \\
\hline
R-232-F-Ob & L'utente deve poter visualizzare il tipo di formato del documento selezionato & \ref{visualizzaTipoFormatoDocumento} \\
\hline
R-233-F-Ob & L'utente deve poter visualizzare il nome del prodotto software che ha generato il documento selezionato & \ref{visualizzaNomeProdottoSoftware} \\
\hline
R-234-F-Ob & L'utente deve poter visualizzare la versione del prodotto software che ha generato il documento selezionato & \ref{visualizzaVersioneProdottoSoftware} \\
\hline
R-235-F-Ob & L'utente deve poter visualizzare il produttore del software che ha generato il documento selezionato & \ref{visualizzaProduttoreSoftware} \\
\hline
R-236-F-Ob & L'utente deve poter visualizzare se il documento selezionato è firmato digitalmente & \ref{visualizzaFirmaDigitaleDocumento} \\
\hline
R-237-F-Ob & L'utente deve poter visualizzare se il documento selezionato è sigillato elettronicamente & \ref{visualizzaSigilloElettronicoDocumento} \\
\hline
R-238-F-Ob & L'utente deve poter visualizzare se il documento selezionato è dotato di marcatura temporale & \ref{visualizzaMarcaturaTemporaleDocumento} \\
\hline
R-239-F-Ob & L'utente deve poter visualizzare se vi è conformità alle copie immagine su supporto informatico del documento selezionato & \ref{visualizzaConformitaCopieImmagineDocumento} \\
\hline
R-240-F-Ob & L'utente deve poter visualizzare la versione del documento selezionato & \ref{visualizzaVersioneDocumento} \\
\hline
R-241-F-Ob & L'utente deve poter visualizzare il nome del documento selezionato & \ref{visualizzaNomeDocumento} \\
\hline
R-242-F-Ob & L'utente deve poter visualizzare il numero di allegati del documento selezionato & \ref{visualizzaNumeroAllegatiDocumento} \\
\hline
R-243-F-Ob & Se il documento ha almeno un allegato, l'utente deve poter visualizzare l'identificativo di ciascun allegato & \ref{visualizzaIdentificativoAllegato} \\
\hline
R-244-F-Ob & Se l'informazione sull'identificativo dell'allegato non è disponibile, il sistema deve mostrare un messaggio di errore & \ref{erroreIdentificativoAllegato} \\
\hline
R-245-F-Ob & Se il documento ha almeno un allegato, l'utente deve poter visualizzare la descrizione di ciascun allegato & \ref{visualizzaDescrizioneAllegato} \\
\hline
R-246-F-Ob & Se l'informazione sulla descrizione dell'allegato non è disponibile, il sistema deve mostrare un messaggio di errore & \ref{erroreDescrizioneAllegato} \\
\hline
R-247-F-Ob & Se il documento non ha allegati, il sistema deve mostrare un messaggio che indica l'assenza degli allegati & \ref{allegatiNonPresenti} \\
\hline
R-248-F-Ob & L'utente deve poter visualizzare il tipo di modifica per ogni modifica del documento selezionato tra: Annullamento, Rettifica, Integrazione e Annotazione & \ref{visualizzaTipoModificaDocumento} \\
\hline
R-249-F-Ob & L'utente deve poter visualizzare le informazioni del soggetto autore di ogni modifica del documento selezionato & \ref{visualizzaSoggettoAutoreModifica} \\
\hline
R-250-F-Ob & L'utente deve poter visualizzare la data e l'ora di ogni modifica del documento selezionato & \ref{visualizzaDataOraModifica} \\
\hline
R-251-F-Ob & L'utente deve poter visualizzare l'identificativo del documento alla versione precedente alla modifica & \ref{visualizzaIdentificativoVersionePrecedenteDocumento} \\
\hline
R-252-F-Ob & L'utente deve poter visualizzare il tipo di aggregazione dell'aggregazione documentale selezionata tra: Fascicolo, Serie Documentale e Serie di Fascicoli & \ref{visualizzaTipoAggregazione} \\
\hline
R-253-F-Ob & L'utente deve poter visualizzare l'identificativo dell'aggregazione documentale selezionata & \ref{visualizzaIdentificativoAggregazione} \\
\hline
R-254-F-Ob & L'utente deve poter visualizzare la tipologia di fascicolo dell'aggregazione documentale selezionata tra: Affare, Attività, Persona Fisica, Persona Giuridica e Procedimento Amministrativo & \ref{visualizzaTipologiaFascicolo} \\
\hline
R-255-F-Ob & L'utente deve poter visualizzare il tipo di assegnazione dell'aggregazione documentale selezionata tra: Per competenza e Per conoscenza & \ref{visualizzaTipoAssegnazioneAggregazione} \\
\hline
R-256-F-Ob & L'utente deve poter visualizzare le informazioni del soggetto assegnatario dell'aggregazione documentale selezionata & \ref{visualizzaInfoSoggettoCoinvolto} \\
\hline
R-257-F-Ob & L'utente deve poter visualizzare la data e l'ora di inizio dell'assegnazione dell'aggregazione documentale selezionata & \ref{visualizzaDataOraInizioAssegnazioneAggregazione} \\
\hline
R-258-F-Ob & L'utente deve poter visualizzare la data e l'ora di fine dell'assegnazione dell'aggregazione documentale selezionata & \ref{visualizzaDataOraFineAssegnazioneAggregazione} \\
\hline
R-259-F-Ob & L'utente deve poter visualizzare la data di apertura dell'aggregazione documentale selezionata & \ref{visualizzaDataAperturaAggregazione} \\
\hline
R-260-F-Ob & L'utente deve poter visualizzare la data di chiusura dell'aggregazione documentale selezionata & \ref{visualizzaDataChiusuraAggregazione} \\
\hline
R-261-F-Ob & L'utente deve poter visualizzare il progressivo dell'aggregazione documentale selezionata & \ref{visualizzaProgressivoAggregazione} \\
\hline
R-262-F-Ob & L'utente deve poter visualizzare l'indice della materia/argomento/struttura per la quale sono catalogati i procedimenti dell'aggregazione documentale selezionata & \ref{visualizzaIndiceProcedimentoAmministrativo} \\
\hline
R-263-F-Ob & L'utente deve poter visualizzare la denominazione del procedimento amministrativo dell'aggregazione documentale selezionata & \ref{visualizzaDenominazioneProcedimentoAmministrativo} \\
\hline
R-264-F-Ob & L'utente deve poter visualizzare il catalogo dei procedimenti come URI di pubblicazione dell'aggregazione documentale selezionata & \ref{visualizzaCatalogoProcedimentiAmministrativi} \\
\hline
R-265-F-Ob & L'utente deve poter visualizzare la lista delle fasi del procedimento amministrativo dell'aggregazione documentale selezionata & \ref{visualizzaFasiProcedimentoAmministrativo} \\
\hline
R-266-F-Ob & Per ogni fase del procedimento amministrativo, il sistema deve visualizzare il tipo di fase tra: Preparatoria, Istruttoria, Consultiva, Decisoria o deliberativa e Integrazione dell'efficacia & \ref{visualizzaTipoFaseProcedimentoAmministrativo} \\
\hline
R-267-F-Ob & Per ogni fase del procedimento amministrativo, il sistema deve visualizzare la data e l'ora di inizio della fase & \ref{visualizzaDataOraInizioFase} \\
\hline
R-268-F-Ob & Per ogni fase del procedimento amministrativo, il sistema deve visualizzare la data e l'ora di fine della fase & \ref{visualizzaDataOraFineFase} \\
\hline
R-269-F-Ob & L'utente deve poter visualizzare l'indice dei documenti contenuti nell'aggregazione documentale selezionata & \ref{visualizzaIndiceDocumentiAggregazione} \\
\hline
R-270-F-Ob & Per ogni voce dell'indice dei documenti dell'aggregazione, il sistema deve visualizzare il tipo di documento contenuto & \ref{visualizzaTipoDocumentiAggregazione} \\
\hline
R-271-F-Ob & Per ogni voce dell'indice dei documenti dell'aggregazione, il sistema deve visualizzare l'identificativo del documento contenuto & \ref{visualizzaIdentificativoDocumentoAggregazione} \\
\hline
R-272-F-Ob & L'utente deve poter visualizzare la posizione fisica dell'aggregazione documentale selezionata & \ref{visualizzaPosizioneFisicaAggregazione} \\
\hline
R-273-F-Ob & L'utente deve poter visualizzare l'identificativo dell'aggregazione primaria dell'aggregazione documentale selezionata & \ref{visualizzaIdentificativoAggregazionePrimaria} \\
\hline
R-274-F-Ob & L'utente deve poter visualizzare il tempo di conservazione dell'aggregazione documentale selezionata & \ref{visualizzaTempoConservazioneAggregazione} \\
\hline
R-275-F-Ob & L'utente deve poter visualizzare il nome di ciascun metadato custom del documento selezionato & \ref{visualizzaNomeMetadatoCustom} \\
\hline
R-276-F-Ob & L'utente deve poter visualizzare il valore di ciascun metadato custom del documento selezionato & \ref{visualizzaValoreMetadatoCustom} \\
\hline
R-277-F-Ob & Se il documento selezionato non dispone di metadati custom, il sistema deve mostrare un messaggio che indica l'assenza dei metadati custom & \ref{metadatiCustomAssenti} \\
\hline

\caption{Requisiti Funzionali}
\label{tab:req-funzionali}
\end{longtable}

\subsection{Requisiti di Qualità}

\rowcolors{2}{gray!15}{white}
\begin{longtable}{|p{2.5cm}|p{9.5cm}|p{2.5cm}|}
\hline
\rowcolor{zpusgreen!30}
\textbf{Codice} & \textbf{Descrizione} & \textbf{Fonti} \\
\hline
\endfirsthead
\rowcolor{zpusgreen!30}
\textbf{Codice} & \textbf{Descrizione} & \textbf{Fonti} \\
\hline
\endhead
R-1-Q-Ob & Il sistema deve garantire un tempo di risposta inferiore a 2 secondi per il caricamento delle interfacce principali dell'applicazione & \href{https://www.math.unipd.it/~tullio/IS-1/2025/Progetto/C3.pdf}{\ul{Capitolato di Progetto}\setulcolor{black}} \\
\hline
R-2-Q-Ob & Il sistema deve garantire un tempo di risposta inferiore a 2 minuti per le ricerche effettuate dall'utente & \href{https://www.math.unipd.it/~tullio/IS-1/2025/Progetto/C3.pdf}{\ul{Capitolato di Progetto}\setulcolor{black}} \\
\hline
R-3-Q-Ob & Il sistema deve fornire un'interfaccia utente intuitiva e user-friendly & \href{https://www.math.unipd.it/~tullio/IS-1/2025/Progetto/C3.pdf}{\ul{Capitolato di Progetto}\setulcolor{black}} \\
\hline
R-4-Q-De & Il sistema deve fornire una sezione dedicata per la consultazione dei report di integrità, presentati in modo chiaro e comprensibile & \href{https://cdn.jsdelivr.net/gh/7-zpus/Docs@main/2_RTB/Verbali/Verbali%20Esterni/2025-12-23-VerbaleEsterno.pdf}{\ul{Verbale Esterno 2025-12-23}\setulcolor{black}} \\
\hline
R-5-Q-Ob & Il sistema essere sottoposto a opportuni test di unità & \href{https://www.math.unipd.it/~tullio/IS-1/2025/Progetto/C3.pdf}{\ul{Capitolato di Progetto}\setulcolor{black}} \\
\hline
R-6-Q-Ob & Lo sviluppo deve seguire le \href{https://cdn.jsdelivr.net/gh/7-zpus/Docs@main/2_RTB/NormeDiProgetto.pdf}{\ul{Norme di Progetto}\setulcolor{black}\ped{v1.0}} & \href{https://cdn.jsdelivr.net/gh/7-zpus/Docs@main/2_RTB/NormeDiProgetto.pdf}{\ul{Norme di Progetto}\setulcolor{black}\ped{v1.0}} \\
\hline
R-7-Q-Ob & Il sistema deve essere utilizzabile da ogni utente senza la necessità di un manuale d'uso & \href{https://cdn.jsdelivr.net/gh/7-zpus/Docs@main/2_RTB/Verbali/Verbali%20Esterni/2025-11-27-VerbaleEsterno.pdf}{\ul{Verbale Esterno 2025-11-27}\setulcolor{black}} \\
\hline
R-8-Q-De & L'interfaccia utente deve rispettare i criteri di accessibilità definiti dalle linee guida WCAG 2.1 livello AA & \href{https://www.math.unipd.it/~tullio/IS-1/2025/Progetto/C3.pdf}{\ul{Capitolato di Progetto}\setulcolor{black}} \\
\hline
\rowcolor{white}
\caption{Requisiti di Qualità}
\label{tab:req-qualita}
\end{longtable}

\subsection{Requisiti di Vincolo}

\rowcolors{2}{gray!15}{white}
\begin{longtable}{|p{2.5cm}|p{9.5cm}|p{2.5cm}|}
\hline
\rowcolor{zpusgreen!30}
\textbf{Codice} & \textbf{Descrizione} & \textbf{Fonti} \\
\hline
\endfirsthead
\rowcolor{zpusgreen!30}
\textbf{Codice} & \textbf{Descrizione} & \textbf{Fonti} \\
\hline
\endhead
R-1-V-Ob & Il sistema deve operare in modalità \textit{portable}, ovvero senza necessità di installazioni o privilegi di amministratore nel sistema ospite & \href{https://www.math.unipd.it/~tullio/IS-1/2025/Progetto/C3.pdf}{\ul{Capitolato di Progetto}\setulcolor{black}} \\
\hline
R-2-V-Ob & Il sistema deve garantire la compatibilità multipiattaforma, assicurando il corretto funzionamento e l'integrità delle funzionalità sui sistemi operativi Windows, MacOs e Linux & \href{https://www.math.unipd.it/~tullio/IS-1/2025/Progetto/C3.pdf}{\ul{Capitolato di Progetto}\setulcolor{black}} \\
\hline
R-3-V-Ob & L'architettura del sistema deve essere basata su componenti modulari e interfacce di astrazione per permettere l'integrazione di nuove metodologie di reperimento dati & \href{https://www.math.unipd.it/~tullio/IS-1/2025/Progetto/C3.pdf}{\ul{Capitolato di Progetto}\setulcolor{black}} \\
\hline
R-4-V-Ob & Il sistema deve garantire la piena funzionalità di ricerca e consultazione in assenza di connettività di rete & \href{https://www.math.unipd.it/~tullio/IS-1/2025/Progetto/C3.pdf}{\ul{Capitolato di Progetto}\setulcolor{black}} \\
\hline
R-5-V-Ob & Il sistema deve deve utilizzare un motore di ricerca \textit{embedded} che non richieda un server standalone & \href{https://www.math.unipd.it/~tullio/IS-1/2025/Progetto/C3.pdf}{\ul{Capitolato di Progetto}\setulcolor{black}} \\
\hline
R-6-V-Ob & Il sistema deve essere sviluppato utilizzando il framework Angular & \href{https://www.math.unipd.it/~tullio/IS-1/2025/Progetto/C3.pdf}{\ul{Capitolato di Progetto}\setulcolor{black}} \\
\hline
R-7-V-Ob & Il sistema deve essere gestito e consegnato tramite il sistema di versionamento del codice GitHub & \href{https://www.math.unipd.it/~tullio/IS-1/2025/Progetto/C3.pdf}{\ul{Capitolato di Progetto}\setulcolor{black}} \\
\hline
R-8-V-Ob & Devono essere prodotti i diagrammi UML relativi ai casi d'uso & \href{https://www.math.unipd.it/~tullio/IS-1/2025/Progetto/C3.pdf}{\ul{Capitolato di Progetto}\setulcolor{black}} \\
\hline
R-9-V-Ob & Devono essere prodotti gli schemi di database e chiamate API utilizzati dal sistema & \href{https://www.math.unipd.it/~tullio/IS-1/2025/Progetto/C3.pdf}{\ul{Capitolato di Progetto}\setulcolor{black}} \\
\hline
R-10-V-Ob & Deve essere prodotta una documentazione tecnica relativa ai casi di test gestiti con relativa reportistica e una lista dei bug risolti & \href{https://www.math.unipd.it/~tullio/IS-1/2025/Progetto/C3.pdf}{\ul{Capitolato di Progetto}\setulcolor{black}} \\
\hline
R-11-V-Ob & Il sistema deve supportare rigorosamente la struttura dei metadati definita dalle Linee Guida AGID & \href{https://cdn.jsdelivr.net/gh/7-zpus/Docs@main/2_RTB/Verbali/Verbali%20-Esterni/2025-11-13-VerbaleEsterno.pdf}{\ul{Verbale Esterno 2025-11-13}\setulcolor{black}} \\
\hline

\caption{Requisiti di Vincolo}
\label{tab:req-vincolo}
\end{longtable}


\subsection{Tracciamento}

\subsubsection{Tracciamento Fonti - Requisiti}

\rowcolors{2}{gray!15}{white}
\begin{longtable}{|p{4cm}|p{10cm}|}
\hline
\rowcolor{zpusgreen!30}
\textbf{Fonti} & \textbf{Requisiti} \\
\hline
\endfirsthead
\rowcolor{zpusgreen!30}
\textbf{Fonti} & \textbf{Requisiti} \\
\hline
\endhead
\ref{classiDocumentali} & R-1-F-Ob \\
\hline
\ref{elencoVuoto} & R-2-F-Ob, R-8-F-Ob, R-12-F-Ob \\
\hline
\ref{classeDocumentale} & R-3-F-Ob \\
\hline
\ref{nomeClasseDocumentale} & R-4-F-Ob \\
\hline
\ref{statoVerificaElemento} & R-5-F-Ob, R-15-F-Ob \\
\hline
\ref{marcaturaTemporaleElemento} & R-6-F-Ob, R-16-F-Ob \\
\hline
\ref{processiClasseDocumentale} & R-7-F-Ob \\
\hline
\ref{processoClasseDocumentale} & R-9-F-Ob \\
\hline
\ref{idProcessoClasse} & R-10-F-Ob \\
\hline
\ref{documentiProcesso} & R-11-F-Ob \\
\hline
\ref{documentoProcesso} & R-13-F-Ob \\
\hline
\ref{nomeDocumentoProcesso} & R-14-F-Ob \\
\hline
\ref{selezionaClasseDocumentale} & R-17-F-Ob \\
\hline
\ref{selezionaProcesso} & R-18-F-Ob \\
\hline
\ref{anteprimaDocumento} & R-19-F-Ob \\
\hline
\ref{formatoDocumentoNonSupportato} & R-20-F-Ob \\
\hline
\ref{ricercaDIP} & R-21-F-Ob, R-22-F-Ob, R-23-F-Ob, R-29-F-Ob, R-32-F-Ob, R-33-F-Ob \\
\hline
\ref{ricercaClasse} & R-22-F-Ob, R-33-F-Ob \\
\hline
\ref{inserimentoNomeClasseDocumentale} & R-22-F-Ob \\
\hline
\ref{ricercaProcesso} & R-23-F-Ob, R-32-F-Ob \\
\hline
\ref{inserimentoIdProcesso} & R-23-F-Ob \\
\hline
\ref{ricercaDIPSemantica} & R-24-F-Op \\
\hline
\ref{indicizzazioneSemantica} & R-25-F-Op, R-26-F-Op \\
\hline
\ref{erroreIndicizzazioneSemantica} & R-27-F-Op \\
\hline
\ref{StatoIndicizzazione} & R-28-F-Op \\
\hline
\ref{campoNonValido} & R-30-F-Ob \\
\hline
\ref{ricercaDIPConFiltri} & \begin{tabular}[t]{@{}l@{\hspace{0.5em}}l}
R-31-F-Ob & R-34-F-Ob \\
R-35-F-Ob & R-36-F-Ob \\
R-37-F-Ob & R-38-F-Ob \\
R-39-F-Ob & R-40-F-Ob \\
R-42-F-Ob & R-43-F-Ob \\
R-44-F-Ob & R-45-F-Ob \\
R-46-F-Ob & R-47-F-Ob \\
R-48-F-Ob & R-49-F-Ob \\
R-50-F-Ob & R-51-F-Ob \\
R-52-F-Ob & R-53-F-Ob \\
R-54-F-Ob & R-55-F-Ob \\
R-56-F-Ob & R-57-F-Ob \\
R-58-F-Ob & R-59-F-Ob \\
R-60-F-Ob & R-61-F-Ob \\
R-62-F-Ob & R-63-F-Ob \\
R-64-F-Ob & R-65-F-Ob \\
R-66-F-Ob & R-67-F-Ob \\
R-68-F-Ob & R-69-F-Ob \\
R-70-F-Ob & R-71-F-Ob \\
R-72-F-Ob & R-73-F-Ob \\
R-74-F-Ob & R-75-F-Ob \\
R-76-F-Ob & R-77-F-Ob \\
R-78-F-Ob & R-79-F-Ob \\
R-80-F-Ob & R-81-F-Ob \\
R-82-F-Ob & R-83-F-Ob \\
R-84-F-Ob & R-85-F-Ob \\
R-86-F-Ob & R-87-F-Ob \\
R-88-F-Ob & R-89-F-Ob \\
R-90-F-Ob & R-91-F-Ob \\
R-92-F-Ob & R-93-F-Ob \\
R-94-F-Ob & R-95-F-Ob \\
R-96-F-Ob & R-97-F-Ob \\
R-98-F-Ob & R-99-F-Ob \\
R-100-F-Ob & R-101-F-Ob \\
R-102-F-Ob & R-103-F-Ob
\end{tabular} \\
\hline
\ref{specificaFiltriRicercaDocumento} & \begin{tabular}[t]{@{}l@{\hspace{0.5em}}l}
R-31-F-Ob & R-35-F-Ob \\
R-36-F-Ob & R-37-F-Ob \\
R-42-F-Ob & R-43-F-Ob \\
R-44-F-Ob & R-45-F-Ob \\
R-46-F-Ob & R-47-F-Ob \\
R-48-F-Ob & R-49-F-Ob \\
R-50-F-Ob & R-51-F-Ob \\
R-52-F-Ob & R-53-F-Ob \\
R-54-F-Ob & R-55-F-Ob \\
R-56-F-Ob & R-57-F-Ob \\
R-58-F-Ob & R-59-F-Ob \\
R-60-F-Ob & R-61-F-Ob \\
R-62-F-Ob & R-63-F-Ob \\
R-64-F-Ob & R-65-F-Ob \\
R-66-F-Ob & R-67-F-Ob \\
R-68-F-Ob & R-69-F-Ob \\
R-70-F-Ob & R-71-F-Ob \\
R-72-F-Ob & R-73-F-Ob \\
R-74-F-Ob & R-75-F-Ob \\
R-76-F-Ob & R-77-F-Ob \\
R-78-F-Ob & R-79-F-Ob \\
R-80-F-Ob & R-81-F-Ob \\
R-82-F-Ob & R-83-F-Ob \\
R-84-F-Ob & R-85-F-Ob \\
R-86-F-Ob & R-87-F-Ob \\
R-88-F-Ob & R-89-F-Ob \\
R-90-F-Ob & R-91-F-Ob \\
R-92-F-Ob & R-93-F-Ob \\
R-94-F-Ob & R-95-F-Ob \\
R-96-F-Ob & R-97-F-Ob \\
R-98-F-Ob & R-99-F-Ob \\
R-100-F-Ob & R-101-F-Ob \\
R-102-F-Ob & R-103-F-Ob
\end{tabular} \\
\hline
\ref{specificaFiltriComuni} & \begin{tabular}[t]{@{}l@{\hspace{0.5em}}l}
R-35-F-Ob & R-42-F-Ob \\
R-45-F-Ob & R-46-F-Ob \\
R-47-F-Ob & R-48-F-Ob \\
R-49-F-Ob & R-50-F-Ob \\
R-51-F-Ob & R-52-F-Ob \\
R-53-F-Ob & R-54-F-Ob \\
R-55-F-Ob & R-56-F-Ob \\
R-57-F-Ob & R-58-F-Ob \\
R-59-F-Ob & R-60-F-Ob \\
R-61-F-Ob & R-62-F-Ob \\
R-63-F-Ob & R-64-F-Ob \\
R-65-F-Ob & R-66-F-Ob \\
R-67-F-Ob & R-68-F-Ob \\
R-69-F-Ob & R-70-F-Ob
\end{tabular} \\
\hline
\ref{addFiltriTipoDocumento} & \begin{tabular}[t]{@{}l@{\hspace{0.5em}}l}
R-36-F-Ob & R-41-F-Ob \\
R-43-F-Ob & R-71-F-Ob \\
R-72-F-Ob & R-73-F-Ob \\
R-74-F-Ob & R-75-F-Ob \\
R-76-F-Ob & R-77-F-Ob \\
R-78-F-Ob & R-79-F-Ob \\
R-80-F-Ob & R-81-F-Ob \\
R-82-F-Ob & R-83-F-Ob \\
R-84-F-Ob & R-85-F-Ob \\
R-86-F-Ob & R-87-F-Ob \\
R-88-F-Ob & R-89-F-Ob \\
R-90-F-Ob & R-91-F-Ob \\
R-92-F-Ob & R-93-F-Ob \\
R-94-F-Ob & R-95-F-Ob \\
R-96-F-Ob & R-97-F-Ob \\
R-98-F-Ob & R-99-F-Ob \\
R-100-F-Ob & R-101-F-Ob
\end{tabular} \\
\hline
\ref{specificaFiltriDI-DAI} & \begin{tabular}[t]{@{}l@{\hspace{0.5em}}l}
R-71-F-Ob & R-72-F-Ob \\
R-73-F-Ob & R-74-F-Ob \\
R-75-F-Ob & R-76-F-Ob \\
R-77-F-Ob & R-78-F-Ob \\
R-79-F-Ob & R-80-F-Ob \\
R-81-F-Ob & R-82-F-Ob \\
R-83-F-Ob & R-84-F-Ob \\
R-85-F-Ob & R-86-F-Ob
\end{tabular} \\
\hline
\ref{addCustomMetadata} & R-37-F-Ob, R-44-F-Ob, R-102-F-Ob, R-103-F-Ob \\
\hline
\ref{visualizzaListaCampiComuni} & R-45-F-Ob \\
\hline
\ref{addChiaveDescrittiva} & R-46-F-Ob, R-47-F-Ob \\
\hline
\ref{addClassificazione} & R-48-F-Ob, R-49-F-Ob \\
\hline
\ref{addTempoConserva} & R-50-F-Ob, R-51-F-Ob, R-52-F-Ob \\
\hline
\ref{addNote} & R-53-F-Ob, R-54-F-Ob \\
\hline
\ref{addTipoDocumento} & R-55-F-Ob, R-56-F-Ob \\
\hline
\ref{addSoggetti} & R-57-F-Ob, R-58-F-Ob \\
\hline
\ref{specificaRuoloSoggetto} & R-59-F-Ob, R-60-F-Ob \\
\hline
\ref{specificaTipoSoggetto} & R-61-F-Ob, R-62-F-Ob \\
\hline
\ref{addDettagliSoggetto} & R-63-F-Ob \\
\hline
\ref{addDettagliPAI} & R-64-F-Ob \\
\hline
\ref{addDettagliPAE} & R-65-F-Ob \\
\hline
\ref{addDettagliAS} & R-66-F-Ob \\
\hline
\ref{addDettagliPG} & R-67-F-Ob \\
\hline
\ref{addDettagliPF} & R-68-F-Ob \\
\hline
\ref{addDettagliRUP} & R-69-F-Ob \\
\hline
\ref{addDettagliSW} & R-70-F-Ob \\
\hline
\ref{selezionaCampiDI-DAI} & R-71-F-Ob \\
\hline
\ref{visualizzaListaCampiDIDAI} & R-72-F-Ob \\
\hline
\ref{addDatiRegistrazione} & R-73-F-Ob, R-74-F-Ob \\
\hline
\ref{addTipologiaDocumentale} & R-75-F-Ob \\
\hline
\ref{addModalitaFormazione} & R-76-F-Ob \\
\hline
\ref{addRiservato} & R-77-F-Ob \\
\hline
\ref{addIdentificativoFormato} & R-78-F-Ob, R-79-F-Ob \\
\hline
\ref{addDatiVerifica} & R-80-F-Ob, R-81-F-Ob \\
\hline
\ref{addNomeDocumento} & R-82-F-Ob \\
\hline
\ref{addVersioneDocumento} & R-83-F-Ob \\
\hline
\ref{addIdentificativoDocumentoPrimario} & R-84-F-Ob \\
\hline
\ref{addTracciatureModificheDocumento} & R-85-F-Ob, R-86-F-Ob \\
\hline
\ref{specificaFiltriAggregazione} & \begin{tabular}[t]{@{}l@{\hspace{0.5em}}l}
R-87-F-Ob & R-88-F-Ob \\
R-89-F-Ob & R-90-F-Ob \\
R-91-F-Ob & R-92-F-Ob \\
R-93-F-Ob & R-94-F-Ob \\
R-95-F-Ob & R-96-F-Ob \\
R-97-F-Ob & R-98-F-Ob \\
R-99-F-Ob & R-100-F-Ob \\
R-101-F-Ob &
\end{tabular} \\
\hline
\ref{visualizzaListaCampiAggregazioneDocumentale} & R-88-F-Ob \\
\hline
\ref{addTipoAggregazione} & R-89-F-Ob, R-90-F-Ob \\
\hline
\ref{addIdAggregazione} & R-91-F-Ob \\
\hline
\ref{addTipologiaFascicolo} & R-92-F-Ob \\
\hline
\ref{addIdAggregazionePrimario} & R-93-F-Ob \\
\hline
\ref{addDataApertura} & R-94-F-Ob \\
\hline
\ref{addDataChiusura} & R-95-F-Ob \\
\hline
\ref{addProcedimentoAmministrativo} & R-96-F-Ob, R-97-F-Ob \\
\hline
\ref{addFasiProcedimentoAmministrativo} & R-98-F-Ob \\
\hline
\ref{addAssegnazione} & R-99-F-Ob, R-100-F-Ob \\
\hline
\ref{addProgressivoAggregazione} & R-101-F-Ob \\
\hline
\ref{addFiltroMetadatoCustom} & R-103-F-Ob \\
\hline
\ref{visualizzaRisultati} & R-104-F-Ob, R-105-F-Ob \\
\hline
\ref{infoRisultatiRicerca} & R-105-F-Ob \\
\hline
\ref{nessunRisultato} & R-106-F-Ob \\
\hline
\ref{formatoNonCorretto} & R-107-F-Ob \\
\hline
\ref{compilaValoreFiltro} & \begin{tabular}[t]{@{}l@{\hspace{0.5em}}l}
R-22-F-Ob & R-23-F-Ob \\
R-47-F-Ob & R-49-F-Ob \\
R-51-F-Ob & R-54-F-Ob \\
R-56-F-Ob & R-64-F-Ob \\
R-65-F-Ob & R-66-F-Ob \\
R-67-F-Ob & R-68-F-Ob \\
R-69-F-Ob & R-70-F-Ob \\
R-74-F-Ob & R-75-F-Ob \\
R-76-F-Ob & R-77-F-Ob \\
R-79-F-Ob & R-81-F-Ob \\
R-82-F-Ob & R-83-F-Ob \\
R-84-F-Ob & R-86-F-Ob \\
R-90-F-Ob & R-91-F-Ob \\
R-92-F-Ob & R-93-F-Ob \\
R-94-F-Ob & R-95-F-Ob \\
R-97-F-Ob & R-98-F-Ob \\
R-100-F-Ob & R-101-F-Ob \\
R-103-F-Ob & R-108-F-Ob \\
R-109-F-Ob &
\end{tabular} \\
\hline


\ref{formatoNonCorretto} & R-109-F-Ob \\ 
\hline
\ref{salvaDocumento} & R-110-F-Ob, R-112-F-Ob\\ 
\hline
\ref{salvaPiuDocumenti} & R-111-F-Ob, R-112-F-Ob\\ 
\hline
\ref{salvataggioFallito} & R-112-F-Ob \\ 
\hline
\ref{stampaSingoloDoc} & R-113-F-Ob, R-115-F-Ob, R-116-F-Ob  \\ 
\hline
\ref{stampaInsiemeDoc} & R-114-F-Ob, R-115-F-Ob, R-116-F-Ob \\ 
\hline
\ref{stampaFallita} & R-115-F-Ob \\ 
\hline
\ref{stampaNonDisponibile} & R-116-F-Ob \\ 
\hline
\ref{verificaIntegritaDIPCompleto} & R-117-F-Ob, R-118-F-Ob \\ 
\hline
\ref{verificaIntegritaClasseDocumentale} & R-119-F-Ob, R-120-F-Ob \\ 
\hline
\ref{verificaIntegritaProcesso} & R-121-F-Ob,R-122-F-Ob \\ 
\hline
\ref{verificaIntegritaDocumento} & R-123-F-Ob, R-124-F-Ob \\ 
\hline


\ref{visualizzazioneReportIntegritaDIPCompleto} & R-125-F-Ob \\
\hline
\ref{visualizzazioneNumeroClassiVerificate} & R-126-F-Ob \\
\hline
\ref{visualizzazioneNumeroClassiIntegre} & R-127-F-Ob \\
\hline
\ref{visualizzazioneNumeroClassiCorrotte} & R-128-F-Ob \\
\hline
\ref{visualizzazioneListaClassiCorrotte} & R-129-F-Ob \\
\hline
\ref{visualizzazioneDataEOraVerificaDIP} & R-130-F-Ob \\
\hline
\ref{visualizzazioneReportIntegritaClasseDocumentale} & R-131-F-Ob \\
\hline
\ref{visualizzazioneNumeroProcessiVerificati} & R-132-F-Ob \\
\hline
\ref{visualizzazioneNumeroProcessiIntegri} & R-133-F-Ob \\
\hline
\ref{visualizzazioneNumeroProcessiCorrotti} & R-134-F-Ob \\
\hline
\ref{visualizzazioneListaProcessiCorrotti} & R-135-F-Ob \\
\hline
\ref{visualizzazioneDataEOraVerificaClasse} & R-136-F-Ob \\
\hline
\ref{visualizzazioneReportIntegritaProcesso} & R-137-F-Ob \\
\hline
\ref{visualizzazioneNumeroDocumentiVerificati} & R-138-F-Ob \\
\hline
\ref{visualizzazioneNumeroDocumentiIntegri} & R-139-F-Ob \\
\hline
\ref{visualizzazioneNumeroDocumentiCorrotti} & R-140-F-Ob \\
\hline
\ref{visualizzazioneListaDocumentiCorrotti} & R-141-F-Ob \\
\hline
\ref{visualizzazioneDataEOraVerificaProcesso} & R-142-F-Ob \\
\hline
\ref{visualizzazioneReportIntegritaDocumento} & R-143-F-Ob \\
\hline
\ref{visualizzazioneNomeDocumento} & R-144-F-Ob \\
\hline
\ref{visualizzazioneStatoVerificaDocumento} & R-145-F-Ob \\
\hline
\ref{visualizzazioneDataEOraVerificaDocumento} & R-146-F-Ob \\
\hline
\ref{visualizzazioneDettagliErroreDocumento} & R-147-F-Ob \\
\hline
\ref{convertiReportVerificaPDF} & R-148-F-Ob \\
\hline
\ref{erroreGenerazionePDF} & R-149-F-Ob \\
\hline
\ref{scaricaFile} & R-150-F-Ob, R-151-F-Ob, R-152-F-Ob \\
\hline
\ref{erroreScaricamentoFile} & R-152-F-Ob, R-153-F-Ob \\
\hline
\ref{visualizzaInfoAiP} & R-154-F-Ob \\
\hline
\ref{visualizzaClasseDocumentaleAiP} & R-155-F-Ob \\
\hline
\ref{visualizzaUUIDAiP} & R-156-F-Ob \\
\hline



\ref{visualizzaInfoProcessoConservazioneAiP} & R-157-F-Ob \\
\hline
\ref{visualizzaDataInizioProcessoSessione} & R-158-F-Ob, R-169-F-Ob, R-180-F-Ob \\
\hline
\ref{visualizzaDataFineProcessoSessione} & R-159-F-Ob, R-170-F-Ob, R-181-F-Ob \\
\hline
\ref{erroreDataFineProcessoSessione} & R-160-F-Ob, R-171-F-Ob, R-182-F-Ob \\
\hline
\ref{visualizzaUUIDUtenteAttivatore} & R-161-F-Ob, R-172-F-Ob, R-183-F-Ob \\
\hline
\ref{visualizzaUUIDUtenteTerminatore} & R-162-F-Ob, R-173-F-Ob, R-184-F-Ob \\
\hline
\ref{erroreUUIDUtenteTerminatore} & R-163-F-Ob, R-174-F-Ob, R-185-F-Ob \\
\hline
\ref{visualizzaNomeCanaleAttivazione} & R-164-F-Ob, R-175-F-Ob, R-186-F-Ob \\
\hline
\ref{visualizzaNomeCanaleTerminazione} & R-165-F-Ob, R-176-F-Ob, R-187-F-Ob \\
\hline
\ref{erroreNomeCanaleTerminazione} & R-166-F-Ob, R-177-F-Ob, R-188-F-Ob \\
\hline
\ref{visualizzaStatoProcessoSessione} & R-167-F-Ob, R-178-F-Ob, R-189-F-Ob \\
\hline
\ref{visualizzaInfoSessioneVersamento} & R-168-F-Ob, R-169-F-Ob, R-170-F-Ob, R-171-F-Ob, R-172-F-Ob, R-173-F-Ob, R-174-F-Ob, R-175-F-Ob, R-176-F-Ob, R-177-F-Ob, R-178-F-Ob \\
\hline
\ref{visualizzaInfoSessioneConservazione} & R-179-F-Ob, R-180-F-Ob, R-181-F-Ob, R-182-F-Ob, R-183-F-Ob, R-184-F-Ob, R-185-F-Ob, R-186-F-Ob, R-187-F-Ob, R-188-F-Ob, R-189-F-Ob \\
\hline
\ref{visualizzaDescrizioneDocumento} & R-190-F-Ob \\
\hline
\ref{visualizzaListaSoggettiCoinvolti} & R-191-F-Ob \\
\hline
\ref{visualizzaInfoSoggettoCoinvolto} & R-256-F-Ob \\
\hline
\ref{visualizzaNomeSoggetto} & R-194-F-Ob, R-204-F-Ob \\
\hline
\ref{visualizzaCognomeSoggetto} & R-195-F-Ob, R-203-F-Ob \\
\hline
\ref{visualizzaCodiceFiscaleSoggetto} & R-196-F-Ob, R-200-F-Ob, R-205-F-Ob \\
\hline
\ref{visualizzaIndirizziDigitaliSoggetto} & R-197-F-Ob, R-202-F-Ob, R-208-F-Ob, R-212-F-Ob, R-215-F-Ob \\
\hline
\ref{visualizzaDenominazioneOrganizzazioneSoggetto} & R-198-F-Ob, R-206-F-Ob \\
\hline
\ref{visualizzaPartitaIVA} & R-199-F-Ob \\
\hline
\ref{visualizzaDenominazioneUfficioSoggetto} & R-201-F-Ob, R-207-F-Ob, R-214-F-Ob \\
\hline
\ref{visualizzaInfoAS} & R-203-F-Ob, R-204-F-Ob, R-205-F-Ob, R-206-F-Ob, R-207-F-Ob, R-208-F-Ob \\
\hline
\ref{visualizzaInfoPAI} & R-212-F-Ob \\
\hline
\ref{visualizzaDenominazioneAmministrazioneCodiceIPA} & R-209-F-Ob \\
\hline
\ref{visualizzaDenominazioneAmministrazioneAOOCodiceIPAOOO} & R-210-F-Ob \\
\hline
\ref{visualizzaDenominazioneAmministrazioneUORCodiceIPAUOR} & R-211-F-Ob \\
\hline
\ref{visualizzaInfoPAE} & R-214-F-Ob, R-215-F-Ob \\
\hline
\ref{visualizzaDenominazioneAmministrazioneSoggetto} & R-213-F-Ob \\
\hline
\ref{visualizzaDenominazioneSistemaSoggetto} & R-216-F-Ob \\
\hline
\ref{visualizzaRuoloSoggetto} & R-143-F-Ob \\
\hline
\ref{visualizzaTipoSoggetto} & R-144-F-Ob \\
\hline
\ref{visualizzaIndiceClassificazioneDocumento} & R-217-F-Ob \\
\hline
\ref{visualizzaDescrizioneIndiceClassificazioneDocumento} & R-218-F-Ob \\
\hline
\ref{visualizzaURIPianoClassificazioneDocumento} & R-219-F-Ob \\
\hline
\ref{visualizzaTempoConservazioneEffettivoDocumento} & R-220-F-Ob \\
\hline
\ref{erroreTempoConservazioneEffettivoDocumento} & R-221-F-Ob \\
\hline
\ref{visualizzaNoteDocumento} & R-222-F-Ob \\
\hline
\ref{erroreNoteDocumento} & R-223-F-Ob \\
\hline
\ref{visualizzaTipologiaFlussoDocumento} & R-224-F-Ob \\
\hline
\ref{visualizzaTipoRegistroDocumento} & R-225-F-Ob \\
\hline
\ref{visualizzaDataRegistrazioneDocumento} & R-226-F-Ob \\
\hline
\ref{visualizzaNumeroDocumento} & R-227-F-Ob \\
\hline
\ref{visualizzaCodiceIdentificativoRegistroAppartenenzaDocumento} & R-228-F-Ob \\
\hline
\ref{visualizzaTipologiaDocumentaleDocumento} & R-229-F-Ob \\
\hline
\ref{visualizzaModalitaFormazioneDocumento} & R-230-F-Ob \\
\hline
\ref{visualizzaStatoRiservatezzaDocumento} & R-231-F-Ob \\
\hline
\ref{visualizzaTipoFormatoDocumento} & R-232-F-Ob \\
\hline
\ref{visualizzaNomeProdottoSoftware} & R-233-F-Ob \\
\hline
\ref{visualizzaVersioneProdottoSoftware} & R-234-F-Ob \\
\hline
\ref{visualizzaProduttoreSoftware} & R-235-F-Ob \\
\hline
\ref{visualizzaFirmaDigitaleDocumento} & R-236-F-Ob \\
\hline
\ref{visualizzaSigilloElettronicoDocumento} & R-237-F-Ob \\
\hline
\ref{visualizzaMarcaturaTemporaleDocumento} & R-238-F-Ob \\
\hline
\ref{visualizzaConformitaCopieImmagineDocumento} & R-239-F-Ob \\
\hline
\ref{visualizzaVersioneDocumento} & R-240-F-Ob \\
\hline
\ref{visualizzaNomeDocumento} & R-241-F-Ob \\
\hline
\ref{visualizzaNumeroAllegatiDocumento} & R-242-F-Ob \\
\hline
\ref{visualizzaIdentificativoAllegato} & R-243-F-Ob \\
\hline
\ref{erroreIdentificativoAllegato} & R-244-F-Ob \\
\hline
\ref{visualizzaDescrizioneAllegato} & R-245-F-Ob \\
\hline
\ref{erroreDescrizioneAllegato} & R-246-F-Ob \\
\hline
\ref{allegatiNonPresenti} & R-247-F-Ob \\
\hline
\ref{visualizzaTipoModificaDocumento} & R-248-F-Ob \\
\hline
\ref{visualizzaSoggettoAutoreModifica} & R-249-F-Ob \\
\hline
\ref{visualizzaDataOraModifica} & R-250-F-Ob \\
\hline
\ref{visualizzaIdentificativoVersionePrecedenteDocumento} & R-251-F-Ob \\
\hline
\ref{visualizzaTipoAggregazione} & R-252-F-Ob \\
\hline
\ref{visualizzaIdentificativoAggregazione} & R-253-F-Ob \\
\hline
\ref{visualizzaTipologiaFascicolo} & R-254-F-Ob \\
\hline
\ref{visualizzaAssegnazioneAggregazione} & R-256-F-Ob \\
\hline
\ref{visualizzaTipoAssegnazioneAggregazione} & R-255-F-Ob \\
\hline
\ref{visualizzaDataOraInizioAssegnazioneAggregazione} & R-257-F-Ob \\
\hline
\ref{visualizzaDataOraFineAssegnazioneAggregazione} & R-258-F-Ob \\
\hline
\ref{visualizzaDataAperturaAggregazione} & R-259-F-Ob \\
\hline
\ref{visualizzaDataChiusuraAggregazione} & R-260-F-Ob \\
\hline
\ref{visualizzaProgressivoAggregazione} & R-261-F-Ob \\
\hline
\ref{visualizzaIndiceProcedimentoAmministrativo} & R-262-F-Ob \\
\hline
\ref{visualizzaDenominazioneProcedimentoAmministrativo} & R-263-F-Ob \\
\hline
\ref{visualizzaCatalogoProcedimentiAmministrativi} & R-264-F-Ob \\
\hline
\ref{visualizzaFasiProcedimentoAmministrativo} & R-265-F-Ob \\
\hline
\ref{visualizzaTipoFaseProcedimentoAmministrativo} & R-266-F-Ob \\
\hline
\ref{visualizzaDataOraInizioFase} & R-267-F-Ob \\
\hline
\ref{visualizzaDataOraFineFase} & R-268-F-Ob \\
\hline
\ref{visualizzaIndiceDocumentiAggregazione} & R-269-F-Ob \\
\hline
\ref{visualizzaTipoDocumentiAggregazione} & R-270-F-Ob \\
\hline
\ref{visualizzaIdentificativoDocumentoAggregazione} & R-271-F-Ob \\
\hline
\ref{visualizzaPosizioneFisicaAggregazione} & R-272-F-Ob \\
\hline
\ref{visualizzaIdentificativoAggregazionePrimaria} & R-273-F-Ob \\
\hline
\ref{visualizzaTempoConservazioneAggregazione} & R-274-F-Ob \\
\hline
\ref{visualizzaNomeMetadatoCustom} & R-275-F-Ob \\
\hline
\ref{visualizzaValoreMetadatoCustom} & R-276-F-Ob \\
\hline
\ref{metadatiCustomAssenti} & R-277-F-Ob \\
\hline
\href{https://www.math.unipd.it/~tullio/IS-1/2025/Progetto/C3.pdf}{\ul{Capitolato di Progetto}\setulcolor{black}} & \begin{tabular}[t]{@{}l@{\hspace{0.5em}}l} 
R-1-Q-Ob & R-2-Q-Ob \\ 
R-3-Q-Ob & R-5-Q-Ob \\ 
R-8-Q-De & R-1-V-Ob \\  
R-2-V-Ob & R-3-V-Ob \\
R-4-V-Ob & R-5-V-Ob \\
R-6-V-Ob & R-7-V-Ob \\
R-8-V-Ob & R-9-V-Ob \\
R-10-V-Ob &
\end{tabular} \\
\hline
\href{https://cdn.jsdelivr.net/gh/7-zpus/Docs@main/2_RTB/Verbali/Verbali%20Esterni/2025-12-23-VerbaleEsterno.pdf}{\ul{Verbale Esterno 2025-12-23}\setulcolor{black}} & R-4-Q-Ob \\
\hline
\href{https://cdn.jsdelivr.net/gh/7-zpus/Docs@main/2_RTB/NormeDiProgetto.pdf}{\ul{Norme di Progetto}\setulcolor{black}\ped{v1.0}} & R-6-Q-Ob \\
\hline
\href{https://cdn.jsdelivr.net/gh/7-zpus/Docs@main/2_RTB/Verbali/Verbali%20Esterni/2025-11-27-VerbaleEsterno.pdf}{\ul{Verbale Esterno 2025-11-27}\setulcolor{black}} & R-7-Q-Ob \\
\hline
\href{https://cdn.jsdelivr.net/gh/7-zpus/Docs@main/2_RTB/Verbali/Verbali%20-Esterni/2025-11-13-VerbaleEsterno.pdf}{\ul{Verbale Esterno 2025-11-13}\setulcolor{black}} & R-11-V-Ob \\
\hline
\rowcolor{white}
\caption{Tracciamento Fonti - Requisiti}
\label{tab:trace-fonti-req}
\end{longtable}



\subsubsection{Tracciamento Requisiti - Fonti}

\rowcolors{2}{gray!15}{white}
\begin{longtable}{|p{4cm}|p{10cm}|}
\hline
\rowcolor{zpusgreen!30}
\textbf{Requisito} & \textbf{Fonti} \\
\hline
\endfirsthead
\rowcolor{zpusgreen!30}
\textbf{Requisito} & \textbf{Fonti} \\
\hline
\endhead
R-1-F-Ob & \ref{classiDocumentali} \\
\hline
R-2-F-Ob & \ref{elencoVuoto} \\
\hline
R-3-F-Ob & \ref{classeDocumentale} \\
\hline
R-4-F-Ob & \ref{nomeClasseDocumentale} \\
\hline
R-5-F-Ob & \ref{statoVerificaElemento} \\
\hline
R-6-F-Ob & \ref{marcaturaTemporaleElemento} \\
\hline
R-7-F-Ob & \ref{processiClasseDocumentale} \\
\hline
R-8-F-Ob & \ref{elencoVuoto} \\
\hline
R-9-F-Ob & \ref{processoClasseDocumentale} \\
\hline
R-10-F-Ob & \ref{idProcessoClasse} \\
\hline
R-11-F-Ob & \ref{documentiProcesso} \\
\hline
R-12-F-Ob & \ref{elencoVuoto} \\
\hline
R-13-F-Ob & \ref{documentoProcesso} \\
\hline
R-14-F-Ob & \ref{nomeDocumentoProcesso} \\
\hline
R-15-F-Ob & \ref{statoVerificaElemento} \\
\hline
R-16-F-Ob & \ref{marcaturaTemporaleElemento} \\
\hline
R-17-F-Ob & \ref{selezionaClasseDocumentale} \\
\hline
R-18-F-Ob & \ref{selezionaProcesso} \\
\hline
R-19-F-Ob & \ref{anteprimaDocumento} \\
\hline
R-20-F-Ob & \ref{formatoDocumentoNonSupportato} \\
\hline
R-21-F-Ob & \ref{ricercaDIP} \\
\hline
R-22-F-Ob & \begin{tabular}[t]{@{}l@{\hspace{0.5em}}l}
\ref{ricercaDIP} & \ref{ricercaClasse} \\
\ref{inserimentoNomeClasseDocumentale} & \ref{compilaValoreFiltro}
\end{tabular} \\
\hline
R-23-F-Ob & \begin{tabular}[t]{@{}l@{\hspace{0.5em}}l}
\ref{ricercaDIP} & \ref{ricercaProcesso} \\
\ref{inserimentoIdProcesso} & \ref{compilaValoreFiltro}
\end{tabular} \\
\hline
R-24-F-Op & \ref{ricercaDIP}, \ref{ricercaDIPSemantica} \\
\hline
R-25-F-Op & \ref{indicizzazioneSemantica} \\
\hline
R-26-F-Op & \ref{indicizzazioneSemantica} \\
\hline
R-27-F-Op & \ref{erroreIndicizzazioneSemantica} \\
\hline
R-28-F-Op & \ref{StatoIndicizzazione} \\
\hline
R-29-F-Ob & \ref{ricercaDIP} \\
\hline
R-30-F-Ob & \ref{campoNonValido} \\
\hline
R-31-F-Ob & \ref{ricercaDIP}, \ref{ricercaDIPConFiltri}, \ref{specificaFiltriRicercaDocumento} \\
\hline
R-32-F-Ob & \ref{ricercaDIP}, \ref{ricercaProcesso} \\
\hline
R-33-F-Ob & \ref{ricercaDIP}, \ref{ricercaClasse} \\
\hline
R-34-F-Ob & \ref{ricercaDIP}, \ref{ricercaDIPConFiltri} \\
\hline
R-35-F-Ob & \begin{tabular}[t]{@{}l@{\hspace{0.5em}}l}
\ref{ricercaDIP} & \ref{ricercaDIPConFiltri} \\
\ref{specificaFiltriRicercaDocumento} & \ref{specificaFiltriComuni}
\end{tabular} \\
\hline
R-36-F-Ob & \begin{tabular}[t]{@{}l@{\hspace{0.5em}}l}
\ref{ricercaDIP} & \ref{ricercaDIPConFiltri} \\
\ref{specificaFiltriRicercaDocumento} & \ref{addFiltriTipoDocumento}
\end{tabular} \\
\hline
R-37-F-Ob & \begin{tabular}[t]{@{}l@{\hspace{0.5em}}l}
\ref{ricercaDIP} & \ref{ricercaDIPConFiltri} \\
\ref{specificaFiltriRicercaDocumento} & \ref{addCustomMetadata}
\end{tabular} \\
\hline
R-38-F-Ob & \ref{ricercaDIP}, \ref{ricercaDIPConFiltri} \\
\hline
R-39-F-Ob & \ref{ricercaDIP}, \ref{ricercaDIPConFiltri} \\
\hline
R-40-F-Ob & \ref{ricercaDIP}, \ref{ricercaDIPConFiltri} \\
\hline
R-41-F-Ob & \begin{tabular}[t]{@{}l@{\hspace{0.5em}}l}
\ref{ricercaDIP} & \ref{ricercaDIPConFiltri} \\
\ref{specificaFiltriRicercaDocumento} & \ref{addFiltriTipoDocumento}
\end{tabular} \\
\hline
R-42-F-Ob & \begin{tabular}[t]{@{}l@{\hspace{0.5em}}l}
\ref{ricercaDIP} & \ref{ricercaDIPConFiltri} \\
\ref{specificaFiltriRicercaDocumento} & \ref{specificaFiltriComuni}
\end{tabular} \\
\hline
R-43-F-Ob & \begin{tabular}[t]{@{}l@{\hspace{0.5em}}l}
\ref{ricercaDIP} & \ref{ricercaDIPConFiltri} \\
\ref{specificaFiltriRicercaDocumento} & \ref{addFiltriTipoDocumento}
\end{tabular} \\
\hline
R-44-F-Ob & \begin{tabular}[t]{@{}l@{\hspace{0.5em}}l}
\ref{ricercaDIP} & \ref{ricercaDIPConFiltri} \\
\ref{specificaFiltriRicercaDocumento} & \ref{addCustomMetadata}
\end{tabular} \\
\hline
R-45-F-Ob & \begin{tabular}[t]{@{}l@{\hspace{0.5em}}l}
\ref{ricercaDIP} & \ref{ricercaDIPConFiltri} \\
\ref{specificaFiltriRicercaDocumento} & \ref{specificaFiltriComuni} \\
\ref{visualizzaListaCampiComuni} &
\end{tabular} \\
\hline
R-46-F-Ob & \begin{tabular}[t]{@{}l@{\hspace{0.5em}}l}
\ref{ricercaDIP} & \ref{ricercaDIPConFiltri} \\
\ref{specificaFiltriRicercaDocumento} & \ref{specificaFiltriComuni} \\
\ref{addChiaveDescrittiva} &
\end{tabular} \\
\hline
R-47-F-Ob & \begin{tabular}[t]{@{}l@{\hspace{0.5em}}l}
\ref{ricercaDIP} & \ref{ricercaDIPConFiltri} \\
\ref{specificaFiltriRicercaDocumento} & \ref{specificaFiltriComuni} \\
\ref{addChiaveDescrittiva} & \ref{compilaValoreFiltro}
\end{tabular} \\
\hline
R-48-F-Ob & \begin{tabular}[t]{@{}l@{\hspace{0.5em}}l}
\ref{ricercaDIP} & \ref{ricercaDIPConFiltri} \\
\ref{specificaFiltriRicercaDocumento} & \ref{specificaFiltriComuni} \\
\ref{addClassificazione} &
\end{tabular} \\
\hline
R-49-F-Ob & \begin{tabular}[t]{@{}l@{\hspace{0.5em}}l}
\ref{ricercaDIP} & \ref{ricercaDIPConFiltri} \\
\ref{specificaFiltriRicercaDocumento} & \ref{specificaFiltriComuni} \\
\ref{addClassificazione} & \ref{compilaValoreFiltro}
\end{tabular} \\
\hline
R-50-F-Ob & \begin{tabular}[t]{@{}l@{\hspace{0.5em}}l}
\ref{ricercaDIP} & \ref{ricercaDIPConFiltri} \\
\ref{specificaFiltriRicercaDocumento} & \ref{specificaFiltriComuni} \\
\ref{addTempoConserva} &
\end{tabular} \\
\hline
R-51-F-Ob & \begin{tabular}[t]{@{}l@{\hspace{0.5em}}l}
\ref{ricercaDIP} & \ref{ricercaDIPConFiltri} \\
\ref{specificaFiltriRicercaDocumento} & \ref{specificaFiltriComuni} \\
\ref{addTempoConserva} & \ref{compilaValoreFiltro}
\end{tabular} \\
\hline
R-52-F-Ob & \begin{tabular}[t]{@{}l@{\hspace{0.5em}}l}
\ref{ricercaDIP} & \ref{ricercaDIPConFiltri} \\
\ref{specificaFiltriRicercaDocumento} & \ref{specificaFiltriComuni} \\
\ref{addTempoConserva} &
\end{tabular} \\
\hline
R-53-F-Ob & \begin{tabular}[t]{@{}l@{\hspace{0.5em}}l}
\ref{ricercaDIP} & \ref{ricercaDIPConFiltri} \\
\ref{specificaFiltriRicercaDocumento} & \ref{specificaFiltriComuni} \\
\ref{addNote} &
\end{tabular} \\
\hline
R-54-F-Ob & \ref{addNote}, \ref{compilaValoreFiltro} \\
\hline
R-55-F-Ob & \ref{addTipoDocumento} \\
\hline
R-56-F-Ob & \ref{addTipoDocumento}, \ref{compilaValoreFiltro} \\
\hline
R-57-F-Ob & \begin{tabular}[t]{@{}l@{\hspace{0.5em}}l}
\ref{ricercaDIP} & \ref{ricercaDIPConFiltri} \\
\ref{specificaFiltriRicercaDocumento} & \ref{specificaFiltriComuni} \\
\ref{addSoggetti} &
\end{tabular} \\
\hline
R-58-F-Ob & \begin{tabular}[t]{@{}l@{\hspace{0.5em}}l}
\ref{ricercaDIP} & \ref{ricercaDIPConFiltri} \\
\ref{specificaFiltriRicercaDocumento} & \ref{specificaFiltriComuni} \\
\ref{addSoggetti} &
\end{tabular} \\
\hline
R-59-F-Ob & \begin{tabular}[t]{@{}l@{\hspace{0.5em}}l}
\ref{ricercaDIP} & \ref{ricercaDIPConFiltri} \\
\ref{specificaFiltriRicercaDocumento} & \ref{specificaFiltriComuni} \\
\ref{addSoggetti} & \ref{specificaRuoloSoggetto}
\end{tabular} \\
\hline
R-60-F-Ob & \begin{tabular}[t]{@{}l@{\hspace{0.5em}}l}
\ref{ricercaDIP} & \ref{ricercaDIPConFiltri} \\
\ref{specificaFiltriRicercaDocumento} & \ref{specificaFiltriComuni} \\
\ref{addSoggetti} & \ref{specificaRuoloSoggetto}
\end{tabular} \\
\hline
R-61-F-Ob & \begin{tabular}[t]{@{}l@{\hspace{0.5em}}l}
\ref{ricercaDIP} & \ref{ricercaDIPConFiltri} \\
\ref{specificaFiltriRicercaDocumento} & \ref{specificaFiltriComuni} \\
\ref{addSoggetti} & \ref{specificaTipoSoggetto}
\end{tabular} \\
\hline
R-62-F-Ob & \begin{tabular}[t]{@{}l@{\hspace{0.5em}}l}
\ref{ricercaDIP} & \ref{ricercaDIPConFiltri} \\
\ref{specificaFiltriRicercaDocumento} & \ref{specificaFiltriComuni} \\
\ref{addSoggetti} & \ref{specificaTipoSoggetto}
\end{tabular} \\
\hline
R-63-F-Ob & \begin{tabular}[t]{@{}l@{\hspace{0.5em}}l}
\ref{ricercaDIP} & \ref{ricercaDIPConFiltri} \\
\ref{specificaFiltriRicercaDocumento} & \ref{specificaFiltriComuni} \\
\ref{addSoggetti} & \ref{addDettagliSoggetto}
\end{tabular} \\
\hline
R-64-F-Ob & \begin{tabular}[t]{@{}l@{\hspace{0.5em}}l}
\ref{ricercaDIP} & \ref{ricercaDIPConFiltri} \\
\ref{specificaFiltriRicercaDocumento} & \ref{specificaFiltriComuni} \\
\ref{addSoggetti} & \ref{addDettagliSoggetto} \\
\ref{addDettagliPAI} & \ref{compilaValoreFiltro}
\end{tabular} \\
\hline
R-65-F-Ob & \begin{tabular}[t]{@{}l@{\hspace{0.5em}}l}
\ref{ricercaDIP} & \ref{ricercaDIPConFiltri} \\
\ref{specificaFiltriRicercaDocumento} & \ref{specificaFiltriComuni} \\
\ref{addSoggetti} & \ref{addDettagliSoggetto} \\
\ref{addDettagliPAE} & \ref{compilaValoreFiltro}
\end{tabular} \\
\hline
R-66-F-Ob & \begin{tabular}[t]{@{}l@{\hspace{0.5em}}l}
\ref{ricercaDIP} & \ref{ricercaDIPConFiltri} \\
\ref{specificaFiltriRicercaDocumento} & \ref{specificaFiltriComuni} \\
\ref{addSoggetti} & \ref{addDettagliSoggetto} \\
\ref{addDettagliAS} & \ref{compilaValoreFiltro}
\end{tabular} \\
\hline
R-67-F-Ob & \begin{tabular}[t]{@{}l@{\hspace{0.5em}}l}
\ref{ricercaDIP} & \ref{ricercaDIPConFiltri} \\
\ref{specificaFiltriRicercaDocumento} & \ref{specificaFiltriComuni} \\
\ref{addSoggetti} & \ref{addDettagliSoggetto} \\
\ref{addDettagliPG} & \ref{compilaValoreFiltro}
\end{tabular} \\
\hline
R-68-F-Ob & \begin{tabular}[t]{@{}l@{\hspace{0.5em}}l}
\ref{ricercaDIP} & \ref{ricercaDIPConFiltri} \\
\ref{specificaFiltriRicercaDocumento} & \ref{specificaFiltriComuni} \\
\ref{addSoggetti} & \ref{addDettagliSoggetto} \\
\ref{addDettagliPF} & \ref{compilaValoreFiltro}
\end{tabular} \\
\hline
R-69-F-Ob & \begin{tabular}[t]{@{}l@{\hspace{0.5em}}l}
\ref{ricercaDIP} & \ref{ricercaDIPConFiltri} \\
\ref{specificaFiltriRicercaDocumento} & \ref{specificaFiltriComuni} \\
\ref{addSoggetti} & \ref{addDettagliSoggetto} \\
\ref{addDettagliRUP} & \ref{compilaValoreFiltro}
\end{tabular} \\
\hline
R-70-F-Ob & \begin{tabular}[t]{@{}l@{\hspace{0.5em}}l}
\ref{ricercaDIP} & \ref{ricercaDIPConFiltri} \\
\ref{specificaFiltriRicercaDocumento} & \ref{specificaFiltriComuni} \\
\ref{addSoggetti} & \ref{addDettagliSoggetto} \\
\ref{addDettagliSW} & \ref{compilaValoreFiltro}
\end{tabular} \\
\hline
R-71-F-Ob & \begin{tabular}[t]{@{}l@{\hspace{0.5em}}l}
\ref{ricercaDIP} & \ref{ricercaDIPConFiltri} \\
\ref{specificaFiltriRicercaDocumento} & \ref{addFiltriTipoDocumento} \\
\ref{specificaFiltriDI-DAI} & \ref{selezionaCampiDI-DAI}
\end{tabular} \\
\hline
R-72-F-Ob & \begin{tabular}[t]{@{}l@{\hspace{0.5em}}l}
\ref{ricercaDIP} & \ref{ricercaDIPConFiltri} \\
\ref{specificaFiltriRicercaDocumento} & \ref{addFiltriTipoDocumento} \\
\ref{specificaFiltriDI-DAI} & \ref{visualizzaListaCampiDIDAI}
\end{tabular} \\
\hline
R-73-F-Ob & \begin{tabular}[t]{@{}l@{\hspace{0.5em}}l}
\ref{ricercaDIP} & \ref{ricercaDIPConFiltri} \\
\ref{specificaFiltriRicercaDocumento} & \ref{addFiltriTipoDocumento} \\
\ref{specificaFiltriDI-DAI} & \ref{addDatiRegistrazione}
\end{tabular} \\
\hline
R-74-F-Ob & \begin{tabular}[t]{@{}l@{\hspace{0.5em}}l}
\ref{ricercaDIP} & \ref{ricercaDIPConFiltri} \\
\ref{specificaFiltriRicercaDocumento} & \ref{addFiltriTipoDocumento} \\
\ref{specificaFiltriDI-DAI} & \ref{addDatiRegistrazione} \\
\ref{compilaValoreFiltro} &
\end{tabular} \\
\hline
R-75-F-Ob & \begin{tabular}[t]{@{}l@{\hspace{0.5em}}l}
\ref{ricercaDIP} & \ref{ricercaDIPConFiltri} \\
\ref{specificaFiltriRicercaDocumento} & \ref{addFiltriTipoDocumento} \\
\ref{specificaFiltriDI-DAI} & \ref{addTipologiaDocumentale} \\
\ref{compilaValoreFiltro} &
\end{tabular} \\
\hline
R-76-F-Ob & \begin{tabular}[t]{@{}l@{\hspace{0.5em}}l}
\ref{ricercaDIP} & \ref{ricercaDIPConFiltri} \\
\ref{specificaFiltriRicercaDocumento} & \ref{addFiltriTipoDocumento} \\
\ref{specificaFiltriDI-DAI} & \ref{addModalitaFormazione} \\
\ref{compilaValoreFiltro} &
\end{tabular} \\
\hline
R-77-F-Ob & \ref{addRiservato}, \ref{compilaValoreFiltro} \\
\hline
R-78-F-Ob & \begin{tabular}[t]{@{}l@{\hspace{0.5em}}l}
\ref{ricercaDIP} & \ref{ricercaDIPConFiltri} \\
\ref{specificaFiltriRicercaDocumento} & \ref{addFiltriTipoDocumento} \\
\ref{specificaFiltriDI-DAI} & \ref{addIdentificativoFormato}
\end{tabular} \\
\hline
R-79-F-Ob & \begin{tabular}[t]{@{}l@{\hspace{0.5em}}l}
\ref{ricercaDIP} & \ref{ricercaDIPConFiltri} \\
\ref{specificaFiltriRicercaDocumento} & \ref{addFiltriTipoDocumento} \\
\ref{specificaFiltriDI-DAI} & \ref{addIdentificativoFormato} \\
\ref{compilaValoreFiltro} &
\end{tabular} \\
\hline
R-80-F-Ob & \begin{tabular}[t]{@{}l@{\hspace{0.5em}}l}
\ref{ricercaDIP} & \ref{ricercaDIPConFiltri} \\
\ref{specificaFiltriRicercaDocumento} & \ref{addFiltriTipoDocumento} \\
\ref{specificaFiltriDI-DAI} & \ref{addDatiVerifica}
\end{tabular} \\
\hline
R-81-F-Ob & \begin{tabular}[t]{@{}l@{\hspace{0.5em}}l}
\ref{ricercaDIP} & \ref{ricercaDIPConFiltri} \\
\ref{specificaFiltriRicercaDocumento} & \ref{addFiltriTipoDocumento} \\
\ref{specificaFiltriDI-DAI} & \ref{addDatiVerifica} \\
\ref{compilaValoreFiltro} &
\end{tabular} \\
\hline
R-82-F-Ob & \begin{tabular}[t]{@{}l@{\hspace{0.5em}}l}
\ref{ricercaDIP} & \ref{ricercaDIPConFiltri} \\
\ref{specificaFiltriRicercaDocumento} & \ref{addFiltriTipoDocumento} \\
\ref{specificaFiltriDI-DAI} & \ref{addNomeDocumento} \\
\ref{compilaValoreFiltro} &
\end{tabular} \\
\hline
R-83-F-Ob & \begin{tabular}[t]{@{}l@{\hspace{0.5em}}l}
\ref{ricercaDIP} & \ref{ricercaDIPConFiltri} \\
\ref{specificaFiltriRicercaDocumento} & \ref{addFiltriTipoDocumento} \\
\ref{specificaFiltriDI-DAI} & \ref{addVersioneDocumento} \\
\ref{compilaValoreFiltro} &
\end{tabular} \\
\hline
R-84-F-Ob & \begin{tabular}[t]{@{}l@{\hspace{0.5em}}l}
\ref{ricercaDIP} & \ref{ricercaDIPConFiltri} \\
\ref{specificaFiltriRicercaDocumento} & \ref{addFiltriTipoDocumento} \\
\ref{specificaFiltriDI-DAI} & \ref{addIdentificativoDocumentoPrimario} \\
\ref{compilaValoreFiltro} &
\end{tabular} \\
\hline
R-85-F-Ob & \begin{tabular}[t]{@{}l@{\hspace{0.5em}}l}
\ref{ricercaDIP} & \ref{ricercaDIPConFiltri} \\
\ref{specificaFiltriRicercaDocumento} & \ref{addFiltriTipoDocumento} \\
\ref{specificaFiltriDI-DAI} & \ref{addTracciatureModificheDocumento}
\end{tabular} \\
\hline
R-86-F-Ob & \begin{tabular}[t]{@{}l@{\hspace{0.5em}}l}
\ref{ricercaDIP} & \ref{ricercaDIPConFiltri} \\
\ref{specificaFiltriRicercaDocumento} & \ref{addFiltriTipoDocumento} \\
\ref{specificaFiltriDI-DAI} & \ref{addTracciatureModificheDocumento} \\
\ref{compilaValoreFiltro} &
\end{tabular} \\
\hline
R-87-F-Ob & \begin{tabular}[t]{@{}l@{\hspace{0.5em}}l}
\ref{ricercaDIP} & \ref{ricercaDIPConFiltri} \\
\ref{specificaFiltriRicercaDocumento} & \ref{addFiltriTipoDocumento} \\
\ref{specificaFiltriAggregazione} &
\end{tabular} \\
\hline
R-88-F-Ob & \begin{tabular}[t]{@{}l@{\hspace{0.5em}}l}
\ref{ricercaDIP} & \ref{ricercaDIPConFiltri} \\
\ref{specificaFiltriRicercaDocumento} & \ref{addFiltriTipoDocumento} \\
\ref{specificaFiltriAggregazione} & \ref{visualizzaListaCampiAggregazioneDocumentale}
\end{tabular} \\
\hline
R-89-F-Ob & \begin{tabular}[t]{@{}l@{\hspace{0.5em}}l}
\ref{ricercaDIP} & \ref{ricercaDIPConFiltri} \\
\ref{specificaFiltriRicercaDocumento} & \ref{addFiltriTipoDocumento} \\
\ref{specificaFiltriAggregazione} & \ref{addTipoAggregazione}
\end{tabular} \\
\hline
R-90-F-Ob & \begin{tabular}[t]{@{}l@{\hspace{0.5em}}l}
\ref{ricercaDIP} & \ref{ricercaDIPConFiltri} \\
\ref{specificaFiltriRicercaDocumento} & \ref{addFiltriTipoDocumento} \\
\ref{specificaFiltriAggregazione} & \ref{addTipoAggregazione} \\
\ref{compilaValoreFiltro} &
\end{tabular} \\
\hline
R-91-F-Ob & \begin{tabular}[t]{@{}l@{\hspace{0.5em}}l}
\ref{ricercaDIP} & \ref{ricercaDIPConFiltri} \\
\ref{specificaFiltriRicercaDocumento} & \ref{addFiltriTipoDocumento} \\
\ref{specificaFiltriAggregazione} & \ref{addIdAggregazione} \\
\ref{compilaValoreFiltro} &
\end{tabular} \\
\hline
R-92-F-Ob & \begin{tabular}[t]{@{}l@{\hspace{0.5em}}l}
\ref{ricercaDIP} & \ref{ricercaDIPConFiltri} \\
\ref{specificaFiltriRicercaDocumento} & \ref{addFiltriTipoDocumento} \\
\ref{specificaFiltriAggregazione} & \ref{addTipologiaFascicolo} \\
\ref{compilaValoreFiltro} &
\end{tabular} \\
\hline
R-93-F-Ob & \begin{tabular}[t]{@{}l@{\hspace{0.5em}}l}
\ref{ricercaDIP} & \ref{ricercaDIPConFiltri} \\
\ref{specificaFiltriRicercaDocumento} & \ref{addFiltriTipoDocumento} \\
\ref{specificaFiltriAggregazione} & \ref{addIdAggregazionePrimario} \\
\ref{compilaValoreFiltro} &
\end{tabular} \\
\hline
R-94-F-Ob & \begin{tabular}[t]{@{}l@{\hspace{0.5em}}l}
\ref{ricercaDIP} & \ref{ricercaDIPConFiltri} \\
\ref{specificaFiltriRicercaDocumento} & \ref{addFiltriTipoDocumento} \\
\ref{specificaFiltriAggregazione} & \ref{addDataApertura} \\
\ref{compilaValoreFiltro} &
\end{tabular} \\
\hline
R-95-F-Ob & \begin{tabular}[t]{@{}l@{\hspace{0.5em}}l}
\ref{ricercaDIP} & \ref{ricercaDIPConFiltri} \\
\ref{specificaFiltriRicercaDocumento} & \ref{addFiltriTipoDocumento} \\
\ref{specificaFiltriAggregazione} & \ref{addDataChiusura} \\
\ref{compilaValoreFiltro} &
\end{tabular} \\
\hline
R-96-F-Ob & \begin{tabular}[t]{@{}l@{\hspace{0.5em}}l}
\ref{ricercaDIP} & \ref{ricercaDIPConFiltri} \\
\ref{specificaFiltriRicercaDocumento} & \ref{addFiltriTipoDocumento} \\
\ref{specificaFiltriAggregazione} & \ref{addProcedimentoAmministrativo}
\end{tabular} \\
\hline
R-97-F-Ob & \begin{tabular}[t]{@{}l@{\hspace{0.5em}}l}
\ref{ricercaDIP} & \ref{ricercaDIPConFiltri} \\
\ref{specificaFiltriRicercaDocumento} & \ref{addFiltriTipoDocumento} \\
\ref{specificaFiltriAggregazione} & \ref{addProcedimentoAmministrativo} \\
\ref{compilaValoreFiltro} &
\end{tabular} \\
\hline
R-98-F-Ob & \begin{tabular}[t]{@{}l@{\hspace{0.5em}}l}
\ref{ricercaDIP} & \ref{ricercaDIPConFiltri} \\
\ref{specificaFiltriRicercaDocumento} & \ref{addFiltriTipoDocumento} \\
\ref{specificaFiltriAggregazione} & \ref{addProcedimentoAmministrativo} \\
\ref{addFasiProcedimentoAmministrativo} & \ref{compilaValoreFiltro}
\end{tabular} \\
\hline
R-99-F-Ob & \begin{tabular}[t]{@{}l@{\hspace{0.5em}}l}
\ref{ricercaDIP} & \ref{ricercaDIPConFiltri} \\
\ref{specificaFiltriRicercaDocumento} & \ref{addFiltriTipoDocumento} \\
\ref{specificaFiltriAggregazione} & \ref{addAssegnazione}
\end{tabular} \\
\hline
R-100-F-Ob & \begin{tabular}[t]{@{}l@{\hspace{0.5em}}l}
\ref{ricercaDIP} & \ref{ricercaDIPConFiltri} \\
\ref{specificaFiltriRicercaDocumento} & \ref{addFiltriTipoDocumento} \\
\ref{specificaFiltriAggregazione} & \ref{addAssegnazione} \\
\ref{compilaValoreFiltro} &
\end{tabular} \\
\hline
R-101-F-Ob & \begin{tabular}[t]{@{}l@{\hspace{0.5em}}l}
\ref{ricercaDIP} & \ref{ricercaDIPConFiltri} \\
\ref{specificaFiltriRicercaDocumento} & \ref{addFiltriTipoDocumento} \\
\ref{specificaFiltriAggregazione} & \ref{addProgressivoAggregazione} \\
\ref{compilaValoreFiltro} &
\end{tabular} \\
\hline
R-102-F-Ob & \begin{tabular}[t]{@{}l@{\hspace{0.5em}}l}
\ref{ricercaDIP} & \ref{ricercaDIPConFiltri} \\
\ref{specificaFiltriRicercaDocumento} & \ref{addCustomMetadata}
\end{tabular} \\
\hline
R-103-F-Ob & \begin{tabular}[t]{@{}l@{\hspace{0.5em}}l}
\ref{ricercaDIP} & \ref{ricercaDIPConFiltri} \\
\ref{specificaFiltriRicercaDocumento} & \ref{addCustomMetadata} \\
\ref{addFiltroMetadatoCustom} & \ref{compilaValoreFiltro}
\end{tabular} \\
\hline
R-104-F-Ob & \ref{ricercaDIP}, \ref{visualizzaRisultati} \\
\hline
R-105-F-Ob & \ref{ricercaDIP}, \ref{visualizzaRisultati}, \ref{infoRisultatiRicerca} \\
\hline
R-106-F-Ob & \ref{ricercaDIP}, \ref{nessunRisultato} \\
\hline
R-107-F-Ob & \ref{formatoNonCorretto} \\
\hline


R-108-F-Ob, & \ref{compilaValoreFiltro} \\ 
\hline
R-109-F-Ob & \ref{formatoNonCorretto}, \ref{compilaValoreFiltro} \\ 
\hline
R-110-F-Ob & \ref{salvaDocumento} \\ 
\hline
R-111-F-Ob & \ref{salvaPiuDocumenti} \\ 
\hline
R-112-F-Ob & \ref{salvataggioFallito}, \ref{salvaDocumento}, \ref{salvaPiuDocumenti} \\ 
\hline
R-113-F-Ob & \ref{stampaSingoloDoc} \\ 
\hline
R-114-F-Ob & \ref{stampaInsiemeDoc} \\ 
\hline
R-115-F-Ob & \ref{stampaFallita}, \ref{stampaSingoloDoc}, \ref{stampaInsiemeDoc} \\ 
\hline
R-116-F-Ob & \ref{stampaNonDisponibile}, \ref{stampaSingoloDoc}, \ref{stampaInsiemeDoc} \\ 
\hline
R-117-F-Ob & \ref{verificaIntegritaDIPCompleto} \\ 
\hline
R-118-F-Ob & \ref{verificaIntegritaDIPCompleto} \\ 
\hline
R-119-F-Ob& \ref{verificaIntegritaClasseDocumentale}\\ 
\hline
R-120-F-Ob & \ref{verificaIntegritaClasseDocumentale} \\ 
\hline
R-121-F-Ob & \ref{verificaIntegritaProcesso} \\ 
\hline
R-122-F-Ob & \ref{verificaIntegritaProcesso} \\ 
\hline
R-123-F-Ob & \ref{verificaIntegritaDocumento} \\ 
\hline
R-124-F-Ob & \ref{verificaIntegritaDocumento} \\ 
\hline




R-125-F-Ob & \ref{visualizzazioneReportIntegritaDIPCompleto} \\
\hline
R-126-F-Ob & \ref{visualizzazioneNumeroClassiVerificate} \\
\hline
R-127-F-Ob & \ref{visualizzazioneNumeroClassiIntegre} \\
\hline
R-128-F-Ob & \ref{visualizzazioneNumeroClassiCorrotte} \\
\hline
R-129-F-Ob & \ref{visualizzazioneListaClassiCorrotte} \\
\hline
R-130-F-Ob & \ref{visualizzazioneDataEOraVerificaDIP} \\
\hline
R-131-F-Ob & \ref{visualizzazioneReportIntegritaClasseDocumentale} \\
\hline
R-132-F-Ob & \ref{visualizzazioneNumeroProcessiVerificati} \\
\hline
R-133-F-Ob & \ref{visualizzazioneNumeroProcessiIntegri} \\
\hline
R-134-F-Ob & \ref{visualizzazioneNumeroProcessiCorrotti} \\
\hline
R-135-F-Ob & \ref{visualizzazioneListaProcessiCorrotti} \\
\hline
R-136-F-Ob & \ref{visualizzazioneDataEOraVerificaClasse} \\
\hline
R-137-F-Ob & \ref{visualizzazioneReportIntegritaProcesso} \\
\hline
R-138-F-Ob & \ref{visualizzazioneNumeroDocumentiVerificati} \\
\hline
R-139-F-Ob & \ref{visualizzazioneNumeroDocumentiIntegri} \\
\hline
R-140-F-Ob & \ref{visualizzazioneNumeroDocumentiCorrotti} \\
\hline
R-141-F-Ob & \ref{visualizzazioneListaDocumentiCorrotti} \\
\hline
R-142-F-Ob & \ref{visualizzazioneDataEOraVerificaProcesso} \\
\hline
R-143-F-Ob & \ref{visualizzazioneReportIntegritaDocumento} \\
\hline
R-144-F-Ob & \ref{visualizzazioneNomeDocumento} \\
\hline
R-145-F-Ob & \ref{visualizzazioneStatoVerificaDocumento} \\
\hline
R-146-F-Ob & \ref{visualizzazioneDataEOraVerificaDocumento} \\
\hline
R-147-F-Ob & \ref{visualizzazioneDettagliErroreDocumento} \\
\hline
R-148-F-Ob & \ref{convertiReportVerificaPDF} \\
\hline
R-149-F-Ob & \ref{erroreGenerazionePDF} \\
\hline
R-150-F-Ob & \ref{scaricaFile} \\
\hline
R-151-F-Ob & \ref{scaricaFile} \\
\hline
R-152-F-Ob & \ref{scaricaFile}, \ref{erroreScaricamentoFile} \\
\hline
R-153-F-Ob & \ref{erroreScaricamentoFile} \\
\hline
R-154-F-Ob & \ref{visualizzaInfoAiP} \\
\hline
R-155-F-Ob & \ref{visualizzaClasseDocumentaleAiP} \\
\hline
R-156-F-Ob & \ref{visualizzaUUIDAiP} \\
\hline




R-157-F-Ob & \ref{visualizzaInfoProcessoConservazioneAiP} \\
\hline
R-158-F-Ob & \ref{visualizzaDataInizioProcessoSessione} \\
\hline
R-159-F-Ob & \ref{visualizzaDataFineProcessoSessione} \\
\hline
R-160-F-Ob & \ref{erroreDataFineProcessoSessione} \\
\hline
R-161-F-Ob & \ref{visualizzaUUIDUtenteAttivatore} \\
\hline
R-162-F-Ob & \ref{visualizzaUUIDUtenteTerminatore} \\
\hline
R-163-F-Ob & \ref{erroreUUIDUtenteTerminatore} \\
\hline
R-164-F-Ob & \ref{visualizzaNomeCanaleAttivazione} \\
\hline
R-165-F-Ob & \ref{visualizzaNomeCanaleTerminazione} \\
\hline
R-166-F-Ob & \ref{erroreNomeCanaleTerminazione} \\
\hline
R-167-F-Ob & \ref{visualizzaStatoProcessoSessione} \\
\hline
R-168-F-Ob & \ref{visualizzaInfoSessioneVersamento} \\
\hline
R-169-F-Ob & \ref{visualizzaInfoSessioneVersamento}, \ref{visualizzaDataInizioProcessoSessione} \\
\hline
R-170-F-Ob & \ref{visualizzaInfoSessioneVersamento}, \ref{visualizzaDataFineProcessoSessione} \\
\hline
R-171-F-Ob & \ref{visualizzaInfoSessioneVersamento}, \ref{erroreDataFineProcessoSessione} \\
\hline
R-172-F-Ob & \ref{visualizzaInfoSessioneVersamento}, \ref{visualizzaUUIDUtenteAttivatore} \\
\hline
R-173-F-Ob & \ref{visualizzaInfoSessioneVersamento}, \ref{visualizzaUUIDUtenteTerminatore} \\
\hline
R-174-F-Ob & \ref{visualizzaInfoSessioneVersamento}, \ref{erroreUUIDUtenteTerminatore} \\
\hline
R-175-F-Ob & \ref{visualizzaInfoSessioneVersamento}, \ref{visualizzaNomeCanaleAttivazione} \\
\hline
R-176-F-Ob & \ref{visualizzaInfoSessioneVersamento}, \ref{visualizzaNomeCanaleTerminazione} \\
\hline
R-177-F-Ob & \ref{visualizzaInfoSessioneVersamento}, \ref{erroreNomeCanaleTerminazione} \\
\hline
R-178-F-Ob & \ref{visualizzaInfoSessioneVersamento}, \ref{visualizzaStatoProcessoSessione} \\
\hline
R-179-F-Ob & \ref{visualizzaInfoSessioneConservazione} \\
\hline
R-180-F-Ob & \ref{visualizzaInfoSessioneConservazione}, \ref{visualizzaDataInizioProcessoSessione} \\
\hline
R-181-F-Ob & \ref{visualizzaInfoSessioneConservazione}, \ref{visualizzaDataFineProcessoSessione} \\
\hline
R-182-F-Ob & \ref{visualizzaInfoSessioneConservazione}, \ref{erroreDataFineProcessoSessione} \\
\hline
R-183-F-Ob & \ref{visualizzaInfoSessioneConservazione}, \ref{visualizzaUUIDUtenteAttivatore} \\
\hline
R-184-F-Ob & \ref{visualizzaInfoSessioneConservazione}, \ref{visualizzaUUIDUtenteTerminatore} \\
\hline
R-185-F-Ob & \ref{visualizzaInfoSessioneConservazione}, \ref{erroreUUIDUtenteTerminatore} \\
\hline
R-186-F-Ob & \ref{visualizzaInfoSessioneConservazione}, \ref{visualizzaNomeCanaleAttivazione} \\
\hline
R-187-F-Ob & \ref{visualizzaInfoSessioneConservazione}, \ref{visualizzaNomeCanaleTerminazione} \\
\hline
R-188-F-Ob & \ref{visualizzaInfoSessioneConservazione}, \ref{erroreNomeCanaleTerminazione} \\
\hline
R-189-F-Ob & \ref{visualizzaInfoSessioneConservazione}, \ref{visualizzaStatoProcessoSessione} \\
\hline
R-190-F-Ob & \ref{visualizzaDescrizioneDocumento} \\
\hline
R-191-F-Ob & \ref{visualizzaListaSoggettiCoinvolti} \\
\hline
R-192-F-Ob & \ref{visualizzaRuoloSoggetto} \\
\hline
R-193-F-Ob & \ref{visualizzaTipoSoggetto} \\
\hline
R-194-F-Ob & \ref{visualizzaNomeSoggetto} \\
\hline
R-195-F-Ob & \ref{visualizzaCognomeSoggetto} \\
\hline
R-196-F-Ob & \ref{visualizzaCodiceFiscaleSoggetto} \\
\hline
R-197-F-Ob & \ref{visualizzaIndirizziDigitaliSoggetto} \\
\hline
R-198-F-Ob & \ref{visualizzaDenominazioneOrganizzazioneSoggetto} \\
\hline
R-199-F-Ob & \ref{visualizzaPartitaIVA} \\
\hline
R-200-F-Ob & \ref{visualizzaCodiceFiscaleSoggetto} \\
\hline
R-201-F-Ob & \ref{visualizzaDenominazioneUfficioSoggetto} \\
\hline
R-202-F-Ob & \ref{visualizzaIndirizziDigitaliSoggetto} \\
\hline
R-203-F-Ob & \ref{visualizzaInfoAS}, \ref{visualizzaCognomeSoggetto} \\
\hline
R-204-F-Ob & \ref{visualizzaInfoAS}, \ref{visualizzaNomeSoggetto} \\
\hline
R-205-F-Ob & \ref{visualizzaInfoAS}, \ref{visualizzaCodiceFiscaleSoggetto} \\
\hline
R-206-F-Ob & \ref{visualizzaInfoAS}, \ref{visualizzaDenominazioneOrganizzazioneSoggetto} \\
\hline
R-207-F-Ob & \ref{visualizzaInfoAS}, \ref{visualizzaDenominazioneUfficioSoggetto} \\
\hline
R-208-F-Ob & \ref{visualizzaInfoAS}, \ref{visualizzaIndirizziDigitaliSoggetto} \\
\hline
R-209-F-Ob & \ref{visualizzaDenominazioneAmministrazioneCodiceIPA} \\
\hline
R-210-F-Ob & \ref{visualizzaDenominazioneAmministrazioneAOOCodiceIPAOOO} \\
\hline
R-211-F-Ob & \ref{visualizzaDenominazioneAmministrazioneUORCodiceIPAUOR} \\
\hline
R-212-F-Ob & \ref{visualizzaInfoPAI}, \ref{visualizzaIndirizziDigitaliSoggetto} \\
\hline
R-213-F-Ob & \ref{visualizzaDenominazioneAmministrazioneSoggetto} \\
\hline
R-214-F-Ob & \ref{visualizzaInfoPAE}, \ref{visualizzaDenominazioneUfficioSoggetto} \\
\hline
R-215-F-Ob & \ref{visualizzaInfoPAE}, \ref{visualizzaIndirizziDigitaliSoggetto} \\
\hline
R-216-F-Ob & \ref{visualizzaDenominazioneSistemaSoggetto} \\
\hline
R-217-F-Ob & \ref{visualizzaIndiceClassificazioneDocumento} \\
\hline
R-218-F-Ob & \ref{visualizzaDescrizioneIndiceClassificazioneDocumento} \\
\hline
R-219-F-Ob & \ref{visualizzaURIPianoClassificazioneDocumento} \\
\hline
R-220-F-Ob & \ref{visualizzaTempoConservazioneEffettivoDocumento} \\
\hline
R-221-F-Ob & \ref{erroreTempoConservazioneEffettivoDocumento} \\
\hline
R-222-F-Ob & \ref{visualizzaNoteDocumento} \\
\hline
R-223-F-Ob & \ref{erroreNoteDocumento} \\
\hline
R-224-F-Ob & \ref{visualizzaTipologiaFlussoDocumento} \\
\hline
R-225-F-Ob & \ref{visualizzaTipoRegistroDocumento} \\
\hline
R-226-F-Ob & \ref{visualizzaDataRegistrazioneDocumento} \\
\hline
R-227-F-Ob & \ref{visualizzaNumeroDocumento} \\
\hline
R-228-F-Ob & \ref{visualizzaCodiceIdentificativoRegistroAppartenenzaDocumento} \\
\hline
R-229-F-Ob & \ref{visualizzaTipologiaDocumentaleDocumento} \\
\hline
R-230-F-Ob & \ref{visualizzaModalitaFormazioneDocumento} \\
\hline
R-231-F-Ob & \ref{visualizzaStatoRiservatezzaDocumento} \\
\hline
R-232-F-Ob & \ref{visualizzaTipoFormatoDocumento} \\
\hline
R-233-F-Ob & \ref{visualizzaNomeProdottoSoftware} \\
\hline
R-234-F-Ob & \ref{visualizzaVersioneProdottoSoftware} \\
\hline
R-235-F-Ob & \ref{visualizzaProduttoreSoftware} \\
\hline
R-236-F-Ob & \ref{visualizzaFirmaDigitaleDocumento} \\
\hline
R-237-F-Ob & \ref{visualizzaSigilloElettronicoDocumento} \\
\hline
R-238-F-Ob & \ref{visualizzaMarcaturaTemporaleDocumento} \\
\hline
R-239-F-Ob & \ref{visualizzaConformitaCopieImmagineDocumento} \\
\hline
R-240-F-Ob & \ref{visualizzaVersioneDocumento} \\
\hline
R-241-F-Ob & \ref{visualizzaNomeDocumento} \\
\hline
R-242-F-Ob & \ref{visualizzaNumeroAllegatiDocumento} \\
\hline
R-243-F-Ob & \ref{visualizzaIdentificativoAllegato} \\
\hline
R-244-F-Ob & \ref{erroreIdentificativoAllegato} \\
\hline
R-245-F-Ob & \ref{visualizzaDescrizioneAllegato} \\
\hline
R-246-F-Ob & \ref{erroreDescrizioneAllegato} \\
\hline
R-247-F-Ob & \ref{allegatiNonPresenti} \\
\hline
R-248-F-Ob & \ref{visualizzaTipoModificaDocumento} \\
\hline
R-249-F-Ob & \ref{visualizzaSoggettoAutoreModifica} \\
\hline
R-250-F-Ob & \ref{visualizzaDataOraModifica} \\
\hline
R-251-F-Ob & \ref{visualizzaIdentificativoVersionePrecedenteDocumento} \\
\hline
R-252-F-Ob & \ref{visualizzaTipoAggregazione} \\
\hline
R-253-F-Ob & \ref{visualizzaIdentificativoAggregazione} \\
\hline
R-254-F-Ob & \ref{visualizzaTipologiaFascicolo} \\
\hline
R-255-F-Ob & \ref{visualizzaTipoAssegnazioneAggregazione} \\
\hline
R-256-F-Ob & \ref{visualizzaAssegnazioneAggregazione}, \ref{visualizzaInfoSoggettoCoinvolto} \\
\hline
R-257-F-Ob & \ref{visualizzaDataOraInizioAssegnazioneAggregazione} \\
\hline
R-258-F-Ob & \ref{visualizzaDataOraFineAssegnazioneAggregazione} \\
\hline
R-259-F-Ob & \ref{visualizzaDataAperturaAggregazione} \\
\hline
R-260-F-Ob & \ref{visualizzaDataChiusuraAggregazione} \\
\hline
R-261-F-Ob & \ref{visualizzaProgressivoAggregazione} \\
\hline
R-262-F-Ob & \ref{visualizzaIndiceProcedimentoAmministrativo} \\
\hline
R-263-F-Ob & \ref{visualizzaDenominazioneProcedimentoAmministrativo} \\
\hline
R-264-F-Ob & \ref{visualizzaCatalogoProcedimentiAmministrativi} \\
\hline
R-265-F-Ob & \ref{visualizzaFasiProcedimentoAmministrativo} \\
\hline
R-266-F-Ob & \ref{visualizzaTipoFaseProcedimentoAmministrativo} \\
\hline
R-267-F-Ob & \ref{visualizzaDataOraInizioFase} \\
\hline
R-268-F-Ob & \ref{visualizzaDataOraFineFase} \\
\hline
R-269-F-Ob & \ref{visualizzaIndiceDocumentiAggregazione} \\
\hline
R-270-F-Ob & \ref{visualizzaTipoDocumentiAggregazione} \\
\hline
R-271-F-Ob & \ref{visualizzaIdentificativoDocumentoAggregazione} \\
\hline
R-272-F-Ob & \ref{visualizzaPosizioneFisicaAggregazione} \\
\hline
R-273-F-Ob & \ref{visualizzaIdentificativoAggregazionePrimaria} \\
\hline
R-274-F-Ob & \ref{visualizzaTempoConservazioneAggregazione} \\
\hline
R-275-F-Ob & \ref{visualizzaNomeMetadatoCustom} \\
\hline
R-276-F-Ob & \ref{visualizzaValoreMetadatoCustom} \\
\hline
R-277-F-Ob & \ref{metadatiCustomAssenti} \\
\hline
R-1-Q-Ob & \href{https://www.math.unipd.it/~tullio/IS-1/2025/Progetto/C3.pdf}{\ul{Capitolato di Progetto}\setulcolor{black}} \\
\hline
R-2-Q-Ob & \href{https://www.math.unipd.it/~tullio/IS-1/2025/Progetto/C3.pdf}{\ul{Capitolato di Progetto}\setulcolor{black}} \\
\hline
R-3-Q-Ob & \href{https://www.math.unipd.it/~tullio/IS-1/2025/Progetto/C3.pdf}{\ul{Capitolato di Progetto}\setulcolor{black}} \\
\hline
R-4-Q-De & \href{https://cdn.jsdelivr.net/gh/7-zpus/Docs@main/2_RTB/Verbali/Verbali%20Esterni/2025-12-23-VerbaleEsterno.pdf}{\ul{Verbale Esterno 2025-12-23}\setulcolor{black}} \\
\hline
R-5-Q-Ob & \href{https://www.math.unipd.it/~tullio/IS-1/2025/Progetto/C3.pdf}{\ul{Capitolato di Progetto}\setulcolor{black}} \\
\hline
R-6-Q-Ob & \href{https://cdn.jsdelivr.net/gh/7-zpus/Docs@main/2_RTB/NormeDiProgetto.pdf}{\ul{Norme di Progetto}\setulcolor{black}\ped{v1.0}} \\
\hline
R-7-Q-Ob & \href{https://cdn.jsdelivr.net/gh/7-zpus/Docs@main/2_RTB/Verbali/Verbali%20Esterni/2025-11-27-VerbaleEsterno.pdf}{\ul{Verbale Esterno 2025-11-27}\setulcolor{black}} \\
\hline
R-8-Q-De & \href{https://www.math.unipd.it/~tullio/IS-1/2025/Progetto/C3.pdf}{\ul{Capitolato di Progetto}\setulcolor{black}} \\
\hline
R-1-V-Ob & \href{https://www.math.unipd.it/~tullio/IS-1/2025/Progetto/C3.pdf}{\ul{Capitolato di Progetto}\setulcolor{black}} \\
\hline
R-2-V-Ob & \href{https://www.math.unipd.it/~tullio/IS-1/2025/Progetto/C3.pdf}{\ul{Capitolato di Progetto}\setulcolor{black}} \\
\hline
R-3-V-Ob & \href{https://www.math.unipd.it/~tullio/IS-1/2025/Progetto/C3.pdf}{\ul{Capitolato di Progetto}\setulcolor{black}} \\
\hline
R-4-V-Ob & \href{https://www.math.unipd.it/~tullio/IS-1/2025/Progetto/C3.pdf}{\ul{Capitolato di Progetto}\setulcolor{black}} \\
\hline
R-5-V-Ob & \href{https://www.math.unipd.it/~tullio/IS-1/2025/Progetto/C3.pdf}{\ul{Capitolato di Progetto}\setulcolor{black}} \\
\hline
R-6-V-Ob & \href{https://www.math.unipd.it/~tullio/IS-1/2025/Progetto/C3.pdf}{\ul{Capitolato di Progetto}\setulcolor{black}} \\
\hline
R-7-V-Ob & \href{https://www.math.unipd.it/~tullio/IS-1/2025/Progetto/C3.pdf}{\ul{Capitolato di Progetto}\setulcolor{black}} \\
\hline
R-8-V-Ob & \href{https://www.math.unipd.it/~tullio/IS-1/2025/Progetto/C3.pdf}{\ul{Capitolato di Progetto}\setulcolor{black}} \\
\hline
R-9-V-Ob & \href{https://www.math.unipd.it/~tullio/IS-1/2025/Progetto/C3.pdf}{\ul{Capitolato di Progetto}\setulcolor{black}} \\
\hline
R-10-V-Ob & \href{https://www.math.unipd.it/~tullio/IS-1/2025/Progetto/C3.pdf}{\ul{Capitolato di Progetto}\setulcolor{black}} \\
\hline
R-11-V-Ob & \href{https://cdn.jsdelivr.net/gh/7-zpus/Docs@main/2_RTB/Verbali/Verbali%20-Esterni/2025-11-13-VerbaleEsterno.pdf}{\ul{Verbale Esterno 2025-11-13}\setulcolor{black}} \\
\hline
\rowcolor{white}
\caption{Tracciamento Requisiti - Fonti}
\label{tab:trace-req-fonti}
\end{longtable}

\newpage
\subsection{Riepilogo}
\begin{table}[h]
    \centering
    \rowcolors{2}{gray!15}{white}
    \begin{tabular}{|l|c|c|c|c|}
        \hline
        \rowcolor{zpusgreen!30}
        \textbf{Tipologia} & \textbf{Obbligatori} & \textbf{Desiderabili} & \textbf{Opzionali} & \textbf{Totale} \\
        \hline
        Funzionali & 272 & 0 & 5 & 277 \\
        \hline
        Qualità & 6 & 2 & 0 & 8 \\
        \hline
        Vincolo & 11 & 0 & 0 & 11 \\
        \hline
        \textbf{Totale} & 289 & 2 & 5 & \textbf{296} \\
        \hline
    \end{tabular}
    \caption{Riepilogo dei Requisiti}
    \label{tab:riepilogo-requisiti}
\end{table}
\end{document}