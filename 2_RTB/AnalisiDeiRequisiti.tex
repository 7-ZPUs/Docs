\documentclass[12pt,a4paper]{article}
\usepackage[utf8]{inputenc}
\usepackage[italian]{babel}
\usepackage{geometry}
\usepackage{graphicx}
\usepackage{hyperref}
\usepackage{longtable}
\usepackage{array}
\usepackage{booktabs}
\usepackage{enumitem}
\usepackage{fancyhdr}
\usepackage{lastpage}
\usepackage{xcolor}
\usepackage{titlesec}
\usepackage{lastpage}
\geometry{
    top=2.5cm,
    bottom=2.5cm,
    left=2.5cm,
    right=2.5cm
}


\pagestyle{fancy}
\fancyhf{} 
\fancyhead[L]{\small Analisi dei Requisiti}
\fancyhead[R]{} 
\fancyfoot[]{\small 7-ZPUs}
\fancyfoot[C]{\small Pagina \thepage\ di \pageref{LastPage}}
\fancyfoot[R]{}
\renewcommand{\headrulewidth}{0.4pt}
\renewcommand{\footrulewidth}{0.4pt}
\hypersetup{
    colorlinks=true,
    linkcolor=black,
    filecolor=black,
    urlcolor=black,
    citecolor=black,
    pdftitle={Analisi dei Requisiti},
    pdfpagemode=FullScreen
}


\newcommand{\docDate}{\today}
\newcommand{\projectName}{DIPReader}
\newcommand{\groupName}{7-ZPUs Team}
\newcommand{\companyName}{Sanmarco Informatica}


\definecolor{sectioncolor}{RGB}{0,0,0}
\titleformat{\section}
  {\color{sectioncolor}\normalfont\Large\bfseries}
  {\thesection}{1em}{}
\titleformat{\subsection}
  {\color{sectioncolor}\normalfont\large\bfseries}
  {\thesubsection}{1em}{}

\begin{document}


\begin{titlepage}
    \centering
    \vspace*{2cm}
    
    {\Huge\bfseries Analisi dei Requisiti\par}
    \vspace{1cm}
    {\Large \projectName\par}
    \vspace{2cm}   
    \vfill
    {\large \groupName\par}
    \vspace{0.5cm}
    {\large Università degli Studi di Padova\par}
    {\large Corso di Ingegneria del Software\par}
    {\large A.A. 2025/2026\par}
\end{titlepage}

% tabella versionamento
\newpage

\section*{Registro dei Cambiamenti}
\begin{longtable}{|p{1.8cm}|p{2cm}|p{3cm}|p{3cm}|p{5cm}|}
\hline
\textbf{Versione} & \textbf{Data} & \textbf{Autore} & \textbf{Verificatore} & \textbf{Dettaglio} \\
\hline
\endfirsthead
\hline
\textbf{Versione} & \textbf{Data} & \textbf{Autore} & \textbf{Verificatore} & \textbf{Descrizione della Modifica} \\
\hline
\endhead

\hline
\endfoot
1.0.0 & 07/11/2025 & Georgescu Diana & Laoud Zakaria & Creazione del template e stesura iniziale. \\
\hline
\end{longtable}

\newpage
\tableofcontents
\newpage
\listoffigures
\newpage
\listoftables

% corpo documento
\newpage
\section{Introduzione}

\subsection{Scopo del documento}
Questo documento si pone l'obiettivo di delineare in modo chiaro le caratteristiche del software da realizzare, partendo dall'analisi dei bisogni e delle aspettative della proponente. L'elaborazione dei requisiti trae origine dallo studio preliminare del capitolato, al fine di individuare gli attori coinvolti e le funzionalità attese.\\
 
Il presente documento sarà utilizzato come punto di riferimento per tutto l'arco dello sviluppo del prodotto, dalla progettazione alla validazione, e permetterà il tracciamento di ogni decisione progettuale, consentendoci di soddisfare le aspettative della proponente.\\

Il documento di Analisi dei Requisiti è redatto dagli \textit{Analisti} del team di progetto ed è destinato principalmente a tre categorie di soggetti. In primo luogo, al \textbf{Committente}, che attraverso la sua consultazione può verificare che i requisiti siano stati correttamente compresi e formalizzati in linea con le proprie aspettative. In secondo luogo, al Team di \textbf{Progettisti e Programmatori}, per i quali il documento rappresenta una guida di riferimento essenziale durante la fase di sviluppo del Sistema software. Infine, al \textbf{Team di Verificatori}, che si baserà sulle informazioni contenute nel suddetto documento per definire i casi di test e verificare la conformità del prodotto alle specifiche.

Questo documento sarà, inoltre, a disposizione degli Amministratori e dei Responsabili di Progetto, allo scopo di ottenere una visione chiara e completa delle caratteristiche e delle funzionalità previste per il Sistema.

\subsection{Prospettiva del prodotto}
DA DESCRIVERE

\subsection{Funzioni del prodotto}
Anche qua DA CAPIRE ( elenco delle funzionalità richieste, meglio se dopo l'incontro con la proponente )

\subsection{Caratteristiche dell'utente}
Descrizione caratteristiche utenti 


\subsection{Riferimenti}

\subsubsection{Riferimenti normativi}
\begin{itemize}
    \item Capitolato d'appalto: \textbf{C3 - DIPReader}\\
    \url{https://www.math.unipd.it/~tullio/IS-1/2025/Progetto/C3.pdf}

    \item Norme di Progetto: \textbf{LINK DA INSERIRE}

    \item Regolamento progetto didattico:\\
    \url{https://www.math.unipd.it/~tullio/IS-1/2025/Dispense/PD1.pdf}
\end{itemize}

\subsubsection{Riferimenti informativi}
\begin{itemize}
    \item ISO/IEC/IEEE 29148:2018 - Systems and software engineering — Life cycle processes — Requirements engineering
    \item Slide del corso - Analisi dei requisiti: \\
    \url{https://www.math.unipd.it/~tullio/IS-1/2025/Dispense/T05.pdf}
    \item Diagrammi dei Casi d'Uso: \\
    \url{https://www.math.unipd.it/~rcardin/swea/2023/Diagrammi\%20delle\%20Classi.pdf}
\end{itemize}

\newpage
\section{Descrizione}

\subsection{Obiettivi del prodotto}
DA FARE

\subsection{Vincoli generali}
vincoli es: tecnologici, normativi, altro

\newpage
\section{Casi d'uso}

\subsection{Introduzione}
Per facilitare la comprensione dei casi d'uso, questi saranno descritti da un grafico UML e un testo per visualizzare gli obiettivi del prodotto.

La descrizione testuale deve contenere le seguenti informazioni:
\begin{itemize}
    \item \textbf{Attori (principali o secondari)}: rappresentano un ruolo che un'entità esterna al sistema assume quando interagisce con esso per raggiungere un obiettivo.
    \item \textbf{Precondizioni}: condizioni che devono essere vere nello stato del sistema prima che il caso d'uso inizi la sua esecuzione.
    \item \textbf{Postcondizioni}: condizioni che devono essere vere nello stato del sistema dopo che il caso d'uso è terminato.
    \item \textbf{Scenario principale}: interazioni tra attore e sistema che porta al raggiungimento dell'obiettivo del caso d'uso con successo
    \item \textbf{Scenari alternativi}: variazione rispetto al flusso principale.
\end{itemize}
  DA VEDERE MEGLIO 
  
\subsection{Attori}
Descrizione attori del sistema con relativo diagramma

\begin{figure}[h]
    \centering
    \caption{Diagramma}
    \label{fig:attori}
\end{figure}

\subsubsection{Attori principali}
\begin{itemize}
    \item \textbf{Attore a}: descrizione
    \item \textbf{Attore b}: descrizione
\end{itemize}

\subsubsection{Attori secondari}
\begin{itemize}
    \item \textbf{Attore Secondario}: descrizione
\end{itemize}

\subsection{Lista casi d'uso}

\subsubsection{UC1 - Nome Caso d'Uso}
\begin{figure}[h]
    \centering
    \caption{UC1}
    \label{fig:uc1}
\end{figure}

\begin{itemize}
    \item \textbf{Attore principale}: Nome attore
    \item \textbf{Precondizioni}:
        \begin{itemize}
            \item Precondizione 1
            \item Precondizione 2
        \end{itemize}
    \item \textbf{Postcondizioni}:
        \begin{itemize}
            \item Postcondizione 1
            \item Postcondizione 2
        \end{itemize}
    \item \textbf{Scenario principale}:
        \begin{enumerate}
            \item ---
            \item ---
            \item ---
        \end{enumerate}
    \item \textbf{Scenari alternativi}:
        \begin{itemize}
            \item ---
        \end{itemize}
\end{itemize}

\textit{da mantenere la medesima struttura per ogni UC}

\newpage
\section{Requisiti}

\subsection{Introduzione}
I requisiti vengono classificati secondo le seguenti categorie:
\begin{itemize}
    \item \textbf{Requisiti Funzionali (F)}: descrivono le funzionalità del sistema
    \item \textbf{Requisiti di Qualità (Q)}: descrivono le caratteristiche qualitative del sistema
    \item \textbf{Requisiti di Vincolo (V)}: descrivono i vincoli tecnologici e normativi
\end{itemize}

Ogni requisito è identificato da un codice univoco nella forma:
\begin{center}
\texttt{R-[ID]-[Tipo]-[Priorità]}
\end{center}

Dove:
\begin{itemize}
    \item \textbf{ID}: numero progressivo del requisito
    \item \textbf{Tipo}: F (Funzionale), Q (Qualità), V (Vincolo)
    \item \textbf{Priorità}: Ob (Obbligatorio), De (Desiderabile), Op (Opzionale)
\end{itemize}

\subsection{Requisiti Funzionali}

\begin{longtable}{|p{2.5cm}|p{8cm}|p{3cm}|}
\hline
\textbf{Codice} & \textbf{Descrizione} & \textbf{Fonti} \\
\hline
\endfirsthead

\hline
\textbf{Codice} & \textbf{Descrizione} & \textbf{Fonti} \\
\hline
\endhead

\hline
\endfoot
\hline
\caption{Requisiti Funzionali}
\label{tab:req-funzionali}
\end{longtable}

\subsection{Requisiti di Qualità}

\begin{longtable}{|p{2.5cm}|p{8cm}|p{3cm}|}
\hline
\textbf{Codice} & \textbf{Descrizione} & \textbf{Fonti} \\
\hline
\endfirsthead

\hline
\textbf{Codice} & \textbf{Descrizione} & \textbf{Fonti} \\
\hline
\endhead

\hline
\endfoot
\hline
\caption{Requisiti di Qualità}
\label{tab:req-qualita}
\end{longtable}

\subsection{Requisiti di Vincolo}

\begin{longtable}{|p{2.5cm}|p{8cm}|p{3cm}|}
\hline
\textbf{Codice} & \textbf{Descrizione} & \textbf{Fonti} \\
\hline
\endfirsthead

\hline
\textbf{Codice} & \textbf{Descrizione} & \textbf{Fonti} \\
\hline
\endhead

\hline
\endfoot

\caption{Requisiti di Vincolo}
\label{tab:req-vincolo}
\end{longtable}

\subsection{Tracciamento}

\subsubsection{Tracciamento Fonti - Requisiti}

\begin{longtable}{|p{4cm}|p{10cm}|}
\hline
\textbf{Fonte} & \textbf{Requisiti} \\
\hline
\endfirsthead

\hline
\textbf{Fonte} & \textbf{Requisiti} \\
\hline
\endhead

\hline
\endfoot

\hline
\caption{Tracciamento Fonti - Requisiti}
\label{tab:trace-fonti-req}
\end{longtable}

\subsubsection{Tracciamento Requisiti - Fonti}

\begin{longtable}{|p{3cm}|p{11cm}|}
\hline
\textbf{Requisito} & \textbf{Fonti} \\
\hline
\endfirsthead

\hline
\textbf{Requisito} & \textbf{Fonti} \\
\hline
\endhead

\hline
\endfoot

\caption{Tracciamento Requisiti - Fonti}
\label{tab:trace-req-fonti}
\end{longtable}

\subsection{Riepilogo}

\begin{table}[h]
\centering
\begin{tabular}{|l|c|c|c|c|}
\hline
\textbf{Tipologia} & \textbf{Obbligatori} & \textbf{Desiderabili} & \textbf{Opzionali} & \textbf{Totale} \\
\hline
Funzionali & X & Y & Z & N \\
\hline
Qualità & X & Y & Z & N \\
\hline
Vincolo & X & Y & Z & N \\
\hline
\textbf{Totale} & X & Y & Z & \textbf{N} \\
\hline
\end{tabular}
\caption{Riepilogo dei Requisiti}
\label{tab:riepilogo-requisiti}
\end{table}

\newpage

\end{document}