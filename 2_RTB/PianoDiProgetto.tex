\documentclass[a4paper,12pt]{article}

\usepackage[utf8]{inputenc}
\usepackage[T1]{fontenc}
\usepackage{lmodern}
\usepackage[italian]{babel}
\renewcommand{\rmdefault}{lmss}
\usepackage{float}
\usepackage{microtype}
\usepackage{geometry}
\usepackage{setspace}
\usepackage{enumitem}
\usepackage{titlesec}
\usepackage{tocloft}
\usepackage{graphicx}
\usepackage{hyperref}
\usepackage{xcolor,soul}
\usepackage{fancyhdr}
\usepackage{changepage}
\hypersetup{
    colorlinks=true,
    linkcolor=black,
    filecolor=magenta,      
    urlcolor=cyan,
}

\pagestyle{fancy}
\setlength{\headwidth}{\textwidth}
\fancyhfoffset[L,R]{0pt}
\lhead{\rightmark}
\rhead{7-ZPUs}
\lfoot{Piano di Progetto}
\rfoot{\thepage}
\cfoot{}
\renewcommand{\headrulewidth}{0.8pt}
\renewcommand{\footrulewidth}{0.8pt}

\renewcommand{\contentsname}{Indice}

\geometry{margin=2.5cm}
\setlength{\headheight}{14.49998pt}
\setstretch{1.2}

\titleformat{\section}{\large\bfseries}{\thesection}{1em}{}
\titleformat{\subsection}{\mdseries\bfseries}{\thesubsection}{1em}{}

\begin{document}

\begin{center}
    \includegraphics[width=9.5cm]{../assets/logo7ZPUs.jpg}\\
    \small\hspace{10cm} 7zpus.swe@gmail.com\\
    \vspace{0.5cm}
    \Large \textbf{Piano di Progetto}\\
\end{center}

\vspace{0.3cm}
\hrule
\vspace{0.5cm}

\tableofcontents

\newpage

\section*{Tabella di Versionamento}
\begin{table}[H]
\centering
\begin{tabular}{|c|c|c|c|c|}
    \hline
    \textbf{Versione} & \textbf{Data} & \textbf{Autore}  & \textbf{Verificatore} & \textbf{Descrizione} \\
    \hline
    0.1.0 & 2025/11/23 & Aaron Gingillino & Verificatore & Redazione sprint 1 \\
    \hline

\end{tabular}
\caption{Tabella di versionamento del documento}
\end{table}

\newpage

\section{Introduzione}
\subsection{Scopo del documento}
Il piano di progetto è un documento il cui obbiettivo è descrivere lo svolgimento del progetto, questo tramite le apposite sezioni: 
\begin{itemize}
    \item Analisi dei rischi
    \item Modello di sviluppo
    \item Pianificazione a lungo termine
    \item Pianificazione a breve termine
\end{itemize}
\subsection{Scopo del prodotto}
Il progetto riguarda lo sviluppo di un software "DIPReader". Il software, tramite una interfaccia utente munita di filtri impostabili dall'utente, permette di poter trovare agilmente informazioni all'interno dei file DIP.
\subsection{Glossario}
Al fine di ovviare ad ambiguità linguistiche è stato adottato un Glossario. Ogni qualvolta che un termine tecnico viene segnato con "\textsuperscript{G}" la sua definizione è disponibile nel glossario.

\subsection{Riferimenti}
\subsubsection{Riferimenti normativi}
\begin{itemize}
    \item \href{https://www.math.unipd.it/~tullio/IS-1/2025/Dispense/PD1.pdf}{\ul{Regolamento progetto didattico}} \setulcolor{black} \ped{(ultimo accesso: 2025/11/23)}
    \item \href{https://www.math.unipd.it/~tullio/IS-1/2025/Progetto/C3.pdf}{\ul{Capitolato d’appalto C3 - DIPReader}} \setulcolor{black} \ped{(ultimo accesso: 2025/12/01)}
    \item Norme di Progetto v1.0.0
\end{itemize}
\subsubsection{Riferimenti informativi}
\begin{itemize}
    \item Ian Sommerville, \textit{Software Engineering}, 9ª Edizione, Pearson, 2011.
    \item \href{https://www.math.unipd.it/~tullio/IS-1/2025/Dispense/T04.pdf}{\ul{Dispense "Gestione di progetto"}} \setulcolor{black} \ped{(ultimo accesso: 2025/12/01)}
    \item \href{https://www.math.unipd.it/~tullio/IS-1/2025/Dispense/T02.pdf}{\ul{Dispense "Processi di ciclo di vita del software"}} \setulcolor{black} \ped{(ultimo accesso: 2025/12/01)}
    \item Glossario v1.0.0
\end{itemize}

\section{Analisi dei rischi}
\subsection{Introduzione}
Ogni rischio presentato in questa sezione è caratterizzato da:
\begin{itemize}
    \item \textbf{Identificativo}
    \item \textbf{Probabilità di occorrenza}: può assumere i valori \textbf{Molto bassa} ($<10\%$), \textbf{Bassa}($10-25\%$), \textbf{Moderata}($25-50\%$), \textbf{Alta}($50-75\%$), \textbf{Molto alta}($>75\%$)
    \item \textbf{Effetti}: può assumere i valori: \textbf{Critico}, \textbf{Grave}, \textbf{Tollerabile}, \textbf{Irrilevante}
    \item \textbf{Strategia}: Per strategia si intendono tutte quelle misure atte a:
    \begin{itemize}
        \item Evitare il rischio, nel caso fosse possibile
        \item Mitigare il rischio, riducendone gli effetti sul progetto nel caso non fosse stato possibile evitarlo
        \item Gestire il rischio, nel caso in cui si verifichi il \textit{Worst Case Scenario}\textsuperscript{G}
    \end{itemize}
    \item \textbf{Indicatori}
\end{itemize}
\subsection{Rischi Tecnologici}
\subsubsection{Definizione}
Il rischio tecnologico riguarda tutto ciò che concerne le tecnologie parte del prodotto software, possono includere database, framework, linguaggi di programmazione, ecc... Viene inclusi per convenienza anche gli strumenti di amministrazione del progetto (e.g. GitHub).

\subsubsection{RT01 - Problemi tecnici con gli strumenti di sviluppo}
\textbf{Probabilità di occorrenza}: Moderata (50\%)\\
\textbf{Effetti}: Grave\\
\textbf{Descrizione}: L'inesperienza del gruppo con le tecnologie da utilizzare potrebbe portare a difficoltà tecniche e rallentamenti nello sviluppo.\\
\textbf{Strategia}: Prevedere sessioni di formazione iniziali con il supporto occasionale dell'azienda proponente per familiarizzare con gli strumenti e le tecnologie.\\
\textbf{Indicatori}: Ritardi nell'implementazione, errori tecnici frequenti, richieste di aiuto ripetute.
\subsection{Rischi Individuali}
\subsubsection{Definizione}
Il rischio individuale riguarda tutto ciò che concerne i membri del gruppo, come la loro disponibilità

\subsubsection{RI01 - Calo di produttività del team}
\textbf{Probabilità di occorrenza}: Molto alta (80\%)\\
\textbf{Effetti}: Tollerabile\\
\textbf{Descrizione}: Il gruppo prevede un calo dell'attività nel periodo natalizio, a causa di impegni personali e festività, che si sovrappone al periodo di sessione d'esame.\\
\textbf{Strategia}: Pianificare in anticipo le attività più critiche prima del periodo di calo, e le restanti tenendo conto del periodo di ridotta attività.\\
\textbf{Indicatori}: Diminuzione delle ore lavorate settimanali, ritardi nelle consegne pianificate.
\subsection{Rischi Organizzativi}
\subsubsection{Definizione}
Il rischio organizzativo riguarda tutto ciò che concerne l'organizzazione del progetto e del lavoro.

\subsubsection{RO01 - Sforamento dei costi preventivati}
\textbf{Probabilità di occorrenza}: Bassa (20\%)\\
\textbf{Effetti}: Grave\\
\textbf{Descrizione}: Il superamento del budget preventivato potrebbe compromettere la fattibilità del progetto, portando a ritardi nella consegna o alla necessità di ridurre alcune funzionalità previste.\\
\textbf{Strategia}: Monitorare costantemente l'allocazione delle ore rispetto alla pianificazione iniziale, effettuare stand-up meetings periodici e prevedere margini temporali per imprevisti.\\
\textbf{Indicatori}: Superamento del 10\% delle ore preventivate per attività, ritardi accumulati nelle milestone.

\subsubsection{RO02 - Mancata comunicazione e collaborazione tra i membri del team}
\textbf{Probabilità di occorrenza}: Molto bassa (5\%)\\
\textbf{Effetti}: Critico\\
\textbf{Descrizione}: La mancanza di comunicazione efficace può portare a incomprensioni, duplicazione degli sforzi e ritardi nel completamento delle attività.\\
\textbf{Strategia}: Stabilire canali di comunicazione chiari e regolari e una routine di aggiornamenti pianificati per garantire che tutti i membri del team siano allineati sugli obiettivi e le responsabilità.\\
\textbf{Indicatori}: Mancata partecipazione ai meeting, conflitti tra membri, lavori duplicati.

\subsubsection{RO03 - Mancata comunicazione con l'azienda proponente}
\textbf{Probabilità di occorrenza}: Bassa (20\%)\\
\textbf{Effetti}: Critico\\
\textbf{Descrizione}: La mancanza di feedback regolari dall'azienda proponente potrebbe portare a uno sviluppo rallentato e implementazione di funzionalità non allineate con le loro aspettative.\\
\textbf{Strategia}: Stabilire un calendario di incontri regolari con l'azienda proponente per garantire un flusso costante di comunicazione e feedback.\\
\textbf{Indicatori}: Ritardo nelle risposte dell'azienda, mancanza di feedback sui deliverables.

\subsubsection{RO04 - Mancato rispetto delle norme e documenti di progetto interni}
\label{RO04}
\textbf{Probabilità di occorrenza}: Bassa (15\%)\\
\textbf{Effetti}: Grave\\
\textbf{Descrizione}: La mancata aderenza ai documenti di progetto potrebbe portare a discrepanze tra i lavori svolti dai componenti del gruppo.\\
\textbf{Strategia}: Ruolo attivo di amministratori e verificatori per garantire il rispetto delle norme e dei documenti di progetto interni. Predisporre il riferimento alle norme attualmente in vigore e, in caso di modifiche, fissare una data di entrata in vigore delle nuove norme. Infine, fornire strumenti e tempo necessari all'apprendimento di queste ultime.\\
\textbf{Indicatori}: Documenti non conformi alle norme, discrepanze negli stili di codifica, mancato utilizzo degli strumenti concordati.
\subsection{Rischi legati ai requisiti}
\subsubsection{Definizione}
Il rischio che può derivare per esempio da una errata analisi di un requisito, o dal cambiamento di esso.

\section{Modello di Sviluppo}
\subsection{Scelta del modello di sviluppo}
Il metodo di sviluppo che il gruppo ha ritenuto più adatto per il progetto, vista la sua natura, è il metodo Agile, nello specifico il framework Scrum.
\subsection{Motivazioni}
Le motivazioni che hanno portato alla scelta di questo modello sono:
\begin{itemize}
    \item Principio Fail Fast: il gruppo vuole essere in grado di ottenere feedback frequenti e rapidi in modo tale da identificare e correggere eventuali non conformità a ciò che la proponente si aspetta. Questo principio è generalizzabile a qualsiasi elemento del progetto (requisiti, Way of Working...).
    \item Rotazione dei ruoli: con Agile i periodi sono divisi in Sprint, elemento che si integra in modo ottimale con la rotazione dei ruoli, che coinciderà naturalmente con l'inizio e la fine di uno sprint (non per forza lo stesso).
    \item Retrospettive: a differenza di altri modelli, Agile dedica un momento specifico alla riflessione sul come si è lavorato, rendendo più strutturato e disciplinato il miglioramento continuo del Way of Working.
    \item Pianificazione delle attività: con Agile le attività sono pianificate per periodi brevi, questo permette di adattarsi più facilmente a cambiamenti nei requisiti o nelle priorità del progetto.
\end{itemize}
\section{Pianificazione a lungo termine}
\subsection{Preventivo complessivo}
Viene di seguito riportata la tabella già esposta nel documento di \href{https://cdn.jsdelivr.net/gh/7-zpus/Docs@main/1_Candidatura/PreventivoCostiEAssunzioneImpegni.pdf}{Preventivo Costi e Assunzione Impegni} come riferimento.
\begin{table}[H]
    \begin{adjustwidth}{-4cm}{-4cm}
    \centering
    \begin{tabular}{|c|c|c|c|c|c|c|c|}
        \hline
        \textbf{Membro} & \textbf{Re} & \textbf{Am} & \textbf{An} & \textbf{Pg} & \textbf{Pr} & \textbf{Ve} & \textbf{Totale} \\
        \hline
        Fattoni Antonio & 10 & 8 & 13 & 17 & 24 & 23 & 95 \\
        \hline
        Georgescu Diana & 10 & 10 & 14 & 18 & 20 & 23 & 95 \\
        \hline
        Gingillino Aaron & 10 & 8 & 14 & 19 & 21 & 23 & 95 \\
        \hline
        Laoud Zakaria & 9 & 8 & 14 & 20 & 21 & 23 & 95 \\
        \hline
        Rocco Matteo Alberto & 12 & 9 & 12 & 17 & 22 & 23 & 95 \\
        \hline
        Soligo Lorenzo & 12 & 9 & 13 & 17 & 21 & 23 & 95 \\
        \hline
        Vigolo Davide & 9 & 10 & 13 & 17 & 24 & 22 & 95 \\
        \hline
    \end{tabular}
    \caption{Distribuzione delle ore per membro e ruolo}
    \end{adjustwidth}
\end{table}
\newpage
Partendo da questa stima di ripartizione oraria, comprensiva sia delle ore relative alla RTB, che alla PB, si stimano i costi riportati di seguito.
\begin{table}[H]
    \label{tab:preventivo_complessivo}
    \begin{adjustwidth}{-4cm}{-4cm}
    \centering
    \begin{tabular}{|c|c|c|}
        \hline
        \textbf{Ruolo} & \textbf{Ore totali} & \textbf{Costo totale ruolo (€)}  \\
        \hline
        Responsabile & 72 & 2160,00 \\
        \hline
        Amministratore & 62 & 1240,00 \\
        \hline
        Analista & 93 & 2325,00 \\
        \hline
        Progettista & 125 & 3125,00 \\
        \hline
        Programmatore & 153 & 2295,00 \\
        \hline
        Verificatore & 160 & 2400,00 \\
        \hline
        \textbf{Totale} & \textbf{665} & \textbf{13545,00} \\
        \hline
    \end{tabular}
    \caption{Preventivo complessivo dei costi per ruolo}
    \end{adjustwidth}
\end{table}

Al termine di ogni periodo di lavoro (sprint), sarà sottratto di volta in volta il monte ore svolto, partendo da questo preventivo. La data termine per la consegna del progetto è stata fissata al \textbf{10/04/2026}.
\subsection{Attività previste per la RTB}
\subsubsection{Milestone X}
\subsubsection{Milestone X+1}
\subsection{Attività previste per la PB}
\subsubsection{Milestone X}
\subsubsection{Milestone X+1}
\section{Pianificazione a breve termine}
\subsection{Sprint 1}
Data inizio: 2025/11/10\\
Data fine attesa: 2025/11/21\\
Data fine effettiva: 2025/11/21
\subsubsection{Attività previste}
\begin{table}[H]
    \begin{adjustwidth}{-4cm}{-4cm}
    \centering
    \begin{tabular}{|p{6cm}|l|c|c|c|}
        \hline
        \textbf{Attività} &  \textbf{Jira Issue} &\textbf{Data inizio} & \textbf{Data fine}& \textbf{Ruoli assegnati} \\
        \hline
        Aggiornamento Norme di Progetto&  \href{https://7zpus.atlassian.net/browse/DIPR-84}{DIPR-84}&2025/11/14& 2025/11/19& Responsable (3h) Verificatore (0.5h)\\
        \hline
        Studio del Materiale Aziendale&  \href{https://7zpus.atlassian.net/browse/DIPR-75}{DIPR-75}&2025/11/14& 2025/12/21& N/A\\
        \hline
        Preparazione Slide&  \href{https://7zpus.atlassian.net/browse/DIPR-76}{DIPR-76}&2025/11/14& 2025/11/17& N/A\\
        \hline
        Riunione Aziendale 2025/11/13&  \href{https://7zpus.atlassian.net/browse/DIPR-70}{DIPR-70}&2025/11/14& 2025/11/17& N/A\\
        \hline
 Preparazione Documento per Sanmarco& \href{https://7zpus.atlassian.net/browse/DIPR-66}{DIPR-66}& 2025/11/14& 2025/11/20&Analista (0.5h) Verificatore (0.5h)\\\hline
 Riunione Settimanale 2025-11-14& \href{https://7zpus.atlassian.net/browse/DIPR-34}{DIPR-34}& 2025/11/14& 2025/11/20&N/A\\\hline
 Aggiornamento DashBoard  di Progetto e dello Sprint& \href{https://7zpus.atlassian.net/browse/DIPR-88}{DIPR-88}& 2025/11/14& 2025/11/17&N/A\\\hline
 Test di opzioni di merge& \href{https://7zpus.atlassian.net/browse/DIPR-89}{DIPR-89}& 2025/11/14& 2025/11/19&N/A\\\hline
 Test di Convenzione di Branching& \href{https://7zpus.atlassian.net/browse/DIPR-92}{DIPR-92}& 2025/11/14& 2025/11/21&N/A\\\hline
    \end{tabular}
    \caption{Attività previste per lo Sprint 1}
    \end{adjustwidth}
\end{table}
\subsubsection{Rischi attesi}
Nessuno
\subsubsection{Preventivo di periodo}
Qui viene indicato, tramite tabella, il preventivo delle ore per ogni ruolo.
\begin{table}[H]
    \begin{adjustwidth}{-4cm}{-4cm}
    \centering
    \begin{tabular}{|c|c|c|c|c|c|c|c|}
        \hline
        \textbf{Membro}& \textbf{Re} & \textbf{Am}  & \textbf{Pr} & \textbf{An} & \textbf{Pg} & \textbf{Ve} & \textbf{Totale} \\
        \hline
        Fattoni Antonio & - & - & - & - & - & 0.5& 0.5\\
        \hline
        Georgescu Diana& - & - & - & - & - & - & - \\
        \hline
        Gingillino Aaron& - & - & - & - & - & - & - \\
        \hline
        Laoud Zakaria& - & - & - & 0.5& - & - & 0.5\\
        \hline
        Rocco Matteo Alberto& 3& -& -& -& -& -&3\\
        \hline
        Soligo Lorenzo& -& -& -& -& -& 0.5&0.5\\
        \hline
        Vigolo Davide& -& -& -& -& -& -&-\\\hline
        \hline
        \textbf{Totale} & \textbf{3}& \textbf{-} & -& \textbf{0.5}& \textbf{-}& \textbf{1}& \textbf{4.5}\\
        \hline
    \end{tabular}
    \caption{Preventivo di periodo per lo Sprint 1}
    \end{adjustwidth}
\end{table}
\subsubsection{Consuntivo di periodo}
\begin{table}[H]
    \begin{adjustwidth}{-4cm}{-4cm}
    \centering
    \begin{tabular}{|c|c|c|c|}
        \hline
        \textbf{Attività} & \textbf{Data inizio} & \textbf{Data fine}& \textbf{Ruoli assegnati} \\
        \hline
        \href{https://7zpus.atlassian.net/browse/DIPR-84}{DIPR-84}& 2025/11/14& 2025/11/19& Responsable (3h) \textcolor{red}{(+1)} Verificatore (0.5h)\\
        \hline
        \href{https://7zpus.atlassian.net/browse/DIPR-75}{DIPR-75}& 2025/11/14& 2025/12/21& N/A\\
        \hline
        \href{https://7zpus.atlassian.net/browse/DIPR-76}{DIPR-76}& 2025/11/14& 2025/11/17& N/A\\
        \hline
        \href{https://7zpus.atlassian.net/browse/DIPR-70}{DIPR-70}& 2025/11/14& 2025/11/17& N/A\\
        \hline
         \href{https://7zpus.atlassian.net/browse/DIPR-66}{DIPR-66}& 2025/11/14& 2025/11/20&Analista (0.5h) Verificatore (0.5h)\\\hline
         \href{https://7zpus.atlassian.net/browse/DIPR-29}{DIPR-29}& 2025/11/14& 2025/11/17&N/A\\\hline
         \href{https://7zpus.atlassian.net/browse/DIPR-34}{DIPR-34}& 2025/11/14& 2025/11/20&N/A\\\hline
         \href{https://7zpus.atlassian.net/browse/DIPR-88}{DIPR-88}& 2025/11/14& 2025/11/17&N/A\\\hline
         \href{https://7zpus.atlassian.net/browse/DIPR-89}{DIPR-89}& 2025/11/14& 2025/11/19&N/A\\\hline
         \href{https://7zpus.atlassian.net/browse/DIPR-92}{DIPR-92}& 2025/11/14& 2025/11/21&N/A\\\hline
    \end{tabular}
    \caption{Consuntivo delle attività dello Sprint 1}
    \end{adjustwidth}
\end{table}
\begin{table}[H]
    \begin{adjustwidth}{-4cm}{-4cm}
    \centering
    \begin{tabular}{|c|c|c|c|c|c|c|c|}
        \hline
        \textbf{Membro} & \textbf{Re} & \textbf{Am}  & \textbf{Pr} & \textbf{An} & \textbf{Pg} & \textbf{Ve} & \textbf{Totale} \\
        \hline
        Fattoni Antonio & - & - & - & - & - & 0.5& 0.5\\
        \hline
        Georgescu Diana& - & - & - & - & - & - & - \\
        \hline
        Gingillino Aaron& - & - & - & - & - & - & - \\
        \hline
        Laoud Zakaria& - & - & - & 0.5& - & - & 0.5\\
        \hline
 Rocco Matteo Alberto& \textbf{4} \textcolor{red}{+1}& -& -& -& -& -&3\\
 \hline
 Soligo Lorenzo& -& -& -& -& -& 0.5&0.5\\
 \hline
 Vigolo Davide& -& -& -& -& -& -&-\\\hline
        \hline
        \textbf{Totale} & \textbf{4}& \textbf{-} & -& \textbf{0.5}& \textbf{-}& \textbf{1}&  5.5 \textcolor{red}{+1}\\
        \hline
    \end{tabular}
    \caption{Consuntivo delle ore per membro dello Sprint 1}
    \end{adjustwidth}
\end{table}
\subsubsection{Rischi incontrati}
Nessuno

\subsubsection{Aggiornamento del preventivo a finire e dei rischi}
\begin{table}[H]
    \begin{adjustwidth}{-4cm}{-4cm}
    \centering
    \begin{tabular}{|c|c|c|c|c|}
        \hline
        \textbf{Ruolo} & \textbf{Ore / costo preventivati} & \textbf{Ore / costo consuntivo} & \textbf{Differenza} & \textbf{Preventivo a finire} \\
        \hline
        Responsabile & 3h / 90€& 4h / 120€& \textcolor{red}{+1h / +30€}& 68h / 2040€\\
        \hline
        Amministratore & - & - & - & 62h / 1240€\\
        \hline
        Analista & 0.5h / 25€& 0.5h / 25€& -& 92.5h / 2.312,5€\\
        \hline
        Progettista & -& -& - & 125h / 3125€\\
        \hline
        Programmatore & -& -& -& 153h / 2295€\\
        \hline
        Verificatore & 1h / 15€& 1h / 15€& - & 159h / 2385€\\
        \hline
        \textbf{Totale} & \textbf{4.5h / 130 €}& \textbf{5.5h / 160€}& \textcolor{red}{+1h / +30€}& \textbf{659.5h / 13.397,5€}\\
        \hline
    \end{tabular}
    \caption{Aggiornamento del preventivo a finire}
    \end{adjustwidth}
\end{table}
\subsection{Sprint 2}
Data inizio: 2025/11/22\\
Data fine attesa: 2025/12/05\\
Data fine effettiva: 2025/12/05
\subsubsection{Attività previste}
\begin{table}[H]
    \begin{adjustwidth}{-4cm}{-4cm}
    \centering
    \small
        \begin{tabular}{|p{5cm}|l|c|c|p{5cm}|}
            \hline
            \textbf{Attività} & \textbf{Jira Issue} & \textbf{Data Inizio} & \textbf{Data Fine} & \textbf{Ruoli Assegnati} \\
            \hline
            Aggiornamento Task & \href{https://7zpus.atlassian.net/browse/DIPR-104}{DIPR-104} & 21 nov & 5 dic & Amministratore (30m) LS\\
            \hline
            Riunione Settimanale 2025-11-21 & \href{https://7zpus.atlassian.net/browse/DIPR-35}{DIPR-35} & 21 nov & 28 nov & Responsabile (2h 30m) DV + AF, Verificatore (15m) DV\\
            \hline
            Prima Scrittura Piano di Progetto & \href{https://7zpus.atlassian.net/browse/DIPR-100}{DIPR-100} & 21 nov & 5 dic & Responsabile (1h 30m) DV\\
            \hline
            Riunione Settimanale 2025-11-28 & \href{https://7zpus.atlassian.net/browse/DIPR-36}{DIPR-36} & 21 nov & 5 dic & Responsabile (2h30m) DV + AF, Verificatore (15m) DV\\
            \hline
            Riunione Aziendale 2025-11-27 & \href{https://7zpus.atlassian.net/browse/DIPR-110}{DIPR-110} & 21 nov & 5 dic & Responsabile (1h30m) DV + LS, Verificatore (15m) ZL\\
            \hline
            Adattamento del sistema alle nuove norme di branching& \href{https://7zpus.atlassian.net/browse/DIPR-105}{DIPR-105} & 21 nov & 5 dic & Amministratore (2h) LS, Verificatore (30m) AF\\
            \hline
            Redazione Sprint 1 & \href{https://7zpus.atlassian.net/browse/DIPR-114}{DIPR-114} & 21 nov & 5 dic & Responsabile (30m) AG, Verificatore (15m) MR\\
            \hline
            Elaborazione Use Case & \href{https://7zpus.atlassian.net/browse/DIPR-115}{DIPR-115} & 21 nov & 5 dic & Analista (8h) Tutti\\
            \hline
            Aggiornamento norme di branching & \href{https://7zpus.atlassian.net/browse/DIPR-116}{DIPR-116} & 21 nov & 5 dic & Amministratore (45m) LS, Verificatore (15m) MR\\
            \hline
    \end{tabular}
    \caption{Attività previste per lo Sprint 2}
    \end{adjustwidth}
\end{table}
\subsubsection{Rischi attesi}
Nessuno
\subsubsection{Preventivo di periodo}
Qui viene indicato, tramite tabella, il preventivo delle ore per ogni ruolo.
\begin{table}[H]
    \begin{adjustwidth}{-4cm}{-4cm}
    \centering
    \begin{tabular}{|c|c|c|c|c|c|c|c|}
        \hline
        \textbf{Membro}& \textbf{Re} & \textbf{Am}  & \textbf{Pr} & \textbf{An} & \textbf{Pg} & \textbf{Ve} & \textbf{Totale} \\
        \hline
        Fattoni Antonio & 1 & - & - & 2 & - & 0.5 & 3.5\\
        \hline
        Georgescu Diana& - & - & - & 2 & - & - & 2 \\
        \hline
        Gingillino Aaron& 0.5 & - & - & 2 & - & - & 2.5 \\
        \hline
        Laoud Zakaria& - & - & - & 2 & - & 0.25 & 2.25\\
        \hline
        Rocco Matteo Alberto & - & - & - & 2 & - & 0.5 & 2.5\\
        \hline
        Soligo Lorenzo& 0.5 & 3.25 & - & 2 & - & - & 5.75\\
        \hline
        Vigolo Davide & 6.5 & - & - & 2 & - & 0.5 & 9 \\\hline
        \hline
        \textbf{Totale} & \textbf{8.5}& \textbf{3.25} & \textbf{-} & \textbf{14}& \textbf{-}& \textbf{1.75}& \textbf{27.5}\\
        \hline
    \end{tabular}
    \caption{Preventivo di periodo per lo Sprint 2}
    \end{adjustwidth}
\end{table}
\subsubsection{Consuntivo di periodo}
\begin{table}[H]
    \begin{adjustwidth}{-4cm}{-4cm}
    \centering
    \small
        \begin{tabular}{|p{5cm}|l|c|c|p{5cm}|}
            \hline
            \textbf{Attività} & \textbf{Jira Issue} & \textbf{Data Inizio} & \textbf{Data Fine} & \textbf{Ruoli Assegnati} \\
            \hline
            Aggiornamento Task & \href{https://7zpus.atlassian.net/browse/DIPR-104}{DIPR-104} & 21 nov & 5 dic & Amministratore (30m) LS\\
            \hline
            Riunione Settimanale 2025-11-21 & \href{https://7zpus.atlassian.net/browse/DIPR-35}{DIPR-35} & 21 nov & 28 nov & Responsabile (2h 30m) DV + AF, Verificatore (15m) DV\\
            \hline
            Prima Scrittura Piano di Progetto & \href{https://7zpus.atlassian.net/browse/DIPR-100}{DIPR-100} & 21 nov & 5 dic & Responsabile (1h 30m) DV\\
            \hline
            Riunione Settimanale 2025-11-28 & \href{https://7zpus.atlassian.net/browse/DIPR-36}{DIPR-36} & 21 nov & 5 dic & Responsabile (2h30m) DV + AF, Verificatore (15m) DV\\
            \hline
            Riunione Aziendale 2025-11-27 & \href{https://7zpus.atlassian.net/browse/DIPR-110}{DIPR-110} & 21 nov & 5 dic & Responsabile (1h30m) DV + LS, Verificatore (15m) ZL\\
            \hline
            Adattamento del sistema alle nuove norme di branching& \href{https://7zpus.atlassian.net/browse/DIPR-105}{DIPR-105} & 21 nov & 5 dic & Amministratore (2h) LS, Verificatore (30m) AF\\
            \hline
            Redazione Sprint 1 & \href{https://7zpus.atlassian.net/browse/DIPR-114}{DIPR-114} & 21 nov & 5 dic & Responsabile (30m) AG, Verificatore (15m) MR\\
            \hline
            Elaborazione Use Case & \href{https://7zpus.atlassian.net/browse/DIPR-115}{DIPR-115} & 21 nov & 5 dic & Analista (8h) Tutti\\
            \hline
            Aggiornamento norme di branching & \href{https://7zpus.atlassian.net/browse/DIPR-116}{DIPR-116} & 21 nov & 5 dic & Amministratore (45m) LS, Verificatore (15m) MR\\
            \hline
    \end{tabular}
    \caption{Consuntivo delle attività dello Sprint 2}
    \end{adjustwidth}
\end{table}
\begin{table}[H]
    \begin{adjustwidth}{-4cm}{-4cm}
    \centering
    \begin{tabular}{|c|c|c|c|c|c|c|c|}
        \hline
        \textbf{Membro}& \textbf{Re} & \textbf{Am}  & \textbf{Pr} & \textbf{An} & \textbf{Pg} & \textbf{Ve} & \textbf{Totale} \\
        \hline
        Fattoni Antonio & 1 & - & - & 2 & - & 0.5 & 3.5\\
        \hline
        Georgescu Diana& - & - & - & 2 & - & - & 2 \\
        \hline
        Gingillino Aaron& 0.5 & - & - & 2 & - & - & 2.5 \\
        \hline
        Laoud Zakaria& - & - & - & 2 & - & 0.25 & 2.25\\
        \hline
        Rocco Matteo Alberto & - & - & - & 2 & - & 0.5 & 2.5\\
        \hline
        Soligo Lorenzo& 0.5 & 3.25 & - & 5 \textcolor{red}{(+3)} & - & - & 8.75\\
        \hline
        Vigolo Davide & 6.5 & - & - & 2 & - & 0.5 & 9 \\\hline
        \hline
        \textbf{Totale} & \textbf{8.5}& \textbf{3.25} & \textbf{-} & \textbf{17}& \textbf{-}& \textbf{1.75}& \textbf{30.50}\textcolor{red}{(+3)}\\
        \hline
    \end{tabular}
    \caption{Consuntivo delle ore per membro dello Sprint 2}
    \end{adjustwidth}
\end{table}
\subsubsection{Rischi incontrati}
Durante questo sprint sono stati incontrati i seguenti rischi:
\begin{itemize}
    \item \hyperref[RO04]{RO04} - Mancato rispetto delle norme e documenti di progetto interni: questo rischio si è manifestato a causa del cambiamento delle norme di branching e la gestione delle task jira associate. Questo a portato al rapido cambiamento delle norme e alla poca uniformità del lavoro svolto. E' stata applicata la misura preventiva, in particolare fornendo supporto ai membri del team non ancora allineati alle nuove norme.
    \begin{itemize}
        \item Il rischio non è stato previsto a causa della sottostima del problema, in quanto si pensava che il cambiamento non sempre ben definito delle norme di branching non avrebbe avuto un impatto così significativo sul lavoro del team.
        \item La strategia di contenimento/prevenzione ha funzionato, in particolare è stato efficace il supporto dagli autori delle norme verso gli altri membri del gruppo tramite un video esplicativo. Si ritiene tuttavia necessario stabilire da ora in poi una data di entrata in vigore delle norme, tenendo conto il tempo di adattamento e apprendimento del team.
    \end{itemize}
\end{itemize}



\subsubsection{Aggiornamento del preventivo a finire e dei rischi}
\begin{table}[H]
    \begin{adjustwidth}{-4cm}{-4cm}
    \centering
    \begin{tabular}{|c|c|c|c|c|}
        \hline
        \textbf{Ruolo} & \textbf{Ore / costo preventivati} & \textbf{Ore / costo consuntivo} & \textbf{Differenza} & \textbf{Preventivo a finire} \\
        \hline
        Responsabile & 8.5h / 255€ & 8.5h / 255€& - & 59.5h / 1.785€\\
        \hline
        Amministratore & 3.25h / 65€ & 3.25h / 65€ & - & 58.75h / 1.175€\\
        \hline
        Analista & 14h / 350€ & 17h / 425€& \textcolor{red}{+3h / +75€} & 75.5h / 1.887,5€\\
        \hline
        Progettista & - & - & - & 125h / 3125€\\
        \hline
        Programmatore & - & - & - & 153h / 2295€\\
        \hline
        Verificatore & 1.75h / 26,25€ & 1.75h / 26,25€ & - & 157.25h / 2.358,75€\\
        \hline
        \textbf{Totale} & \textbf{27.5h / 696,25 €}& \textbf{30.5h / 771,25€}& \textcolor{red}{+3h / +75€}& \textbf{629h / 12.626,25€}\\
        \hline
    \end{tabular}
    \caption{Aggiornamento del preventivo a finire}
    \end{adjustwidth}
\end{table}
\vfill
\begin{flushright}
    \textit{7-ZPUs}
\end{flushright}

\end{document}