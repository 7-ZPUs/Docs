\documentclass[a4paper,12pt]{article}

\usepackage[utf8]{inputenc}
\usepackage[T1]{fontenc}
\usepackage[italian]{babel}
\usepackage{lmodern}
\renewcommand*\familydefault{\sfdefault}
\usepackage{float}
\usepackage{microtype}
\usepackage{geometry}
\usepackage{setspace}
\usepackage{enumitem}
\usepackage{titlesec}
\usepackage{chngpage}
\usepackage{tocloft}
\usepackage{graphicx}
\usepackage{fancyhdr}
\usepackage{xcolor}
\usepackage{color,soul}
\usepackage[most]{tcolorbox}
\usepackage[colorlinks=true]{hyperref}



\hypersetup{
    linkcolor=black,
    urlcolor=blue
}

\definecolor{lightblack}{gray}{0.35}
\newcommand{\glossario}[1]{\textit{#1}\textsubscript{\textbf{\textit{\textcolor{lightblack}{G}}}}}

\pagestyle{fancy}
\setlength{\headwidth}{\textwidth}
\fancyhfoffset[L,R]{0pt}
\lhead{\rightmark}
\rhead{7-ZPUs}
\lfoot{Norme di Progetto}
\rfoot{\thepage}
\cfoot{}
\renewcommand{\headrulewidth}{0.8pt}
\renewcommand{\footrulewidth}{0.8pt}

\renewcommand{\contentsname}{Indice}

\geometry{margin=2.5cm}
\setstretch{1.1}

\titleformat{\section}{\Large\bfseries}{\thesection}{1em}{}
\titleformat{\subsection}{\large\bfseries}{\thesubsection}{1em}{}
\titleformat{\subsubsection}{\normalsize\bfseries}{\thesubsubsection}{1em}{}
\titleformat{\paragraph}{\large\bfseries}{\theparagraph}{1em}{}
\titleformat{\subparagraph}{\normalsize\bfseries}{\thesubparagraph}{1em}{}

\begin{document}

\begin{center}
    \includegraphics[width=9.5cm]{../assets/logo7ZPUs.jpg}\\
    \small\hspace{10cm} 7zpus.swe@gmail.com\\
    \vspace{0.5cm}
    \Large \textbf{Norme di Progetto}\\
\end{center}

\vspace{0.3cm}
\hrule
\vspace{0.5cm}

\tableofcontents


\section*{Tabella di Versionamento}
\begin{table}[H]
    \begin{adjustwidth}{-4cm}{-4cm}
    \centering
    \begin{tabular}{|c|c|c|c|c|}
        \hline
        \textbf{Versione} & \textbf{Data} & \textbf{Autore}  & \textbf{Verificatore} & \textbf{Descrizione} \\
        \hline
        0.1 & 16/11/2025 & Rocco Matteo A. & Soligo Lorenzo & \begin{tabular}[c]{@{}c@{}} Creazione e stesura sezioni \\Introduzione e Processo di fornitura \end{tabular} \\
        \hline
    \end{tabular}
    \end{adjustwidth}
\end{table}

\newpage

\section{Introduzione}

\subsection{Scopo}
Questo documento ha l'obiettivo di definire e normare il \glossario{Way of Working}, ovvero le regole di lavoro che ogni membro del gruppo deve rispettare durante lo svolgimento delle \glossario{attività di progetto} volte allo sviluppo dell'applicativo software \glossario{\textbf{DIPReader}}, proposto dall'azienda \glossario{Sanmarco Informatica}. A ciascun membro è richiesto di seguirle nella sua interezza per poter lavorare in maniera quanto più efficace ed efficiente, oltre che omogenea.
Data la natura incrementale di redazione del documento, il \glossario{responsabile di progetto} in carica ha il compito di mantenere aggiornate le presenti norme e eventuali riferimenti ad altri documenti contenuti al loro interno. 

\subsection{Glossario}
Ogni termine tecnico o con particolare significato nell'ambito dell'\glossario{Ingegneria del Software} utilizzato nella documentazione di progetto viene definito nell'apposito documento \href{https://cdn.jsdelivr.net/gh/7-zpus/Docs@norme_in_lavorazione/2_RTB/Glossario.pdf}{\ul{Glossario 1.0}\setulcolor{black}}\ped{(ultimo accesso: 17/11/2025)}. 

\subsection{Riferimenti}
Il gruppo ha deciso di redarre il presente documento in conformità con lo \glossario{standard} ISO/IEC 12207:1995, ricorrendo occasionalmente ad approfondimenti contenuti nella sua versione più attuale ISO/IEC/IEEE 12207:2017 per includere dettagli aggiuntivi relativi agli approcci \glossario{agili} e \glossario{iterativi} che contraddistinguono lo \glossario{sviluppo software} moderno. 

\subsubsection{Riferimenti Normativi}
\begin{itemize}
    \item \href{https://www.math.unipd.it/~tullio/IS-1/2009/Approfondimenti/ISO_12207-1995.pdf}{\ul{Standard ISO/IEC 12207:1995}\setulcolor{black}} \ped{(ultimo accesso: 17/11/2025)}
    \item Standard ISO/IEC/IEEE 12207:2017
    \item Standard ISO/IEC/IEEE 24765:2017
    \item \href{https://www.math.unipd.it/~tullio/IS-1/2025/Progetto/C3.pdf}{\ul{Capitolato C3: DIPReader}\setulcolor{black}} \ped{(ultimo accesso: 13/11/2025)}
    \item \href{https://www.math.unipd.it/~tullio/IS-1/2025/Dispense/PD1.pdf}{\ul{Regolamento di Progetto Didattico a.a. 2025/2026}\setulcolor{black}} \ped{(ultimo accesso: 17/11/2025)}
\end{itemize}

\subsubsection{Riferimenti Informativi}
\begin{itemize}
    \item Dispense del corso di Ingegneria del Software 2025/2026:
    \begin{itemize}
        \item \href{https://www.math.unipd.it/~tullio/IS-1/2025/Dispense/T01.pdf}{\ul{https://www.math.unipd.it/~tullio/IS-1/2025/Dispense/T01.pdf}\setulcolor{black}} \ped{(ultimo accesso: 17/11/2025)}
        \item \href{https://www.math.unipd.it/~tullio/IS-1/2025/Dispense/T02.pdf}{\ul{https://www.math.unipd.it/~tullio/IS-1/2025/Dispense/T02.pdf}\setulcolor{black}} \ped{(ultimo accesso: 17/11/2025)}
        \item \href{https://www.math.unipd.it/~tullio/IS-1/2025/Dispense/T03.pdf}{\ul{https://www.math.unipd.it/~tullio/IS-1/2025/Dispense/T03.pdf}\setulcolor{black}} \ped{(ultimo accesso: 17/11/2025)}
        \item \href{https://www.math.unipd.it/~tullio/IS-1/2025/Dispense/T04.pdf}{\ul{https://www.math.unipd.it/~tullio/IS-1/2025/Dispense/T04.pdf}\setulcolor{black}} \ped{(ultimo accesso: 17/11/2025)}
        \item \href{https://www.math.unipd.it/~tullio/IS-1/2025/Dispense/T05.pdf}{\ul{https://www.math.unipd.it/~tullio/IS-1/2025/Dispense/T05.pdf}\setulcolor{black}} \ped{(ultimo accesso: 17/11/2025)}
        \item \href{https://www.math.unipd.it/~tullio/IS-1/2025/Dispense/T06.pdf}{\ul{https://www.math.unipd.it/~tullio/IS-1/2025/Dispense/T06.pdf}\setulcolor{black}} \ped{(ultimo accesso: 17/11/2025)}
        \item \href{https://www.math.unipd.it/~tullio/IS-1/2025/Dispense/T07.pdf}{\ul{https://www.math.unipd.it/~tullio/IS-1/2025/Dispense/T07.pdf}\setulcolor{black}} \ped{(ultimo accesso: 17/11/2025)}
        \item \href{https://www.math.unipd.it/~tullio/IS-1/2025/Dispense/T08.pdf}{\ul{https://www.math.unipd.it/~tullio/IS-1/2025/Dispense/T08.pdf}\setulcolor{black}} \ped{(ultimo accesso: 17/11/2025)}
        \item \href{https://www.math.unipd.it/~tullio/IS-1/2025/Dispense/T09.pdf}{\ul{https://www.math.unipd.it/~tullio/IS-1/2025/Dispense/T09.pdf}\setulcolor{black}} \ped{(ultimo accesso: 17/11/2025)}
        \item \href{https://www.math.unipd.it/~tullio/IS-1/2025/Dispense/T10.pdf}{\ul{https://www.math.unipd.it/~tullio/IS-1/2025/Dispense/T10.pdf}\setulcolor{black}} \ped{(ultimo accesso: 17/11/2025)}
        \item \href{https://www.math.unipd.it/~tullio/IS-1/2025/Dispense/T11.pdf}{\ul{https://www.math.unipd.it/~tullio/IS-1/2025/Dispense/T11.pdf}\setulcolor{black}} \ped{(ultimo accesso: 17/11/2025)}
    \end{itemize}
    \item \href{https://www.agid.gov.it/it/sicurezza/cert-pa/linee-guida-sviluppo-del-software-sicuro}{\ul{Linee Guida Sviluppo Sicuro AGID (Agenzia per l'Italia Digitale)}\setulcolor{black}}
    \item  \href{https://www.agid.gov.it/it/linee-guida#index-3}{\ul{Linee Guida sulla formazione, gestione e conservazione dei documenti informatici AGID}\setulcolor{black}}
    \item \href{https://www.lorenzopantieri.net/LaTeX_files/LaTeXpedia.pdf}{\ul{Documentazione \LaTeX{} by Lorenzo Pantieri}\setulcolor{black}} \ped{(ultimo accesso: 17/11/2025)}
    \item \href{https://confluence.atlassian.com/jira}{\ul{Documentazione Jira}\setulcolor{black}}
\end{itemize}

\section{Processi primari}

\subsection{Processo di Fornitura}
Il \glossario{processo} di fornitura contiene le attività e i compiti svolti dal \glossario{fornitore}. Per implementare correttamente il processo il gruppo si impegna a svolgere le seguenti attività.

\subsubsection{Attività di processo}

\subparagraph{Avvio}
Il fornitore analizza i \glossario{requisiti} necessari alla proposta di fornitura, tenendo considerazione di eventuali vincoli organizzativi e normativi.
\subparagraph{Preparazione della proposta di fornitura}
Il fornitore prepara la proposta di fornitura in risposta alle richieste del committente e definisce i termini in cui si articola la proposta.
\subparagraph{Accordo}
Proponente e fornitore entrano nella fase di definizione dell'accordo di fornitura del prodotto software, prevedendo possibilità di negoziazione della fornitura da parte del fornitore.
\subparagraph{Pianificazione}
Il fornitore rielabora l'analisi dei requisiti fondamentali per definire il \glossario{framework} entro il quale il prodotto verrà sviluppato e gestito, in modo tale da garantire un processo di qualità durante lo sviluppo. Si impegna inoltre a definire il modello del ciclo di vita del prodotto adatto alla complessità del progetto e ai relativi rischi che potrebbero insorgere. Tutte queste decisioni convergono nel Piano di Progetto.
\subparagraph{Esecuzione e controllo}
Il fornitore si impegna a sviluppare il prodotto secondo il Piano di Progetto, avendo cura di controllare che i processi di siano stati eseguiti correttamente.
\subparagraph{Verifica e validazione}
Il fornitore stabilisce con la proponente le modalità di rendicontazione dello stato di avanzamento del prodotto e rende disponibili i documenti che dimostrino la verifica e validazione dei processi secondo i requisiti precedentemente individuati.
\subparagraph{Consegna e terminazione}
Il fornitore consegna il prodotto finale al proponente e ne espone le funzionalità.

\subsubsection{Accordi con l'azienda proponente}
I capitolati presentati dalle proponenti vengono analizzati e viene redatto il documento di \href{https://cdn.jsdelivr.net/gh/7-zpus/Docs@main/1_Candidatura/AnalisiCapitolati.pdf}{\ul{Analisi dei capitolati}\setulcolor{black}}, nel quale sono delineati i bisogni e i principali vincoli a cui attenersi per la fornitura del prodotto finale.
Il fornitore espone ai committenti di fornitura, ovvero i Professori Vardanega Tullio e Cardin Riccardo, la Lettera di Presentazione della proposta di fornitura che descrive il preventivo di costi, cronogramma di sviluppo, suddivisione del lavoro e i ruoli coinvolti.

La \glossario{proponente}, in qualità di \glossario{stakeholder}, esercita il diritto di ricevere la rendicontazione professionale e approfondita del lavoro svolto dal gruppo fornitore, perciò si instaura un accordo per delineare le modalità di comunicazione e il contenuto di tale rendicontazione. 
È previsto l'aggiornamento costante e tempestivo della proponente per quanto riguarda la pianificazione degli obiettivi e delle tempistiche di sviluppo individuate dal fornitore. Ogni qual volta vi siano modifiche di notevole interesse esterno dal gruppo fornitore verranno comunicate all'azienda proponente attraverso appositi canali di comunicazione sincrona o asincrona.

Il fornitore e la proponente hanno accordato lo svolgimento di un incontro di verifica dello stato di avanzamento lavori (\glossario{SAL}) in modalità sincrona ogni due settimane, in cui discutere l'andamento del lavoro e chiarire eventuali dubbi da parte del fornitore o segnalazioni di difformità dai requisiti iniziali della proponente. È inoltre sempre disponibile la comunicazione via email per questioni minori e di facile risoluzione.
La consegna del prodotto è suddivisa in due \glossario{milestone} principali: \glossario{RTB} (Requirements and Technology Baseline) e \glossario{PB} (Product Baseline).

\subsubsection{Documentazione fornita}

\subparagraph{Analisi dei requisiti}
Nel documento di \href{https://cdn.jsdelivr.net/gh/7-zpus/Docs@main/2_RTB/AnalisiDeiRequisiti.pdf}{\ul{Analisi dei requisiti}\setulcolor{black}} \ped{(ultimo accesso: 17/11/2025)} sono riportati i bisogni e i vincoli a cui attenersi per la realizzazione del prodotto finale. L'obiettivo è definire in maniera non ambigua i \glossario{casi d'uso} (\textit{Use Cases}) e i requisiti (\textit{Requirements}) del software. Il documento è diviso nelle seguenti sezioni:
\begin{enumerate}
    \item Introduzione
    \item Descrizione
    \item Definizione dei casi d'uso
    \item Definizione dei requisiti
\end{enumerate}

\subparagraph{Glossario}
Il Glossario è il documento che raccoglie ogni termine di carattere tecnico, nomenclature e acronimi con particolare significato nell'ambito dell'Ingegneria del Software utilizzato nella documentazione di progetto. La definizione dei termini di glossario è coadiuvata dal contenuto dello standard ISO/IEC/IEEE 24765/2017.

\subparagraph{Piano di progetto}
Il \href{https://cdn.jsdelivr.net/gh/7-zpus/Docs@norme_in_lavorazione/2_RTB/PianoDiProgetto.pdf}{\ul{Piano di progetto v1.0}\setulcolor{black}} \ped{(ultimo accesso: 17/11/2025)} è il documento che espone all'esterno il lavoro di sviluppo svolto seguendo le procedure delineate all'interno di questo documento. Fornisce una guida dettagliata alla pianificazione, esecuzione e consuntivo delle attività completate in ciascuna \glossario{sprint}. Il documento è diviso nelle seguenti sezioni:
\begin{enumerate}
    \item Introduzione
    \item Analisi dei rischi e mitigazione
    \item Modello di sviluppo
    \item Pianificazione dei costi e suddivisione ruoli
    \item Preventivo di periodo
    \item Consuntivo di periodo
    \item Retrospettiva
\end{enumerate}

\subparagraph{Piano di qualifica}
Il piano di qualifica descrive gli obiettivi di qualità dei processi che il fornitore si impegna a soddisfare per consegnare un prodotto finale di qualità. Le metriche di valutazione vengono determinate dall'analisi dei requisiti e dalle indicazioni date dalla proponente, suddivise in base all'applicazione sui processi o sul prodotto.
Le metriche stabilite vengono poi misurate attraverso opportuni test e verifiche, di cui vengono riportate le specifiche.
Il documento include una sezione di rendicontazione per la valutazione dei processi e la valutazione del prodotto, in cui riportare l'attinenza alle metriche ottenuta rispetto agli obiettivi e di conseguenza valutare azioni correttive in caso si verifichino eventuali problemi (\glossario{cruscotto di qualità}). Il documento è diviso nelle seguenti sezioni:
\begin{enumerate}
    \item Qualità dei processi
    \item Qualità del prodotto
    \item Specifiche di test e verifica
    \item Cruscotto di qualità
\end{enumerate}


\subparagraph{Lettera di presentazione}
La lettera di presentazione è il documento necessario alla candidatura per la milestone di revisione di avanzamento \glossario{RTB} (\textit{Requirements and Technology Baseline}). Essa contiene le informazioni sul repository di progetto, il puntatore al \glossario{Proof of Concept}, il consuntivo di spesa e preventivo a finire del progetto.

\subsubsection{Strumenti}
\begin{itemize}
    \item \glossario{Git-Hub}
    \item \glossario{Jira}
\end{itemize}


\subsection{Processo di sviluppo}

\subsubsection{Attività di processo}


\subsection{Processo operativo}

\subsection{Processo di manutenzione}


\section{Processi di supporto}

\subsection{Processo di documentazione}

\subsection{Processo di gestione della configurazione}

\subsection{Processo di valutazione della qualità}

\subsection{Processo di verifica}

\subsection{Processo di validazione}

\subsection{Processo di Joint Review}

\subsection{Risoluzione dei problemi}

\subsection{Gestione della qualità}

\section{Processi organizzativi}

\subsection{Gestione}

\subsection{Infrastruttura}

\subsection{Miglioramento}

\subsection{Allenamento}

\section{Metriche della qualità}



\vfill
\begin{flushright}
    \textit{7-ZPUs}
\end{flushright}

\end{document}