\documentclass[a4paper,12pt]{article}

\usepackage[utf8]{inputenc}
\usepackage[T1]{fontenc}
\usepackage[italian, provide=*]{babel}
\usepackage{lmodern}
\renewcommand*\familydefault{\sfdefault}
\usepackage{float}
\usepackage{microtype}
\usepackage{geometry}
\usepackage{setspace}
\usepackage{enumitem}
\usepackage{titlesec}
\usepackage{chngpage}
\usepackage{tocloft}
\usepackage{graphicx}
\usepackage{fancyhdr}
\usepackage{xcolor}
\usepackage{color,soul}
\usepackage[most]{tcolorbox}
\usepackage[colorlinks=true]{hyperref}



\hypersetup{
    linkcolor=black,
    urlcolor=blue
}

\definecolor{lightblack}{gray}{0.35}
\newcommand{\glossario}[1]{\textit{#1}\textsubscript{\textbf{\textit{\textcolor{lightblack}{G}}}}}
\newcommand{\ped}[1]{\textsubscript{#1}}

\pagestyle{fancy}
\setlength{\headwidth}{\textwidth}
\fancyhfoffset[L,R]{0pt}
\lhead{\rightmark}
\rhead{7-ZPUs}
\lfoot{Norme di Progetto}
\rfoot{\thepage}
\cfoot{}
\renewcommand{\headrulewidth}{0.8pt}
\renewcommand{\footrulewidth}{0.8pt}

\renewcommand{\contentsname}{Indice}

\geometry{margin=2.5cm}
\setstretch{1.1}

\titleformat{\section}{\Large\bfseries}{\thesection}{1em}{}
\titleformat{\subsection}{\large\bfseries}{\thesubsection}{1em}{}
\titleformat{\subsubsection}{\normalsize\bfseries}{\thesubsubsection}{1em}{}
\titleformat{\paragraph}{\large\bfseries}{\theparagraph}{1em}{}
\titleformat{\subparagraph}{\normalsize\bfseries}{\thesubparagraph}{1em}{}

\begin{document}

\begin{center}
    \includegraphics[width=9.5cm]{../assets/logo7ZPUs.jpg}\\
    \small\hspace{10cm} 7zpus.swe@gmail.com\\
    \vspace{0.5cm}
    \Large \textbf{Norme di Progetto}\\
\end{center}

\vspace{0.3cm}
\hrule
\vspace{0.5cm}

\tableofcontents

\section*{Tabella di Versionamento}
\begin{table}[H]
    \begin{adjustwidth}{-4cm}{-4cm}
        \centering
        \begin{tabular}{|c|c|c|c|c|}
            \hline
            \textbf{Versione} & \textbf{Data} & \textbf{Autore} & \textbf{Verificatore} & \textbf{Descrizione}                                                                                         \\
            \hline
            0.1               & 16/11/2025    & Rocco Matteo A. & Soligo Lorenzo        & \begin{tabular}[c]{@{}c@{}} Creazione e stesura sezioni \\Introduzione e Processo di fornitura \end{tabular} \\
            \hline
        \end{tabular}
    \end{adjustwidth}
\end{table}

\newpage

\section{Introduzione}

\subsection{Scopo}
Questo documento ha l'obiettivo di definire e normare il \glossario{Way of
    Working}, ovvero le regole di lavoro che ogni membro del gruppo deve rispettare
durante lo svolgimento delle \glossario{attività di progetto} volte allo
sviluppo dell'applicativo software \glossario{\textbf{DIPReader}}, proposto
dall'azienda \glossario{Sanmarco Informatica}. A ciascun membro è richiesto di
seguirle integralmente per poter lavorare in maniera quanto più efficace ed
efficiente, oltre che omogenea. Data la natura incrementale di redazione del
documento, il \glossario{responsabile di progetto} in carica ha il compito di
mantenere aggiornate le presenti norme e eventuali riferimenti ad altri
documenti contenuti al loro interno.

\subsection{Glossario}
Ogni termine tecnico o con particolare significato nell'ambito
dell'\glossario{Ingegneria del Software} utilizzato nella documentazione di
progetto viene definito nell'apposito documento
\href{https://cdn.jsdelivr.net/gh/7-zpus/Docs@norme_in_lavorazione/2_RTB/Glossario.pdf}{\ul{Glossario
        1.0}\setulcolor{black}}\ped{(ultimo accesso: 17/11/2025)}.

\subsection{Riferimenti}
Il gruppo ha deciso di redigere il presente documento in conformità con lo
\glossario{standard} ISO/IEC 12207:1995, integrando occasionalmente con
approfondimenti contenuti nella sua versione più attuale ISO/IEC/IEEE
12207:2017 per includere dettagli aggiuntivi relativi agli approcci
\glossario{agili} e \glossario{iterativi} che contraddistinguono lo
\glossario{sviluppo software} moderno.

\subsubsection{Riferimenti Normativi}
\begin{itemize}
    \item \href{https://www.math.unipd.it/~tullio/IS-1/2009/Approfondimenti/ISO_12207-1995.pdf}{\ul{Standard ISO/IEC 12207:1995}\setulcolor{black}} \ped{(ultimo accesso: 17/11/2025)}
    \item \href{https://www.iso.org/standard/63712.html}{\ul{Standard ISO/IEC/IEEE 12207:2017}}
    \item \href{https://www.iso.org/standard/71952.html}{\ul{Standard ISO/IEC/IEEE 24765:2017}}
    \item \href{https://www.math.unipd.it/~tullio/IS-1/2025/Progetto/C3.pdf}{\ul{Capitolato C3: DIPReader}\setulcolor{black}} \ped{(ultimo accesso: 13/11/2025)}
    \item \href{https://www.math.unipd.it/~tullio/IS-1/2025/Dispense/PD1.pdf}{\ul{Regolamento di Progetto Didattico a.a. 2025/2026}\setulcolor{black}} \ped{(ultimo accesso: 17/11/2025)}
\end{itemize}

\subsubsection{Riferimenti Informativi}
\begin{itemize}
    \item Dispense del corso di Ingegneria del Software 2025/2026:
          \begin{itemize}
              \item \href{https://www.math.unipd.it/~tullio/IS-1/2025/Dispense/T01.pdf}{\ul{https://www.math.unipd.it/~tullio/IS-1/2025/Dispense/T01.pdf}\setulcolor{black}} \ped{(ultimo accesso: 17/11/2025)}
              \item \href{https://www.math.unipd.it/~tullio/IS-1/2025/Dispense/T02.pdf}{\ul{https://www.math.unipd.it/~tullio/IS-1/2025/Dispense/T02.pdf}\setulcolor{black}} \ped{(ultimo accesso: 17/11/2025)}
              \item \href{https://www.math.unipd.it/~tullio/IS-1/2025/Dispense/T03.pdf}{\ul{https://www.math.unipd.it/~tullio/IS-1/2025/Dispense/T03.pdf}\setulcolor{black}} \ped{(ultimo accesso: 17/11/2025)}
              \item \href{https://www.math.unipd.it/~tullio/IS-1/2025/Dispense/T04.pdf}{\ul{https://www.math.unipd.it/~tullio/IS-1/2025/Dispense/T04.pdf}\setulcolor{black}} \ped{(ultimo accesso: 17/11/2025)}
              \item \href{https://www.math.unipd.it/~tullio/IS-1/2025/Dispense/T05.pdf}{\ul{https://www.math.unipd.it/~tullio/IS-1/2025/Dispense/T05.pdf}\setulcolor{black}} \ped{(ultimo accesso: 17/11/2025)}
              \item \href{https://www.math.unipd.it/~tullio/IS-1/2025/Dispense/T06.pdf}{\ul{https://www.math.unipd.it/~tullio/IS-1/2025/Dispense/T06.pdf}\setulcolor{black}} \ped{(ultimo accesso: 17/11/2025)}
              \item \href{https://www.math.unipd.it/~tullio/IS-1/2025/Dispense/T07.pdf}{\ul{https://www.math.unipd.it/~tullio/IS-1/2025/Dispense/T07.pdf}\setulcolor{black}} \ped{(ultimo accesso: 17/11/2025)}
              \item \href{https://www.math.unipd.it/~tullio/IS-1/2025/Dispense/T08.pdf}{\ul{https://www.math.unipd.it/~tullio/IS-1/2025/Dispense/T08.pdf}\setulcolor{black}} \ped{(ultimo accesso: 17/11/2025)}
              \item \href{https://www.math.unipd.it/~tullio/IS-1/2025/Dispense/T09.pdf}{\ul{https://www.math.unipd.it/~tullio/IS-1/2025/Dispense/T09.pdf}\setulcolor{black}} \ped{(ultimo accesso: 17/11/2025)}
              \item \href{https://www.math.unipd.it/~tullio/IS-1/2025/Dispense/T10.pdf}{\ul{https://www.math.unipd.it/~tullio/IS-1/2025/Dispense/T10.pdf}\setulcolor{black}} \ped{(ultimo accesso: 17/11/2025)}
              \item \href{https://www.math.unipd.it/~tullio/IS-1/2025/Dispense/T11.pdf}{\ul{https://www.math.unipd.it/~tullio/IS-1/2025/Dispense/T11.pdf}\setulcolor{black}} \ped{(ultimo accesso: 17/11/2025)}
          \end{itemize}
    \item \href{https://www.agid.gov.it/it/sicurezza/cert-pa/linee-guida-sviluppo-del-software-sicuro}{\ul{Linee Guida Sviluppo Sicuro AGID (Agenzia per l'Italia Digitale)}\setulcolor{black}}
    \item  \href{https://www.agid.gov.it/it/linee-guida#index-3}{\ul{Linee Guida sulla formazione, gestione e conservazione dei documenti informatici AGID}\setulcolor{black}}
    \item \href{https://www.lorenzopantieri.net/LaTeX_files/LaTeXpedia.pdf}{\ul{Documentazione \LaTeX{} by Lorenzo Pantieri}\setulcolor{black}} \ped{(ultimo accesso: 17/11/2025)}
    \item \href{https://confluence.atlassian.com/jira}{\ul{Documentazione Jira}\setulcolor{black}}
\end{itemize}

\section{Processi Primari}

\subsection{Processo di Fornitura}
Il \glossario{processo} di fornitura contiene le attività e i compiti svolti
dal \glossario{fornitore}. Per implementare correttamente il processo il gruppo
si impegna a svolgere le seguenti attività.

\subsubsection{Attività di processo}

\subparagraph{Avvio}
Il fornitore analizza i \glossario{requisiti} necessari alla proposta di
fornitura, tenendo considerazione di eventuali vincoli organizzativi e
normativi.
\subparagraph{Preparazione della proposta di fornitura}
Il fornitore prepara la proposta di fornitura in risposta alle richieste del
committente e definisce i termini in cui si articola la proposta.
\subparagraph{Accordo}
Proponente e fornitore entrano nella fase di definizione dell'accordo di
fornitura del prodotto software, prevedendo possibilità di negoziazione della
fornitura da parte del fornitore.
\subparagraph{Pianificazione}
Il fornitore rielabora l'analisi dei requisiti fondamentali per definire il
\glossario{framework} entro il quale il prodotto verrà sviluppato e gestito, in
modo tale da garantire un processo di qualità durante lo sviluppo. Si impegna
inoltre a definire il modello del ciclo di vita del prodotto adatto alla
complessità del progetto e ai relativi rischi che potrebbero insorgere. Tutte
queste decisioni convergono nel Piano di Progetto.
\subparagraph{Esecuzione e controllo}
Il fornitore si impegna a sviluppare il prodotto secondo il Piano di Progetto,
avendo cura di controllare che i processi siano stati eseguiti correttamente.
\subparagraph{Verifica e validazione}
Il fornitore stabilisce con la proponente le modalità di rendicontazione dello
stato di avanzamento del prodotto e rende disponibili i documenti che
dimostrino la verifica e validazione dei processi secondo i requisiti
precedentemente individuati.
\subparagraph{Consegna e terminazione}
Il fornitore consegna il prodotto finale al proponente e ne espone le
funzionalità.

\subsubsection{Accordi con l'azienda proponente}
I capitolati presentati dalle proponenti vengono analizzati e viene redatto il
documento di
\href{https://cdn.jsdelivr.net/gh/7-zpus/Docs@main/1_Candidatura/AnalisiCapitolati.pdf}{\ul{Analisi
        dei capitolati}\setulcolor{black}}, nel quale sono delineati i bisogni e i
principali vincoli a cui attenersi per la fornitura del prodotto finale. Il
fornitore espone ai committenti di fornitura, ovvero i Professori Vardanega
Tullio e Cardin Riccardo, la Lettera di Presentazione della proposta di
fornitura che descrive il preventivo di costi, cronogramma di sviluppo,
suddivisione del lavoro e i ruoli coinvolti.

La \glossario{proponente}, in qualità di \glossario{stakeholder}, esercita il
diritto di ricevere la rendicontazione professionale e approfondita del lavoro
svolto dal gruppo fornitore, perciò si instaura un accordo per delineare le
modalità di comunicazione e il contenuto di tale rendicontazione. È previsto
l'aggiornamento costante e tempestivo della proponente per quanto riguarda la
pianificazione degli obiettivi e delle tempistiche di sviluppo individuate dal
fornitore. Ogni qualvolta vi siano modifiche di notevole interesse esterno dal
gruppo fornitore verranno comunicate all'azienda proponente attraverso appositi
canali di comunicazione sincrona o asincrona.

Il fornitore e la proponente hanno accordato lo svolgimento di un incontro di
verifica dello stato di avanzamento lavori (\glossario{SAL}) in modalità
sincrona ogni due settimane, in cui discutere l'andamento del lavoro e chiarire
eventuali dubbi da parte del fornitore o segnalazioni di difformità dai
requisiti iniziali della proponente. È inoltre sempre disponibile la
comunicazione via email per questioni minori e di facile risoluzione. La
consegna del prodotto è suddivisa in due \glossario{milestone} principali:
\glossario{RTB} (Requirements and Technology Baseline) e \glossario{PB}
(Product Baseline).

\subsubsection{Documentazione fornita}

\subparagraph{Analisi dei requisiti}
Nel documento di
\href{https://cdn.jsdelivr.net/gh/7-zpus/Docs@main/2_RTB/AnalisiDeiRequisiti.pdf}{\ul{Analisi
        dei requisiti}\setulcolor{black}} \ped{(ultimo accesso: 17/11/2025)} sono
riportati i bisogni e i vincoli a cui attenersi per la realizzazione del
prodotto finale. L'obiettivo è definire in maniera non ambigua i
\glossario{casi d'uso} (\textit{Use Cases}) e i requisiti
(\textit{Requirements}) del software. Il documento è diviso nelle seguenti
sezioni:
\begin{enumerate}
    \item Introduzione
    \item Descrizione
    \item Definizione dei casi d'uso
    \item Definizione dei requisiti
\end{enumerate}

\subparagraph{Glossario}
Il Glossario è il documento che raccoglie ogni termine di carattere tecnico,
nomenclature e acronimi con particolare significato nell'ambito dell'Ingegneria
del Software utilizzato nella documentazione di progetto. La definizione dei
termini di glossario è coadiuvata dal contenuto dello standard ISO/IEC/IEEE
24765/2017.

\subparagraph{Piano di progetto}
Il
\href{https://cdn.jsdelivr.net/gh/7-zpus/Docs@main/2_RTB/PianoDiProgetto.pdf}{\ul{Piano
        di progetto v1.0}\setulcolor{black}} \ped{(ultimo accesso: 17/11/2025)} è il
documento che espone all'esterno il lavoro di sviluppo svolto seguendo le
procedure delineate all'interno di questo documento. Fornisce una guida
dettagliata alla pianificazione, esecuzione e consuntivo delle attività
completate in ciascuna \glossario{sprint}. Il documento è diviso nelle seguenti
sezioni:
\begin{enumerate}
    \item Introduzione
    \item Analisi dei rischi e mitigazione
    \item Modello di sviluppo
    \item Pianificazione dei costi e suddivisione ruoli
    \item Preventivo di periodo
    \item Consuntivo di periodo
    \item Retrospettiva
\end{enumerate}

\subparagraph{Piano di qualifica}
Il piano di qualifica descrive gli obiettivi di qualità dei processi che il
fornitore si impegna a soddisfare per consegnare un prodotto finale di qualità.
Le metriche di valutazione vengono determinate dall'analisi dei requisiti e
dalle indicazioni date dalla proponente, suddivise in base all'applicazione sui
processi o sul prodotto. Le metriche stabilite vengono poi misurate attraverso
opportuni test e verifiche, di cui vengono riportate le specifiche. Il
documento include una sezione di rendicontazione per la valutazione dei
processi e la valutazione del prodotto, in cui riportare l'attinenza alle
metriche ottenuta rispetto agli obiettivi e di conseguenza valutare azioni
correttive in caso si verifichino eventuali problemi (\glossario{cruscotto di
    qualità}). Il documento è diviso nelle seguenti sezioni:
\begin{enumerate}
    \item Qualità dei processi
    \item Qualità del prodotto
    \item Specifiche di test e verifica
    \item Cruscotto di qualità
\end{enumerate}

\subparagraph{Lettera di presentazione}
La lettera di presentazione è il documento necessario alla candidatura per la
milestone di revisione di avanzamento \glossario{RTB} (\textit{Requirements and
    Technology Baseline}). Essa contiene le informazioni sul repository di
progetto, il puntatore al \glossario{Proof of Concept}, il consuntivo di spesa
e preventivo a finire del progetto.

\subsubsection{Strumenti}
\begin{itemize}
    \item \glossario{GitHub} per la gestione della documentazione di progetto e mezzo comunicativo nella fase di fornitura
    \item \glossario{Jira} per la suddivisione e il monitoraggio delle attività di progetto
    \item Discord per la comunicazione sincrona tra i membri del gruppo
    \item Gmail per la comunicazione asincrona con l'azienda proponente
\end{itemize}

\subsection{Processo di sviluppo}

\subsubsection{Attività di processo}

\subsection{Processo operativo}

\subsection{Processo di manutenzione}

\section{Processi di Supporto}
I processi di supporto sono volti a garantire l'efficacia e l'efficienza dei
processi primari.

\subsection{Processo di documentazione}
Il processo di documentazione è parte integrante del Progetto in quanto permette il tracciamento delle decisioni prese, delle attività svolte e dei risultati ottenuti. Tutto ciò al fine di favorire il lavoro asincrono tra membri del gruppo e promuovere il principio \glossario{Agile} di continuo miglioramento e adattamento tramite \glossario{feedback}.
\subsubsection{Strumenti a supporto}
Per la gestione della documentazione di progetto il gruppo utilizza i seguenti strumenti:
\begin{itemize}
    \item \glossario{GitHub}: repository centrale per la gestione della documentazione. Permette il versionamento dei documenti e grazie alle \glossario{pull request} anche la gestione di \glossario{Verifica} e \glossario{Approvazione} finale dei documenti.
    \item \glossario{\LaTeX}: linguaggio di markup utilizzato per la stesura dei documenti di progetto, in quanto permette di ottenere una formattazione professionale e uniforme tra i vari documenti. Per una stesura efficiente dei documenti il Team si è dotato di modelli predefiniti \ped{(Decisione del \href{https://cdn.jsdelivr.net/gh/7-zpus/Docs@main/2_RTB/Verbali/Verbali\%20Interni/2025-11-07-VerbaleInterno.pdf}{2025-11-07})}.
    \item \glossario{Jira}: strumento di gestione delle attività di progetto, utilizzato per tracciare le attività di documentazione e assegnarle ai membri del gruppo.
\end{itemize}

\subsubsection{Attività di processo}
Le attività principali del processo di documentazione sono:
\begin{itemize}
    \item \textbf{Pianificazione della documentazione}: definizione delle linee guida per la stesura dei documenti, inclusi formati, modelli e standard di qualità e assegnazione di della redazione ai membri del gruppo. Più in \ref{pianificazioneDocs}
    \item \textbf{Produzione della documentazione}: redazione dei documenti di progetto seguendo le linee guida stabilite, assicurando chiarezza, coerenza e completezza delle informazioni.Più in \ref{produzioneDocs}
    \item \textbf{Revisione e approvazione}: ogni documento redatto viene sottoposto a un processo di revisione interna da parte di un membro del gruppo diverso dall'autore.
   \end{itemize}
e seguono il seguente flusso:
\begin{figure}[H]
    \centering
    \includegraphics[width=0.7\textwidth]{../assets/ProcessoDocumentazione.png}
    \caption{Flusso del processo di documentazione}
\end{figure}
\subsubsection{Pianificazione della documentazione}\label{pianificazioneDocs}
La pianificazione della documentazione avviene contestualmente alla pianificazione delle attività di progetto. \\Durante la pianificazione di ogni \glossario{sprint}, il \glossario{responsabile di progetto} assegna le attività di documentazione ai membri del gruppo, tenendo conto delle competenze e della disponibilità di ciascuno. Le scadenze per la consegna dei documenti sono stabilite in modo da garantire che la documentazione sia sempre aggiornata e disponibile per la consultazione da parte del gruppo e di eventuali attori esterni (Azienda \glossario{proponente}, \glossario{committente}).
\\Per una più efficiente scrittura dei documenti, soprattutto di tutti quei documenti periodici (Verbali Interni, Verbali Esterni, Diario di Bordo) sono presenti modelli standard approvati in \href{https://github.com/7-ZPUs/Docs/tree/main/assets}{/assets}. L'aggiornamento di tali standard deve essere argomento di Verbali Interni e risultato di una discussione e succesiva decisione presa in tale sede.
\subsubsection{Produzione della documentazione}\label{produzioneDocs}
La produzione della documentazione, assegnata durante la pianificazione, è visibile all'assegnatario come \glossario{Work item} grazie all'estensione Jira di \ref*{fff}VSCode. Grazie a quest'ultima è possibile creare direttamente il Branch di lavoro che si baserà sulla feature branch principale. UNa volta completata la stesura, seguendo i modelli standard sopracitati, l'autore del documento crea una \glossario{(PR) Pull Request} verso la feature branch principale, assegnando come revisore il membro del gruppo designato, diverso da se. \\
A questo punto:
\begin{itemize}
\item Se il revisore \textbf{approva la PR}, questa branch viene automaticamente eliminata, il work item viene marcato come completato in Jira e l'assegnatario può proseguire con gli altri compiti a lui assegnati.
\item Se il revisore richiede modifiche \textbf{la PR viene bocciata} e l'assegnatario deve procedere con le modifiche richieste. Una volta completate, l'assegnatario notifica il revisore che procederà con una nuova revisione. Questo ciclo si ripete fino a quando la PR non viene approvata.
\end{itemize}
L'integrazione con Jira permette di controllare lo stato di avanzamento dei Work Item, la rendicontazione delle ore lavorate e la gestione delle scadenze.
Risulta quindi \textbf{obbligatorio} l'utilizzo di Smart Commit per tutti i commit, compresi quelli di Pull Request. Più in \ref{Jira}.
\subsubsection{Revisione e approvazione}
Ogni documento redatto viene sottoposto a un processo di revisione interna che ne accerta la correttezza contenutistica, formale, e stilistica. La revisione viene effettuata da un membro del gruppo diverso dall'autore del documento che al termine del processo può:
\begin{itemize}
    \item \textbf{Approvare il documento.} Questo porta all'apertura di una Pull Request verso la branch principale revisionata dal \glossario{responsabile} che quindi confermerà l'integrazione ai documenti approvati in via definitiva.
    \item \textbf{Richiedere modifiche.} In questo caso il \glossario{responsabile} fornisce un feedback dettagliato all'autore del documento, indicando le aree che necessitano di miglioramenti o correzioni. L'autore apporta le modifiche richieste e il documento viene nuovamente sottoposto a revisione.
\end{itemize}
A questo punto il responsabile effettua il merge della feature nel rame principale (verbali_in_lavorazione, norme_in_lavorazione, etc), base per la versione ufficiale di rilascio corrispondente alla \glossario{milestone}.
\subsection{Processo di garanzia della qualità}

\subsection{Processo di verifica}

\subsection{Processo di validazione}

\subsection{Processo di revisione congiunta}

\subsection{Processo di risoluzione dei problemi}

\subsection{Gestione della qualità}

\section{Processi Organizzativi}

\subsection{Gestione}

\subsection{Infrastruttura}

\subsection{Miglioramento}

\subsection{Formazione}

\section{Metriche della qualità}

\vfill
\begin{flushright}
    \textit{7-ZPUs}
\end{flushright}

\end{document}