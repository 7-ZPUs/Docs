\documentclass[11pt,a4paper]{article}
\usepackage[italian]{babel}
\usepackage[utf8]{inputenc}
\usepackage[T1]{fontenc}
\usepackage{geometry}
\usepackage{xcolor}
\usepackage[most]{tcolorbox}
\usepackage{graphicx}
\usepackage{amssymb}
\usepackage{wasysym}
\usepackage{hyperref}
\usepackage{enumitem}
\usepackage[explicit]{titlesec}
\usepackage{fancyhdr}

% Colori ZPUS - Verde, Nero, Bianco
\definecolor{zpusgreen}{RGB}{0, 150, 0}
\definecolor{zpusdarkgreen}{RGB}{0, 100, 0}
\definecolor{zpusblack}{RGB}{0, 0, 0}
\definecolor{zpuswhite}{RGB}{255, 255, 255}
\definecolor{zpuslightgray}{RGB}{245, 245, 245}

% Stili per i box migliorati
\newtcolorbox{headerbox}{
    colback=zpusgreen,
    colframe=zpusdarkgreen,
    arc=0pt,
    boxrule=0pt,
    left=0pt,
    right=0pt,
    top=8pt,
    bottom=8pt,
    fontupper=\color{zpuswhite}\bfseries\large,
    center
}

\newtcolorbox{infobox}{
    colback=zpuslightgray,
    colframe=zpusgreen,
    arc=4pt,
    boxrule=2pt,
    left=6pt,
    right=6pt,
    top=8pt,
    bottom=8pt,
    fontupper=\color{zpusblack}
}

\newtcolorbox{stepbox}{
    colback=zpuswhite,
    colframe=zpusgreen,
    arc=4pt,
    boxrule=1pt,
    left=6pt,
    right=6pt,
    top=8pt,
    bottom=8pt,
    fontupper=\color{zpusblack}
}

\newtcolorbox{highlightbox}{
    colback=zpusgreen!10,
    colframe=zpusdarkgreen,
    arc=4pt,
    boxrule=2pt,
    left=12pt,
    right=12pt,
    top=12pt,
    bottom=12pt,
    fontupper=\color{zpusblack}\bfseries,
    center
}

% Stili per le sezioni
\titleformat{\section}
  {\normalfont\LARGE\bfseries\color{zpusgreen}}
  {}
  {0em}
  {#1 \titlerule[2pt]}

\titleformat{\subsection}
  {\normalfont\Large\bfseries\color{zpusdarkgreen}}
  {}
  {0em}
  {#1}

% Intestazione e piè di pagina
\pagestyle{fancy}
\fancyhf{}
\fancyhead[L]{\color{zpusgreen}\textbf{ZPUS Engineering Team}}
\fancyhead[R]{\color{zpusgreen}\textbf{Guida Jira}}
\fancyfoot[C]{\color{zpusgreen}\thepage}
\renewcommand{\headrule}{\color{zpusgreen}\hrule width\hsize height 2pt}

% Simboli compatibili senza fontawesome
\newcommand{\info}{\textcolor{zpusgreen}{\textbf{\textbullet}}}
\newcommand{\task}{\textcolor{zpusgreen}{\textbf{>>}}}
\newcommand{\workflow}{\textcolor{zpusgreen}{\textbf{$\rightleftarrows$}}}
\newcommand{\commit}{\textcolor{zpusgreen}{\textbf{>}}}
\newcommand{\timeicon}{\textcolor{zpusgreen}{\textbf{\clock}}}
\newcommand{\plus}{\textcolor{zpusgreen}{\textbf{+}}}
\newcommand{\search}{\textcolor{zpusgreen}{\textbf{$\bigcirc$}}}
\newcommand{\play}{\textcolor{zpusgreen}{\textbf{$\blacktriangleright$}}}
\newcommand{\checkmarkicon}{\textcolor{zpusgreen}{\textbf{$\checkmark$}}}
\newcommand{\bulb}{\textcolor{zpusgreen}{\textbf{!}}}
\newcommand{\rocket}{\textcolor{zpusgreen}{\textbf{$\blacktriangle$}}}
\newcommand{\email}{\textcolor{zpusgreen}{\textbf{@}}}
\newcommand{\github}{\textcolor{zpusgreen}{\textbf{\#}}}

% Impostazioni generali
\geometry{margin=1.2in}
\setlength{\parindent}{0pt}
\setlist[itemize]{leftmargin=*,noitemsep}
\setlist[enumerate]{leftmargin=*,noitemsep}

\begin{document}

% Intestazione
\begin{headerbox}
    {\Huge \textbf{GUIDA JIRA}} \\
    \vspace{0.3cm}
    {\Large ZPUS Engineering Team}
\end{headerbox}

\vspace{1cm}

\begin{infobox}
    \textbf{Informazione}: Questa guida ti accompagnerà nell'utilizzo quotidiano di Jira con focus su gestione work item, workflow, smart commit e work log. Segui le indicazioni per ottimizzare il tuo workflow!
    Si presuppone la visione del Video Tutorial e l'utilizzo dell'estensione VSCode per Jira.
\end{infobox}

\section{\task~Amministrazione Work Item}

\begin{stepbox}
    \subsection{\textbf{\color{zpusgreen}Crea e Gestisci Ticket:}}
    \begin{itemize}
        \item \plus{} \textbf{Creazione}: Usa il pulsante ``Crea'' in alto a destra
        \item \task{} \textbf{Tipologia}: Seleziona il tipo appropriato (Task/Bug/Story)
        \item \textbf{\color{zpusgreen}Descrizione}: Compila i campi obbligatori con descrizione chiara
        \item \textbf{\color{zpusgreen}Assegnazione}: Assegna al membro del team corretto
        \item \textbf{\color{zpusgreen}Gerarchia}: Imposta la Gerarchia con il campo Linked Issues, selezionando child of o parent of in base alla situazione.
        \item \textbf{\color{zpusgreen}Altri Link}: Aggiungi ulteriori Linked Issues se necessario (relates to, blocks, ecc.)
    \end{itemize}
    I campi Linked Issues sono automatizzati, una volta impostata una relazione da un lato viene creata in automatico nell'altro Work Item.
\end{stepbox}

\vspace{0.5cm}

\begin{stepbox}
    \subsection{\textbf{\color{zpusgreen}Organizzazione Avanzata:}}
    \begin{itemize}
        \item \search{} \textbf{Ricerca}: Usa i filtri per trovare ticket specifici
        \item \textbf{\color{zpusgreen}Board}: Trascina tra le colonne per aggiornare lo stato
        \item \textbf{\color{zpusgreen}Allegati}: Allega file con drag-and-drop o Ctrl+V
        \item \textbf{\color{zpusgreen}Commenti}: Usa i commenti per aggiornamenti in tempo reale
        \item \textbf{\color{zpusgreen}Hierarchy Board}: Visualizza la gerarchia dei work item tramite la board dedicata. Le relazioni devono essere unicamente tra Relatives (Parent of/Child of) diretti.
        \item \textbf{\color{zpusgreen}Integrazione con VSCode}: Utilizza l'estensione per una gestione più fluida dei ticket direttamente dall'editor. Qui vedrai direttamente i work item assegnati a te.
    \end{itemize}
\end{stepbox}

\section{\workflow~Workflow dei Work Item}

\begin{highlightbox}
       \includegraphics[width=14cm]{../assets/workflowWorkItem.png}\\
\end{highlightbox}

\begin{stepbox}
    \textbf{\color{zpusgreen}Best Practices del Workflow:}
    \begin{itemize}
        \item \play{} \textbf{Inizia solo quando lavori}: Al primo commit di un Work Item, un'Automazione lo sposta da To DO a In Progress
        \item \commit{} \textbf{Code Review obbligatoria}: Usa questa fase per revisioni codice.
        \item \checkmarkicon{} \textbf{Completa tutto}: Segna ``Done'' solo quando tutto è completato e testato. La chiusura spetta al Verificatore.
        \item \textbf{Aggiornamenti giornalieri}: Aggiorna lo stato man mano che avanzi.
    \end{itemize}
\end{stepbox}

\section*{\commit~Smart Commit}

\begin{stepbox}
    \textbf{\color{zpusgreen}Sintassi Smart Commit:}
    \begin{verbatim}
# Chiudi ticket con commento:
[DIPR-123] #comment Messaggio commit

# Log tempo lavoro:
[DIPR-123] #time 2h 30m #comment Implementazione feature

# Transizione automatica:
[DIPR-123] #in-progress #comment Fix bug critico
\end{verbatim}
\end{stepbox}

\vspace{0.5cm}

\begin{stepbox}
    \textbf{\color{zpusgreen}Comandi Principali:}
    \begin{itemize}
        \item \timeicon{} \texttt{\#time Xh Ym} -- Registra tempo lavoro
        \item \texttt{\#comment} -- Aggiungi commento automatico
        \item \textbf{\color{zpusgreen} Transizioni}
              \begin{itemize}
                  \item \texttt{\#nome-transizione} -- Cambia stato del ticket seguendo la transizione
                  \item \texttt{\#done}, \texttt{\#in-progress}, \texttt{\#in-review} -- Transizioni rapide.\\ \textbf{\color{zpusgreen}NON CONSIGLIATE}
              \end{itemize}
    \end{itemize}
\end{stepbox}

\section*{\timeicon~Work Log}

\begin{stepbox}
    \textbf{\color{zpusgreen}Registrazione Professionale:}
    \begin{itemize}
        \item \textbf{\color{zpusgreen}Log Work}: Clicca sull'icona ``Log Work'' nel ticket
        \item \textbf{\color{zpusgreen}Data/Ora}: Inserisci data e ora spesa accuratamente
        \item \textbf{\color{zpusgreen}Note Descrittive}: Aggiungi note dettagliate sul lavoro svolto
        \item \textbf{\color{zpusgreen}Unità Consistenti}: Usa la stessa unità di misura (ore/giorni)
    \end{itemize}
\end{stepbox}

\vspace{0.5cm}

\begin{stepbox}
    \textbf{\color{zpusgreen}Suggerimenti Avanzati:}
    \begin{itemize}
        \item \textbf{\color{zpusgreen}Regolarità}: Logga il tempo giornalmente per maggiore accuratezza
        \item \textbf{\color{zpusgreen}Suddivisione}: Dividi attività complesse in più log
        \item \textbf{\color{zpusgreen}Report}: Rivedi i report sotto ``Tempo di Lavoro'' in progetti
        \item \textbf{\color{zpusgreen}Mobile}: Usa l'app mobile per log rapidi
    \end{itemize}
\end{stepbox}

\begin{infobox}
    \textbf{\bulb{} Pro Tip}: Integra Jira con gli strumenti di sviluppo (Git, VS Code) per un workflow più fluido e automatico! Massimizza l'efficienza con le integrazioni.
\end{infobox}

% Sezione finale
\begin{center}
    \begin{tcolorbox}[
            width=0.8\textwidth,
            colback=zpusgreen,
            colframe=zpusdarkgreen,
            arc=6pt,
            boxrule=0pt,
            left=12pt,
            right=12pt,
            top=12pt,
            bottom=12pt,
            fontupper=\color{zpuswhite}\bfseries\large
        ]
        {\Huge \rocket} \\
        \textbf{Ready to Jira?} \\
        Inizia ora a ottimizzare il tuo workflow!
    \end{tcolorbox}
\end{center}

\vspace{1cm}

\begin{center}
    \color{zpusgreen}
    \rule{0.3\textwidth}{1pt} \\
    \textbf{ZPUS Engineering Team} \\
    \vspace{0.2cm}
    \email{} \texttt{7-zpus.swe@gmail.com} \\
    \github{} \texttt{github.com/7-ZPUs} \\
\end{center}

\end{document}