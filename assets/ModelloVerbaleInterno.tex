\documentclass[a4paper,12pt]{article}
\usepackage[utf8]{inputenc}
\usepackage[T1]{fontenc} % per i caratteri accentati corretti in PDF
\usepackage[italian]{babel}
\usepackage{lmodern}
\renewcommand*\familydefault{\sfdefault}
\usepackage{float}
\usepackage{geometry}
\usepackage{xcolor}
\usepackage[most]{tcolorbox}
\usepackage{amssymb}
\usepackage{wasysym}
\usepackage{setspace}
\usepackage{chngpage}
\usepackage{enumitem}
\usepackage{titlesec}
\usepackage{tocloft}
\usepackage{graphicx}
\usepackage{hyperref}
\usepackage{fancyhdr}
\hypersetup{
    colorlinks=true,
    linkcolor=black,
    filecolor=magenta,      
    urlcolor=cyan,
}

% Colori ZPUS - Verde, Nero, Bianco
\definecolor{zpusgreen}{RGB}{4, 138, 55}
\definecolor{zpusdarkgreen}{RGB}{0, 100, 0}
\definecolor{zpusblack}{RGB}{0, 0, 0}
\definecolor{zpuswhite}{RGB}{255, 255, 255}
\definecolor{zpuslightgray}{RGB}{245, 245, 245}

% Stili per i box migliorati
\newtcolorbox{headerbox}{
    colback=zpusgreen,
    colframe=zpusdarkgreen,
    arc=0pt,
    boxrule=0pt,
    left=0pt,
    right=0pt,
    top=8pt,
    bottom=8pt,
    fontupper=\color{zpuswhite}\bfseries\large,
    center
}

\newtcolorbox{infobox}{
    colback=zpuslightgray,
    colframe=zpusgreen,
    arc=4pt,
    boxrule=2pt,
    left=6pt,
    right=6pt,
    top=8pt,
    bottom=8pt,
    fontupper=\color{zpusblack}
}

\newtcolorbox{stepbox}{
    colback=zpuswhite,
    colframe=zpusgreen,
    arc=4pt,
    boxrule=1pt,
    left=6pt,
    right=6pt,
    top=8pt,
    bottom=8pt,
    fontupper=\color{zpusblack}
}

\newtcolorbox{highlightbox}{
    colback=zpusgreen!10,
    colframe=zpusdarkgreen,
    arc=4pt,
    boxrule=2pt,
    left=12pt,
    right=12pt,
    top=12pt,
    bottom=12pt,
    fontupper=\color{zpusblack}\bfseries,
    center
}

\pagestyle{fancy}
\setlength{\headwidth}{\textwidth}
\fancyhfoffset[L,R]{0pt}
\lhead{}
\rhead{7-ZPUs}
\lfoot{}
\rfoot{\thepage}
\cfoot{}
\renewcommand{\headrulewidth}{0.8pt}
\renewcommand{\footrulewidth}{0.8pt}

\renewcommand{\contentsname}{Indice}

\geometry{margin=2.5cm}
\setstretch{1.2}

\titleformat{\section}{\large\bfseries}{\thesection}{1em}{}
\titleformat{\subsection}{\mdseries\bfseries}{\thesubsection}{1em}{}

\begin{document}

\begin{center}
    \includegraphics[width=9.5cm]{../../../assets/logo7zpus.jpg}\\
    \small\hspace{10cm} 7zpus.swe@gmail.com\\
    \Large \textbf{Verbale Interno Gruppo di Progetto}\\
    \vspace{0.5cm}
\end{center}

\noindent
\textbf{Data:} 07/11/2025 \\
\textbf{Durata:} 2 ore\\
\textbf{Luogo:} Incontro online (Discord)

\vspace{0.3cm}
\hrule
\vspace{0.5cm}

\tableofcontents

\newpage


\section*{Tabella di Versionamento}
\begin{table}[H]
    \begin{adjustwidth}{-1cm}{-1cm} % modificare ogni volta in base alla larghezza della tabella per centrarla!!!
    \centering
\begin{tabular}{|c|c|c|c|c|}
    \hline
    \textbf{Versione} & \textbf{Data} & \textbf{Autore}  & \textbf{Verificatore} & \textbf{Descrizione} \\
    \hline
    0.1 & 07/11/2025 & Autore & Verificatore & Creazione del verbale e stesura iniziale \\
    \hline
\end{tabular}
    \end{adjustwidth}
\end{table}

\section*{Partecipanti}
\begin{itemize}[noitemsep]
    \item Fattoni Antonio 
    \item Georgescu Diana
    \item Gingilino Aaron
    \item Laoud Zakaria
    \item Rocco Matteo Alberto
    \item Soligo Lorenzo
    \item Vigolo Davide
\end{itemize}

\section{Ordine del Giorno}
\begin{enumerate}[noitemsep]
    \item Lorem ipsum dolor sit amet
    \item Lorem ipsum dolor sit amet
\end{enumerate}


\section{Svolgimento e Discussione}

\subsection{Eventuali aggiornamenti o collegamenti al verbale precedente}
Nel senso che può essere che sia necessario spiegare se qualcosa è cambiato rispetto alle decisioni del verbale precedente e magari non riteniamo necessario riportarlo nuovamente come ordine del giorno con relativa sottosezione, nuova decisione eccetera.

\subsection{Definizione ruoli}
I ruoli necessari in questa sprint sono:
\begin{itemize}[noitemsep]
    \item ???
    \item ???
    \item ???
    \item ???
\end{itemize}

I ruoli assegnati a ciascun membro per questo sprint sono i seguenti:
\begin{table}[H]
    \begin{adjustwidth}{-1cm}{-1cm}
    \centering
\begin{tabular}{|c|c|}
    \hline
    \textbf{Membro} & \textbf{Ruolo} \\
    \hline
    Fattoni Antonio & ??? \\
    \hline
    Georgescu Diana & ??? \\
    \hline
    Gingilino Aaron & ??? \\
    \hline
    Laoud Zakaria & ??? \\
    \hline
    Rocco Matteo A. & ??? \\
    \hline
    Soligo Lorenzo & ??? \\
    \hline
    Vigolo Davide & ??? \\
    \hline
\end{tabular}
    \end{adjustwidth}
\end{table}

\subsection{Sottosezioni corrispondenti agli ordini del giorno}
Creare una sottosezione per ogni punto da trattare.

\section{Eventuali argomenti non affrontati}
Possiamo inserirci o gli ordini del giorno che non si è riusciti a discutere completamente (e quindi la sezione degli ordini del giorno sono più una dichiarazione di intenti di cosa discutere più che un sommario degli argomenti trattati) oppure argomenti che sappiamo saranno prossimimente trattati.

\section{Decisioni}
\begin{enumerate}[noitemsep]
    \item Lorem ipsum dolor sit amet
    \item Lorem ipsum dolor sit amet
    \item Lorem ipsum dolor sit amet
    \item Lorem ipsum dolor sit amet
    \item Lorem ipsum dolor sit amet
    \item Lorem ipsum dolor sit amet
\end{enumerate}

\vspace{2cm}

\begin{infobox}
Considerare di volta in volta a fine stesura se/dove inserire dei newpage.
\end{infobox}


\section*{Tabella delle decisioni}
\begin{table}[H]
    \begin{adjustwidth}{-4cm}{-4cm} % modificare ogni volta in base alla larghezza della tabella per centrarla!!!
    \centering
\begin{tabular}{|c|c|c|c|c|}
    \hline
    \textbf{Decisione} & \textbf{To Do} & \textbf{Jira Issue} & \textbf{Membro assegnato} & \textbf{Verificatore} \\
    \hline
    \#0 & Redazione del verbale & \href{https://7zpus.atlassian.net/browse/DIPR-43}{DIPR-43} & Membro & Verificatore \\
    \hline
    \#2 & Decisione & \href{https://7zpus.atlassian.net/browse/DIPR-61}{DIPR-61} & Membro & Verificatore \\
    \hline
    \#3 & Decisione & \href{https://7zpus.atlassian.net/browse/DIPR-20}{DIPR-20} & Membro & Verificatore \\
    \hline
    \#4 & Decisione & \href{https://7zpus.atlassian.net/browse/DIPR-24}{DIPR-24} & Membro & \begin{tabular}[c]{@{}c@{}} Rocco Matteo A.\\ Verificatore \end{tabular} \\
    \hline
    \#4 & Decisione & \href{https://7zpus.atlassian.net/browse/DIPR-25}{DIPR-25} & Membro & \begin{tabular}[c]{@{}c@{}} Rocco Matteo A.\\ Verificatore \end{tabular} \\
    \hline
    \#4 & \begin{tabular}[c]{@{}c@{}} Argomento che occupa \\troppo spazio \end{tabular} & \href{https://7zpus.atlassian.net/browse/DIPR-26}{DIPR-26} & Membro & \begin{tabular}[c]{@{}c@{}} Verificatore 1\\ Verificatore 2 \end{tabular} \\
    \hline
    \#5 & \begin{tabular}[c]{@{}c@{}} Argomento che occupa \\troppo spazio \end{tabular} & \begin{tabular}[c]{@{}c@{}} \href{https://7zpus.atlassian.net/browse/DIPR-56}{DIPR-56}\\\href{https://7zpus.atlassian.net/browse/DIPR-57}{DIPR-57} \end{tabular} & Membro & Verificatore \\
    \hline
    \#6 & \begin{tabular}[c]{@{}c@{}} Argomento che occupa \\troppo spazio  \end{tabular} & \href{https://7zpus.atlassian.net/browse/DIPR-53}{DIPR-53} & Membro & Verificatore\\
    \hline
    \#7 & \begin{tabular}[c]{@{}c@{}} Argomento che occupa \\troppo spazio \end{tabular} & \begin{tabular}[c]{@{}c@{}} \href{https://7zpus.atlassian.net/browse/DIPR-51}{DIPR-51}\\\href{https://7zpus.atlassian.net/browse/DIPR-52}{DIPR-52} \end{tabular} & Membro & Verificatore \\
    \hline
\end{tabular}
    \end{adjustwidth}
\end{table}

\vfill
\begin{flushright}
    \textit{7-ZPUs}
\end{flushright}

\end{document}
